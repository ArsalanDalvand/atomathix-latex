
% ---------------------------------------------------------------------
% Copyright (c) 2026 Arsalan Dalvand & Reyhaneh Darvishi.
% Licensed under CC BY-NC-SA 4.0.
% See LICENSE file for details.
% ---------------------------------------------------------------------

\documentclass[12pt,a4paper,twoside,openright]{book}
\tracinglostchars=0
\usepackage{amsmath, amssymb, amsthm}
\usepackage{graphicx}
\usepackage{float}
\usepackage[svgnames]{xcolor}
\usepackage{calc}
\usepackage{etoolbox}
\usepackage{geometry}
\geometry{
    top=30mm, bottom=30mm, 
    left=25mm, right=25mm,
    headheight=40pt
}
\usepackage[
    colorlinks=true,
    linkcolor=NavyBlue,
    urlcolor=DeepPink,
    citecolor=DarkGreen,
    pdfauthor={Reyhaneh Darvishi and Arsalan Dalvand},
    pdftitle={مبانی ریاضی مقدماتی},
    pdfsubject={جزوه اتمیک درسی ریاضی - Copyright (c) 2026 CC BY-NC-SA 4.0}, 
    pdfkeywords={لاتک, ریاضی, دالوند, درویشی},
    pdfcreator={Arsalan Dalvand and Reyhaneh Darvishi}
]{hyperref}
\usepackage[nameinlink]{cleveref}
\usepackage[type={CC}, modifier={by-nc-sa}, version={4.0}]{doclicense}
\usepackage{tikz}
\usetikzlibrary{calc, positioning, shapes.geometric, backgrounds, shadows}
\usepackage{booktabs}
\usepackage{array}
\usepackage{multirow}
\usepackage{longtable}
\usepackage{titlesec}
\usepackage{fancyhdr}
\usepackage{qrcode}
\newcounter{none} 
\def\LTcaptype{table}
\providecommand{\tightlist}{%
  \setlength{\itemsep}{0pt}\setlength{\parskip}{0pt}}
\usepackage[most]{tcolorbox}
\tcbset{
    enhanced, 
    boxrule=0.5mm, 
    arc=3mm, 
    left=4mm, right=4mm, top=3mm, bottom=3mm,
    fonttitle=\bfseries\large,
    before skip=5mm, after skip=5mm
}
\newtcolorbox{tldr}[1]{
    colback=Honeydew, colframe=ForestGreen, title={$\blacktriangleleft$ #1} 
}
\newtcolorbox{theorembox}[1]{
    colback=AliceBlue, colframe=RoyalBlue, title={قضیه: #1}, attach boxed title to top right={yshift=-3mm, xshift=-3mm}
}
\newtcolorbox{warning}[1]{
    colback=MistyRose, colframe=Crimson, title={هشدار: #1}
}
\newtcolorbox{note}[1]{
    colback=LightYellow, colframe=Orange, title={نکته: #1}, coltitle=black
}
\newtcolorbox{tip}[1]{
    colback=MintCream, colframe=Teal, title={#1}
}
\newtcolorbox{info}[1]{
    colback=WhiteSmoke, colframe=DimGray, title={#1}
}
\newtcolorbox{quotebox}[1]{
    colback=GhostWhite, colframe=LightGray, title={نقل قول}, coltitle=black
}
\usepackage{xepersian}
\settextfont[Path=./fonts/, Extension=.ttf, UprightFont=Sahel, BoldFont=Sahel-Bold, ItalicFont=Sahel, BoldItalicFont=Sahel-Bold]{Sahel}
\setlatintextfont[Path=./fonts/, Extension=.ttf, UprightFont=Sahel, BoldFont=Sahel-Bold]{Sahel}
\setdigitfont[Path=./fonts/, Extension=.ttf]{Sahel}
\linespread{1.5} 
\setlength{\parindent}{0pt} 
\setlength{\parskip}{0.8em} 
\setcounter{secnumdepth}{1} 
\setcounter{tocdepth}{1}
\def\chaptername{فصل}
\def\contentsname{فهرست مطالب}
\titleformat{\section}
  {\Large\bfseries\color{DarkGreen}}
  {\thesection}{1em}{}
\pagestyle{fancy}
\fancyhf{}
\fancyhead[RE,LO]{\bfseries\thepage} 
\fancyhead[LE]{\small \leftmark}
\fancyhead[RO]{\small \rightmark}
\renewcommand{\headrulewidth}{1pt}
\renewcommand{\headrule}{\hbox to\headwidth{\color{NavyBlue}\leaders\hrule height \headrulewidth\hfill}}
\title{\textbf{\Huge مبانی ریاضی مقدماتی}}
\author{\Large ارسلان دالوند \and \Large ریحانه درویشی}\date{\today}
\begin{document}
\begin{titlepage}
    \newgeometry{top=1cm, bottom=1cm, left=1cm, right=1cm}


    \begin{tikzpicture}[remember picture, overlay]
        \fill[white] (current page.south west) rectangle (current page.north east);

        \fill[NavyBlue] (current page.north east) rectangle ($(current page.south east) + (-2.5cm, 0)$);
        \fill[Orange] ($(current page.north east) + (-2.5cm, 0)$) rectangle ($(current page.south east) + (-2.6cm, 0)$);


        \begin{scope}[shift={(current page.north west)},opacity=0.4, color=NavyBlue]
            
            \tikzstyle{node_style}=[circle, fill=NavyBlue!10, draw=NavyBlue, thick, inner sep=2.5pt]
            \tikzstyle{edge_style}=[draw=NavyBlue!25, line width=1.2pt]


            

            \node[node_style] (N1) at (2, -1.5) {$\pi$};
            \node[node_style] (N2) at (6, -0.8) {};
            \node[node_style] (N3) at (10, -1.2) {$\forall$};
            \node[node_style] (N4) at (15, -0.5) {};
            \node[node_style] (N5) at (18, -1.5) {$\in$};

            \node[node_style] (N6) at (1, -4) {};
            \node[node_style] (N7) at (4, -3) {$\Sigma$};
            \node[node_style] (N8) at (8, -5) {$\infty$};
            \node[node_style] (N9) at (13, -2.5) {};
            \node[node_style] (N10) at (16, -4.5) {$\subset$};
            \node[node_style] (N12) at (3, -6.5) {};
            \node[node_style] (N13) at (7, -7) {$\emptyset$};
            \node[node_style] (N14) at (10, -6.5) {};
            \node[node_style] (N15) at (17, -7.5) {$\int$};

            \draw[edge_style, opacity=0.4] (N1) -- (N7) -- (N6);
            \draw[edge_style, opacity=0.4] (N1) -- (N2) -- (N3) -- (N4) -- (N5);
            \draw[edge_style, opacity=0.4] (N2) -- (N7) -- (N3);
            \draw[edge_style, opacity=0.4] (N3) -- (N8);
            \draw[edge_style, opacity=0.4] (N9) -- (N3);
            \draw[edge_style, opacity=0.4] (N9) -- (N4);
            \draw[edge_style, opacity=0.4] (N9) -- (N10);
            \draw[edge_style, opacity=0.4] (N4) -- (N10) -- (N5);
            \draw[edge_style, opacity=0.4] (N10) -- (N5);
            \draw[edge_style, opacity=0.4] (N6) -- (N12) -- (N7);
            \draw[edge_style, opacity=0.4] (N8) -- (N13) -- (N7);
            \draw[edge_style, opacity=0.4] (N8) -- (N14);
            \draw[edge_style, opacity=0.4] (N10) -- (N15);

            \draw[edge_style, opacity=0.4] (N1) -- (0, -0.5);
            \draw[edge_style, opacity=0.4] (N12) -- (0, -8);
            \draw[edge_style, opacity=0.4] (N3) -- (11, 0);
            \draw[edge_style, opacity=0.4] (N2) -- (4, 0);
            \draw[edge_style, opacity=0.4] (N4) -- (17, 0);


            
        \end{scope}
    \end{tikzpicture}

    \begin{center}
        
        \includegraphics[width=3cm]{IUST-University-Logo.pdf} 
        
        \vspace{2.5cm} 
        
        {\fontsize{35}{45}\selectfont \bfseries \color{NavyBlue} مبانی ریاضی مقدماتی} \\
        \vspace{0.7cm}
        {\Large \bfseries \color{DarkGreen} (مبانی منطق، مجموعه و نظریه اعداد)}
        
        \vspace{1.5cm}
        
        \textcolor{Orange}{\rule{11cm}{2pt}}
        
        \vspace{2cm}
        
        {\large \textbf{بر اساس درس‌گفتارهای:}} \\
        \vspace{0.3cm}
        {\Large استاد هنگامه تمیمی}
        
        \vspace{2cm}
        
        {\large \textbf{گردآوری و تألیف:}} \\
        \vspace{0.5cm}
        {\Large ریحانه درویشی \hspace{2cm} ارسلان دالوند}
        
        \vfill
        
        \begin{tcolorbox}[
            enhanced,
            colback=AliceBlue!95!white, 
            colframe=NavyBlue, 
            width=8.5cm, 
            arc=3mm, 
            boxrule=0.8mm,
            halign=center,
            drop shadow
        ]
            \textbf{\large اکوسیستم آموزشی Atomathix} \\
            \vspace{0.3cm}
            \qrcode[height=2.2cm]{https://atomathix.ir/set-theo/} \\
            \vspace{0.2cm}
            \href{https://atomathix.ir/set-theo/}{\lr{\Large \textbf{atomathix.ir/set-theo/}}}
        \end{tcolorbox}
        
        \vspace{0.5cm}
        {\small زمستان ۱۴۰۴}
    \end{center}

    \restoregeometry
\end{titlepage}
\frontmatter
\thispagestyle{empty}
\vspace*{\fill}
\begin{tcolorbox}[colback=white, colframe=gray, title={حقوق و دسترسی}]
    این اثر توسط \textbf{ریحانه درویشی و ارسلان دالوند} کدنویسی و تالیف شده است.
    \vspace{0.2cm}
    \begin{latin}
        \doclicenseThis 
    \end{latin}
\end{tcolorbox}
\vspace*{\fill}
\clearpage
\thispagestyle{empty}
\vspace*{7cm}
\begin{center}
    \large 
    \textbf{تقدیم به} \\
    \vspace{0.8cm}
    {\Large \textit{الماس‌های نایاب زندگی‌تان،}} \\
    \vspace{0.5cm}
    \textit{و آنان که تعالیِ خود را در یاری رساندن به دیگران می‌جویند.}
\end{center}
\clearpage
\chapter*{پیشگفتار: تلفیق سنت و تکنولوژی}
\addcontentsline{toc}{chapter}{پیشگفتار}
این کتاب که پیش روی شماست، تلاشی است برای آشتی دادن «دقت و ساختار ریاضیات کلاسیک» با «شیوه تفکر مدرن و شبکه‌ای».
محتوای این اثر بر اساس درس‌گفتارهای استاد \textbf{هنگامه تمیمی} در درس «مبانی ریاضی مقدماتی» تدوین شده است. ساختار علمی و آموزشی کتاب، مدیون تدریس ایشان است.
\section*{معماری این کتاب}
آنچه این اثر را متمایز می‌کند، روش تألیف آن است. این کتاب بر پایه متدولوژی \textbf{«یادداشت‌برداری اتمیک» \lr{(Atomic Notes)}} و سیستم \textbf{Zettelkasten} بنا شده است. ما بر این باوریم که دانش ریاضی، یک رشته خطی نیست، بلکه شبکه‌ای درهم‌تنیده از مفاهیم است.
به همین منظور، نسخه کامل و پویای این اثر در وبسایت \href{https://atomathix.ir/set-theo/}{\textbf{\lr{Atomathix.ir/set-theo/}}} توسعه داده می‌شود. در حالی که نسخه چاپی (همین کتاب) برای مطالعه عمیق، تدریس در کلاس و مرور ساختارمند (خطی) طراحی شده است، نسخه وبسایت امکانات زیر را فراهم می‌کند:
\begin{itemize}
    \item \textbf{نقشه دانش \lr{(Graph View)}:} مشاهده ارتباطات پنهان بین قضایا و تعاریف به صورت بصری.
    \item \textbf{لینک‌دهی دوجانبه:} امکان پرش سریع بین مفاهیم مرتبط و پیش‌نیازها.
    \item \textbf{جستجوی معنایی:} یافتن سریع تمام قضایایی که به یک مفهوم خاص اشاره دارند.
\end{itemize}
لذا به دانشجویان عزیز پیشنهاد می‌شود برای درک عمیق‌تر ارتباطات بین‌مفهومی، از نقشه ذهنی موجود در وبسایت نیز بهره ببرند.
\section*{سپاسگزاری}
ضمن تشکر مجدد از استاد محترم که راهنمای ما بودند، امیدواریم این مجموعه (وبسایت و کتاب) بتواند گامی کوچک در جهت مدرن‌سازی منابع آموزشی ریاضی بردارد.
\vspace{2cm}
\hfill \textbf{ریحانه درویشی و ارسلان دالوند} \\
\hfill زمستان ۱۴۰۴
\tableofcontents
\mainmatter
\cleardoublepage
\thispagestyle{empty}
\begin{tcolorbox}[colback=AliceBlue, colframe=NavyBlue, title={راهنمای استفاده از اکوسیستم Atomathix}]
    دانشجوی گرامی، این محتوا در دو قالب ارائه شده است:
    \begin{enumerate}
        \item \textbf{کتاب فیزیکی/PDF:   } 
        زمانی که می‌خواهید درس را یاد بگیرید، اثبات‌ها را خط‌به‌خط دنبال کنید و برای امتحان آماده شوید، از این نسخه استفاده کنید. ترتیب فصول بر اساس منطق آموزشی چیده شده است.
        \item \textbf{وبسایت Atomathix.ir  :}
        زمانی که می‌خواهید بدانید «این قضیه کجا کاربرد دارد؟» یا «پیش‌نیاز این تعریف چیست؟»، به وبسایت مراجعه کنید. روی گراف مفاهیم زوم کنید و مسیر یادگیری شخصی خود را بسازید.
    \end{enumerate}
\end{tcolorbox}
\clearpage
\titleformat{\chapter}[display]
  {\normalfont\huge\bfseries} 
  {
    \begin{tikzpicture}
      \draw[NavyBlue, line width=3pt] (1.5, 0) -- (-13, 0);
      \node[
          circle, 
          draw=Orange,
          line width=2pt,
          fill=NavyBlue, 
          minimum size=2.8cm,
          inner sep=0pt,
          drop shadow
      ] (chapnode) at (0,0) {
          \color{white}\fontsize{50}{60}\selectfont\bfseries\thechapter
      };
      \node[
          fill=white, 
          text=NavyBlue, 
          rounded corners=3pt, 
          inner sep=3pt,
          anchor=south
      ] at (chapnode.north) {\small \bfseries فصل};
      \fill[Orange] (-13, 0) circle (4pt);
    \end{tikzpicture}
  }
  {10pt}
  {
    \vspace{-0.5cm} 
    \raggedright 
    \fontsize{40}{50}\selectfont 
    \color{NavyBlue} 
  }
  [] 
\chapter{منطق مقدماتی}
\clearpage
% ---------------------------------------------------------------------
% Copyright (c) 2026 Arsalan Dalvand & Reyhaneh Darvishi.
% Licensed under CC BY-NC-SA 4.0.
% See LICENSE file for details.
% ---------------------------------------------------------------------

\section{قضیه ۱: قوانین جمع و اختصار (با تکیه بر گزاره‌های عطفی و
فصلی)}\label{قضیه-۱---قوانین-جمع-و-اختصار}
\begin{tldr}{خلاصه سریع}
این قضیه مجوزهای ورود و خروج اطلاعات در منطق است:
\begin{itemize}
\tightlist
\item
  \textbf{جمع \lr{(Addition):}} اگر یک حقیقت (\(p\)) دارید، می‌توانید آن
  را با هر چیز دیگری در یک «ترکیب فصلی» (یا) جمع کنید.
\item
  \textbf{اختصار \lr{(Simplification):}} اگر یک «ترکیب عطفی» (و) دارید،
  می‌توانید اجزای آن را جدا کرده و به عنوان حقیقت استفاده کنید.
\end{itemize}
\end{tldr}
\subsection{۱. تعاریف پایه: گزاره‌های عطفی و
فصلی}\label{ux62aux639ux627ux631ux6ccux641-ux67eux627ux6ccux647-ux6afux632ux627ux631ux647ux647ux627ux6cc-ux639ux637ux641ux6cc-ux648-ux641ux635ux644ux6cc}
برای درک این قضیه، ابتدا باید تعاریف دقیق عملگرهای \(\wedge\) و \(\vee\)
را طبق جداول ارزش بدانیم.
\subsubsection{\texorpdfstring{الف) ترکیب عطفی \lr{(Conjunction) }-
عملگر
\(\wedge\)}{الف) ترکیب عطفی - عملگر \textbackslash wedge}}\label{ux627ux644ux641-ux62aux631ux6a9ux6ccux628-ux639ux637ux641ux6cc-conjunction---ux639ux645ux644ux6afux631-wedge}
رابط \(\wedge\) بین دو گزاره \(p\) و \(q\) قرار می‌گیرد و گزاره مرکب
\(p \wedge q\) را می‌سازد. این ترکیب تنها زمانی \textbf{راست \lr{(T)}}
است که \textbf{هر دو مؤلفه} راست باشند.
{\def\LTcaptype{none}
\begin{longtable}[]{@{}ccc@{}}
\toprule\noalign{}
\(p\) & \(q\) & \(p \wedge q\) \\
\midrule\noalign{}
\endhead
\bottomrule\noalign{}
\endlastfoot
\lr{T} & \lr{T} & \textbf{\lr{T}} \\
\lr{T} & \lr{F} & \lr{F} \\
\lr{F} & \lr{T} & \lr{F} \\
\lr{F} & \lr{F} & \lr{F} \\
\emph{(جدول ۲ - ارزش‌های ترکیب عطفی)} & & \\
\end{longtable}
}
\subsubsection{\texorpdfstring{ب) ترکیب فصلی \lr{(Disjunction) }- عملگر
\(\vee\)}{ب) ترکیب فصلی - عملگر \textbackslash vee}}\label{ux628-ux62aux631ux6a9ux6ccux628-ux641ux635ux644ux6cc-disjunction---ux639ux645ux644ux6afux631-vee}
رابط \(\vee\) برای تشکیل گزاره مرکب \(p \vee q\) به کار می‌رود. این رابط
به معنای «یای شمول» است؛ یعنی اگر \textbf{حداقل یکی} از مؤلفه‌ها راست
باشد، کل گزاره راست است (فقط وقتی دروغ است که هر دو دروغ باشند).
{\def\LTcaptype{none}
\begin{longtable}[]{@{}ccc@{}}
\toprule\noalign{}
\(p\) & \(q\) & \(p \vee q\) \\
\midrule\noalign{}
\endhead
\bottomrule\noalign{}
\endlastfoot
\lr{T} & \lr{T} & \lr{T} \\
\lr{T} & \lr{F} & \lr{T} \\
\lr{F} & \lr{T} & \lr{T} \\
\lr{F} & \lr{F} & \textbf{\lr{F}} \\
\emph{(جدول ۴ - ارزش‌های ترکیب فصلی)} & & \\
\end{longtable}
}
\begin{center}\rule{0.5\linewidth}{0.5pt}\end{center}
\subsection{۲. متن ریاضی
قضیه}\label{ux645ux62aux646-ux631ux6ccux627ux636ux6cc-ux642ux636ux6ccux647}
فرض کنید \(p\) و \(q\) دو گزاره دلبخواه باشند.
\begin{theorembox}{قضیه ۱}
\textbf{الف) قانون جمع \lr{(Addition):}} \[p \Rightarrow (p \vee q)\]
\textbf{ب) قانون اختصار \lr{(Simplification):}}
\[(p \wedge q) \Rightarrow p\]
\end{theorembox}
\subsection{۳. تحلیل و اثبات با جدول
ارزش}\label{ux62aux62dux644ux6ccux644-ux648-ux627ux62bux628ux627ux62a-ux628ux627-ux62cux62fux648ux644-ux627ux631ux632ux634}
\subsubsection{\texorpdfstring{اثبات قانون اختصار
(\((p \wedge q) \Rightarrow p\))}{اثبات قانون اختصار ((p \textbackslash wedge q) \textbackslash Rightarrow p)}}\label{ux627ux62bux628ux627ux62a-ux642ux627ux646ux648ux646-ux627ux62eux62aux635ux627ux631-p-wedge-q-rightarrow-p}
به جدول ۲ (ترکیب عطفی) نگاه کنید.
\begin{enumerate}
\def\labelenumi{\arabic{enumi}.}
\tightlist
\item
  فرض می‌کنیم مقدم استدلال یعنی \((p \wedge q)\) درست باشد.
\item
  طبق \textbf{جدول ۲}، تنها حالتی که \(p \wedge q\) مقدار
  \textbf{\lr{T}} دارد، ردیف اول است.
\item
  در ردیف اول، ارزش \(p\) نیز حتماً \textbf{\lr{T}} است.
\item
  بنابراین، راستیِ \((p \wedge q)\) لزوماً راستیِ \(p\) را تضمین می‌کند.
\end{enumerate}
\subsubsection{\texorpdfstring{اثبات قانون جمع
(\(p \Rightarrow (p \vee q)\))}{اثبات قانون جمع (p \textbackslash Rightarrow (p \textbackslash vee q))}}\label{ux627ux62bux628ux627ux62a-ux642ux627ux646ux648ux646-ux62cux645ux639-p-rightarrow-p-vee-q}
به جدول ۴ (ترکیب فصلی) نگاه کنید.
\begin{enumerate}
\def\labelenumi{\arabic{enumi}.}
\tightlist
\item
  فرض می‌کنیم \(p\) درست \lr{(T) }باشد.
\item
  در \textbf{جدول ۴}، ردیف‌هایی که \(p\) در آن‌ها \textbf{\lr{T}} است را
  بررسی می‌کنیم (ردیف ۱ و ۲).
\item
  در هر دو ردیف، مقدار ستون \(p \vee q\) برابر با \textbf{\lr{T}} است
  (چون در ترکیب فصلی، وجود یک راست کافی است).
\item
  پس اگر \(p\) راست باشد، ترکیب فصلی آن با هر گزاره دیگر (\(q\)) نیز
  راست است.
\end{enumerate}
\subsection{\texorpdfstring{۴. شبکه ارتباطی با سایر قضایا
\lr{(Analytic Map)}}{۴. شبکه ارتباطی با سایر قضایا }}\label{ux634ux628ux6a9ux647-ux627ux631ux62aux628ux627ux637ux6cc-ux628ux627-ux633ux627ux6ccux631-ux642ux636ux627ux6ccux627-analytic-map}
قضیه ۱ (جمع و اختصار) زیربنای بسیاری از مفاهیم بعدی در منطق است. تحلیل
ارتباطات آن با سایر بخش‌های کتاب به شرح زیر است:
\subsubsection{\texorpdfstring{۱. ارتباط با
\autoref{قضیه-۲---هم‌ارزی‌های-منطقی-پایه}
(هم‌ارزی‌ها)}{۱. ارتباط با  (هم‌ارزی‌ها)}}\label{ux627ux631ux62aux628ux627ux637-ux628ux627-ux642ux636ux6ccux647-ux6f2-ux647ux645ux627ux631ux632ux6ccux647ux627}
\begin{itemize}
\tightlist
\item
  \textbf{جابجایی \lr{(Commutativity):}} در قضیه ۱ گفتیم
  \((p \wedge q) \Rightarrow p\). طبق
  \textbf{\autoref{قضیه-۲---هم‌ارزی‌های-منطقی-پایه}}، داریم
  \(p \wedge q \equiv q \wedge p\). بنابراین قانون اختصار برای مؤلفه دوم
  هم معتبر می‌شود: \((q \wedge p) \Rightarrow q\).
\item
  \textbf{عکس نقیض \lr{(Contrapositive):}} طبق
  \textbf{\autoref{قضیه-۲---هم‌ارزی‌های-منطقی-پایه}}، هر شرطی با عکس نقیضش
  هم‌ارز است. عکس نقیض قانون جمع (\(p \Rightarrow p \vee q\)) می‌شود:
  \(\sim(p \vee q) \Rightarrow \sim p\). این پایه و اساس
  \textbf{\autoref{قضیه-۳---قوانین-دمورگان}} است که می‌گوید نفیِ ترکیب
  فصلی، مستلزم نفیِ تک‌تک اجزاست.
\end{itemize}
\subsubsection{\texorpdfstring{۲. ارتباط با
\autoref{قضیه-۴---قوانین-شرکت-پذیری-و-پخش-پذیری}\textbar قضیه ۴ (قانون
تعدی)}{۲. ارتباط با \textbar قضیه ۴ (قانون تعدی)}}\label{ux627ux631ux62aux628ux627ux637-ux628ux627-ux642ux636ux6ccux647-ux6f4---ux642ux648ux627ux646ux6ccux646-ux634ux631ux6a9ux62a-ux67eux630ux6ccux631ux6cc-ux648-ux67eux62eux634-ux67eux630ux6ccux631ux6ccux642ux636ux6ccux647-ux6f4-ux642ux627ux646ux648ux646-ux62aux639ux62fux6cc}
\begin{itemize}
\tightlist
\item
  \textbf{زنجیره‌سازی استدلال:}
  \textbf{\autoref{قضیه-۴---قوانین-شرکت-پذیری-و-پخش-پذیری}} قانون تعدی
  را بیان می‌کند: \((p \to q) \wedge (q \to r) \Rightarrow (p \to r)\).
  ما می‌توانیم با استفاده از قانون اختصار (قضیه ۱)، مقدمات یک استدلال
  مرکب را جدا کرده و سپس با قانون تعدی به نتایج جدید برسیم.
\end{itemize}
\subsubsection{\texorpdfstring{۳. ارتباط با
\autoref{قضیه-۶---قواعد-استنتاج} (قیاس استثنایی و
دفع)}{۳. ارتباط با  (قیاس استثنایی و دفع)}}\label{ux627ux631ux62aux628ux627ux637-ux628ux627-ux642ux636ux6ccux647-ux6f6---ux642ux648ux627ux639ux62f-ux627ux633ux62aux646ux62aux627ux62c-ux642ux6ccux627ux633-ux627ux633ux62aux62bux646ux627ux6ccux6cc-ux648-ux62fux641ux639}
\begin{itemize}
\tightlist
\item
  \textbf{موتور محرک استنتاج:} \textbf{قضیه ۶} روش‌های اصلی نتیجه‌گیری
  (مثل اگر \(p \to q\) و \(p\) آنگاه \(q\)) را معرفی می‌کند. قضیه ۱
  معمولاً \textbf{پیش‌نیاز} استفاده از قضیه ۶ است؛ به این صورت که ابتدا با
  «قانون اختصار» داده‌های مسئله را استخراج می‌کنیم و سپس در قالب‌های قیاسی
  قضیه ۶ قرار می‌دهیم.
\end{itemize}
\subsubsection{۴. ارتباط عمیق با سورها (تعمیم‌یافته‌ی قضیه
۱)}\label{ux627ux631ux62aux628ux627ux637-ux639ux645ux6ccux642-ux628ux627-ux633ux648ux631ux647ux627-ux62aux639ux645ux6ccux645ux6ccux627ux641ux62aux647ux6cc-ux642ux636ux6ccux647-ux6f1}
این قضیه ارتباط ساختاری مستقیمی با مبحث سورها دارد:
\begin{itemize}
\tightlist
\item
  \textbf{سور عمومی (\(\forall\)) و قانون اختصار:} متن کتاب بیان می‌کند
  که \((\forall x) p(x)\) در یک دامنه محدود معادل
  \(p(a_1) \wedge p(a_2) \wedge \dots\) است. بنابراین، \textbf{قانون
  اختصار} در اینجا تعمیم می‌یابد به: «اگر حکمی برای \textbf{همه} درست
  باشد، برای \textbf{تک‌تک} اعضا هم درست است»
  (\((\forall x) p(x) \Rightarrow p(a_i)\)).
\item
  \textbf{سور وجودی (\(\exists\)) و قانون جمع:} متن کتاب بیان می‌کند که
  \((\exists x) p(x)\) معادل \(p(a_1) \vee p(a_2) \vee \dots\) است.
  بنابراین، \textbf{قانون جمع} تعمیم می‌یابد به: «اگر حکمی برای
  \textbf{یک نفر} (\(a_i\)) درست باشد، پس برای \textbf{حداقل یک نفر}
  (\(\exists\)) درست است» (\(p(a_i) \Rightarrow (\exists x) p(x)\)).
\end{itemize}

\clearpage
% ---------------------------------------------------------------------
% Copyright (c) 2026 Arsalan Dalvand & Reyhaneh Darvishi.
% Licensed under CC BY-NC-SA 4.0.
% See LICENSE file for details.
% ---------------------------------------------------------------------

\section{قضیه ۲: هم‌ارزی‌های منطقی
پایه}\label{قضیه-۲---هم‌ارزی‌های-منطقی-پایه}
\begin{tldr}{خلاصه سریع}
این قضیه مجموعه ابزارهای جبری برای تغییر شکل گزاره‌ها بدون تغییر ارزش
راستی آن‌هاست. مهم‌ترین بخش آن «قانون عکس نقیض» است که پایه بسیاری از
اثبات‌های ریاضی (برهان غیرمستقیم) را تشکیل می‌دهد.
\end{tldr}
\subsection{۱. متن ریاضی
قضیه}\label{ux645ux62aux646-ux631ux6ccux627ux636ux6cc-ux642ux636ux6ccux647}
فرض کنید \(p\) و \(q\) دو گزاره دلبخواه باشند.قوانین زیر همواره
برقرارند:
\begin{theorembox}{قضیه ۲}
\textbf{الف) قانون نفی مضاعف \lr{(Double Negation):}}
\[\sim(\sim p) \equiv p\] \textbf{ب) قانون جابجایی
\lr{(Commutative Laws):}} \[p \vee q \equiv q \vee p\]
\[p \wedge q \equiv q \wedge p\] \textbf{ج) قانون خودتوانی
\lr{(Idempotent Laws):}} \[p \vee p \equiv p\] \[p \wedge p \equiv p\]
\textbf{د) قانون عکس نقیض \lr{(Contrapositive Law):}}
\[(p \rightarrow q) \equiv (\sim q \rightarrow \sim p)\]
\end{theorembox}
\subsection{۲. اثبات و تحلیل (با جدول
ارزش)}\label{ux627ux62bux628ux627ux62a-ux648-ux62aux62dux644ux6ccux644-ux628ux627-ux62cux62fux648ux644-ux627ux631ux632ux634}
برهان قسمت‌های الف، ب و ج بدیهی است. اما قسمت (د) یعنی قانون عکس نقیض،
نیاز به اثبات دقیق با جدول ارزش دارد. ما ارزش گزاره دو شرطی
\((p \to q) \leftrightarrow (\sim q \to \sim p)\) را بررسی می‌کنیم. اگر
این گزاره در تمام حالات «راست» باشد، دو طرف با هم هم‌ارز هستند.
{\def\LTcaptype{none}
\begin{longtable}[]{@{}
  >{\centering\arraybackslash}p{(\linewidth - 12\tabcolsep) * \real{0.1429}}
  >{\centering\arraybackslash}p{(\linewidth - 12\tabcolsep) * \real{0.1429}}
  >{\centering\arraybackslash}p{(\linewidth - 12\tabcolsep) * \real{0.1429}}
  >{\centering\arraybackslash}p{(\linewidth - 12\tabcolsep) * \real{0.1429}}
  >{\centering\arraybackslash}p{(\linewidth - 12\tabcolsep) * \real{0.1429}}
  >{\centering\arraybackslash}p{(\linewidth - 12\tabcolsep) * \real{0.1429}}
  >{\centering\arraybackslash}p{(\linewidth - 12\tabcolsep) * \real{0.1429}}@{}}
\toprule\noalign{}
\begin{minipage}[b]{\linewidth}\centering
\(p\)
\end{minipage} & \begin{minipage}[b]{\linewidth}\centering
\(q\)
\end{minipage} & \begin{minipage}[b]{\linewidth}\centering
\((p \to q)\)
\end{minipage} & \begin{minipage}[b]{\linewidth}\centering
\(\leftrightarrow\)
\end{minipage} & \begin{minipage}[b]{\linewidth}\centering
\((\sim q \to \sim p)\)
\end{minipage} & \begin{minipage}[b]{\linewidth}\centering
\(\sim q\)
\end{minipage} & \begin{minipage}[b]{\linewidth}\centering
\(\sim p\)
\end{minipage} \\
\midrule\noalign{}
\endhead
\bottomrule\noalign{}
\endlastfoot
\lr{T} & \lr{T} & \lr{T} & \textbf{\lr{T}} & \lr{T} & \lr{F} & \lr{F} \\
\lr{T} & \lr{F} & \lr{F} & \textbf{\lr{T}} & \lr{F} & \lr{T} & \lr{F} \\
\lr{F} & \lr{T} & \lr{T} & \textbf{\lr{T}} & \lr{T} & \lr{F} & \lr{T} \\
\lr{F} & \lr{F} & \lr{T} & \textbf{\lr{T}} & \lr{T} & \lr{T} & \lr{T} \\
\emph{جدول ۹ - اثبات قانون عکس نقیض} & & & & & & \\
\end{longtable}
}
\begin{info}{تحلیل اثبات}
۱. ستون سوم نشان‌دهنده ارزش گزاره شرطی اصلی (\(p \to q\)) است. ۲. ستون
پنجم نشان‌دهنده ارزش عکس نقیض (\(\sim q \to \sim p\)) است که با توجه به
ستون‌های \(\sim q\) و \(\sim p\) محاسبه شده است. ۳. با مقایسه ستون ۳ و ۵،
می‌بینیم که در هر ۴ حالت منطقی، ارزش‌ها دقیقاً یکسان هستند. ۴. نتیجه: ستون
چهارم (\(\leftrightarrow\)) تماماً \textbf{\lr{T }(راستگو)} است، که اثبات
می‌کند این دو عبارت منطقاً هم‌ارز هستند
\end{info}
\subsection{\texorpdfstring{۳. شبکه ارتباطی با سایر قضایا
\lr{(Analytic Map)}}{۳. شبکه ارتباطی با سایر قضایا }}\label{ux634ux628ux6a9ux647-ux627ux631ux62aux628ux627ux637ux6cc-ux628ux627-ux633ux627ux6ccux631-ux642ux636ux627ux6ccux627-analytic-map}
این قضیه نقش حیاتی در دستکاری ساختاری گزاره‌ها در سایر قضایا ایفا می‌کند:
\subsubsection{\texorpdfstring{۱. ارتباط با
\autoref{قضیه-۱---قوانین-جمع-و-اختصار} (قوانین جمع و
اختصار)}{۱. ارتباط با  (قوانین جمع و اختصار)}}\label{ux627ux631ux62aux628ux627ux637-ux628ux627-ux642ux636ux6ccux647-ux6f1---ux642ux648ux627ux646ux6ccux646-ux62cux645ux639-ux648-ux627ux62eux62aux635ux627ux631-ux642ux648ux627ux646ux6ccux646-ux62cux645ux639-ux648-ux627ux62eux62aux635ux627ux631}
\begin{itemize}
\tightlist
\item
  \textbf{تعمیم قانون اختصار:} در قضیه ۱ داشتیم
  \((p \wedge q) \Rightarrow p\). با استفاده از \textbf{قانون جابجایی}
  این قضیه (\(p \wedge q \equiv q \wedge p\))، می‌توانیم ثابت کنیم که
  اختصار برای مؤلفه دوم هم معتبر است: \((q \wedge p) \Rightarrow q\).
\end{itemize}
\subsubsection{\texorpdfstring{۲. ارتباط با
\autoref{قضیه-۳---قوانین-دمورگان}
(دمورگان)}{۲. ارتباط با  (دمورگان)}}\label{ux627ux631ux62aux628ux627ux637-ux628ux627-ux642ux636ux64aux647-ux62fux645ux648ux631ux6afux627ux646}
\begin{itemize}
\tightlist
\item
  \textbf{ساختار نفی:} قانون عکس نقیض نوعی «جابجایی همراه با نفی» در
  گزاره‌های شرطی است. این مفهوم در قضیه ۳ توسعه می‌یابد، جایی که نفی وارد
  پرانتز شده و عملگرها را تغییر می‌دهد (نفی ترکیب عطفی/فصلی).
\end{itemize}
\subsubsection{\texorpdfstring{۳. ارتباط با
\autoref{قضیه-۵---قیاس-ها}(قیاس‌های
ذو‌الوجهین)}{۳. ارتباط با (قیاس‌های ذو‌الوجهین)}}\label{ux627ux631ux62aux628ux627ux637-ux628ux627-ux642ux636ux6ccux647-ux6f5---ux642ux6ccux627ux633-ux647ux627ux642ux6ccux627ux633ux647ux627ux6cc-ux630ux648ux627ux644ux648ux62cux647ux6ccux646}
\begin{itemize}
\tightlist
\item
  \textbf{قیاس منفی:} قضیه ۵ (ب) (قیاس ذوالوجهین منفی) مستقیماً بر اساس
  \textbf{قانون عکس نقیض} بنا شده است. در آنجا از نفیِ تالی‌ها
  (\(\sim q \vee \sim s\)) به نفیِ مقدم‌ها (\(\sim p \vee \sim r\))
  می‌رسیم، دقیقاً همان منطقی که \(p \to q\) را به \(\sim q \to \sim p\)
  تبدیل می‌کند.
\end{itemize}
\subsubsection{\texorpdfstring{۴. ارتباط با
\autoref{قضیه-۶---قواعد-استنتاج} (قیاس دفع -
\lr{Modus Tollens)}}{۴. ارتباط با  (قیاس دفع - }}\label{ux627ux631ux62aux628ux627ux637-ux628ux627-ux642ux636ux6ccux647-ux6f6---ux642ux648ux627ux639ux62f-ux627ux633ux62aux646ux62aux627ux62c-ux642ux6ccux627ux633-ux62fux641ux639---modus-tollens}
\begin{itemize}
\tightlist
\item
  \textbf{پایه نظری:} قاعده قیاس دفع
  (\([(p \to q) \wedge \sim q] \Rightarrow \sim p\)) در واقع کاربرد
  مستقیم قانون عکس نقیض در یک استدلال است. ما \(p \to q\) را با هم‌ارز آن
  \(\sim q \to \sim p\) جایگزین می‌کنیم و سپس از قاعده قیاس استثنایی
  \lr{(Modus Ponens) }استفاده می‌کنیم.
\end{itemize}

\clearpage
% ---------------------------------------------------------------------
% Copyright (c) 2026 Arsalan Dalvand & Reyhaneh Darvishi.
% Licensed under CC BY-NC-SA 4.0.
% See LICENSE file for details.
% ---------------------------------------------------------------------

\section{\texorpdfstring{قضیه ۳: قوانین دمورگان
\lr{(De Morgan’s Laws)}}{قضیه ۳: قوانین دمورگان }}\label{قضیه-۳---قوانین-دمورگان}
\begin{tldr}{خلاصه سریع}
این قوانین بیانگر رفتار «نفی» در مواجهه با پرانتزهای شامل ترکیب‌های عطفی
و فصلی هستند. طبق این اصل، توزیع نفی بر روی یک پرانتز، عملگر منطقی درون
آن را معکوس می‌کند (عطف به فصل و برعکس).
\end{tldr}
\subsection{۱. متن ریاضی
قضیه}\label{ux645ux62aux646-ux631ux6ccux627ux636ux6cc-ux642ux636ux6ccux647}
این قضیه که به نام \textbf{آگوستوس دمورگان} (۱۸۷۱-۱۸۰۶) نام‌گذاری شده
است، برای هر دو گزاره دلبخواه \(p\) و \(q\) برقرار است:
\begin{theorembox}{قضیه ۳}
\textbf{الف) نفی ترکیب عطفی:}
\[\sim(p \wedge q) \equiv (\sim p \vee \sim q)\] \textbf{ب) نفی ترکیب
فصلی:} \[\sim(p \vee q) \equiv (\sim p \wedge \sim q)\]
\end{theorembox}
\subsection{۲. اثبات و تحلیل (با جدول
ارزش)}\label{ux627ux62bux628ux627ux62a-ux648-ux62aux62dux644ux6ccux644-ux628ux627-ux62cux62fux648ux644-ux627ux631ux632ux634}
برای اثبات قسمت (الف)، جدول ارزش اختصاری گزاره دو شرطی
\(\sim(p \wedge q) \leftrightarrow (\sim p \vee \sim q)\) را تشکیل
می‌دهیم. اگر ستون رابط اصلی (\(\leftrightarrow\)) تماماً راستگو
\lr{(T) }باشد، هم‌ارزی ثابت می‌شود.
{\def\LTcaptype{none}
\begin{longtable}[]{@{}cccccccc@{}}
\toprule\noalign{}
\(\sim\) & \((p\) & \(\wedge\) & \(q)\) & \(\leftrightarrow\) &
\((\sim p\) & \(\vee\) & \(\sim q)\) \\
\midrule\noalign{}
\endhead
\bottomrule\noalign{}
\endlastfoot
\lr{F} & \lr{T} & \lr{T} & \lr{T} & \textbf{\lr{T}} & \lr{F} & \lr{F} &
\lr{F} \\
\lr{T} & \lr{T} & \lr{F} & \lr{F} & \textbf{\lr{T}} & \lr{F} & \lr{T} &
\lr{T} \\
\lr{T} & \lr{F} & \lr{F} & \lr{T} & \textbf{\lr{T}} & \lr{T} & \lr{T} &
\lr{F} \\
\lr{T} & \lr{F} & \lr{F} & \lr{F} & \textbf{\lr{T}} & \lr{T} & \lr{T} &
\lr{T} \\
\emph{(جدول ۱۰ - اثبات قانون اول دمورگان)} & & & & & & & \\
\end{longtable}
}
\begin{info}{تحلیل مراحل جدول}
۱. در سمت چپ (\(\leftrightarrow\))، ابتدا ارزش \(p \wedge q\) محاسبه شده
و سپس نقیض (\(\sim\)) آن اعمال شده است. ۲. در سمت راست، ابتدا نقیض‌های
\(\sim p\) و \(\sim q\) محاسبه شده و سپس بین آن‌ها ترکیب فصلی (\(\vee\))
برقرار شده است. ۳. ستون وسط (\(\leftrightarrow\)) نشان می‌دهد که در هر ۴
حالت ممکن، ارزش دو طرف یکسان است.
\end{info}
\begin{warning}{توجه!}
نکته حیاتی در اعمال قانون دمورگان، تغییر عملگر است.
\begin{itemize}
\tightlist
\item
  اشتباه رایج: \(\sim (p \wedge q) \equiv \sim p \wedge \sim q\) (توزیع
  نفی بدون تغییر عملگر).
\item
  \textbf{شکل صحیح:}
  \(\sim (p \wedge q) \equiv \sim p \mathbf{\vee} \sim q\) (توزیع نفی
  همراه با تغییر عملگر).
\end{itemize}
\end{warning}
\subsection{\texorpdfstring{۳. شبکه ارتباطی با سایر قضایا
\lr{(Analytic Map)}}{۳. شبکه ارتباطی با سایر قضایا }}\label{ux634ux628ux6a9ux647-ux627ux631ux62aux628ux627ux637ux6cc-ux628ux627-ux633ux627ux6ccux631-ux642ux636ux627ux6ccux627-analytic-map}
قوانین دمورگان نقش کلیدی در ساده‌سازی و تبدیل ساختارهای منطقی در سایر
بخش‌های کتاب دارند:
\subsubsection{\texorpdfstring{۱. ارتباط با
\autoref{قضیه-۲---هم‌ارزی‌های-منطقی-پایه} (عکس
نقیض)}{۱. ارتباط با  (عکس نقیض)}}\label{ux627ux631ux62aux628ux627ux637-ux628ux627-ux642ux636ux6ccux647-ux6f2---ux647ux645ux627ux631ux632ux6ccux647ux627ux6cc-ux645ux646ux637ux642ux6cc-ux67eux627ux6ccux647-ux639ux6a9ux633-ux646ux642ux6ccux636}
\begin{itemize}
\tightlist
\item
  \textbf{تحلیل ساختاری:} در قضیه ۲(قانون عکس نقیض)، دیدیم که
  \((p \to q) \equiv (\sim q \to \sim p)\). اگر گزاره‌های \(p\) یا \(q\)
  خودشان مرکب باشند (مثلاً \(p = a \wedge b\))، برای محاسبه عکس نقیض
  ناچاریم از قانون دمورگان استفاده کنیم تا نفی را به داخل پرانتز ببریم.
\end{itemize}
\subsubsection{\texorpdfstring{۲. ارتباط با
\autoref{قضیه-۱---قوانین-جمع-و-اختصار}(تعمیم
یافته)}{۲. ارتباط با (تعمیم یافته)}}\label{ux627ux631ux62aux628ux627ux637-ux628ux627-ux642ux636ux6ccux647-ux6f1---ux642ux648ux627ux646ux6ccux646-ux62cux645ux639-ux648-ux627ux62eux62aux635ux627ux631ux62aux639ux645ux6ccux645-ux6ccux627ux641ux62aux647}
\begin{itemize}
\tightlist
\item
  \textbf{دوگانگی \lr{(Duality):}} در قضیه ۱ با قوانین حاکم بر «و»
  (اختصار) و «یا» (جمع) آشنا شدیم. دمورگان نشان می‌دهد که این دو عملگر،
  دوگان یکدیگر تحت عملگر نفی هستند. یعنی جهانِ «و» با یک منفی‌سازی به جهان
  «یا» تبدیل می‌شود.
\end{itemize}
\subsubsection{۳. ارتباط با نقیض سورها (تعمیم
نهایی)}\label{ux627ux631ux62aux628ux627ux637-ux628ux627-ux646ux642ux6ccux636-ux633ux648ux631ux647ux627-ux62aux639ux645ux6ccux645-ux646ux647ux627ux6ccux6cc}
\begin{itemize}
\tightlist
\item
  \textbf{دمورگان در مجموعه‌ها:} قوانین دمورگان زیربنای اصلی قواعد نقیض
  سورها هستند که در انتهای فصل ۱ مطرح می‌شوند:
  \begin{itemize}
  \tightlist
  \item
    نفی سور عمومی (کل):
    \(\sim (\forall x) p(x) \equiv (\exists x) \sim p(x)\)
    \begin{itemize}
    \tightlist
    \item
      \emph{(این تعمیمِ قانون
      \(\sim(p \wedge q \wedge \dots) \equiv \sim p \vee \sim q \vee \dots\)
      است)}.
    \end{itemize}
  \item
    نفی سور وجودی (بعضی):
    \(\sim (\exists x) p(x) \equiv (\forall x) \sim p(x)\)
    \begin{itemize}
    \tightlist
    \item
      \emph{(این تعمیمِ قانون
      \(\sim(p \vee q \vee \dots) \equiv \sim p \wedge \sim q \wedge \dots\)
      است)}.
    \end{itemize}
  \end{itemize}
\end{itemize}

\clearpage
% ---------------------------------------------------------------------
% Copyright (c) 2026 Arsalan Dalvand & Reyhaneh Darvishi.
% Licensed under CC BY-NC-SA 4.0.
% See LICENSE file for details.
% ---------------------------------------------------------------------

\section{قضیه ۴: قوانین شرکت‌پذیری، پخش‌پذیری و
تعدی}\label{قضیه-۴---قوانین-شرکت-پذیری-و-پخش-پذیری}
\begin{tldr}{خلاصه سریع}
این قضیه معماریِ جملات مرکب را مدیریت می‌کند. «شرکت‌پذیری» به ما اجازه
می‌دهد پرانتزها را در عملگرهای هم‌جنس نادیده بگیریم، «پخش‌پذیری» تعامل دو
عملگر غیرهم‌جنس را نشان می‌دهد، و «تعدی» زنجیره‌ای از استدلال‌های شرطی را به
یک نتیجه واحد متصل می‌کند.
\end{tldr}
\subsection{۱. متن ریاضی
قضیه}\label{ux645ux62aux646-ux631ux6ccux627ux636ux6cc-ux642ux636ux6ccux647}
فرض کنید \(p\)، \(q\) و \(r\) سه گزاره دلبخواه باشند.
\begin{theorembox}{قضیه ۴}
\textbf{الف) قوانین شرکت‌پذیری \lr{(Associative Laws):}}
\[(p \vee q) \vee r \equiv p \vee (q \vee r)\]
\[(p \wedge q) \wedge r \equiv p \wedge (q \wedge r)\]
\textbf{ب) قوانین پخش‌پذیری \lr{(Distributive Laws):}}
\[p \wedge (q \vee r) \equiv (p \wedge q) \vee (p \wedge r)\]
\[p \vee (q \wedge r) \equiv (p \vee q) \wedge (p \vee r)\]
\textbf{ج) قانون تعدی \lr{(Transitive Law):}}
\[(p \rightarrow q) \wedge (q \rightarrow r) \Rightarrow (p \rightarrow r)\]
\end{theorembox}
\subsection{۲. اثبات و تحلیل (با جدول
ارزش)}\label{ux627ux62bux628ux627ux62a-ux648-ux62aux62dux644ux6ccux644-ux628ux627-ux62cux62fux648ux644-ux627ux631ux632ux634}
برای اثبات این قوانین، چون با ۳ متغیر گزاره‌ای سر و کار داریم، جدول ارزش
باید شامل \(2^3=8\) سطر باشد. در اینجا اثبات \textbf{قانون تعدی} (قسمت
ج) را با استفاده از جدول ارزش بررسی می‌کنیم. ما باید نشان دهیم که گزاره
شرطی نهایی همیشه «راستگو» \lr{(Tautology) }است.
{\def\LTcaptype{none}
\begin{longtable}[]{@{}
  >{\centering\arraybackslash}p{(\linewidth - 16\tabcolsep) * \real{0.1000}}
  >{\centering\arraybackslash}p{(\linewidth - 16\tabcolsep) * \real{0.0750}}
  >{\centering\arraybackslash}p{(\linewidth - 16\tabcolsep) * \real{0.0750}}
  >{\centering\arraybackslash}p{(\linewidth - 16\tabcolsep) * \real{0.0750}}
  >{\centering\arraybackslash}p{(\linewidth - 16\tabcolsep) * \real{0.0750}}
  >{\centering\arraybackslash}p{(\linewidth - 16\tabcolsep) * \real{0.0750}}
  >{\centering\arraybackslash}p{(\linewidth - 16\tabcolsep) * \real{0.1500}}
  >{\centering\arraybackslash}p{(\linewidth - 16\tabcolsep) * \real{0.1500}}
  >{\centering\arraybackslash}p{(\linewidth - 16\tabcolsep) * \real{0.2250}}@{}}
\toprule\noalign{}
\begin{minipage}[b]{\linewidth}\centering
ردیف
\end{minipage} & \begin{minipage}[b]{\linewidth}\centering
\(p\)
\end{minipage} & \begin{minipage}[b]{\linewidth}\centering
\(q\)
\end{minipage} & \begin{minipage}[b]{\linewidth}\centering
\(r\)
\end{minipage} & \begin{minipage}[b]{\linewidth}\centering
\((p \to q)\)
\end{minipage} & \begin{minipage}[b]{\linewidth}\centering
\((q \to r)\)
\end{minipage} & \begin{minipage}[b]{\linewidth}\centering
مقدم: \((p \to q) \wedge (q \to r)\)
\end{minipage} & \begin{minipage}[b]{\linewidth}\centering
تالی: \((p \to r)\)
\end{minipage} & \begin{minipage}[b]{\linewidth}\centering
کل گزاره \((\to)\)
\end{minipage} \\
\midrule\noalign{}
\endhead
\bottomrule\noalign{}
\endlastfoot
1 & \lr{T} & \lr{T} & \lr{T} & \lr{T} & \lr{T} & \textbf{\lr{T}} &
\lr{T} & \textbf{\lr{T}} \\
2 & \lr{T} & \lr{T} & \lr{F} & \lr{T} & \lr{F} & \lr{F} & \lr{F} &
\textbf{\lr{T}} \\
3 & \lr{T} & \lr{F} & \lr{T} & \lr{F} & \lr{T} & \lr{F} & \lr{T} &
\textbf{\lr{T}} \\
4 & \lr{T} & \lr{F} & \lr{F} & \lr{F} & \lr{T} & \lr{F} & \lr{F} &
\textbf{\lr{T}} \\
5 & \lr{F} & \lr{T} & \lr{T} & \lr{T} & \lr{T} & \textbf{\lr{T}} &
\lr{T} & \textbf{\lr{T}} \\
6 & \lr{F} & \lr{T} & \lr{F} & \lr{T} & \lr{F} & \lr{F} & \lr{T} &
\textbf{\lr{T}} \\
7 & \lr{F} & \lr{F} & \lr{T} & \lr{T} & \lr{T} & \textbf{\lr{T}} &
\lr{T} & \textbf{\lr{T}} \\
8 & \lr{F} & \lr{F} & \lr{F} & \lr{T} & \lr{T} & \textbf{\lr{T}} &
\lr{T} & \textbf{\lr{T}} \\
\emph{(جدول ۱۱ - اثبات قانون تعدی)} & & & & & & & & \\
\end{longtable}
}
\begin{info}{تحلیل اثبات}
۱. ستون‌های «مقدم» و «تالی» را محاسبه می‌کنیم. ۲. ستون آخر (کل گزاره)
رابطه شرطی بین مقدم و تالی را بررسی می‌کند. ۳. مشاهده می‌شود که در تمام ۸
حالت ممکن، ارزش نهایی \textbf{\lr{T}} است. این ثابت می‌کند که رابطه تعدی
یک قانون معتبر منطقی است.
\end{info}
\subsection{\texorpdfstring{۳. شبکه ارتباطی با سایر قضایا
\lr{(Analytic Map)}}{۳. شبکه ارتباطی با سایر قضایا }}\label{ux634ux628ux6a9ux647-ux627ux631ux62aux628ux627ux637ux6cc-ux628ux627-ux633ux627ux6ccux631-ux642ux636ux627ux6ccux627-analytic-map}
این قضیه پل ارتباطی بین منطق گزاره‌ها و نظریه مجموعه‌هاست و ابزار اصلی
برای ساختن برهان‌های چندمرحله‌ای می‌باشد.
\subsubsection{\texorpdfstring{۱. ارتباط با
\autoref{قضیه-۲---هم‌ارزی‌های-منطقی-پایه}
(جابجایی)}{۱. ارتباط با  (جابجایی)}}\label{ux627ux631ux62aux628ux627ux637-ux628ux627-ux642ux636ux6ccux647-ux6f2---ux647ux645ux627ux631ux632ux6ccux647ux627ux6cc-ux645ux646ux637ux642ux6cc-ux67eux627ux6ccux647-ux62cux627ux628ux62cux627ux6ccux6cc}
\begin{itemize}
\tightlist
\item
  \textbf{تکمیل ساختار جبری:} قضیه ۲ (جابجایی) می‌گفت ترتیب \(p\) و \(q\)
  مهم نیست (\(p \vee q \equiv q \vee p\)). قضیه ۴ (شرکت‌پذیری) مکمل آن
  است و می‌گوید اولویت پرانتزها هم مهم نیست. ترکیب این دو ویژگی به ما
  اجازه می‌دهد در زنجیره‌ای مثل \(p \vee q \vee r \vee s\)، گزاره‌ها را
  آزادانه جابجا و دسته‌بندی کنیم.
\end{itemize}
\subsubsection{\texorpdfstring{۲. ارتباط با
\autoref{قضیه-۶---قواعد-استنتاج}(قیاس
شرطی)}{۲. ارتباط با (قیاس شرطی)}}\label{ux627ux631ux62aux628ux627ux637-ux628ux627-ux642ux636ux6ccux647-ux6f6---ux642ux648ux627ux639ux62f-ux627ux633ux62aux646ux62aux627ux62cux642ux6ccux627ux633-ux634ux631ux637ux6cc}
\begin{itemize}
\tightlist
\item
  \textbf{نام دیگر:} قانون تعدی در بسیاری از متون (و در قضیه ۶ همین
  کتاب) تحت عنوان \textbf{قیاس شرطی \lr{(Hypothetical Syllogism)}}
  شناخته می‌شود. قضیه ۴ زیربنای نظری آن را با جدول ارزش ثابت می‌کند، و
  قضیه ۶ آن را به عنوان یک ابزار استنتاجی معرفی می‌کند.
\end{itemize}
\subsubsection{۴. ارتباط با نظریه مجموعه‌ها (فصل
۲)}\label{ux627ux631ux62aux628ux627ux637-ux628ux627-ux646ux638ux631ux6ccux647-ux645ux62cux645ux648ux639ux647ux647ux627-ux641ux635ux644-ux6f2}
\begin{itemize}
\tightlist
\item
  \textbf{ایزومورفیسم ساختاری:} قوانین پخش‌پذیری در منطق (\(\wedge\) روی
  \(\vee\)) دقیقاً متناظر با قوانین پخش‌پذیری در مجموعه‌ها (\(\cap\) روی
  \(\cup\)) هستند:
  \begin{itemize}
  \tightlist
  \item
    منطق: \(p \wedge (q \vee r) \equiv (p \wedge q) \vee (p \wedge r)\)
  \item
    مجموعه: \(A \cap (B \cup C) = (A \cap B) \cup (A \cap C)\) این شباهت
    نشان می‌دهد که منطق و نظریه مجموعه‌ها از یک ساختار جبری واحد (جبر بول)
    پیروی می‌کنند.
  \end{itemize}
\end{itemize}

\clearpage
% ---------------------------------------------------------------------
% Copyright (c) 2026 Arsalan Dalvand & Reyhaneh Darvishi.
% Licensed under CC BY-NC-SA 4.0.
% See LICENSE file for details.
% ---------------------------------------------------------------------

\section{\texorpdfstring{قضیه ۵: قیاس‌های ذو‌الوجهین
\lr{(Dilemmas)}}{قضیه ۵: قیاس‌های ذو‌الوجهین }}\label{قضیه-۵---قیاس-ها}
\begin{tldr}{خلاصه سریع}
این قضیه ابزار قدرتمندی برای استدلال در شرایط «چندراهی» است. اگر دو مسیر
شرطی داشته باشیم و بدانیم که در ابتدای یکی از این دو مسیر هستیم (یا در
انتهای آن‌ها نیستیم)، می‌توانیم نتیجه نهایی را پیش‌بینی کنیم. این قضیه
تعمیم‌یافته‌ی قواعد استنتاج ساده (مانند \lr{Modus Ponens) }برای حالات
ترکیبی است.
\end{tldr}
\subsection{۱. متن ریاضی
قضیه}\label{ux645ux62aux646-ux631ux6ccux627ux636ux6cc-ux642ux636ux6ccux647}
فرض کنید \(p\)، \(q\)، \(r\) و \(s\) چهار گزاره دلبخواه باشند. احکام زیر
برقرارند:
\begin{theorembox}{قضیه ۵}
\textbf{الف) قیاس ذو‌الوجهین موجب \lr{(Constructive Dilemma):}}
\[(p \rightarrow q) \wedge (r \rightarrow s) \Rightarrow (p \vee r \rightarrow q \vee s)\]
\textbf{ب) قیاس ذو‌الوجهین منفی \lr{(Destructive Dilemma):}}
\[(p \rightarrow q) \wedge (r \rightarrow s) \Rightarrow (\sim q \vee \sim s \rightarrow \sim p \vee \sim r)\]
\lr{[cite\_start][cite: }161, 162{]}
\end{theorembox}
\subsection{۲. اثبات و تحلیل
ساختاری}\label{ux627ux62bux628ux627ux62a-ux648-ux62aux62dux644ux6ccux644-ux633ux627ux62eux62aux627ux631ux6cc}
کتاب اثبات این قضیه را به خواننده واگذار کرده است، اما برای تسلط کامل،
جدول زیر ساختار منطقی و نحوه عملکرد این قضیه را تحلیل می‌کند.
{\def\LTcaptype{none}
\begin{longtable}[]{@{}
  >{\centering\arraybackslash}p{(\linewidth - 6\tabcolsep) * \real{0.1356}}
  >{\centering\arraybackslash}p{(\linewidth - 6\tabcolsep) * \real{0.3898}}
  >{\centering\arraybackslash}p{(\linewidth - 6\tabcolsep) * \real{0.2542}}
  >{\centering\arraybackslash}p{(\linewidth - 6\tabcolsep) * \real{0.2203}}@{}}
\toprule\noalign{}
\begin{minipage}[b]{\linewidth}\centering
نوع قیاس
\end{minipage} & \begin{minipage}[b]{\linewidth}\centering
داده‌های ورودی (مقدمات)
\end{minipage} & \begin{minipage}[b]{\linewidth}\centering
مکانیزم استدلال
\end{minipage} & \begin{minipage}[b]{\linewidth}\centering
نتیجه (خروجی)
\end{minipage} \\
\midrule\noalign{}
\endhead
\bottomrule\noalign{}
\endlastfoot
\textbf{موجب} & ۱. دو شرط (\(p \to q, r \to s\))۲. وقوع یکی از
\textbf{مقدم‌ها} (\(p \vee r\)) & اگر مقدم‌ها رخ دهند، تالی‌ها به دنبالشان
می‌آیند. & وقوع یکی از \textbf{تالی‌ها} (\(q \vee s\)) \\
\textbf{منفی} & ۱. دو شرط (\(p \to q, r \to s\))۲. نفی یکی از
\textbf{تالی‌ها} (\(\sim q \vee \sim s\)) & اگر تالی‌ها رخ ندهند، مقدم‌ها
هم رخ نداده‌اند (عکس نقیض). & نفی یکی از \textbf{مقدم‌ها}
(\(\sim p \vee \sim r\)) \\
\emph{(جدول تحلیلی - ساختار قیاس‌های ذو‌الوجهین)} & & & \\
\end{longtable}
}
\subsection{\texorpdfstring{۳. شبکه ارتباطی با سایر قضایا
\lr{(Analytic Map)}}{۳. شبکه ارتباطی با سایر قضایا }}\label{ux634ux628ux6a9ux647-ux627ux631ux62aux628ux627ux637ux6cc-ux628ux627-ux633ux627ux6ccux631-ux642ux636ux627ux6ccux627-analytic-map}
این قضیه یک نقطه اتصال مهم در شبکه مفاهیم فصل ۱ است و می‌توان آن را
ترکیبی از قضایای قبلی دانست:
\subsubsection{\texorpdfstring{۱. ارتباط با
\autoref{قضیه-۶---قواعد-استنتاج} (تعمیم
استنتاج)}{۱. ارتباط با  (تعمیم استنتاج)}}\label{ux627ux631ux62aux628ux627ux637-ux628ux627-ux642ux636ux6ccux647-ux6f6---ux642ux648ux627ux639ux62f-ux627ux633ux62aux646ux62aux627ux62c-ux62aux639ux645ux6ccux645-ux627ux633ux62aux646ux62aux627ux62c}
\begin{itemize}
\tightlist
\item
  \textbf{رابطه با قیاس استثنایی \lr{(Modus Ponens):}} قیاس ذو‌الوجهین
  \textbf{موجب}، در واقع ترکیب دو «قیاس استثنایی» است که با عملگر «یا»
  (\(\vee\)) به هم جوش خورده‌اند.
  \begin{itemize}
  \tightlist
  \item
    قیاس استثنایی: \((p \to q) \wedge p \Rightarrow q\)
  \item
    قیاس ذو‌الوجهین موجب:
    \((p \to q) \wedge (r \to s) \wedge (p \vee r) \Rightarrow (q \vee s)\)
  \end{itemize}
\item
  \textbf{رابطه با قیاس دفع \lr{(Modus Tollens):}} قیاس ذو‌الوجهین
  \textbf{منفی}، ترکیب دو «قیاس دفع» است.
  \begin{itemize}
  \tightlist
  \item
    قیاس دفع: \((p \to q) \wedge \sim q \Rightarrow \sim p\)
  \item
    قیاس ذو‌الوجهین منفی:
    \((p \to q) \wedge (r \to s) \wedge (\sim q \vee \sim s) \Rightarrow (\sim p \vee \sim r)\)
  \end{itemize}
\end{itemize}
\subsubsection{۲. ارتباط با قضیه ۲(عکس
نقیض)}\label{ux627ux631ux62aux628ux627ux637-ux628ux627-ux642ux636ux6ccux647-ux6f2ux639ux6a9ux633-ux646ux642ux6ccux636}
\begin{itemize}
\tightlist
\item
  \textbf{تبدیل موجب به منفی:} با استفاده از \textbf{قانون عکس نقیض}
  (\autoref{قضیه-۲---هم‌ارزی‌های-منطقی-پایه}) می‌توانیم قیاس موجب را به
  منفی تبدیل کنیم.
  \begin{itemize}
  \tightlist
  \item
    اگر در فرمولِ موجب، جای \((p \to q)\) را با \((\sim q \to \sim p)\) و
    جای \((r \to s)\) را با \((\sim s \to \sim r)\) عوض کنیم، دقیقاً به
    فرمول قیاس ذو‌الوجهین منفی می‌رسیم. این نشان می‌دهد که این دو قیاس، دو
    روی یک سکه هستند.
  \end{itemize}
\end{itemize}

\clearpage
% ---------------------------------------------------------------------
% Copyright (c) 2026 Arsalan Dalvand & Reyhaneh Darvishi.
% Licensed under CC BY-NC-SA 4.0.
% See LICENSE file for details.
% ---------------------------------------------------------------------

\section{\texorpdfstring{قضیه ۶: قواعد استنتاج اصلی
\lr{(Inference Rules)}}{قضیه ۶: قواعد استنتاج اصلی }}\label{قضیه-۶---قواعد-استنتاج}
\begin{tldr}{خلاصه سریع}
این قضیه «موتورهای محرک» اثبات‌های ریاضی هستند. قیاس استثنایی (تایید شرط)
و قیاس دفع (انکار نتیجه)، دو روش بنیادین برای استخراج حقایق جدید از
حقایق قبلی می‌باشند.
\end{tldr}
\subsection{۱. متن ریاضی
قضیه}\label{ux645ux62aux646-ux631ux6ccux627ux636ux6cc-ux642ux636ux6ccux647}
فرض کنید \(p\) و \(q\) دو گزاره دلبخواه باشند. قواعد زیر، که همگی
\textbf{راستگو \lr{(Tautology)}} هستند، اساس استدلال ریاضی را تشکیل
می‌دهند:
\begin{theorembox}{قضیه ۶}
\textbf{الف) قیاس استثنایی \lr{(Modus Ponens):}}
\[[(p \rightarrow q) \wedge p] \Rightarrow q\] \textbf{ب) قیاس دفع
\lr{(Modus Tollens):}}
\[[(p \rightarrow q) \wedge \sim q] \Rightarrow \sim p\] \textbf{ج)
برهان خلف \lr{(Proof by Contradiction):}}
\[(p \rightarrow q) \leftrightarrow [(p \wedge \sim q) \rightarrow c]\]
\emph{(که در آن \(c\) نماد تناقض است)}
\end{theorembox}
\subsection{۲. اثبات و تحلیل (با جدول
ارزش)}\label{ux627ux62bux628ux627ux62a-ux648-ux62aux62dux644ux6ccux644-ux628ux627-ux62cux62fux648ux644-ux627ux631ux632ux634}
برای اطمینان از صحت «قیاس دفع»، جدول ارزش آن را رسم می‌کنیم. ما باید نشان
دهیم که فرض‌های \([(p \to q) \wedge \sim q]\) منطقاً منجر به نتیجه
\(\sim p\) می‌شوند.
{\def\LTcaptype{none}
\begin{longtable}[]{@{}
  >{\centering\arraybackslash}p{(\linewidth - 12\tabcolsep) * \real{0.0882}}
  >{\centering\arraybackslash}p{(\linewidth - 12\tabcolsep) * \real{0.0882}}
  >{\centering\arraybackslash}p{(\linewidth - 12\tabcolsep) * \real{0.0882}}
  >{\centering\arraybackslash}p{(\linewidth - 12\tabcolsep) * \real{0.0882}}
  >{\centering\arraybackslash}p{(\linewidth - 12\tabcolsep) * \real{0.2353}}
  >{\centering\arraybackslash}p{(\linewidth - 12\tabcolsep) * \real{0.0882}}
  >{\centering\arraybackslash}p{(\linewidth - 12\tabcolsep) * \real{0.3235}}@{}}
\toprule\noalign{}
\begin{minipage}[b]{\linewidth}\centering
\(p\)
\end{minipage} & \begin{minipage}[b]{\linewidth}\centering
\(q\)
\end{minipage} & \begin{minipage}[b]{\linewidth}\centering
\((p \to q)\)
\end{minipage} & \begin{minipage}[b]{\linewidth}\centering
\(\sim q\)
\end{minipage} & \begin{minipage}[b]{\linewidth}\centering
فرض کل: \((p \to q) \wedge \sim q\)
\end{minipage} & \begin{minipage}[b]{\linewidth}\centering
\(\sim p\)
\end{minipage} & \begin{minipage}[b]{\linewidth}\centering
کل استدلال \((\Rightarrow)\)
\end{minipage} \\
\midrule\noalign{}
\endhead
\bottomrule\noalign{}
\endlastfoot
\lr{T} & \lr{T} & \lr{T} & \lr{F} & \lr{F} & \lr{F} & \textbf{\lr{T}} \\
\lr{T} & \lr{F} & \lr{F} & \lr{T} & \lr{F} & \lr{F} & \textbf{\lr{T}} \\
\lr{F} & \lr{T} & \lr{T} & \lr{F} & \lr{F} & \lr{T} & \textbf{\lr{T}} \\
\lr{F} & \lr{F} & \lr{T} & \lr{T} & \textbf{\lr{T}} & \textbf{\lr{T}} &
\textbf{\lr{T}} \\
\end{longtable}
}
\begin{info}{تحلیل اثبات}
۱. ستون پنجم (فرض کل) تنها در \textbf{ردیف آخر} درست است (جایی که مقدم
دروغ و تالی هم دروغ است). ۲. در همین ردیف آخر، نتیجه (\(\sim p\)) نیز
\textbf{درست} است (چون \(p\) دروغ است). ۳. در سایر ردیف‌ها که فرض غلط
است، طبق تعریف استلزام، کل گزاره شرطی به طور پیش‌فرض درست است (انتفای
مقدم). ۴. نتیجه: ستون آخر تماماً \textbf{\lr{T}} است، پس قیاس دفع یک
قانون معتبر است.
\end{info}
\subsection{\texorpdfstring{۳. شبکه ارتباطی با سایر قضایا
\lr{(Analytic Map)}}{۳. شبکه ارتباطی با سایر قضایا }}\label{ux634ux628ux6a9ux647-ux627ux631ux62aux628ux627ux637ux6cc-ux628ux627-ux633ux627ux6ccux631-ux642ux636ux627ux6ccux627-analytic-map}
این قضیه، نقطه همگرایی قضایای قبلی برای تولید نتیجه است:
\subsubsection{\texorpdfstring{۱. ارتباط با
\autoref{قضیه-۲---هم‌ارزی‌های-منطقی-پایه}}{۱. ارتباط با }}\label{ux627ux631ux62aux628ux627ux637-ux628ux627-ux642ux636ux6ccux647-ux6f2---ux647ux645ux627ux631ux632ux6ccux647ux627ux6cc-ux645ux646ux637ux642ux6cc-ux67eux627ux6ccux647}
\begin{itemize}
\tightlist
\item
  \textbf{ریشه نظری قیاس دفع:} قیاس دفع (قسمت ب) در واقع فرزندِ
  \textbf{قانون عکس نقیض} (قضیه ۲) است.
  \begin{itemize}
  \tightlist
  \item
    طبق قضیه ۲ داریم: \((p \to q) \equiv (\sim q \to \sim p)\).
  \item
    حال اگر از «قیاس استثنایی» \lr{(Modus Ponens) }روی
    \((\sim q \to \sim p)\) استفاده کنیم (یعنی \(\sim q\) را داشته
    باشیم)، مستقیماً \(\sim p\) را نتیجه می‌دهد.
  \item
    بنابراین: \textbf{قیاس دفع = عکس نقیض + قیاس استثنایی}.
  \end{itemize}
\end{itemize}
\subsubsection{\texorpdfstring{۲. ارتباط با \autoref{قضیه-۵---قیاس-ها}
(قیاس‌های
ذو‌الوجهین)}{۲. ارتباط با  (قیاس‌های ذو‌الوجهین)}}\label{ux627ux631ux62aux628ux627ux637-ux628ux627-ux642ux636ux6ccux647-ux6f5---ux642ux6ccux627ux633-ux647ux627-ux642ux6ccux627ux633ux647ux627ux6cc-ux630ux648ux627ux644ux648ux62cux647ux6ccux646}
\begin{itemize}
\tightlist
\item
  \textbf{تعمیم ساختاری:} قیاس‌های ذو‌الوجهین نسخه‌های پیشرفته و دوبلِ همین
  قواعد هستند:
  \begin{itemize}
  \tightlist
  \item
    \textbf{قیاس ذو‌الوجهین موجب}، تعمیم «قیاس استثنایی» است (دو شرط و
    انتخاب یکی از مقدم‌ها).
  \item
    \textbf{قیاس ذو‌الوجهین منفی}، تعمیم «قیاس دفع» است (دو شرط و نفی یکی
    از تالی‌ها).
  \end{itemize}
\end{itemize}
\subsubsection{\texorpdfstring{۴. ارتباط با
\autoref{قضیه-۷---قوانین-تناقض}
(تناقض)}{۴. ارتباط با  (تناقض)}}\label{ux627ux631ux62aux628ux627ux637-ux628ux627-ux642ux636ux6ccux647-ux6f7---ux642ux648ux627ux646ux6ccux646-ux62aux646ux627ux642ux636-ux62aux646ux627ux642ux636}
\begin{itemize}
\tightlist
\item
  \textbf{زیربنای برهان خلف:} قسمت (ج) این قضیه (برهان خلف) مستقیماً به
  مفهوم \(c\) (تناقض) وابسته است که ویژگی‌های جبری آن در قضیه ۷ بررسی
  می‌شود. برهان خلف می‌گوید: «اگر فرض \(p\) و نفی \(q\) ما را به یک بن‌بست
  منطقی (\(c\)) برساند، پس راه را اشتباه آمده‌ایم و \(p \to q\) باید درست
  باشد».
\end{itemize}

\clearpage
% ---------------------------------------------------------------------
% Copyright (c) 2026 Arsalan Dalvand & Reyhaneh Darvishi.
% Licensed under CC BY-NC-SA 4.0.
% See LICENSE file for details.
% ---------------------------------------------------------------------

\section{\texorpdfstring{قضیه ۷: قوانین مربوط به راستگو و تناقض
\lr{(Identity and Domination Laws)}}{قضیه ۷: قوانین مربوط به راستگو و تناقض }}\label{قضیه-۷---قوانین-تناقض}
\begin{tldr}{خلاصه سریع}
این قضیه رفتار گزاره‌ها را در تعامل با «ثابت‌های منطقی» یعنی راستگو
(\(t\)) و تناقض (\(c\)) مشخص می‌کند. در اینجا \(t\) شبیه عدد ۱ در ضرب
(خنثی) یا بینهایت در جمع (غالب) عمل می‌کند و \(c\) شبیه ۰ در جمع (خنثی)
یا ۰ در ضرب (غالب) است.
\end{tldr}
\subsection{۱. متن ریاضی
قضیه}\label{ux645ux62aux646-ux631ux6ccux627ux636ux6cc-ux642ux636ux6ccux647}
فرض کنید \(t\) یک گزاره همیشه راست \lr{(Tautology)، }\(c\) یک گزاره
همیشه دروغ \lr{(Contradiction) }و \(p\) یک گزاره دلبخواه باشد.
\begin{theorembox}{قضیه ۷}
\textbf{الف) قوانین همانی \lr{(Identity Laws):}} \[p \wedge t \equiv p\]
\[p \vee c \equiv p\] \textbf{ب) قوانین سلطه \lr{(Domination Laws):}}
\[p \vee t \equiv t\] \[p \wedge c \equiv c\] \textbf{ج) قوانین شرطی
خاص:} \[p \rightarrow t \equiv t\] \[c \rightarrow p \equiv t\]
\end{theorembox}
\subsection{۲. اثبات و تحلیل (با جدول
ارزش)}\label{ux627ux62bux628ux627ux62a-ux648-ux62aux62dux644ux6ccux644-ux628ux627-ux62cux62fux648ux644-ux627ux631ux632ux634}
\subsubsection{\texorpdfstring{اثبات قانون همانی
(\(p \wedge t \equiv p\))}{اثبات قانون همانی (p \textbackslash wedge t \textbackslash equiv p)}}\label{ux627ux62bux628ux627ux62a-ux642ux627ux646ux648ux646-ux647ux645ux627ux646ux6cc-p-wedge-t-equiv-p}
در این جدول، ارزش \(t\) همواره \textbf{\lr{T}} در نظر گرفته می‌شود.
{\def\LTcaptype{none}
\begin{longtable}[]{@{}ccccc@{}}
\toprule\noalign{}
\(p\) & \(\wedge\) & \(t\) & \(\leftrightarrow\) & \(p\) \\
\midrule\noalign{}
\endhead
\bottomrule\noalign{}
\endlastfoot
\lr{T} & \lr{T} & \lr{T} & \textbf{\lr{T}} & \lr{T} \\
\lr{F} & \lr{F} & \lr{T} & \textbf{\lr{T}} & \lr{F} \\
\emph{(جدول اثبات قانون همانی)} & & & & \\
\end{longtable}
}
\begin{info}{تحلیل}
وقتی یکی از طرفین «و» (\(\wedge\))، حقیقت محض (\(t\)) باشد، کل عبارت
تنها زمانی درست است که طرف دیگر (\(p\)) درست باشد. اگر \(p\) غلط باشد،
کل عبارت غلط می‌شود. پس نتیجه دقیقاً تابع \(p\) است.
\end{info}
\subsubsection{\texorpdfstring{اثبات قانون سلطه
(\(p \vee t \equiv t\))}{اثبات قانون سلطه (p \textbackslash vee t \textbackslash equiv t)}}\label{ux627ux62bux628ux627ux62a-ux642ux627ux646ux648ux646-ux633ux644ux637ux647-p-vee-t-equiv-t}
در این جدول نیز ارزش \(t\) همواره \textbf{\lr{T}} است.
{\def\LTcaptype{none}
\begin{longtable}[]{@{}ccccc@{}}
\toprule\noalign{}
\(p\) & \(\vee\) & \(t\) & \(\leftrightarrow\) & \(t\) \\
\midrule\noalign{}
\endhead
\bottomrule\noalign{}
\endlastfoot
\lr{T} & \lr{T} & \lr{T} & \textbf{\lr{T}} & \lr{T} \\
\lr{F} & \lr{T} & \lr{T} & \textbf{\lr{T}} & \lr{T} \\
\emph{(جدول اثبات قانون سلطه)} & & & & \\
\end{longtable}
}
\begin{info}{تحلیل}
در ترکیب فصلی (\(\vee\))، اگر حداقل یک طرف درست باشد، کل عبارت درست است.
چون \(t\) همیشه درست است، حضور آن کافی است تا کل عبارت (\(p \vee t\))
صرف‌نظر از مقدار \(p\)، همیشه درست (\(t\)) شود.
\end{info}
\subsubsection{\texorpdfstring{اثبات قوانین شرطی
(\(c \rightarrow p\))}{اثبات قوانین شرطی (c \textbackslash rightarrow p)}}\label{ux627ux62bux628ux627ux62a-ux642ux648ux627ux646ux6ccux646-ux634ux631ux637ux6cc-c-rightarrow-p}
در این جدول، ارزش \(c\) همواره \textbf{\lr{F}} است.
{\def\LTcaptype{none}
\begin{longtable}[]{@{}ccccc@{}}
\toprule\noalign{}
\(c\) & \(\rightarrow\) & \(p\) & \(\leftrightarrow\) & \(t\) \\
\midrule\noalign{}
\endhead
\bottomrule\noalign{}
\endlastfoot
\lr{F} & \lr{T} & \lr{T} & \textbf{\lr{T}} & \lr{T} \\
\lr{F} & \lr{T} & \lr{F} & \textbf{\lr{T}} & \lr{T} \\
\emph{(جدول اثبات انتفای مقدم)} & & & & \\
\end{longtable}
}
\begin{info}{تحلیل}
در گزاره شرطی، اگر مقدم (\(c\)) نادرست باشد، کل گزاره به صورت خودکار
درست (\(t\)) می‌شود (انتفای مقدم). بنابراین، «از یک دروغ می‌توان هر
نتیجه‌ای گرفت» و کل ساختار همیشه راستگوست.
\end{info}
\subsection{\texorpdfstring{۳. شبکه ارتباطی با سایر قضایا
\lr{(Analytic Map)}}{۳. شبکه ارتباطی با سایر قضایا }}\label{ux634ux628ux6a9ux647-ux627ux631ux62aux628ux627ux637ux6cc-ux628ux627-ux633ux627ux6ccux631-ux642ux636ux627ux6ccux627-analytic-map}
این قضیه ابزارهای جبری قدرتمندی برای ساده‌سازی عبارات منطقی و مجموعه‌ای
فراهم می‌کند:
\subsubsection{\texorpdfstring{۱. ارتباط با
\autoref{قضیه-۶---قواعد-استنتاج} (برهان
خلف)}{۱. ارتباط با  (برهان خلف)}}\label{ux627ux631ux62aux628ux627ux637-ux628ux627-ux642ux636ux6ccux647-ux6f6---ux642ux648ux627ux639ux62f-ux627ux633ux62aux646ux62aux627ux62c-ux628ux631ux647ux627ux646-ux62eux644ux641}
\begin{itemize}
\tightlist
\item
  \textbf{تعریف فرمال تناقض:} در برهان خلف (قضیه ۶-ج) دیدیم که
  \((p \wedge \sim q) \to c\). قضیه ۷ ماهیت جبری \(c\) را تعریف می‌کند.
  مثلاً اگر در یک اثبات به \(A \wedge \sim A\) برسیم، می‌توانیم آن را با
  \(c\) جایگزین کنیم و سپس با استفاده از قانون سلطه
  (\(p \wedge c \equiv c\)) کل شاخه استدلال را باطل کنیم.
\end{itemize}

\clearpage
% ---------------------------------------------------------------------
% Copyright (c) 2026 Arsalan Dalvand & Reyhaneh Darvishi.
% Licensed under CC BY-NC-SA 4.0.
% See LICENSE file for details.
% ---------------------------------------------------------------------

\section{\texorpdfstring{قواعد تسویر
\lr{(Quantification Rules)}}{قواعد تسویر }}\label{قواعد-تسویر}
\begin{tldr}{خلاصه سریع}
این بخش زبان منطق را از «گزاره‌های ساده» به «مجموعه‌ها» گسترش می‌دهد.
\begin{itemize}
\tightlist
\item
  \textbf{سور عمومی (\(\forall\)):} یعنی «برای همه» (مثل عملگر «و» روی
  تمام اعضا).
\item
  \textbf{سور وجودی (\(\exists\)):} یعنی «حداقل یکی هست» (مثل عملگر «یا»
  روی تمام اعضا).
\item
  \textbf{نفی:} نفیِ «همه»، «بعضی» است و نفیِ «بعضی»، «همه» است (با تغییر
  گزاره).
\end{itemize}
\end{tldr}
\subsection{۱. تعاریف
پایه}\label{ux62aux639ux627ux631ux6ccux641-ux67eux627ux6ccux647}
در گزاره‌هایی که مربوط به یک مجموعه (عالم سخن - \lr{Universe) }هستند، از
دو نماد اصلی استفاده می‌کنیم:
\subsubsection{\texorpdfstring{الف) سور عمومی
\lr{(Universal Quantifier)}}{الف) سور عمومی }}\label{ux627ux644ux641-ux633ux648ux631-ux639ux645ux648ux645ux6cc-universal-quantifier}
عبارت «برای تمام \(x\)های در عالم» را با نماد \(\forall x\) نشان می‌دهیم.
\begin{itemize}
\tightlist
\item
  \textbf{نماد:} \((\forall x)(p(x))\)
\item
  \textbf{معنی:} گزاره \(p(x)\) برای تک‌تک اعضای مجموعه مرجع درست است.
\end{itemize}
\subsubsection{\texorpdfstring{ب) سور وجودی
\lr{(Existential Quantifier)}}{ب) سور وجودی }}\label{ux628-ux633ux648ux631-ux648ux62cux648ux62fux6cc-existential-quantifier}
عبارت «حداقل یک \(x\) وجود دارد که\ldots» را با نماد \(\exists x\) نشان
می‌دهیم.
\begin{itemize}
\tightlist
\item
  \textbf{نماد:} \((\exists x)(p(x))\)
\item
  \textbf{معنی:} حداقل یک عضو در مجموعه پیدا می‌شود که \(p(x)\) برای آن
  درست باشد (لازم نیست برای همه درست باشد).
\end{itemize}
\begin{center}\rule{0.5\linewidth}{0.5pt}\end{center}
\subsection{\texorpdfstring{۲. قواعد نقیض سور
\lr{(Quantifier Negation Laws)}}{۲. قواعد نقیض سور }}\label{ux642ux648ux627ux639ux62f-ux646ux642ux6ccux636-ux633ux648ux631-quantifier-negation-laws}
چگونه جملات کلی را منفی کنیم؟ این قواعد بسیار شبیه به
\textbf{\autoref{قضیه-۳---قوانین-دمورگان}} عمل می‌کنند.
\begin{theorembox}{قاعده نقیض سور}
\textbf{۱. نفی سور عمومی:}
\[\sim [(\forall x)(p(x))] \equiv (\exists x)(\sim p(x))\] \emph{(ترجمه:
اگر «همه خوب نیستند»، یعنی «حداقل یک نفر بد است»)}.
\textbf{۲. نفی سور وجودی:}
\[\sim [(\exists x)(p(x))] \equiv (\forall x)(\sim p(x))\] \emph{(ترجمه:
اگر «چنین نیست که کسی آمده باشد»، یعنی «همه نیامده‌اند»)}.
\end{theorembox}
\begin{center}\rule{0.5\linewidth}{0.5pt}\end{center}
\subsection{\texorpdfstring{۳. شبکه ارتباطی با سایر قضایا
\lr{(Analytic Map)}}{۳. شبکه ارتباطی با سایر قضایا }}\label{ux634ux628ux6a9ux647-ux627ux631ux62aux628ux627ux637ux6cc-ux628ux627-ux633ux627ux6ccux631-ux642ux636ux627ux6ccux627-analytic-map}
این بخش نشان می‌دهد که سورها موجودات جدیدی نیستند، بلکه تعمیم‌یافته‌ی همان
عملگرهای منطقی فصل ۱ هستند.
\subsubsection{\texorpdfstring{۱. ارتباط با
\autoref{قضیه-۳---قوانین-دمورگان} (ریشه نفی
سورها)}{۱. ارتباط با  (ریشه نفی سورها)}}\label{ux627ux631ux62aux628ux627ux637-ux628ux627-ux642ux636ux6ccux647-ux6f3---ux642ux648ux627ux646ux6ccux646-ux62fux645ux648ux631ux6afux627ux646-ux631ux6ccux634ux647-ux646ux641ux6cc-ux633ux648ux631ux647ux627}
اگر عالم سخن ما محدود باشد (مثلاً \(\{a_1, a_2, \dots, a_n\}\))، می‌توان
سورها را باز کرد:
\begin{itemize}
\tightlist
\item
  \textbf{سور عمومی \(\equiv\) ترکیب عطفی:}
  \[(\forall x)p(x) \equiv p(a_1) \wedge p(a_2) \wedge \dots \wedge p(a_n)\]
\item
  \textbf{سور وجودی \(\equiv\) ترکیب فصلی:}
  \[(\exists x)p(x) \equiv p(a_1) \vee p(a_2) \vee \dots \vee p(a_n)\]
\end{itemize}
حال اگر از \textbf{\autoref{قضیه-۳---قوانین-دمورگان}} استفاده کنیم:
\begin{itemize}
\tightlist
\item
  نفیِ «ترکیب عطفی» (\(\sim(p \wedge q \dots)\)) تبدیل به «ترکیب فصلی
  نقیض‌ها» (\(\sim p \vee \sim q \dots\)) می‌شود.
\item
  این دقیقاً همان قاعده تبدیل \(\sim \forall\) به \(\exists \sim\) است.
\end{itemize}
\subsubsection{\texorpdfstring{۲. ارتباط با
\autoref{قضیه-۱---قوانین-جمع-و-اختصار} (ساختار
استنتاج)}{۲. ارتباط با  (ساختار استنتاج)}}\label{ux627ux631ux62aux628ux627ux637-ux628ux627-ux642ux636ux6ccux647-ux6f1---ux642ux648ux627ux646ux6ccux646-ux62cux645ux639-ux648-ux627ux62eux62aux635ux627ux631-ux633ux627ux62eux62aux627ux631-ux627ux633ux62aux646ux62aux627ux62c}
\begin{itemize}
\item
  \textbf{قانون اختصار تعمیم‌یافته:} چون \(\forall\) ماهیت «عطفی»
  \lr{(AND) }دارد، قانون اختصار (\((p \wedge q) \to p\)) برای آن صادق
  است. یعنی: \[(\forall x)p(x) \Rightarrow p(a_i)\] \emph{(اگر حکمی برای
  همه درست باشد، برای تک‌تک افراد هم درست است).}
\item
  \textbf{قانون جمع تعمیم‌یافته:} چون \(\exists\) ماهیت «فصلی»
  \lr{(OR) }دارد، قانون جمع (\(p \to p \vee q\)) برای آن صادق است. یعنی:
  \[p(a_i) \Rightarrow (\exists x)p(x)\] *(اگر حکمی برای یک نفر درست
  باشد، پس وجود دارد کسی
\end{itemize}

\clearpage
% ---------------------------------------------------------------------
% Copyright (c) 2026 Arsalan Dalvand & Reyhaneh Darvishi.
% Licensed under CC BY-NC-SA 4.0.
% See LICENSE file for details.
% ---------------------------------------------------------------------

\section{\texorpdfstring{مفهوم استقرای ریاضی
\lr{(Mathematical Induction)}}{مفهوم استقرای ریاضی }}\label{مفهوم-استقرا}
\begin{tldr}{خلاصه سریع}
استقرای ریاضی یک تکنیک قدرتمند برای اثبات قضایایی است که برای «تمام
اعداد طبیعی» بیان می‌شوند. این روش شبیه «ریختن دومینو» است: ۱. ضربه به
اولی (پایه)، ۲. اطمینان از اینکه افتادن هر مهره باعث افتادن مهره بعدی
می‌شود (گام استقرا).
\end{tldr}
\subsection{۱. درک شهودی (اثر
دومینو)}\label{ux62fux631ux6a9-ux634ux647ux648ux62fux6cc-ux627ux62bux631-ux62fux648ux645ux6ccux646ux648}
فرض کنید صفی بی‌نهایت از دومینوها چیده‌ایم. برای اینکه مطمئن شویم
\textbf{همه} دومینوها می‌ریزند، فقط باید دو چیز را ثابت کنیم:
\begin{enumerate}
\def\labelenumi{\arabic{enumi}.}
\tightlist
\item
  \textbf{شرط شروع:} دومینو اول می‌افتد.
\item
  \textbf{مکانیزم انتقال:} هر دومینویی که بیفتد، حتماً دومینو بعدی‌اش را
  می‌اندازد. اگر این دو برقرار باشند، کل صف تا بی‌نهایت خواهد ریخت.
\end{enumerate}
\subsection{۲. متن ریاضی اصل
استقرا}\label{ux645ux62aux646-ux631ux6ccux627ux636ux6cc-ux627ux635ux644-ux627ux633ux62aux642ux631ux627}
اگر \(P(n)\) یک حکم مربوط به عدد طبیعی \(n\) باشد، چنانچه دو شرط زیر
برقرار باشند:
\begin{enumerate}
\def\labelenumi{\arabic{enumi}.}
\tightlist
\item
  \textbf{پایه استقرا:} \(P(1)\) راست باشد.
\item
  \textbf{گام استقرا:} برای هر عدد طبیعی \(k\)، اگر \(P(k)\) راست باشد،
  آنگاه \(P(k+1)\) نیز راست باشد (\(P(k) \Rightarrow P(k+1)\)).
\end{enumerate}
آنگاه \(P(n)\) برای \textbf{هر عدد طبیعی} \(n\) راست است.
\subsection{۳. مثال آموزشی (جمع
اعداد)}\label{ux645ux62bux627ux644-ux622ux645ux648ux632ux634ux6cc-ux62cux645ux639-ux627ux639ux62fux627ux62f}
\textbf{حکم:} ثابت کنید \(1 + 2 + \dots + n = \frac{n(n+1)}{2}\).
\begin{info}{مراحل اثبات}
\textbf{۱. بررسی پایه (\(n=1\)):} \[1 = \frac{1(1+1)}{2} = 1\] (سمت چپ و
راست برابرند، پس درست است).
\textbf{۲. فرض استقرا (\(n=k\)):} فرض می‌کنیم حکم برای \(k\) درست است:
\[1 + \dots + k = \frac{k(k+1)}{2}\]
\textbf{۳. حکم استقرا (\(n=k+1\)):} باید ثابت کنیم تساوی برای \(k+1\) هم
برقرار است. به طرفین فرض استقرا، عدد بعدی یعنی \((k+1)\) را اضافه
می‌کنیم: \[1 + \dots + k + (k+1) = \frac{k(k+1)}{2} + (k+1)\] با مخرج
مشترک گرفتن: \[= \frac{k(k+1) + 2(k+1)}{2} = \frac{(k+1)(k+2)}{2}\] این
دقیقاً همان فرمول حکم برای \(n=k+1\) است. پس حکم ثابت شد.
\end{info}
\subsection{\texorpdfstring{۴. شبکه ارتباطی با سایر قضایا
\lr{(Analytic Map)}}{۴. شبکه ارتباطی با سایر قضایا }}\label{ux634ux628ux6a9ux647-ux627ux631ux62aux628ux627ux637ux6cc-ux628ux627-ux633ux627ux6ccux631-ux642ux636ux627ux6ccux627-analytic-map}
\subsubsection{\texorpdfstring{۱. ارتباط با
\autoref{قضیه-۶---قواعد-استنتاج} (قیاس
استثنایی)}{۱. ارتباط با  (قیاس استثنایی)}}\label{ux627ux631ux62aux628ux627ux637-ux628ux627-ux642ux636ux6ccux647-ux6f6---ux642ux648ux627ux639ux62f-ux627ux633ux62aux646ux62aux627ux62c-ux642ux6ccux627ux633-ux627ux633ux62aux62bux646ux627ux6ccux6cc}
\begin{itemize}
\tightlist
\item
  \textbf{موتور محرک:} در گام دوم استقرا، ما ثابت می‌کنیم
  \(P(k) \Rightarrow P(k+1)\). اما این به تنهایی کافی نیست. وقتی پایه
  استقرا (\(P(1)\)) را ثابت کردیم، عملاً از \textbf{قیاس استثنایی}
  استفاده می‌کنیم:
  \begin{itemize}
  \tightlist
  \item
    مقدم ۱: \(P(1)\) (ثابت شده در پایه)
  \item
    مقدم ۲: \(P(1) \Rightarrow P(2)\) (ثابت شده در گام)
  \item
    نتیجه: \(P(2)\) (و این زنجیره ادامه می‌یابد\ldots).
  \end{itemize}
\end{itemize}
\subsubsection{۲. ارتباط با تعاریف
بازگشتی}\label{ux627ux631ux62aux628ux627ux637-ux628ux627-ux62aux639ux627ux631ux6ccux641-ux628ux627ux632ux6afux634ux62aux6cc}
\begin{itemize}
\tightlist
\item
  \textbf{تعریف بازگشتی:} استقرا ارتباط نزدیکی با تعاریفی دارد که خودشان
  را صدا می‌زنند (مثل تعریف توان \(x^{n+1} = x^n \cdot x\) یا فاکتوریل).
  این تعاریف، مواد اولیه برای اثبات‌های استقرایی هستند.
\end{itemize}

\clearpage
% ---------------------------------------------------------------------
% Copyright (c) 2026 Arsalan Dalvand & Reyhaneh Darvishi.
% Licensed under CC BY-NC-SA 4.0.
% See LICENSE file for details.
% ---------------------------------------------------------------------

\section{\texorpdfstring{قضیه ۸: فرمول ترکیب
\lr{(Combination)}}{قضیه ۸: فرمول ترکیب }}\label{قضیه-۸---ترکیب}
\begin{tldr}{خلاصه سریع}
این قضیه فرمول محاسبه تعداد راه‌های انتخاب \(r\) شیء از بین \(n\) شیء
متمایز را می‌دهد، به شرطی که \textbf{ترتیب مهم نباشد}. به این مقدار،
«ضریب دوجمله‌ای» نیز می‌گویند.
\end{tldr}
\subsection{۱. تعاریف
پیش‌نیاز}\label{ux62aux639ux627ux631ux6ccux641-ux67eux6ccux634ux646ux6ccux627ux632}
قبل از فرمول، نیاز به تعریف فاکتوریل و تعریف بازگشتی ترکیب داریم:
\begin{itemize}
\tightlist
\item
  \textbf{فاکتوریل (\(n!\)):} حاصل‌ضرب اعداد طبیعی از ۱ تا \(n\).
  (قرارداد: \(0! = 1\)).
\item
  \textbf{تعریف بازگشتی \(C(n,r)\):}
  \begin{itemize}
  \tightlist
  \item
    \(C(n,0) = 1\)
  \item
    \(C(n,r) = C(n-1, r) + C(n-1, r-1)\).
  \end{itemize}
\end{itemize}
\subsection{۲. متن ریاضی
قضیه}\label{ux645ux62aux646-ux631ux6ccux627ux636ux6cc-ux642ux636ux6ccux647}
اگر \(n\) و \(r\) اعداد صحیح باشند به‌طوری که \(0 \le r \le n\)، آنگاه:
\begin{theorembox}{قضیه ۸}
\[C(n, r) = \frac{n!}{r!(n-r)!}\]
\end{theorembox}
\subsection{۳. تحلیل}\label{ux62aux62dux644ux6ccux644}
این فرمول نشان می‌دهد که \(C(n,r)\) همیشه یک عدد صحیح است.
\begin{itemize}
\tightlist
\item
  صورت کسر (\(n!\)) کل جایگشت‌هاست.
\item
  مخرج کسر (\(r!\)) ترتیبِ \(r\) شیء انتخاب شده را حذف می‌کند (چون در
  ترکیب، ترتیب مهم نیست).
\item
  عبارت \((n-r)!\) ترتیبِ اشیاء باقی‌مانده را حذف می‌کند.
\end{itemize}
\subsection{\texorpdfstring{۴. شبکه ارتباطی با سایر قضایا
\lr{(Analytic Map)}}{۴. شبکه ارتباطی با سایر قضایا }}\label{ux634ux628ux6a9ux647-ux627ux631ux62aux628ux627ux637ux6cc-ux628ux627-ux633ux627ux6ccux631-ux642ux636ux627ux6ccux627-analytic-map}
\subsubsection{\texorpdfstring{۱. ارتباط با
\autoref{مفهوم-استقرا}}{۱. ارتباط با }}\label{ux627ux631ux62aux628ux627ux637-ux628ux627-ux645ux641ux647ux648ux645-ux627ux633ux62aux642ux631ux627}
\begin{itemize}
\tightlist
\item
  این فرمول را می‌توان با استفاده از تعریف بازگشتی آن و روش
  \textbf{\autoref{مفهوم-استقرا}} اثبات کرد (هرچند کتاب اثبات را واگذار
  کرده است).
\end{itemize}
\subsubsection{\texorpdfstring{۲. ارتباط با
\autoref{قضیه-۹---دو-جمله-ای}}{۲. ارتباط با }}\label{ux627ux631ux62aux628ux627ux637-ux628ux627-ux642ux636ux6ccux647-ux6f9---ux62fux648-ux62cux645ux644ux647-ux627ux6cc}
\begin{itemize}
\tightlist
\item
  \textbf{نقش کلیدی:} این اعداد \(C(n,r)\) دقیقاً همان ضرایبی هستند که در
  بسط پرانتزهای توان‌دار \((x+y)^n\) ظاهر می‌شوند. به همین دلیل به آن‌ها
  «ضرایب دوجمله‌ای» می‌گویند.
\end{itemize}

\clearpage
% ---------------------------------------------------------------------
% Copyright (c) 2026 Arsalan Dalvand & Reyhaneh Darvishi.
% Licensed under CC BY-NC-SA 4.0.
% See LICENSE file for details.
% ---------------------------------------------------------------------

\section{\texorpdfstring{قضیه ۹: قضیه دوجمله‌ای
\lr{(Binomial Theorem)}}{قضیه ۹: قضیه دوجمله‌ای }}\label{قضیه-۹---دو-جمله-ای}
\begin{tldr}{خلاصه سریع}
این قضیه الگوی باز کردن اتحاد \((x+y)^n\) را برای هر توان طبیعی \(n\)
بیان می‌کند. ضرایب هر جمله، همان اعداد ترکیب (\(C(n,r)\)) هستند که در
قضیه ۸ معرفی شدند.
\end{tldr}
\subsection{۱. متن ریاضی
قضیه}\label{ux645ux62aux646-ux631ux6ccux627ux636ux6cc-ux642ux636ux6ccux647}
اگر \(x\) و \(y\) دو متغیر و \(n\) یک عدد طبیعی باشد، آنگاه:
\begin{theorembox}{قضیه ۹}
\[(x+y)^n = C(n,0)x^n + C(n,1)x^{n-1}y + \dots + C(n,r)x^{n-r}y^r + \dots + C(n,n)y^n\]
یا به فرم سیگما (جمع): \[(x+y)^n = \sum_{r=0}^{n} C(n,r) x^{n-r} y^r\]
\end{theorembox}
\subsection{۲. اثبات (با استفاده از
استقرا)}\label{ux627ux62bux628ux627ux62a-ux628ux627-ux627ux633ux62aux641ux627ux62fux647-ux627ux632-ux627ux633ux62aux642ux631ux627}
از روش \textbf{\autoref{مفهوم-استقرا}} استفاده می‌کنیم:
\begin{info}{مراحل اثبات}
\textbf{۱. پایه استقرا (\(n=1\)):}
\[(x+y)^1 = C(1,0)x + C(1,1)y = 1x + 1y = x+y\] (حکم برقرار است).
\textbf{۲. فرض استقرا (\(n=k\)):} فرض می‌کنیم حکم برای \(k\) درست باشد:
\[(x+y)^k = \sum_{r=0}^{k} C(k,r)x^{k-r}y^r\]
\textbf{۳. گام استقرا (\(n=k+1\)):} طرفین فرض را در \((x+y)\) ضرب
می‌کنیم: \[(x+y)^{k+1} = (x+y) [C(k,0)x^k + \dots + C(k,k)y^k]\] با ضرب
کردن \(x\) در تمام جملات و سپس \(y\) در تمام جملات و فاکتورگیری از
توان‌های مشابه \(x^{k+1-r}y^r\)، ضرایب به صورت مجموع دو جمله قبلی ظاهر
می‌شوند: \[C(k+1, r) = C(k, r) + C(k, r-1)\] (این همان تعریف بازگشتی
\textbf{\autoref{قضیه-۸---ترکیب}} است). بدین ترتیب فرمول برای \(k+1\)
ساخته می‌شود.
\end{info}
\subsection{\texorpdfstring{۳. شبکه ارتباطی با سایر قضایا
\lr{(Analytic Map)}}{۳. شبکه ارتباطی با سایر قضایا }}\label{ux634ux628ux6a9ux647-ux627ux631ux62aux628ux627ux637ux6cc-ux628ux627-ux633ux627ux6ccux631-ux642ux636ux627ux6ccux627-analytic-map}
\subsubsection{\texorpdfstring{۱. ارتباط با
\autoref{قضیه-۸---ترکیب}}{۱. ارتباط با }}\label{ux627ux631ux62aux628ux627ux637-ux628ux627-ux642ux636ux6ccux647-ux6f8---ux62aux631ux6a9ux6ccux628}
\begin{itemize}
\tightlist
\item
  \textbf{تامین ضرایب:} قضیه دوجمله‌ای بدون قضیه ۸ معنا ندارد. قضیه ۸
  ابزار محاسبه‌ی «وزن» هر جمله در بسط دوجمله‌ای است.
\end{itemize}
\subsubsection{\texorpdfstring{۲. ارتباط با
\autoref{مفهوم-استقرا}}{۲. ارتباط با }}\label{ux627ux631ux62aux628ux627ux637-ux628ux627-ux645ux641ux647ux648ux645-ux627ux633ux62aux642ux631ux627}
\begin{itemize}
\tightlist
\item
  \textbf{کاربرد عملی:} قضیه ۹ یکی از کلاسیک‌ترین و مهم‌ترین مثال‌های
  کاربرد استقرای ریاضی در جبر است. این نشان می‌دهد که چگونه ابزار منطقی
  (استقرا) برای اثبات حقایق جبری (اتحادها) به کار می‌رود.
\end{itemize}

\clearpage
% ---------------------------------------------------------------------
% Copyright (c) 2026 Arsalan Dalvand & Reyhaneh Darvishi.
% Licensed under CC BY-NC-SA 4.0.
% See LICENSE file for details.
% ---------------------------------------------------------------------

\section{تمرین ۵: قانون توزیع‌پذیری (یا) روی
(و)}\label{تمرین-۵---قانون-توزیع‌پذیری-در-منطق}
\begin{tldr}{خلاصه سریع}
این قانون شبیه ضرب اعداد در پرانتز است: \(a \times (b + c) = ab + ac\).
در منطق هم عملگر \(\lor\) (یا) روی \(\land\) (و) پخش می‌شود.
\[p \lor (q \land r) \equiv (p \lor q) \land (p \lor r)\]
\end{tldr}
\subsection{۱. درک
شهودی}\label{ux62fux631ux6a9-ux634ux647ux648ux62fux6cc}
فرض کن برای استخدام شدن (\(p\)) یا باید «مدرک» (\(q\)) داشته باشی و
«سابقه» (\(r\)). جمله: «یا پارتی دارم (\(p\))، یا (مدرک دارم و سابقه
دارم)». این دقیقاً معادل است با: «(یا پارتی دارم یا مدرک) \textbf{و} (یا
پارتی دارم یا سابقه)». اگر هر کدام از این دو شرطِ داخل پرانتزِ دوم برقرار
نباشد، کل شرط باطل است.
\subsection{۲. اثبات (با جدول
ارزش)}\label{ux627ux62bux628ux627ux62a-ux628ux627-ux62cux62fux648ux644-ux627ux631ux632ux634}
برای اثبات هم‌ارزی منطقی، کافیست نشان دهیم جدول ارزش دو طرف یکسان است.
{\def\LTcaptype{none}
\begin{longtable}[]{@{}
  >{\raggedright\arraybackslash}p{(\linewidth - 14\tabcolsep) * \real{0.0208}}
  >{\raggedright\arraybackslash}p{(\linewidth - 14\tabcolsep) * \real{0.0208}}
  >{\raggedright\arraybackslash}p{(\linewidth - 14\tabcolsep) * \real{0.0208}}
  >{\raggedright\arraybackslash}p{(\linewidth - 14\tabcolsep) * \real{0.0833}}
  >{\raggedright\arraybackslash}p{(\linewidth - 14\tabcolsep) * \real{0.2917}}
  >{\raggedright\arraybackslash}p{(\linewidth - 14\tabcolsep) * \real{0.0833}}
  >{\raggedright\arraybackslash}p{(\linewidth - 14\tabcolsep) * \real{0.0833}}
  >{\raggedright\arraybackslash}p{(\linewidth - 14\tabcolsep) * \real{0.3958}}@{}}
\toprule\noalign{}
\begin{minipage}[b]{\linewidth}\raggedright
\lr{p}
\end{minipage} & \begin{minipage}[b]{\linewidth}\raggedright
\lr{q}
\end{minipage} & \begin{minipage}[b]{\linewidth}\raggedright
\lr{r}
\end{minipage} & \begin{minipage}[b]{\linewidth}\raggedright
\lr{q }\(\land\) \lr{r}
\end{minipage} & \begin{minipage}[b]{\linewidth}\raggedright
\textbf{\lr{LHS: p }\(\lor\) \lr{(q }\(\land\) \lr{r)}}
\end{minipage} & \begin{minipage}[b]{\linewidth}\raggedright
\lr{p }\(\lor\) \lr{q}
\end{minipage} & \begin{minipage}[b]{\linewidth}\raggedright
\lr{p }\(\lor\) \lr{r}
\end{minipage} & \begin{minipage}[b]{\linewidth}\raggedright
\textbf{\lr{RHS: (p }\(\lor\) \lr{q) }\(\land\) \lr{(p }\(\lor\)
\lr{r)}}
\end{minipage} \\
\midrule\noalign{}
\endhead
\bottomrule\noalign{}
\endlastfoot
\lr{T} & \lr{T} & \lr{T} & \lr{T} & \textbf{\lr{T}} & \lr{T} & \lr{T} &
\textbf{\lr{T}} \\
\lr{T} & \lr{T} & \lr{F} & \lr{F} & \textbf{\lr{T}} & \lr{T} & \lr{T} &
\textbf{\lr{T}} \\
\lr{T} & \lr{F} & \lr{T} & \lr{F} & \textbf{\lr{T}} & \lr{T} & \lr{T} &
\textbf{\lr{T}} \\
\lr{T} & \lr{F} & \lr{F} & \lr{F} & \textbf{\lr{T}} & \lr{T} & \lr{T} &
\textbf{\lr{T}} \\
\lr{F} & \lr{T} & \lr{T} & \lr{T} & \textbf{\lr{T}} & \lr{T} & \lr{T} &
\textbf{\lr{T}} \\
\lr{F} & \lr{T} & \lr{F} & \lr{F} & \textbf{\lr{F}} & \lr{T} & \lr{F} &
\textbf{\lr{F}} \\
\lr{F} & \lr{F} & \lr{T} & \lr{F} & \textbf{\lr{F}} & \lr{F} & \lr{T} &
\textbf{\lr{F}} \\
\lr{F} & \lr{F} & \lr{F} & \lr{F} & \textbf{\lr{F}} & \lr{F} & \lr{F} &
\textbf{\lr{F}} \\
\end{longtable}
}
چون ستون \textbf{\lr{LHS}} و \textbf{\lr{RHS}} دقیقاً یکسان هستند، حکم
ثابت شد. \(\blacksquare\)

\clearpage
% ---------------------------------------------------------------------
% Copyright (c) 2026 Arsalan Dalvand & Reyhaneh Darvishi.
% Licensed under CC BY-NC-SA 4.0.
% See LICENSE file for details.
% ---------------------------------------------------------------------

\section{تمرین ۶: افزودن شرط مشترک به
استلزام}\label{تمرین-۶---اثبات-استلزام-شرطی}
\begin{tldr}{خلاصه سریع}
اگر بدانیم «اگر باران بیاید، زمین خیس می‌شود» (\(p \to q\))، می‌توانیم یک
شرط دیگر مثل «هوا سرد است» (\(r\)) را به دو طرف اضافه کنیم. حکم: «اگر
(باران بیاید و سرد باشد)، آنگاه (زمین خیس می‌شود و سرد است)».
\end{tldr}
\subsection{۱. صورت
مسأله}\label{ux635ux648ux631ux62a-ux645ux633ux623ux644ux647}
\begin{info}{سوال}
ثابت کنید:
\[(p \rightarrow q) \implies (p \land r \rightarrow q \land r)\]
\end{info}
\subsection{۲. اثبات
مستقیم}\label{ux627ux62bux628ux627ux62a-ux645ux633ux62aux642ux6ccux645}
\begin{info}{گام‌به‌گام}
فرض می‌کنیم مقدمِ گزاره‌ی اصلی یعنی \((p \rightarrow q)\) \textbf{درست}
باشد. می‌خواهیم نشان دهیم نتیجه هم درست است.
\begin{enumerate}
\def\labelenumi{\arabic{enumi}.}
\tightlist
\item
  فرض کنید سمت چپِ فلشِ دوم درست باشد: یعنی \((p \land r)\) درست است.
\item
  از درست بودن \((p \land r)\) نتیجه می‌گیریم که هم \(p\) درست است و هم
  \(r\) درست است.
\item
  چون \(p\) درست است و طبق فرض اولیه داریم \(p \rightarrow q\)، پس نتیجه
  می‌گیریم \(q\) هم باید درست باشد (قیاس استثنایی).
\item
  الان چی داریم؟
  \begin{itemize}
  \tightlist
  \item
    \(q\) درست است (از مرحله ۳).
  \item
    \(r\) درست است (از مرحله ۲).
  \end{itemize}
\item
  پس ترکیب \((q \land r)\) حتماً درست است.
\item
  چون از درستیِ مقدم (\((p \land r)\)) به درستی تالی (\((q \land r)\))
  رسیدیم، پس استلزام برقرار است.
\end{enumerate}
\(\blacksquare\)
\end{info}

\clearpage
% ---------------------------------------------------------------------
% Copyright (c) 2026 Arsalan Dalvand & Reyhaneh Darvishi.
% Licensed under CC BY-NC-SA 4.0.
% See LICENSE file for details.
% ---------------------------------------------------------------------

\section{تمرین ۷: تعریف باز شده‌ی ترکیب
دوشرطی}\label{تمرین-۷---بازنویسی-ترکیب-دوشرطی}
\begin{tldr}{خلاصه سریع}
عبارت «اگر و تنها اگر» (\(p \leftrightarrow q\)) یعنی این دو گزاره
سرنوشت یکسانی دارند: یا \textbf{هر دو راستگو} هستند (\(p \land q\))، یا
\textbf{هر دو دروغگو} (\(\sim p \land \sim q\)).
\[(p \leftrightarrow q) \equiv (p \land q) \lor (\sim p \land \sim q)\]
\end{tldr}
\subsection{۱. اثبات}\label{ux627ux62bux628ux627ux62a}
\begin{info}{تحلیل حالت‌ها}
گزاره \(p \leftrightarrow q\) زمانی ارزش \textbf{\lr{T}} (درست) دارد که
ارزش \(p\) و \(q\) یکسان باشد. بیایید حالت‌ها را چک کنیم:
\textbf{حالت اول: هر دو درست باشند \lr{(T, T)}}
\begin{itemize}
\tightlist
\item
  سمت چپ: \(T \leftrightarrow T\) (درست).
\item
  سمت راست: \((T \land T) \lor (F \land F) \equiv T \lor F \equiv T\)
  (درست).
\end{itemize}
\textbf{حالت دوم: هر دو نادرست باشند \lr{(F, F)}}
\begin{itemize}
\tightlist
\item
  سمت چپ: \(F \leftrightarrow F\) (درست).
\item
  سمت راست: \((F \land F) \lor (T \land T) \equiv F \lor T \equiv T\)
  (درست).
\end{itemize}
\textbf{حالت سوم: یکی درست و دیگری نادرست \lr{(T, F }یا \lr{F, T)}}
\begin{itemize}
\tightlist
\item
  سمت چپ: \(T \leftrightarrow F\) (نادرست).
\item
  سمت راست: \((T \land F) \lor (F \land T) \equiv F \lor F \equiv F\)
  (نادرست).
\end{itemize}
چون در تمام حالات نتایج یکسان بود، هم‌ارزی برقرار است.
\end{info}

\clearpage
% ---------------------------------------------------------------------
% Copyright (c) 2026 Arsalan Dalvand & Reyhaneh Darvishi.
% Licensed under CC BY-NC-SA 4.0.
% See LICENSE file for details.
% ---------------------------------------------------------------------

\section{تمرین ۱۷: اثبات فرمول مجموع
مکعبات}\label{تمرین-۱۷---فرمول-مجموع-مکعبات}
\begin{tldr}{خلاصه سریع}
مجموع مکعب اعداد طبیعی از ۱ تا \(n\) برابر است با توان‌دومِ فرمولِ مجموعِ
اعداد (یعنی \(\frac{n(n+1)}{2}\) به توان ۲).
\[1^3 + 2^3 + \dots + n^3 = \left[ \frac{n(n+1)}{2} \right]^2 = \frac{n^2(n+1)^2}{4}\]
\end{tldr}
\subsection{۱. صورت
تمرین}\label{ux635ux648ux631ux62a-ux62aux645ux631ux6ccux646}
\begin{info}{سوال}
با استفاده از استقرای ریاضی ثابت کنید برای هر عدد طبیعی \(n\):
\[1^3 + 2^3 + 3^3 + \dots + n^3 = \frac{n^2(n+1)^2}{4}\]
\end{info}
\subsection{۲. اثبات با
استقراء}\label{ux627ux62bux628ux627ux62a-ux628ux627-ux627ux633ux62aux642ux631ux627ux621}
\begin{info}{گام‌های اثبات}
\textbf{گام ۱: پایه استقراء (\(n=1\))}
\begin{itemize}
\tightlist
\item
  سمت چپ: \(1^3 = 1\)
\item
  سمت راست: \(\frac{1^2(1+1)^2}{4} = \frac{1 \times 4}{4} = 1\)
\item
  چون \(1=1\)، حکم برای \(n=1\) برقرار است.
\end{itemize}
\textbf{گام ۲: فرض استقراء} فرض می‌کنیم حکم برای \(n\) برقرار باشد:
\[S_n = \frac{n^2(n+1)^2}{4}\]
\textbf{گام ۳: گام استقراء (اثبات برای \(n+1\))} باید نشان دهیم اگر به
مجموع قبلی، جمله بعدی یعنی \((n+1)^3\) را اضافه کنیم، فرمول باز هم کار
می‌کند.
\[S_{n+1} = S_n + (n+1)^3\]
جایگذاری فرض استقراء: \[= \frac{n^2(n+1)^2}{4} + (n+1)^3\]
فاکتورگیری از \((n+1)^2\) (مشترک در هر دو جمله):
\[= (n+1)^2 \left[ \frac{n^2}{4} + (n+1) \right]\]
مخرج مشترک گرفتن داخل کروشه:
\[= (n+1)^2 \left[ \frac{n^2 + 4(n+1)}{4} \right]\]
ساده‌سازی صورت کسر: \[= (n+1)^2 \left[ \frac{n^2 + 4n + 4}{4} \right]\]
عبارت \(n^2+4n+4\) اتحاد مربع کامل \((n+2)^2\) است:
\[= \frac{(n+1)^2 (n+2)^2}{4}\]
\textbf{نتیجه:} این دقیقا همان فرمول اصلی است که در آن به جای \(n\)،
مقدار \(n+1\) قرار گرفته است. پس حکم ثابت شد. \(\blacksquare\)
\end{info}

\clearpage
% ---------------------------------------------------------------------
% Copyright (c) 2026 Arsalan Dalvand & Reyhaneh Darvishi.
% Licensed under CC BY-NC-SA 4.0.
% See LICENSE file for details.
% ---------------------------------------------------------------------

\section{تمرین ۱۸: قضیه نیکوماخوس (رابطه توان ۲ و
۳)}\label{تمرین-۱۸---رابطه-مجموع-اعداد-و-مجموع-مکعبات}
\begin{tldr}{خلاصه سریع}
یک حقیقت بسیار زیبا در ریاضیات: «مربعِ مجموعِ اعداد» برابر است با «مجموعِ
مکعبِ اعداد». \[(1 + 2 + \dots + n)^2 = 1^3 + 2^3 + \dots + n^3\]
\end{tldr}
\subsection{۱. صورت
تمرین}\label{ux635ux648ux631ux62a-ux62aux645ux631ux6ccux646}
\begin{info}{سوال}
ثابت کنید برای تمام اعداد طبیعی \(n\):
\[(1 + 2 + 3 + \dots + n)^2 = 1^3 + 2^3 + 3^3 + \dots + n^3\]
\end{info}
\subsection{۲. حل
تشریحی}\label{ux62dux644-ux62aux634ux631ux6ccux62dux6cc}
\begin{info}{استراتژی حل}
به جای اثبات مستقیم با استقراء (که ممکن است طولانی شود)، از نتایج
تمرین‌های قبلی استفاده می‌کنیم. ما فرمولِ داخل پرانتز (مجموع اعداد) و فرمولِ
سمت راست (مجموع مکعبات) را جداگانه می‌دانیم. کافیست نشان دهیم این دو با
هم سازگارند.
\end{info}
\begin{info}{اثبات جبری}
\begin{enumerate}
\def\labelenumi{\arabic{enumi}.}
\item
  \textbf{محاسبه سمت چپ \lr{(LHS):}} می‌دانیم مجموع اعداد طبیعی (تصاعد
  حسابی) برابر است با \(\frac{n(n+1)}{2}\). (تمرین ۴ کتاب) پس سمت چپ
  می‌شود:
  \[\text{LHS} = \left( \sum_{i=1}^n i \right)^2 = \left( \frac{n(n+1)}{2} \right)^2\]
  با توان رساندن صورت و مخرج: \[= \frac{n^2(n+1)^2}{4}\]
\item
  \textbf{محاسبه سمت راست \lr{(RHS):}} طبق تمرین ۱۷ (فایل قبلی)، ثابت
  کردیم که مجموع مکعبات برابر است با:
  \[\text{RHS} = \sum_{i=1}^n i^3 = \frac{n^2(n+1)^2}{4}\]
\item
  \textbf{نتیجه‌گیری:} چون \(\text{LHS} = \text{RHS}\)، پس تساوی برقرار
  است. \[(1 + \dots + n)^2 = 1^3 + \dots + n^3\] \(\blacksquare\)
\end{enumerate}
\end{info}

\clearpage
\chapter{مفهوم مجموعه}
\clearpage
% ---------------------------------------------------------------------
% Copyright (c) 2026 Arsalan Dalvand & Reyhaneh Darvishi.
% Licensed under CC BY-NC-SA 4.0.
% See LICENSE file for details.
% ---------------------------------------------------------------------

\section{مفاهیم بنیادین نظریه
مجموعه‌ها}\label{پیشنیاز---مفاهیم-بنیادین-مجموعه‌ها}
\begin{tldr}{خلاصه سریع}
این بخش به تعریف دقیق مفاهیم پایه‌ای می‌پردازد که سنگ‌بنای تمام قضایای فصل
۲ هستند: «زیرمجموعه» بر پایه استلزام منطقی، «مجموعه توانی» به عنوان فضای
تمام زیرمجموعه‌ها، و «خانواده‌های ایندکس‌دار» که تعمیم عملیات جبری با
استفاده از سورها هستند.
\end{tldr}
\subsection{۱. زیرمجموعه و مجموعه
تهی}\label{ux632ux6ccux631ux645ux62cux645ux648ux639ux647-ux648-ux645ux62cux645ux648ux639ux647-ux62aux647ux6cc}
درک دقیق این مفاهیم نیازمند گذر از تعاریف شهودی به تعاریف مبتنی بر منطق
گزاره‌ها (فصل ۱) است.
\subsubsection{الف) تعریف صوری
زیرمجموعه}\label{ux627ux644ux641-ux62aux639ux631ux6ccux641-ux635ux648ux631ux6cc-ux632ux6ccux631ux645ux62cux645ux648ux639ux647}
مجموعه \(A\) را زیرمجموعه \(B\) می‌نامیم (\(A \subseteq B\)) هرگاه شرط
عضویت در \(A\)، عضویت در \(B\) را ایجاب کند. این تعریف با استفاده از
گزاره شرطی و سور عمومی بیان می‌شود:
\begin{tldr}{تعریف ۱: زیرمجموعه}
\[A \subseteq B \iff \forall x (x \in A \rightarrow x \in B)\]
\end{tldr}
\textbf{تحلیل منطقی (ارتباط با جدول ارزش):}
\begin{itemize}
\tightlist
\item
  اگر \(A\) مجموعه تهی باشد (\(\emptyset\))، گزاره \(x \in A\) برای هر
  \(x\) همواره \textbf{نادرست \lr{(F)}} است.
\item
  در منطق گزاره‌ها، استلزام \(F \rightarrow Q\) (انتفای مقدم) همواره
  \textbf{راست \lr{(T)}} است.
\item
  بنابراین، شرط زیرمجموعه بودن برای \(\emptyset\) نسبت به هر مجموعه
  دلبخواه \(B\) به صورت «تُهی‌مایه» \lr{(Vacuously) }برقرار است.
\end{itemize}
\subsubsection{ب) تعریف اصل موضوعی
تهی}\label{ux628-ux62aux639ux631ux6ccux641-ux627ux635ux644-ux645ux648ux636ux648ux639ux6cc-ux62aux647ux6cc}
مجموعه تهی (\(\emptyset\))، مجموعه‌ای است که فاقد عضو باشد. این تعریف را
می‌توان با استفاده از اصل تصریح \lr{(Specification) }و یک گزاره همواره
تناقض (مانند \(x \neq x\)) صورتمندی کرد:
\begin{tldr}{تعریف ۲: تهی}
\[\emptyset = \{ x \mid x \neq x \}\]
\end{tldr}
\begin{center}\rule{0.5\linewidth}{0.5pt}\end{center}
\subsection{\texorpdfstring{۲. مجموعه توانی
\lr{(Power Set)}}{۲. مجموعه توانی }}\label{ux645ux62cux645ux648ux639ux647-ux62aux648ux627ux646ux6cc-power-set}
مجموعه توانی، گذر از «عنصر» به «مجموعه» است. این مجموعه، فضای حالت تمام
زیرمجموعه‌های ممکن را می‌سازد.
\subsubsection{تعریف
ریاضی}\label{ux62aux639ux631ux6ccux641-ux631ux6ccux627ux636ux6cc}
برای هر مجموعه \(A\)، مجموعه توانی آن که با \(\mathcal{P}(A)\) یا
\(2^A\) نمایش داده می‌شود، عبارت است از مجموعه تمام زیرمجموعه‌های \(A\):
\begin{tldr}{تعریف ۳: مجموعه توانی}
\[\mathcal{P}(A) = \{ S \mid S \subseteq A \}\]
\end{tldr}
\textbf{نکته کاردینالیتی:} اگر \(|A| = n\) باشد، آنگاه
\(|\mathcal{P}(A)| = 2^n\). این ویژگی با استفاده از بسط دوجمله‌ای
\((1+1)^n\) و مجموع ضرایب ترکیب \(\sum C(n,r)\) قابل اثبات است.
\begin{center}\rule{0.5\linewidth}{0.5pt}\end{center}
\subsection{\texorpdfstring{۳. خانواده‌های ایندکس‌دار
\lr{(Indexed Families)}}{۳. خانواده‌های ایندکس‌دار }}\label{ux62eux627ux646ux648ux627ux62fux647ux647ux627ux6cc-ux627ux6ccux646ux62fux6a9ux633ux62fux627ux631-indexed-families}
زمانی که با مجموعه‌های نامتناهی یا تعداد دلخواهی از مجموعه‌ها سر و کار
داریم، عملیات اجتماع و اشتراک باینری (دوتایی) کافی نیستند. در اینجا از
یک \textbf{مجموعه اندیس (\(\Gamma\))} استفاده می‌کنیم که به هر عضو آن
(\(\gamma \in \Gamma\)) یک مجموعه \(A_\gamma\) اختصاص داده شده است.
\subsubsection{تعمیم عملیات با
سورها}\label{ux62aux639ux645ux6ccux645-ux639ux645ux644ux6ccux627ux62a-ux628ux627-ux633ux648ux631ux647ux627}
تعاریف اجتماع و اشتراک تعمیم‌یافته، ترجمه مستقیم سورهای وجودی
(\(\exists\)) و عمومی (\(\forall\)) در نظریه مجموعه‌ها هستند:
\begin{tldr}{تعاریف ۴ و ۵: عملیات تعمیم‌یافته}
\textbf{۱. اجتماع \lr{(Generalized Union):}} معادل سور وجودی
\[x \in \bigcup_{\gamma \in \Gamma} A_\gamma \iff \exists \gamma \in \Gamma, (x \in A_\gamma)\]
\emph{(ترجمه: \(x\) عضو اجتماع است اگر \textbf{حداقل یک} اندیس موجود
باشد که \(x\) در مجموعه متناظر آن باشد).}
\textbf{۲. اشتراک \lr{(Generalized Intersection):}} معادل سور عمومی
\[x \in \bigcap_{\gamma \in \Gamma} A_\gamma \iff \forall \gamma \in \Gamma, (x \in A_\gamma)\]
\emph{(ترجمه: \(x\) عضو اشتراک است اگر \textbf{به ازای تمام} اندیس‌های
موجود، \(x\) در مجموعه‌های متناظر باشد).}
\end{tldr}
\begin{center}\rule{0.5\linewidth}{0.5pt}\end{center}
\subsection{\texorpdfstring{۴. شبکه ارتباطی با قضایای فصل
\lr{(Analytic Map)}}{۴. شبکه ارتباطی با قضایای فصل }}\label{ux634ux628ux6a9ux647-ux627ux631ux62aux628ux627ux637ux6cc-ux628ux627-ux642ux636ux627ux6ccux627ux6cc-ux641ux635ux644-analytic-map}
این تعاریف، زیربنای اثبات قضایای اصلی فصل ۲ و ۳ هستند:
\subsubsection{\texorpdfstring{۱. زیرمجموعه \(\leftarrow\)
\autoref{قضیه-۱---شمول-تهی}}{۱. زیرمجموعه \textbackslash leftarrow }}\label{ux632ux6ccux631ux645ux62cux645ux648ux639ux647-leftarrow-ux642ux636ux6ccux647-ux6f1---ux634ux645ux648ux644-ux62aux647ux6cc}
\begin{itemize}
\tightlist
\item
  اثبات اینکه «تهی زیرمجموعه هر مجموعه‌ای است»، مستقیماً از تحلیل جدول
  ارزش گزاره شرطی در تعریف ۱ (انتفای مقدم) استخراج می‌شود.
\end{itemize}
\subsubsection{\texorpdfstring{۲. استلزام منطقی \(\leftarrow\)
\autoref{قضیه-۲---تعدی-در-مجموعه-ها}}{۲. استلزام منطقی \textbackslash leftarrow }}\label{ux627ux633ux62aux644ux632ux627ux645-ux645ux646ux637ux642ux6cc-leftarrow-ux642ux636ux6ccux647-ux6f2---ux62aux639ux62fux6cc-ux62fux631-ux645ux62cux645ux648ux639ux647-ux647ux627}
\begin{itemize}
\tightlist
\item
  خاصیت تعدی زیرمجموعه‌ها
  (\(A \subseteq B \wedge B \subseteq C \Rightarrow A \subseteq C\))،
  بازتاب مستقیم قانون تعدی در منطق
  (\(p \to q \wedge q \to r \Rightarrow p \to r\)) است که در تعریف
  زیرمجموعه نهفته است.
\end{itemize}
\subsubsection{\texorpdfstring{۳. مجموعه توانی \(\leftarrow\)
\autoref{قضیه-۳---تعداد-اعضای-مجموعه-توانی}}{۳. مجموعه توانی \textbackslash leftarrow }}\label{ux645ux62cux645ux648ux639ux647-ux62aux648ux627ux646ux6cc-leftarrow-ux642ux636ux6ccux647-ux6f3---ux62aux639ux62fux627ux62f-ux627ux639ux636ux627ux6cc-ux645ux62cux645ux648ux639ux647-ux62aux648ux627ux646ux6cc}
\begin{itemize}
\tightlist
\item
  قضیه ۳ که تعداد زیرمجموعه‌ها را شمارش می‌کند، بر پایه تعریف مجموعه توانی
  و ارتباط آن با ضرایب دوجمله‌ای (فصل ۱) بنا شده است.
\end{itemize}
\subsubsection{\texorpdfstring{۴. سورهای عمومی \(\leftarrow\)
\autoref{قضیه-۷.---تعمیم-اشتراک-و-اجتماع}}{۴. سورهای عمومی \textbackslash leftarrow }}\label{ux633ux648ux631ux647ux627ux6cc-ux639ux645ux648ux645ux6cc-leftarrow-ux642ux636ux6ccux647-ux6f7.---ux62aux639ux645ux6ccux645-ux627ux634ux62aux631ux627ux6a9-ux648-ux627ux62cux62aux645ux627ux639}
\begin{itemize}
\tightlist
\item
  تعریف اشتراک با سور \(\forall\) گره خورده است.\_در قضیه ۷، وقتی
  \(\Gamma = \emptyset\) باشد، شرط «به ازای هر \(\gamma \in \emptyset\)»
  به دلیل دروغ بودن مقدم، برای تمام عناصر جهان (\(U\)) درست می‌شود. این
  پدیده \lr{(Vacuous Truth) }باعث می‌شود اشتراک روی تهی برابر با مجموعه
  مرجع گردد.
\end{itemize}
\subsubsection{\texorpdfstring{۵. سورها و نقیض \(\leftarrow\)
\autoref{قضیه-۸---تعمیم-دمورگان}}{۵. سورها و نقیض \textbackslash leftarrow }}\label{ux633ux648ux631ux647ux627-ux648-ux646ux642ux6ccux636-leftarrow-ux642ux636ux6ccux647-ux6f8---ux62aux639ux645ux6ccux645-ux62fux645ux648ux631ux6afux627ux646}
\begin{itemize}
\tightlist
\item
  \_از آنجا که اجتماع با \(\exists\) و اشتراک با \(\forall\) تعریف
  می‌شوند، قوانین دمورگان برای مجموعه‌ها (\((\cup A_i)' = \cap A_i'\)
  دقیقاً همان قوانین \textbf{نقیض سورها} در منطق
  (\(\sim \exists \equiv \forall \sim\)) هستند.
\end{itemize}

\clearpage
% ---------------------------------------------------------------------
% Copyright (c) 2026 Arsalan Dalvand & Reyhaneh Darvishi.
% Licensed under CC BY-NC-SA 4.0.
% See LICENSE file for details.
% ---------------------------------------------------------------------

\section{\texorpdfstring{تعریف مجموعه توانی
\lr{(Power Set)}}{تعریف مجموعه توانی }}\label{پیشنیاز---تعریف-مجموعه-توانی}
\begin{tldr}{خلاصه سریع}
مجموعه‌ای که اعضایش، «تمام زیرمجموعه‌های ممکن» مجموعه اصلی هستند. آن را با
\(\mathcal{P}(A)\) یا \(2^A\) نشان می‌دهند.
\end{tldr}
\subsection{۱. مثال
ساده}\label{ux645ux62bux627ux644-ux633ux627ux62fux647}
اگر \(A = \{1, 2\}\) باشد، زیرمجموعه‌هایش عبارتند از:
\begin{itemize}
\tightlist
\item
  \(\emptyset\) (هیچ‌کدام)
\item
  \(\{1\}\)
\item
  \(\{2\}\)
\item
  \(\{1, 2\}\) (هر دو) پس مجموعه توانی می‌شود:
  \[\mathcal{P}(A) = \{ \emptyset, \{1\}, \{2\}, \{1, 2\} \}\]
\end{itemize}
\subsection{۲. تعریف
ریاضی}\label{ux62aux639ux631ux6ccux641-ux631ux6ccux627ux636ux6cc}
\begin{tldr}{تعریف}
\[\mathcal{P}(A) = \{ S \mid S \subseteq A \}\]
\end{tldr}

\clearpage
% ---------------------------------------------------------------------
% Copyright (c) 2026 Arsalan Dalvand & Reyhaneh Darvishi.
% Licensed under CC BY-NC-SA 4.0.
% See LICENSE file for details.
% ---------------------------------------------------------------------

\section{قضیه ۱: شمول مجموعه
تهی}\label{قضیه-۱---شمول-تهی}
\begin{tldr}{خلاصه سریع}
این قضیه بیان می‌کند که «مجموعه تهی» (\(\emptyset\)) عنصرِ کمین
\lr{(Minimal Element) }در رابطه شمول است. یعنی \(\emptyset\) زیرمجموعه
هر مجموعه دلبخواهی محسوب می‌شود. صدق این گزاره بر پایه مفهوم منطقی «صدق
تُهی‌مایه» \lr{(Vacuous Truth) }استوار است.
\end{tldr}
\subsection{۱. متن ریاضی
قضیه}\label{ux645ux62aux646-ux631ux6ccux627ux636ux6cc-ux642ux636ux6ccux647}
برای هر مجموعه دلبخواه \(A\):
\begin{theorembox}{قضیه ۱}
\[\forall A, (\emptyset \subseteq A)\]
\end{theorembox}
\subsection{\texorpdfstring{۲. اثبات صوری
\lr{(Formal Proof)}}{۲. اثبات صوری }}\label{ux627ux62bux628ux627ux62a-ux635ux648ux631ux6cc-formal-proof}
اثبات این قضیه کاربرد مستقیم تعاریف منطق گزاره‌ها در نظریه مجموعه‌هاست و
بر پایه تحلیل گزاره شرطی بنا شده است.
\begin{info}{مراحل اثبات}
۱. طبق \textbf{\autoref{پیشنیاز---مفاهیم-بنیادین-مجموعه‌ها}}، باید ثابت
کنیم گزاره شرطی زیر به ازای هر \(x\) صادق است:
\[(x \in \emptyset) \rightarrow (x \in A)\] ۲. طبق
\textbf{\autoref{پیشنیاز---مفاهیم-بنیادین-مجموعه‌ها}}، گزاره
\(x \in \emptyset\) همواره \textbf{نادرست} \lr{(False) }است. در منطق،
این گزاره را با نماد تناقض (\(c\)) نشان می‌دهیم. ۳. بنابراین ساختار گزاره
شرطی به فرم \(c \rightarrow P\) تبدیل می‌شود (که \(P\) گزاره \(x \in A\)
است
\end{info}

\clearpage
% ---------------------------------------------------------------------
% Copyright (c) 2026 Arsalan Dalvand & Reyhaneh Darvishi.
% Licensed under CC BY-NC-SA 4.0.
% See LICENSE file for details.
% ---------------------------------------------------------------------

\section{قضیه ۲: خاصیت تعدی در شمول
مجموعه‌ها}\label{قضیه-۲---تعدی-در-مجموعه-ها}
\begin{tldr}{خلاصه سریع}
این قضیه بیانگر ویژگی «تعدی» \lr{(Transitivity) }در رابطه زیرمجموعه بودن
(\(\subseteq\)) است. این ویژگی نشان می‌دهد که ساختار شمول مجموعه‌ها، یک
ساختار سلسله‌مراتب منطقی است که مستقیماً از ویژگی تعدی در استلزام منطقی
پیروی می‌کند.
\end{tldr}
\subsection{۱. متن ریاضی
قضیه}\label{ux645ux62aux646-ux631ux6ccux627ux636ux6cc-ux642ux636ux6ccux647}
برای هر سه مجموعه دلبخواه \(A\)، \(B\) و \(C\):
\begin{theorembox}{قضیه ۲}
\[(A \subseteq B) \wedge (B \subseteq C) \Rightarrow (A \subseteq C)\]
\end{theorembox}
\subsection{\texorpdfstring{۲. اثبات صوری
\lr{(Formal Proof)}}{۲. اثبات صوری }}\label{ux627ux62bux628ux627ux62a-ux635ux648ux631ux6cc-formal-proof}
اثبات این قضیه، ترجمه مستقیم \textbf{قانون تعدی در منطق گزاره‌ها} به زبان
نظریه مجموعه‌هاست.
\begin{info}{مراحل اثبات}
۱. طبق \textbf{\autoref{پیشنیاز---مفاهیم-بنیادین-مجموعه‌ها}}، باید ثابت
کنیم: \[\forall x (x \in A \rightarrow x \in C)\] ۲. فرض می‌کنیم مقدمات
حکم برقرار باشند:
\begin{itemize}
\tightlist
\item
  فرض ۱: \(A \subseteq B \iff \forall x (x \in A \rightarrow x \in B)\)
\item
  فرض ۲: \(B \subseteq C \iff \forall x (x \in B \rightarrow x \in C)\)
  ۳. گزاره‌های اتمیک زیر را در نظر می‌گیریم:
\item
  \(p: x \in A\)
\item
  \(q: x \in B\)
\item
  \(r: x \in C\) ۴. بنابراین ما دو گزاره شرطی داریم:
  \((p \rightarrow q)\) و \((q \rightarrow r)\). ۵. طبق
  \textbf{\autoref{قضیه-۴---قوانین-شرکت-پذیری-و-پخش-پذیری}} (قانون تعدی
  یا قیاس شرطی):
  \[(p \rightarrow q) \wedge (q \rightarrow r) \Rightarrow (p \rightarrow r)\]
  ۶. نتیجه می‌شود که \((x \in A \rightarrow x \in C)\) برای هر \(x\)
  برقرار است. ۷. پس طبق تعریف، \(A \subseteq C\).
\end{itemize}
\end{info}
\subsection{\texorpdfstring{۳. شبکه ارتباطی با سایر قضایا
\lr{(Analytic Map)}}{۳. شبکه ارتباطی با سایر قضایا }}\label{ux634ux628ux6a9ux647-ux627ux631ux62aux628ux627ux637ux6cc-ux628ux627-ux633ux627ux6ccux631-ux642ux636ux627ux6ccux627-analytic-map}
این قضیه یک اصل ساختاری مهم است که در سراسر نظریه مجموعه‌ها تکرار می‌شود:
\subsubsection{\texorpdfstring{۱. ارتباط با
\autoref{قضیه-۴---قوانین-شرکت-پذیری-و-پخش-پذیری}}{۱. ارتباط با }}\label{ux627ux631ux62aux628ux627ux637-ux628ux627-ux642ux636ux6ccux647-ux6f4---ux642ux648ux627ux646ux6ccux646-ux634ux631ux6a9ux62a-ux67eux630ux6ccux631ux6cc-ux648-ux67eux62eux634-ux67eux630ux6ccux631ux6cc}
\begin{itemize}
\tightlist
\item
  \textbf{ایزومورفیسم منطق و مجموعه:} همان‌طور که در اثبات دیدیم، خاصیت
  تعدی زیرمجموعه‌ها (\(A \subseteq B \subseteq C\)) دقیقاً تصویر آینه‌ای
  خاصیت تعدی استلزام (\(p \to q \to r\)) است. این نشان می‌دهد که رابطه
  \(\subseteq\) در مجموعه‌ها، همان رفتار \(\rightarrow\) در منطق را دارد.
\end{itemize}
\subsubsection{\texorpdfstring{۲. ارتباط با
\autoref{قضیه-۱---شمول-تهی}}{۲. ارتباط با }}\label{ux627ux631ux62aux628ux627ux637-ux628ux627-ux642ux636ux6ccux647-ux6f1---ux634ux645ux648ux644-ux62aux647ux6cc}
\begin{itemize}
\tightlist
\item
  \textbf{سازگاری در کران پایین:} طبق قضیه ۱، \(\emptyset \subseteq A\).
  حال اگر \(A \subseteq B\) باشد، طبق قضیه تعدی باید
  \(\emptyset \subseteq B\) باشد. این نتیجه با قضیه ۱ سازگار است و نشان
  می‌دهد که سلسله‌مراتب مجموعه‌ها از «تهی» شروع شده و با خاصیت تعدی به بالا
  گسترش می‌یابد.
\end{itemize}
\subsubsection{۳. پیش‌زمینه برای فصل ۳
(رابطه‌ها)}\label{ux67eux6ccux634ux632ux645ux6ccux646ux647-ux628ux631ux627ux6cc-ux641ux635ux644-ux6f3-ux631ux627ux628ux637ux647ux647ux627}
\begin{itemize}
\tightlist
\item
  \textbf{رابطه ترتیب جزئی \lr{(Partial Order):}} در فصل آینده خواهیم
  دید که هر رابطه‌ای که سه ویژگی «بازتابی» (\(A \subseteq A\))،
  «پادتقارنی» و «تعدی» (همین قضیه) را داشته باشد، یک ترتیب جزئی است.
  بنابراین، رابطه شمول (\(\subseteq\)) یک ترتیب جزئی روی مجموعه توانی
  است.
\end{itemize}

\clearpage
% ---------------------------------------------------------------------
% Copyright (c) 2026 Arsalan Dalvand & Reyhaneh Darvishi.
% Licensed under CC BY-NC-SA 4.0.
% See LICENSE file for details.
% ---------------------------------------------------------------------

\section{قضیه ۳: کاردینالیتی مجموعه
توانی}\label{قضیه-۳---تعداد-اعضای-مجموعه-توانی}
\begin{tldr}{خلاصه سریع}
این قضیه رابطه نمایی بین اندازه یک مجموعه و اندازه مجموعه توانی آن را
بیان می‌کند. اگر مجموعه‌ای \(n\) عضو داشته باشد، فضای حالت زیرمجموعه‌های آن
(مجموعه توانی) \(2^n\) عضو خواهد داشت.
\end{tldr}
\subsection{۱. متن ریاضی
قضیه}\label{ux645ux62aux646-ux631ux6ccux627ux636ux6cc-ux642ux636ux6ccux647}
فرض کنید \(A\) یک مجموعه متناهی با \(n\) عضو باشد (یعنی \(|A| = n\)). در
این صورت تعداد اعضای مجموعه توانی \(A\) برابر است با:
\begin{theorembox}{قضیه ۳}
\[|\mathcal{P}(A)| = 2^n\]
\end{theorembox}
\subsection{۲. اثبات‌های
صوری}\label{ux627ux62bux628ux627ux62aux647ux627ux6cc-ux635ux648ux631ux6cc}
\subsubsection{اثبات اول: تناظر یک‌به‌یک با رشته‌های
باینری}\label{ux627ux62bux628ux627ux62a-ux627ux648ux644-ux62aux646ux627ux638ux631-ux6ccux6a9ux628ux647ux6ccux6a9-ux628ux627-ux631ux634ux62aux647ux647ux627ux6cc-ux628ux627ux6ccux646ux631ux6cc}
این اثبات بر اساس ساختن یک تناظر یک‌به‌یک \lr{(Bijection) }بین
زیرمجموعه‌های \(A\) و توابع مشخصه \lr{(Characteristic Functions) }بنا شده
است.
\begin{info}{اثبات ترکیبیاتی}
۱. فرض کنید \(A = \{a_1, a_2, \dots, a_n\}\). ۲. هر زیرمجموعه
\(S \subseteq A\) را می‌توان با یک دنباله دوتایی
\lr{(Binary Sequence) }به طول \(n\) نمایش داد، به طوری که \(i\)-امین
مؤلفه ۱ باشد اگر \(a_i \in S\) و ۰ باشد اگر \(a_i \notin S\). ۳. برای هر
عنصر \(a_i\) دقیقاً ۲ حالت وجود دارد (حضور یا عدم حضور). ۴. طبق اصل ضرب
در ترکیبیات، تعداد کل حالت‌های ممکن برای ساختن این دنباله‌ها برابر است با:
\[\underbrace{2 \times 2 \times \dots \times 2}_{n \text{ times}} = 2^n\]
۵. بنابراین تعداد زیرمجموعه‌ها نیز \(2^n\) است.
\end{info}
\subsubsection{اثبات دوم: استفاده از بسط
دوجمله‌ای}\label{ux627ux62bux628ux627ux62a-ux62fux648ux645-ux627ux633ux62aux641ux627ux62fux647-ux627ux632-ux628ux633ux637-ux62fux648ux62cux645ux644ux647ux627ux6cc}
این اثبات از افراز مجموعه توانی بر اساس «اندازه زیرمجموعه‌ها» استفاده
می‌کند.
\begin{info}{اثبات جبری}
۱. می‌دانیم تعداد زیرمجموعه‌های \(k\)-عضوی از یک مجموعه \(n\)-عضوی برابر
با ترکیب \(k\) از \(n\) یا \(C(n,k)\) است. ۲. تعداد کل اعضای
\(\mathcal{P}(A)\) برابر است با مجموع تعداد زیرمجموعه‌های ۰ عضوی، ۱ عضوی،
\ldots{} تا \(n\) عضوی: \[|\mathcal{P}(A)| = \sum_{k=0}^{n} C(n, k)\] ۳.
طبق \textbf{\autoref{قضیه-۹---دو-جمله-ای}} در فصل ۱، داریم:
\[(x+y)^n = \sum_{k=0}^{n} C(n, k) x^{n-k} y^k\] ۴. با جایگذاری \(x=1\)
و \(y=1\) در اتحاد فوق:
\[(1+1)^n = \sum_{k=0}^{n} C(n, k) (1)^{n-k} (1)^k = \sum_{k=0}^{n} C(n, k)\]
۵. نتیجه: \(|\mathcal{P}(A)| = 2^n\).
\end{info}

\clearpage
% ---------------------------------------------------------------------
% Copyright (c) 2026 Arsalan Dalvand & Reyhaneh Darvishi.
% Licensed under CC BY-NC-SA 4.0.
% See LICENSE file for details.
% ---------------------------------------------------------------------

\section{\texorpdfstring{قضیه ۴: قوانین جبر مجموعه‌ها
\lr{(Set Algebra Laws)}}{قضیه ۴: قوانین جبر مجموعه‌ها }}\label{قضیه-۴---جبر-مجموعه-ها}
\begin{tldr}{خلاصه سریع}
این قضیه ساختار جبری حاکم بر مجموعه‌ها را تبیین می‌کند. این قوانین نشان
می‌دهند که \((\mathcal{P}(X), \cup, \cap, ', \emptyset, X)\) یک
\textbf{جبر بول \lr{(Boolean Algebra)}} است که دقیقاً ایزومورف با جبر
گزاره‌ها در منطق ریاضی می‌باشد.
\end{tldr}
\subsection{۱. متن ریاضی
قضیه}\label{ux645ux62aux646-ux631ux6ccux627ux636ux6cc-ux642ux636ux6ccux647}
فرض کنید \(X\) مجموعه مرجع باشد و \(A, B, C\) زیرمجموعه‌هایی از \(X\)
باشند. قوانین زیر همواره برقرارند:
\begin{theorembox}{قضیه ۴}
\textbf{الف) قوانین یکه \lr{(Identity Laws):}} \[A \cup \emptyset = A\]
\[A \cap X = A\]
\textbf{ب) قوانین خودتوانی \lr{(Idempotent Laws):}} \[A \cup A = A\]
\[A \cap A = A\]
\textbf{ج) قوانین جابجایی \lr{(Commutative Laws):}}
\[A \cup B = B \cup A\] \[A \cap B = B \cap A\]
\textbf{د) قوانین شرکت‌پذیری \lr{(Associative Laws):}}
\[A \cup (B \cup C) = (A \cup B) \cup C\]
\[A \cap (B \cap C) = (A \cap B) \cap C\]
\textbf{ه) قوانین پخش‌پذیری \lr{(Distributive Laws):}}
\[A \cap (B \cup C) = (A \cap B) \cup (A \cap C)\]
\[A \cup (B \cap C) = (A \cup B) \cap (A \cup C)\]
\end{theorembox}
\subsection{\texorpdfstring{۲. اثبات صوری
\lr{(Formal Proof)}}{۲. اثبات صوری }}\label{ux627ux62bux628ux627ux62a-ux635ux648ux631ux6cc-formal-proof}
اثبات این قوانین مبتنی بر ترجمه گزاره‌های مجموعه‌ای به گزاره‌های منطقی و
استفاده از هم‌ارزی‌های فصل ۱ است. در اینجا اثبات قانون شرکت‌پذیری اجتماع
(بخش د) به عنوان نمونه ارائه می‌شود.
\begin{info}{اثبات \(A \cup (B \cup C) = (A \cup B) \cup C\)}
برای اثبات تساوی دو مجموعه، نشان می‌دهیم تابع گزاره‌نمای عضویت
(\(x \in S\)) برای هر دو طرف هم‌ارز است:
۱. تعریف طرف چپ:
\[x \in A \cup (B \cup C) \iff x \in A \vee (x \in B \cup C)\]
\[\iff x \in A \vee (x \in B \vee x \in C)\]
۲. اعمال قانون شرکت‌پذیری در منطق (طبق
\textbf{\autoref{قضیه-۴---قوانین-شرکت-پذیری-و-پخش-پذیری}}):
\[\iff (x \in A \vee x \in B) \vee x \in C\]
۳. بازگردانی به تعریف مجموعه: \[\iff x \in (A \cup B) \vee x \in C\]
\[\iff x \in (A \cup B) \cup C\]
۴. نتیجه: چون شرط عضویت برای هر \(x\) در دو مجموعه یکسان است، پس دو
مجموعه برابرند.
\end{info}
\subsection{\texorpdfstring{۳. شبکه ارتباطی با سایر قضایا
\lr{(Analytic Map)}}{۳. شبکه ارتباطی با سایر قضایا }}\label{ux634ux628ux6a9ux647-ux627ux631ux62aux628ux627ux637ux6cc-ux628ux627-ux633ux627ux6ccux631-ux642ux636ux627ux6ccux627-analytic-map}
این قضیه تجلی مستقیم قوانین منطق در دنیای مجموعه‌هاست:
\subsubsection{\texorpdfstring{۱. ارتباط با
\autoref{قضیه-۴---قوانین-شرکت-پذیری-و-پخش-پذیری} (ساختار
مشترک)}{۱. ارتباط با  (ساختار مشترک)}}\label{ux627ux631ux62aux628ux627ux637-ux628ux627-ux642ux636ux6ccux647-ux6f4---ux642ux648ux627ux646ux6ccux646-ux634ux631ux6a9ux62a-ux67eux630ux6ccux631ux6cc-ux648-ux67eux62eux634-ux67eux630ux6ccux631ux6cc-ux633ux627ux62eux62aux627ux631-ux645ux634ux62aux631ux6a9}
\begin{itemize}
\tightlist
\item
  \textbf{ایزومورفیسم:} قوانین شرکت‌پذیری و پخش‌پذیری در مجموعه‌ها، تصویر
  دقیق قوانین متناظر در منطق هستند. اگر جای \(\cup\) را با \(\vee\) و
  جای \(\cap\) را با \(\wedge\) عوض کنیم، دقیقاً به فرمول‌های منطقی فصل ۱
  می‌رسیم. این نشان می‌دهد که منطق و نظریه مجموعه‌ها دو مدل مختلف از یک
  ساختار جبری واحد (جبر بول) هستند.
\end{itemize}
\subsubsection{\texorpdfstring{۲. ارتباط با
\autoref{قضیه-۷---قوانین-تناقض} (یکه و
خودتوانی)}{۲. ارتباط با  (یکه و خودتوانی)}}\label{ux627ux631ux62aux628ux627ux637-ux628ux627-ux642ux636ux6ccux647-ux6f7---ux642ux648ux627ux646ux6ccux646-ux62aux646ux627ux642ux636-ux6ccux6a9ux647-ux648-ux62eux648ux62fux62aux648ux627ux646ux6cc}
\begin{itemize}
\tightlist
\item
  \textbf{تناظر ثابت‌ها:} قوانین یکه \lr{(Identity) }در مجموعه‌ها
  (\(A \cup \emptyset = A\)) متناظر با قوانین همانی در منطق
  (\(p \vee c \equiv p\)) هستند. در این تناظر، مجموعه تهی
  (\(\emptyset\)) نقش تناقض (\(c\)) و مجموعه مرجع (\(X\)) نقش راستگو
  (\(t\)) را بازی می‌کند.
\item
  \textbf{خودتوانی:} قوانین \(A \cup A = A\) نیز مستقیماً از معادل منطقی
  \(p \vee p \equiv p\) (قضیه ۲ فصل ۱) نشأت می‌گیرند.
\end{itemize}
\subsubsection{\texorpdfstring{۳. ارتباط با
\autoref{قضیه-۹---تعمیم-پخش-پذیری}
(تعمیم)}{۳. ارتباط با  (تعمیم)}}\label{ux627ux631ux62aux628ux627ux637-ux628ux627-ux642ux636ux6ccux647-ux6f9---ux62aux639ux645ux6ccux645-ux67eux62eux634-ux67eux630ux6ccux631ux6cc-ux62aux639ux645ux6ccux645}
\begin{itemize}
\tightlist
\item
  \textbf{گسترش به خانواده‌ها:} قوانین پخش‌پذیری بیان شده در اینجا (بخش ه)
  محدود به دو یا سه مجموعه هستند. در قضیه ۹ همین فصل، این قوانین برای
  خانواده‌های نامتناهی از مجموعه‌ها تعمیم داده می‌شوند
  (\(A \cap (\bigcup B_i) = \bigcup (A \cap B_i)\)).
\end{itemize}

\clearpage
% ---------------------------------------------------------------------
% Copyright (c) 2026 Arsalan Dalvand & Reyhaneh Darvishi.
% Licensed under CC BY-NC-SA 4.0.
% See LICENSE file for details.
% ---------------------------------------------------------------------

\section{قضیه ۵: ویژگی‌های جبری متمم و رابطه آن با
شمول}\label{قضیه-۵---متمم-و-زیرمجموعه}
\begin{tldr}{خلاصه سریع}
این قضیه رفتار عملگر «متمم» (\('\)) را در جبر مجموعه‌ها تبیین می‌کند.
مهم‌ترین بخش آن، اثبات هم‌ارزی بین «شمول دو مجموعه» و «شمول متمم‌های آن‌ها
به صورت معکوس» است که پایه‌ی بسیاری از استدلال‌های غیرمستقیم در توپولوژی و
آنالیز می‌باشد.
\end{tldr}
\subsection{۱. متن ریاضی
قضیه}\label{ux645ux62aux646-ux631ux6ccux627ux636ux6cc-ux642ux636ux6ccux647}
فرض کنید \(U\) مجموعه مرجع \lr{(Universal Set) }باشد و
\(A, B \subseteq U\). احکام زیر برقرارند:
\begin{theorembox}{قضیه ۵}
\textbf{الف) قانون نفی مضاعف \lr{(Involution):}} \[(A')' = A\]
\textbf{ب) متمم‌های کرانی:}
\[\emptyset' = U \quad , \quad U' = \emptyset\] \textbf{ج) قوانین مکمل
\lr{(Complement Laws):}}
\[A \cup A' = U \quad , \quad A \cap A' = \emptyset\] \textbf{د) قانون
عکس نقیض مجموعه‌ای \lr{(Contraposition):}}
\[A \subseteq B \iff B' \subseteq A'\]
\end{theorembox}
\subsection{\texorpdfstring{۲. اثبات صوری
\lr{(Formal Proof)}}{۲. اثبات صوری }}\label{ux627ux62bux628ux627ux62a-ux635ux648ux631ux6cc-formal-proof}
\subsubsection{اثبات قسمت (د): عکس
نقیض}\label{ux627ux62bux628ux627ux62a-ux642ux633ux645ux62a-ux62f-ux639ux6a9ux633-ux646ux642ux6ccux636}
این اثبات نشان می‌دهد که ساختار ترتیب شمول (\(\subseteq\)) تحت عملگر
متمم، معکوس می‌شود.
\begin{info}{اثبات}
طبق تعریف زیرمجموعه، باید نشان دهیم گزاره سمت چپ منطقاً با سمت راست هم‌ارز
است:
۱. تعریف طرف چپ (\(A \subseteq B\)):
\[\forall x (x \in A \rightarrow x \in B)\]
۲. استفاده از \textbf{قانون عکس نقیض در منطق}
(\autoref{قضیه-۲---هم‌ارزی‌های-منطقی-پایه}): می‌دانیم
\((p \rightarrow q) \equiv (\sim q \rightarrow \sim p)\). با قرار دادن
\(p: x \in A\) و \(q: x \in B\):
\[\forall x (x \notin B \rightarrow x \notin A)\]
۳. ترجمه به زبان متمم‌ها (\(x \notin S \iff x \in S'\)):
\[\forall x (x \in B' \rightarrow x \in A')\]
۴. تطبیق با تعریف زیرمجموعه برای طرف راست: این دقیقاً تعریف
\(B' \subseteq A'\) است.
\end{info}
\subsubsection{اثبات قسمت (الف): نفی
مضاعف}\label{ux627ux62bux628ux627ux62a-ux642ux633ux645ux62a-ux627ux644ux641-ux646ux641ux6cc-ux645ux636ux627ux639ux641}
\begin{info}{اثبات}
\[x \in (A')' \iff x \notin A' \iff \sim(x \notin A) \iff \sim(\sim(x \in A))\]
طبق قانون نفی مضاعف در منطق (\(\sim \sim p \equiv p\)): \[\iff x \in A\]
\end{info}
\subsection{\texorpdfstring{۳. شبکه ارتباطی با سایر قضایا
\lr{(Analytic Map)}}{۳. شبکه ارتباطی با سایر قضایا }}\label{ux634ux628ux6a9ux647-ux627ux631ux62aux628ux627ux637ux6cc-ux628ux627-ux633ux627ux6ccux631-ux642ux636ux627ux6ccux627-analytic-map}
این قضیه، ترجمه‌ی دقیق اصول «منطق کلاسیک» به «جبر مجموعه‌ها» است:
\subsubsection{\texorpdfstring{۱. ارتباط با
\autoref{قضیه-۲---هم‌ارزی‌های-منطقی-پایه} (ریشه
منطقی)}{۱. ارتباط با  (ریشه منطقی)}}\label{ux627ux631ux62aux628ux627ux637-ux628ux627-ux642ux636ux6ccux647-ux6f2-ux641ux635ux644-ux6f1-ux631ux6ccux634ux647-ux645ux646ux637ux642ux6cc}
\begin{itemize}
\tightlist
\item
  \textbf{تناظر یک‌به‌یک:} بند (الف) این قضیه دقیقاً معادل قانون
  \(\sim(\sim p) \equiv p\) است. بند (د) دقیقاً معادل قانون
  \((p \to q) \equiv (\sim q \to \sim p)\) است. این نشان می‌دهد که
  متمم‌گیری در مجموعه‌ها، ایزومورف با نقیض‌گیری در منطق است.
\end{itemize}
\subsubsection{\texorpdfstring{۲. ارتباط با
\autoref{قضیه-۷---قوانین-تناقض} (ثابت‌های
منطقی)}{۲. ارتباط با  (ثابت‌های منطقی)}}\label{ux627ux631ux62aux628ux627ux637-ux628ux627-ux642ux636ux6ccux647-ux6f7---ux642ux648ux627ux646ux6ccux646-ux62aux646ux627ux642ux636-ux62bux627ux628ux62aux647ux627ux6cc-ux645ux646ux637ux642ux6cc}
\begin{itemize}
\tightlist
\item
  \textbf{جبر بولی:} در بند (ج) دیدیم که \(A \cup A' = U\) و
  \(A \cap A' = \emptyset\). این‌ها معادل قوانین منطقی «طرد شق ثالث»
  (\(p \vee \sim p \equiv t\)) و «عدم تناقض»
  (\(p \wedge \sim p \equiv c\)) هستند. در اینجا \(U\) نقش راستگو
  (\(t\)) و \(\emptyset\) نقش تناقض (\(c\)) را ایفا می‌کند.
\end{itemize}
\subsubsection{\texorpdfstring{۳. ارتباط با
\autoref{قضیه-۱---شمول-تهی}}{۳. ارتباط با }}\label{ux627ux631ux62aux628ux627ux637-ux628ux627-ux642ux636ux6ccux647-ux6f1---ux634ux645ux648ux644-ux62aux647ux6cc}
\begin{itemize}
\tightlist
\item
  \textbf{سازگاری در کران‌ها:} طبق قضیه ۱، \(\emptyset \subseteq A\). اگر
  از قانون عکس نقیض (بند د همین قضیه) استفاده کنیم، نتیجه می‌شود
  \(A' \subseteq \emptyset'\). طبق بند (ب)، \(\emptyset' = U\). پس
  \(A' \subseteq U\). این نتیجه با تعریف مجموعه مرجع (که شامل همه
  زیرمجموعه‌هاست) کاملاً سازگار است.
\end{itemize}

\clearpage
% ---------------------------------------------------------------------
% Copyright (c) 2026 Arsalan Dalvand & Reyhaneh Darvishi.
% Licensed under CC BY-NC-SA 4.0.
% See LICENSE file for details.
% ---------------------------------------------------------------------

\section{\texorpdfstring{قضیه ۶: قوانین دمورگان در نظریه مجموعه‌ها
\lr{(De Morgan’s Laws)}}{قضیه ۶: قوانین دمورگان در نظریه مجموعه‌ها }}\label{قضیه-۶---دمورگان-در-مجموعه-ها}
\begin{tldr}{خلاصه سریع}
این قضیه بیانگر اصل «دوگانگی» \lr{(Duality) }بین عملگرهای اجتماع و
اشتراک تحت تأثیر عملگر متمم است. طبق این قوانین، متممِ اجتماع برابر با
اشتراک متمم‌هاست و بالعکس. این روابط، ایزومورفیسم کامل بین جبر مجموعه‌ها و
منطق گزاره‌ها را نشان می‌دهند.
\end{tldr}
\subsection{۱. متن ریاضی
قضیه}\label{ux645ux62aux646-ux631ux6ccux627ux636ux6cc-ux642ux636ux6ccux647}
فرض کنید \(U\) مجموعه مرجع باشد و \(A, B \subseteq U\). هم‌ارزی‌های
مجموعه‌ای زیر برقرارند:
\begin{theorembox}{قضیه ۶}
\textbf{الف) متمم اجتماع:} \[(A \cup B)' = A' \cap B'\] \textbf{ب) متمم
اشتراک:} \[(A \cap B)' = A' \cup B'\]
\end{theorembox}
\subsection{\texorpdfstring{۲. اثبات صوری
\lr{(Formal Proof)}}{۲. اثبات صوری }}\label{ux627ux62bux628ux627ux62a-ux635ux648ux631ux6cc-formal-proof}
اثبات این قضیه مبتنی بر ترجمه گزاره‌های عضویت به منطق صوری و استفاده از
\textbf{قوانین دمورگان در منطق} است.
\subsubsection{اثبات قسمت
(الف)}\label{ux627ux62bux628ux627ux62a-ux642ux633ux645ux62a-ux627ux644ux641}
نشان می‌دهیم تابع گزاره‌نمای عضویت برای دو طرف تساوی هم‌ارز است:
\begin{info}{اثبات}
\[x \in (A \cup B)'\] ۱. طبق تعریف متمم: \[\iff x \notin (A \cup B)\]
\[\iff \sim (x \in A \cup B)\] ۲. طبق تعریف اجتماع:
\[\iff \sim (x \in A \lor x \in B)\] ۳. طبق
\textbf{\autoref{قضیه-۳---قوانین-دمورگان}} (دمورگان منطقی
\(\sim(p \lor q) \equiv \sim p \land \sim q\)):
\[\iff \sim(x \in A) \land \sim(x \in B)\] ۴. طبق تعریف متمم:
\[\iff (x \in A') \land (x \in B')\] ۵. طبق تعریف اشتراک:
\[\iff x \in (A' \cap B')\]
نتیجه: \((A \cup B)' = A' \cap B'\).
\end{info}
\subsubsection{اثبات قسمت
(ب)}\label{ux627ux62bux628ux627ux62a-ux642ux633ux645ux62a-ux628}
روند مشابهی طی می‌شود با این تفاوت که از قانون منطقی
\(\sim(p \land q) \equiv \sim p \lor \sim q\) استفاده می‌کنیم.
\begin{info}{اثبات}
\[x \in (A \cap B)' \iff \sim(x \in A \land x \in B)\]
\[\iff \sim(x \in A) \lor \sim(x \in B)\]
\[\iff x \in A' \lor x \in B'\] \[\iff x \in A' \cup B'\]
\end{info}
\subsection{\texorpdfstring{۳. شبکه ارتباطی با سایر قضایا
\lr{(Analytic Map)}}{۳. شبکه ارتباطی با سایر قضایا }}\label{ux634ux628ux6a9ux647-ux627ux631ux62aux628ux627ux637ux6cc-ux628ux627-ux633ux627ux6ccux631-ux642ux636ux627ux6ccux627-analytic-map}
این قضیه پل ارتباطی حیاتی برای انتقال ویژگی‌های منطق کلاسیک به ساختارهای
جبری و توپولوژیک است:
\subsubsection{\texorpdfstring{۱. ارتباط با
\autoref{قضیه-۳---قوانین-دمورگان} (ریشه
منطقی)}{۱. ارتباط با  (ریشه منطقی)}}\label{ux627ux631ux62aux628ux627ux637-ux628ux627-ux642ux636ux6ccux647-ux6f3-ux641ux635ux644-ux6f1-ux631ux6ccux634ux647-ux645ux646ux637ux642ux6cc}
\begin{itemize}
\tightlist
\item
  \textbf{ایزومورفیسم ساختاری:} قضیه ۶ تصویر دقیق قضیه ۳ در جبر
  مجموعه‌هاست. تناظر زیر به طور کامل برقرار است:
  \begin{itemize}
  \tightlist
  \item
    اجتماع (\(\cup\)) \(\leftrightarrow\) فصل (\(\lor\))
  \item
    اشتراک (\(\cap\)) \(\leftrightarrow\) عطف (\(\land\))
  \item
    متمم (\('\)) \(\leftrightarrow\) نقیض (\(\sim\))
  \end{itemize}
\end{itemize}
\subsubsection{\texorpdfstring{۲. ارتباط با
\autoref{قضیه-۵---متمم-و-زیرمجموعه}}{۲. ارتباط با }}\label{ux627ux631ux62aux628ux627ux637-ux628ux627-ux642ux636ux6ccux647-ux6f5---ux645ux62aux645ux645-ux648-ux632ux6ccux631ux645ux62cux645ux648ux639ux647}
\begin{itemize}
\tightlist
\item
  \textbf{پایه جبری:} اثبات قضیه ۶ به شدت به تعریف دقیق متمم و نقیض
  گزاره‌ها که در قضیه ۵ بررسی شد، وابسته است. همچنین، ترکیب قضیه ۶ با
  قضیه ۵ (قانون عکس نقیض \(A \subseteq B \iff B' \subseteq A'\))
  ابزارهای قدرتمندی برای ساده‌سازی عبارات پیچیده مجموعه‌ای فراهم می‌کند.
\end{itemize}
\subsubsection{\texorpdfstring{۳. ارتباط با
\autoref{قضیه-۸---تعمیم-دمورگان}
(تعمیم)}{۳. ارتباط با  (تعمیم)}}\label{ux627ux631ux62aux628ux627ux637-ux628ux627-ux642ux636ux6ccux647-ux6f8---ux62aux639ux645ux6ccux645-ux62fux645ux648ux631ux6afux627ux646-ux62aux639ux645ux6ccux645}
\begin{itemize}
\tightlist
\item
  \textbf{حالت خاص:} قضیه ۶ حالت محدود \lr{(Finite Case) }برای دو مجموعه
  است. \textbf{\autoref{قضیه-۸---تعمیم-دمورگان}} این مفهوم را برای
  خانواده‌های نامتناهی از مجموعه‌ها گسترش می‌دهد:
  \((\bigcup A_i)' = \bigcap A_i'\). در آنجا، نقش عملگرهای \(\lor\) و
  \(\land\) به سورهای \(\exists\) و \(\forall\) تبدیل می‌شود.
\end{itemize}

\clearpage
% ---------------------------------------------------------------------
% Copyright (c) 2026 Arsalan Dalvand & Reyhaneh Darvishi.
% Licensed under CC BY-NC-SA 4.0.
% See LICENSE file for details.
% ---------------------------------------------------------------------

\section{قضیه ۷: رفتار حدی اجتماع و اشتراک (خانواده
تهی)}\label{قضیه-۷.---تعمیم-اشتراک-و-اجتماع}
\begin{tldr}{خلاصه سریع}
این قضیه نتایج عملیات تعمیم‌یافته مجموعه‌ای را در «شرایط مرزی» بررسی
می‌کند. زمانی که مجموعه اندیس تهی باشد (\(\Gamma = \emptyset\))، اجتماع
برابر با عنصر خنثیِ جمع (تهی) و اشتراک برابر با عنصر خنثیِ ضرب (مجموعه
مرجع) می‌شود. این نتایج بر پایه مفهوم منطقی «صدق تُهی‌مایه»
\lr{(Vacuous Truth) }استوار هستند.
\end{tldr}
\subsection{۱. متن ریاضی
قضیه}\label{ux645ux62aux646-ux631ux6ccux627ux636ux6cc-ux642ux636ux6ccux647}
فرض کنید \(\{A_\gamma\}_{\gamma \in \Gamma}\) یک خانواده از زیرمجموعه‌های
مجموعه مرجع \(U\) باشد. اگر مجموعه اندیس تهی باشد
(\(\Gamma = \emptyset\))، آنگاه:
\begin{theorembox}{قضیه ۷}
\textbf{الف) اجتماع روی تهی:}
\[\bigcup_{\gamma \in \emptyset} A_\gamma = \emptyset\] \textbf{ب)
اشتراک روی تهی:} \[\bigcap_{\gamma \in \emptyset} A_\gamma = U\]
\end{theorembox}
\subsection{\texorpdfstring{۲. اثبات صوری
\lr{(Formal Proof)}}{۲. اثبات صوری }}\label{ux627ux62bux628ux627ux62a-ux635ux648ux631ux6cc-formal-proof}
اثبات این قضیه نیازمند تحلیل دقیق تعاریف سورهای وجودی و عمومی در دامنه
تهی است.
\subsubsection{اثبات قسمت
(الف)}\label{ux627ux62bux628ux627ux62a-ux642ux633ux645ux62a-ux627ux644ux641}
\begin{info}{اثبات}
طبق \textbf{\autoref{پیشنیاز---مفاهیم-بنیادین-مجموعه‌ها}}:
\[x \in \bigcup_{\gamma \in \emptyset} A_\gamma \iff \exists \gamma \in \emptyset, (x \in A_\gamma)\]
گزاره «\(\exists \gamma \in \emptyset\)» همواره \textbf{نادرست} (تناقض)
است، زیرا هیچ عضوی در \(\emptyset\) وجود ندارد که بخواهد در شرطی صدق
کند. بنابراین، هیچ \(x\) در جهان وجود ندارد که در این مجموعه باشد.
\[\forall x, x \notin \bigcup_{\gamma \in \emptyset} A_\gamma \implies \bigcup_{\gamma \in \emptyset} A_\gamma = \emptyset\]
\end{info}
\subsubsection{اثبات قسمت
(ب)}\label{ux627ux62bux628ux627ux62a-ux642ux633ux645ux62a-ux628}
\begin{info}{اثبات}
طبق \textbf{\autoref{پیشنیاز---مفاهیم-بنیادین-مجموعه‌ها}}:
\[x \in \bigcap_{\gamma \in \emptyset} A_\gamma \iff \forall \gamma (\gamma \in \emptyset \rightarrow x \in A_\gamma)\]
در گزاره شرطیِ داخل پرانتز، مقدم (\(\gamma \in \emptyset\)) همواره
\textbf{نادرست} است. طبق اصول منطق ریاضی (انتفای مقدم)، هر گزاره شرطی با
مقدم نادرست، دارای ارزش \textbf{راست \lr{(True)}} است (مستقل از ارزش
تالی). بنابراین شرط عضویت برای \textbf{تمام} \(x\)های موجود در مجموعه
مرجع \(U\) صادق است (صدق تهی‌مایه).
\[\forall x \in U, x \in \bigcap_{\gamma \in \emptyset} A_\gamma \implies \bigcap_{\gamma \in \emptyset} A_\gamma = U\]
\end{info}
\subsection{\texorpdfstring{۳. شبکه ارتباطی با سایر قضایا
\lr{(Analytic Map)}}{۳. شبکه ارتباطی با سایر قضایا }}\label{ux634ux628ux6a9ux647-ux627ux631ux62aux628ux627ux637ux6cc-ux628ux627-ux633ux627ux6ccux631-ux642ux636ux627ux6ccux627-analytic-map}
این قضیه یکی از زیباترین نمونه‌های تعامل منطق و نظریه مجموعه‌هاست:
\subsubsection{\texorpdfstring{۱. ارتباط با
\autoref{قضیه-۷---قوانین-تناقض} (منطق
گزاره‌ها)}{۱. ارتباط با  (منطق گزاره‌ها)}}\label{ux627ux631ux62aux628ux627ux637-ux628ux627-ux642ux636ux6ccux647-ux6f7---ux642ux648ux627ux646ux6ccux646-ux62aux646ux627ux642ux636-ux645ux646ux637ux642-ux6afux632ux627ux631ux647ux647ux627}
\begin{itemize}
\tightlist
\item
  \textbf{انتفای مقدم:} اثبات بخش (ب) تماماً متکی بر قانون منطقی
  \(c \rightarrow p \equiv t\) است که در فصل ۱ بررسی شد. در اینجا
  \(\gamma \in \emptyset\) نقش تناقض (\(c\)) را بازی می‌کند و باعث می‌شود
  کل گزاره راستگو (\(t\)) شود.
\end{itemize}
\subsubsection{\texorpdfstring{۲. ارتباط با
\autoref{پیشنیاز---مفاهیم-بنیادین-مجموعه‌ها}}{۲. ارتباط با }}\label{ux627ux631ux62aux628ux627ux637-ux628ux627-ux67eux6ccux634ux646ux6ccux627ux632---ux645ux641ux627ux647ux6ccux645-ux628ux646ux6ccux627ux62fux6ccux646-ux645ux62cux645ux648ux639ux647ux647ux627}
\begin{itemize}
\tightlist
\item
  \textbf{وابستگی تعریفی:} این قضیه بدون استفاده از تعاریف دقیق اجتماع و
  اشتراک مبتنی بر سورها (که در فایل پیشنیاز آمده است) قابل اثبات نیست.
  تعاریف شهودی در این نقطه کور \lr{(Empty Index) }کارایی ندارند.
\end{itemize}
\subsubsection{\texorpdfstring{۳. ارتباط با
\autoref{قضیه-۱---شمول-تهی}}{۳. ارتباط با }}\label{ux627ux631ux62aux628ux627ux637-ux628ux627-ux642ux636ux6ccux647-ux6f1---ux634ux645ux648ux644-ux62aux647ux6cc}
\begin{itemize}
\tightlist
\item
  \textbf{وحدت رویه:} همان منطقی که در قضیه ۱ باعث می‌شود
  \(\emptyset \subseteq A\) باشد (چون شرط عضویت در تهی دروغ است)، در
  اینجا باعث می‌شود اشتراک روی تهی برابر \(U\) شود. هر دو قضیه از
  ویژگی‌های دامنه تهی در گزاره‌های شرطی بهره می‌برند.
\end{itemize}
\subsubsection{\texorpdfstring{۴. ارتباط با
\autoref{قضیه-۵---متمم-و-زیرمجموعه}}{۴. ارتباط با }}\label{ux627ux631ux62aux628ux627ux637-ux628ux627-ux642ux636ux6ccux647-ux6f5---ux645ux62aux645ux645-ux648-ux632ux6ccux631ux645ux62cux645ux648ux639ux647}
\begin{itemize}
\tightlist
\item
  \textbf{تناظر جبری:} در قضیه ۵ دیدیم که \(U' = \emptyset\) و
  \(\emptyset' = U\). اگر قوانین دمورگان تعمیم‌یافته
  (\autoref{قضیه-۸---تعمیم-دمورگان}) را روی این قضیه اعمال کنیم، به
  نتیجه سازگاری می‌رسیم:
  \[(\bigcup_{\emptyset} A_\gamma)' = \bigcap_{\emptyset} A_\gamma' \implies (\emptyset)' = U\]
  که تاییدی بر صحت قضیه ۷ است.
\end{itemize}

\clearpage
% ---------------------------------------------------------------------
% Copyright (c) 2026 Arsalan Dalvand & Reyhaneh Darvishi.
% Licensed under CC BY-NC-SA 4.0.
% See LICENSE file for details.
% ---------------------------------------------------------------------

\section{\texorpdfstring{قضیه ۸: تعمیم قوانین دمورگان
\lr{(Generalized De Morgan’s Laws)}}{قضیه ۸: تعمیم قوانین دمورگان }}\label{قضیه-۸---تعمیم-دمورگان}
\begin{tldr}{خلاصه سریع}
این قضیه قوانین دمورگان را از حالت متناهی (دو مجموعه) به حالت نامتناهی
(خانواده‌های ایندکس‌دار) گسترش می‌دهد. این اصل بیانگر «دوگانگی»
\lr{(Duality) }میان سورهای وجودی و عمومی در ساختار مجموعه‌هاست: متممِ
اجتماع (وجود) به اشتراک (عمومیت) تبدیل می‌شود و بالعکس.
\end{tldr}
\subsection{۱. متن ریاضی
قضیه}\label{ux645ux62aux646-ux631ux6ccux627ux636ux6cc-ux642ux636ux6ccux647}
فرض کنید \(\{A_\gamma\}_{\gamma \in \Gamma}\) یک خانواده دلبخواه از
زیرمجموعه‌های مجموعه مرجع \(U\) باشد (که \(\Gamma\) مجموعه اندیس است). در
این صورت:
\begin{theorembox}{قضیه ۸}
\textbf{الف) متمم اجتماع تعمیم‌یافته:}
\[(\bigcup_{\gamma \in \Gamma} A_\gamma)' = \bigcap_{\gamma \in \Gamma} A_\gamma'\]
\textbf{ب) متمم اشتراک تعمیم‌یافته:}
\[(\bigcap_{\gamma \in \Gamma} A_\gamma)' = \bigcup_{\gamma \in \Gamma} A_\gamma'\]
\end{theorembox}
\subsection{\texorpdfstring{۲. اثبات صوری
\lr{(Formal Proof)}}{۲. اثبات صوری }}\label{ux627ux62bux628ux627ux62a-ux635ux648ux631ux6cc-formal-proof}
اثبات این قضیه مبتنی بر ترجمه تعاریف مجموعه‌ای به \textbf{منطق سورها} و
استفاده از قوانین نقیض سور \lr{(Quantifier Negation) }است.
\subsubsection{اثبات قسمت
(الف)}\label{ux627ux62bux628ux627ux62a-ux642ux633ux645ux62a-ux627ux644ux641}
نشان می‌دهیم که گزاره‌نمای عضویت برای دو طرف تساوی، منطقاً هم‌ارز است:
\begin{info}{اثبات}
\[x \in (\bigcup_{\gamma \in \Gamma} A_\gamma)'\] ۱. طبق تعریف متمم:
\[\iff \sim (x \in \bigcup_{\gamma \in \Gamma} A_\gamma)\] ۲. طبق
\textbf{\autoref{پیشنیاز---مفاهیم-بنیادین-مجموعه‌ها}} (معادل سور وجودی):
\[\iff \sim (\exists \gamma \in \Gamma, x \in A_\gamma)\] ۳. طبق
\textbf{\autoref{قواعد-تسویر}} در فصل ۱
(\(\sim \exists x P(x) \equiv \forall x \sim P(x)\)):
\[\iff \forall \gamma \in \Gamma, \sim(x \in A_\gamma)\] ۴. طبق تعریف
متمم (\(x \notin A \iff x \in A'\)):
\[\iff \forall \gamma \in \Gamma, (x \in A_\gamma')\] ۵. طبق
\textbf{\autoref{پیشنیاز---مفاهیم-بنیادین-مجموعه‌ها}} (معادل سور عمومی):
\[\iff x \in \bigcap_{\gamma \in \Gamma} A_\gamma'\]
نتیجه: دو مجموعه با هم برابرند.
\end{info}
\emph{(اثبات قسمت ب مشابه است، با این تفاوت که از نقیض سور عمومی استفاده
می‌شود: \(\sim \forall \equiv \exists \sim\))}.
\subsection{\texorpdfstring{۳. شبکه ارتباطی با سایر قضایا
\lr{(Analytic Map)}}{۳. شبکه ارتباطی با سایر قضایا }}\label{ux634ux628ux6a9ux647-ux627ux631ux62aux628ux627ux637ux6cc-ux628ux627-ux633ux627ux6ccux631-ux642ux636ux627ux6ccux627-analytic-map}
این قضیه نقطه اوج همگرایی بین منطق و نظریه مجموعه‌ها در این فصل است:
\subsubsection{\texorpdfstring{۱. ارتباط با
\autoref{قضیه-۶---دمورگان-در-مجموعه-ها} (حالت
خاص)}{۱. ارتباط با  (حالت خاص)}}\label{ux627ux631ux62aux628ux627ux637-ux628ux627-ux642ux636ux6ccux647-ux6f6---ux62fux645ux648ux631ux6afux627ux646-ux62fux631-ux645ux62cux645ux648ux639ux647-ux647ux627-ux62dux627ux644ux62a-ux62eux627ux635}
\begin{itemize}
\tightlist
\item
  \textbf{توسعه دامنه:} قضیه ۶ بیان می‌کرد \((A \cup B)' = A' \cap B'\).
  قضیه ۸ نشان می‌دهد که این قانون محدود به دو مجموعه نیست و برای هر تعداد
  مجموعه (حتی ناشمارا) برقرار است. در واقع قضیه ۶، حالتی است که مجموعه
  اندیس \(\Gamma = \{1, 2\}\) باشد.
\end{itemize}
\subsubsection{\texorpdfstring{۲. ارتباط با
\autoref{قواعد-تسویر}}{۲. ارتباط با }}\label{ux627ux631ux62aux628ux627ux637-ux628ux627-ux642ux648ux627ux639ux62f-ux646ux642ux6ccux636-ux633ux648ux631-ux641ux635ux644-ux6f1}
\begin{itemize}
\tightlist
\item
  \textbf{ایزومورفیسم بنیادی:} اثبات قضیه ۸ نشان می‌دهد که عملیات
  مجموعه‌ای زیر دقیقاً تصاویر عملیات منطقی هستند:
  \begin{itemize}
  \tightlist
  \item
    \(\bigcup\) (اجتماع) \(\longleftrightarrow\) \(\exists\) (سور وجودی)
  \item
    \(\bigcap\) (اشتراک) \(\longleftrightarrow\) \(\forall\) (سور عمومی)
  \item
    \$' \$ (متمم) \(\longleftrightarrow\) \(\sim\) (نقیض) بنابراین قانون
    \((\cup A)' = \cap A'\) دقیقاً ترجمه مجموعه ایِ قانون منطقی
    \(\sim (\exists x) \equiv (\forall x) \sim\) است.
  \end{itemize}
\end{itemize}
\subsubsection{\texorpdfstring{۳. ارتباط با
\autoref{قضیه-۵---متمم-و-زیرمجموعه}}{۳. ارتباط با }}\label{ux627ux631ux62aux628ux627ux637-ux628ux627-ux642ux636ux6ccux647-ux6f5---ux645ux62aux645ux645-ux648-ux632ux6ccux631ux645ux62cux645ux648ux639ux647}
\begin{itemize}
\tightlist
\item
  \textbf{ابزار اثبات:} در گام چهارم اثبات، از هم‌ارزی
  \(x \notin A_\gamma \iff x \in A_\gamma'\) استفاده کردیم که نتیجه
  مستقیم تعریف متمم در قضیه ۵ است.
\end{itemize}
\subsubsection{\texorpdfstring{۴. ارتباط با
\autoref{قضیه-۷---قوانین-تناقض}}{۴. ارتباط با }}\label{ux627ux631ux62aux628ux627ux637-ux628ux627-ux642ux636ux6ccux647-ux6f7---ux642ux648ux627ux646ux6ccux646-ux62aux646ux627ux642ux636}
\begin{itemize}
\tightlist
\item
  \textbf{سازگاری در مرزها:} اگر \(\Gamma = \emptyset\) باشد:
  \begin{itemize}
  \tightlist
  \item
    سمت چپ تساوی (الف): \((\bigcup_\emptyset)' = (\emptyset)' = U\) (طبق
    قضیه ۵-ب).
  \item
    سمت راست تساوی (الف): \(\bigcap_\emptyset (A') = U\) (طبق قضیه ۷-ب).
  \item
    این نشان می‌دهد که قضیه ۸ حتی برای خانواده‌های تهی نیز سازگار و معتبر
    است.
  \end{itemize}
\end{itemize}

\clearpage
% ---------------------------------------------------------------------
% Copyright (c) 2026 Arsalan Dalvand & Reyhaneh Darvishi.
% Licensed under CC BY-NC-SA 4.0.
% See LICENSE file for details.
% ---------------------------------------------------------------------

\section{\texorpdfstring{قضیه ۹: تعمیم قوانین پخش‌پذیری
\lr{(Generalized Distributive Laws)}}{قضیه ۹: تعمیم قوانین پخش‌پذیری }}\label{قضیه-۹---تعمیم-پخش-پذیری}
\begin{tldr}{خلاصه سریع}
این قضیه نشان می‌دهد که قوانین پخش‌پذیری (توزیع‌پذیری) محدود به تعداد
متناهی از مجموعه‌ها نیستند. عملگر اشتراک روی «اجتماعِ یک خانواده نامتناهی»
پخش می‌شود و عملگر اجتماع روی «اشتراکِ یک خانواده نامتناهی» توزیع می‌گردد.
این ویژگی اساس تعریف «جبر سیگما» در نظریه اندازه است.
\end{tldr}
\subsection{۱. متن ریاضی
قضیه}\label{ux645ux62aux646-ux631ux6ccux627ux636ux6cc-ux642ux636ux6ccux647}
فرض کنید \(A\) یک مجموعه و \(\{B_\gamma\}_{\gamma \in \Gamma}\) یک
خانواده دلبخواه از مجموعه‌ها باشد. در این صورت:
\begin{theorembox}{قضیه ۹}
\textbf{الف) پخش اشتراک بر اجتماع:}
\[A \cap (\bigcup_{\gamma \in \Gamma} B_\gamma) = \bigcup_{\gamma \in \Gamma} (A \cap B_\gamma)\]
\textbf{ب) پخش اجتماع بر اشتراک:}
\[A \cup (\bigcap_{\gamma \in \Gamma} B_\gamma) = \bigcap_{\gamma \in \Gamma} (A \cup B_\gamma)\]
\end{theorembox}
\subsection{\texorpdfstring{۲. اثبات صوری
\lr{(Formal Proof)}}{۲. اثبات صوری }}\label{ux627ux62bux628ux627ux62a-ux635ux648ux631ux6cc-formal-proof}
اثبات این قضیه بر پایه «قوانین پخش‌پذیری منطق گزاره‌ها» و تعامل آن با
سورها بنا شده است. ما اثبات قسمت (الف) را بررسی می‌کنیم.
\begin{info}{اثبات}
باید نشان دهیم گزاره‌نمای عضویت (\(x \in S\)) برای طرفین تساوی هم‌ارز است:
\[x \in A \cap (\bigcup_{\gamma \in \Gamma} B_\gamma)\] ۱. طبق تعریف
اشتراک:
\[\iff (x \in A) \wedge (x \in \bigcup_{\gamma \in \Gamma} B_\gamma)\]
۲. طبق \textbf{\autoref{پیشنیاز---مفاهیم-بنیادین-مجموعه‌ها}} (سور وجودی):
\[\iff (x \in A) \wedge (\exists \gamma \in \Gamma, x \in B_\gamma)\] ۳.
نکته منطقی کلیدی: از آنجا که گزاره \((x \in A)\) مستقل از اندیس
\(\gamma\) است، می‌توانیم آن را به داخل سور وجودی ببریم (قانون پخش‌پذیری
منطق مسور):
\[p \wedge (\exists \gamma, q_\gamma) \iff \exists \gamma, (p \wedge q_\gamma)\]
\[\iff \exists \gamma \in \Gamma, (x \in A \wedge x \in B_\gamma)\] ۴.
طبق تعریف اشتراک:
\[\iff \exists \gamma \in \Gamma, (x \in A \cap B_\gamma)\] ۵. طبق تعریف
اجتماع تعمیم‌یافته:
\[\iff x \in \bigcup_{\gamma \in \Gamma} (A \cap B_\gamma)\]
نتیجه: دو مجموعه برابرند. (اثبات قسمت ب مشابه است و از پخش‌پذیری \(\vee\)
روی \(\forall\) استفاده می‌کند).
\end{info}
\subsection{\texorpdfstring{۳. شبکه ارتباطی با سایر قضایا
\lr{(Analytic Map)}}{۳. شبکه ارتباطی با سایر قضایا }}\label{ux634ux628ux6a9ux647-ux627ux631ux62aux628ux627ux637ux6cc-ux628ux627-ux633ux627ux6ccux631-ux642ux636ux627ux6ccux627-analytic-map}
این قضیه توسعه‌یافته‌ی مفاهیم جبری فصل ۱ و ۲ در ابعاد نامتناهی است:
\subsubsection{\texorpdfstring{۱. ارتباط با
\autoref{قضیه-۴---جبر-مجموعه-ها} (حالت
متناهی)}{۱. ارتباط با  (حالت متناهی)}}\label{ux627ux631ux62aux628ux627ux637-ux628ux627-ux642ux636ux6ccux647-ux6f4---ux62cux628ux631-ux645ux62cux645ux648ux639ux647-ux647ux627-ux62dux627ux644ux62a-ux645ux62aux646ux627ux647ux6cc}
\begin{itemize}
\tightlist
\item
  \textbf{تعمیم:} قضیه ۴ (بخش ه) بیان می‌کرد که
  \(A \cap (B \cup C) = (A \cap B) \cup (A \cap C)\). قضیه ۹ دقیقا همان
  قانون است با این تفاوت که تعداد مجموعه‌های داخل پرانتز از ۲ تا به
  بی‌نهایت (به تعداد اعضای \(\Gamma\)) افزایش یافته است.
\end{itemize}
\subsubsection{\texorpdfstring{۲. ارتباط با
\autoref{قضیه-۴---قوانین-شرکت-پذیری-و-پخش-پذیری} (ریشه
منطقی)}{۲. ارتباط با  (ریشه منطقی)}}\label{ux627ux631ux62aux628ux627ux637-ux628ux627-ux642ux636ux6ccux647-ux6f4-ux641ux635ux644-ux6f1-ux631ux6ccux634ux647-ux645ux646ux637ux642ux6cc}
\begin{itemize}
\tightlist
\item
  \textbf{پایه استدلال:} اثبات قضیه ۹ تماماً متکی بر ساختار منطقی
  \(p \wedge (q \vee r) \equiv (p \wedge q) \vee (p \wedge r)\) است. در
  نظریه مجموعه‌ها، سور وجودی (\(\exists\)) نقش تعمیم‌یافته‌ی «یا»
  (\(\vee\)) را بازی می‌کند؛ بنابراین پخش شدن \(\wedge\) روی \(\exists\)
  در اثبات بالا، بازتاب مستقیم پخش شدن \(\wedge\) روی \(\vee\) در منطق
  است.
\end{itemize}
\subsubsection{\texorpdfstring{۳. ارتباط با
\autoref{قضیه-۸---تعمیم-دمورگان}}{۳. ارتباط با }}\label{ux627ux631ux62aux628ux627ux637-ux628ux627-ux642ux636ux6ccux647-ux6f8---ux62aux639ux645ux6ccux645-ux62fux645ux648ux631ux6afux627ux646}
\begin{itemize}
\tightlist
\item
  \textbf{مکمل‌سازی:} اگر از طرفین تساوی‌های قضیه ۹ متمم بگیریم و از
  \textbf{\autoref{قضیه-۸---تعمیم-دمورگان}} استفاده کنیم، به دوگانِ
  \lr{(Dual) }یکدیگر تبدیل می‌شوند. یعنی متمم‌گیری از فرمول (الف) و
  استفاده از دمورگان، ما را به فرمول (ب) می‌رساند (با جایگزینی \(A\) با
  \(A'\) و \(B_\gamma\) با \(B_\gamma'\)).
\end{itemize}
\subsubsection{۴. کاربرد در توپولوژی (فصول
پیشرفته)}\label{ux6a9ux627ux631ux628ux631ux62f-ux62fux631-ux62aux648ux67eux648ux644ux648ux698ux6cc-ux641ux635ux648ux644-ux67eux6ccux634ux631ux641ux62aux647}
\begin{itemize}
\tightlist
\item
  \textbf{پیوستگی:} در توپولوژی، این قضیه نقش حیاتی دارد. مثلاً تعریف
  تابع پیوسته (\(f^{-1}(\cup U_\alpha) = \cup f^{-1}(U_\alpha)\)) و خواص
  تصویر معکوس توابع، دقیقاً از همین ساختار پخش‌پذیری پیروی می‌کنند.
\end{itemize}

\clearpage
% ---------------------------------------------------------------------
% Copyright (c) 2026 Arsalan Dalvand & Reyhaneh Darvishi.
% Licensed under CC BY-NC-SA 4.0.
% See LICENSE file for details.
% ---------------------------------------------------------------------

\section{قضیه ۱۰: عدم وجود مجموعه جهانی
مطلق}\label{قضیه-۱۰---عدم-وجود-مجموعه-جهانی}
\begin{tldr}{خلاصه سریع}
این قضیه بیان می‌کند که چیزی به نام «مجموعه همه مجموعه‌ها» وجود ندارد. اگر
چنین مجموعه‌ای وجود داشته باشد، ریاضیات دچار فروپاشی منطقی می‌شود. این
حکم، راه حلی برای گریز از پارادوکس راسل است.
\end{tldr}
\subsection{۱. متن ریاضی
قضیه}\label{ux645ux62aux646-ux631ux6ccux627ux636ux6cc-ux642ux636ux6ccux647}
هیچ مجموعه‌ای به نام \(\mathcal{U}\) (مجموعه جهانی) وجود ندارد که شامل
تمام مجموعه‌ها باشد.
\begin{theorembox}{قضیه ۱۰}
\[\nexists \mathcal{U} : \forall S, (S \in \mathcal{U})\]
\end{theorembox}
\subsection{۲. اثبات صوری (برهان
خلف)}\label{ux627ux62bux628ux627ux62a-ux635ux648ux631ux6cc-ux628ux631ux647ux627ux646-ux62eux644ux641}
این اثبات از \textbf{\autoref{مفهوم-پارادوکس-راسل}} به عنوان موتور محرک
استفاده می‌کند.
\begin{info}{مراحل اثبات}
۱. \textbf{فرض خلف:} فرض کنید مجموعه جهانی \(\mathcal{U}\) وجود دارد که
شامل تمام مجموعه‌هاست. ۲. طبق اصل تصریح
\lr{(Axiom of Separation)، }می‌توانیم زیرمجموعه‌ای از \(\mathcal{U}\) را
با یک ویژگی خاص جدا کنیم. زیرمجموعه \(R\) را چنین تعریف می‌کنیم:
\[R = \{ S \in \mathcal{U} \mid S \notin S \}\] ۳. چون \(R\) یک
زیرمجموعه خوش‌تعریف از \(\mathcal{U}\) است و \(\mathcal{U}\) شامل همه
مجموعه‌هاست، پس خود \(R\) نیز باید عضوی از \(\mathcal{U}\) باشد
(\(R \in \mathcal{U}\)). ۴. اکنون وضعیت عضویت \(R\) در خودش را بررسی
می‌کنیم:
\begin{itemize}
\tightlist
\item
  اگر \(R \in R \implies R \notin R\) (طبق تعریف \(R\)).
\item
  اگر \(R \notin R \implies R \in R\) (طبق تعریف \(R\)). ۵. به تناقض
  \((R \in R \iff R \notin R)\) می‌رسیم. ۶. \textbf{نتیجه:} فرض اولیه
  (وجود \(\mathcal{U}\)) باطل است.
\end{itemize}
\end{info}
\subsection{\texorpdfstring{۳. شبکه ارتباطی با سایر قضایا
\lr{(Analytic Map)}}{۳. شبکه ارتباطی با سایر قضایا }}\label{ux634ux628ux6a9ux647-ux627ux631ux62aux628ux627ux637ux6cc-ux628ux627-ux633ux627ux6ccux631-ux642ux636ux627ux6ccux627-analytic-map}
این قضیه مرزهای نظریه مجموعه‌ها را مشخص می‌کند:
\subsubsection{\texorpdfstring{۱. ارتباط با
\autoref{مفهوم-پارادوکس-راسل}}{۱. ارتباط با }}\label{ux627ux631ux62aux628ux627ux637-ux628ux627-ux645ux641ux647ux648ux645-ux67eux627ux631ux627ux62fux648ux6a9ux633-ux631ux627ux633ux644}
\begin{itemize}
\tightlist
\item
  \textbf{رابطه علت و معلولی:} \autoref{مفهوم-پارادوکس-راسل} «مشکل» را
  نشان می‌دهد و قضیه ۱۰ «راه حل» (حذف مجموعه جهانی) را ارائه می‌دهد. بدون
  تعریف \(R\) در پارادوکس راسل، اثبات این قضیه ممکن نیست.
\end{itemize}
\subsubsection{\texorpdfstring{۲. ارتباط با
\autoref{پیشنیاز---تعریف-مجموعه-توانی}}{۲. ارتباط با }}\label{ux627ux631ux62aux628ux627ux637-ux628ux627-ux67eux6ccux634ux646ux6ccux627ux632---ux62aux639ux631ux6ccux641-ux645ux62cux645ux648ux639ux647-ux62aux648ux627ux646ux6cc}
\begin{itemize}
\tightlist
\item
  \textbf{قضیه کانتور:} قضیه ۱۰ با قضیه کانتور (که می‌گوید
  \(|A| < |\mathcal{P}(A)|\)) هم‌خوانی دارد. اگر \(\mathcal{U}\) وجود
  داشت، باید \(\mathcal{P}(\mathcal{U})\) زیرمجموعه‌ای از \(\mathcal{U}\)
  می‌شد (چون \(\mathcal{U}\) شامل همه چیز است). این یعنی اندازه کل کوچکتر
  از جزء می‌شود که محال است.
\end{itemize}
\subsubsection{\texorpdfstring{۳. ارتباط با
\autoref{قضیه-۱---شمول-تهی}}{۳. ارتباط با }}\label{ux627ux631ux62aux628ux627ux637-ux628ux627-ux642ux636ux6ccux647-ux6f1---ux634ux645ux648ux644-ux62aux647ux6cc}
\begin{itemize}
\tightlist
\item
  \textbf{تضاد در کران‌ها:} در \autoref{قضیه-۱---شمول-تهی} ثابت کردیم
  «کوچکترین» مجموعه (\(\emptyset\)) وجود دارد و یکتاست. قضیه ۱۰ ثابت
  می‌کند که «بزرگترین» مجموعه (\(\mathcal{U}\)) وجود ندارد. ساختار
  مجموعه‌ها از پایین بسته و از بالا باز است.
\end{itemize}

\clearpage
% ---------------------------------------------------------------------
% Copyright (c) 2026 Arsalan Dalvand & Reyhaneh Darvishi.
% Licensed under CC BY-NC-SA 4.0.
% See LICENSE file for details.
% ---------------------------------------------------------------------

\section{\texorpdfstring{مفهوم پارادوکس راسل
\lr{(Russell’s Paradox)}}{مفهوم پارادوکس راسل }}\label{مفهوم-پارادوکس-راسل}
\begin{tldr}{خلاصه سریع}
پارادوکس راسل نشان می‌دهد که «اصل تصریح نامحدود»
\lr{(Unrestricted Comprehension) }در نظریه مجموعه‌های کانتور منجر به
تناقض می‌شود. اگر اجازه دهیم هر ویژگی دلخواهی یک مجموعه بسازد، می‌توانیم
مجموعه‌ای مثل \(R\) بسازیم که همزمان باید «عضو خودش باشد» و «عضو خودش
نباشد».
\end{tldr}
\subsection{۱. ساختار صوری
پارادوکس}\label{ux633ux627ux62eux62aux627ux631-ux635ux648ux631ux6cc-ux67eux627ux631ux627ux62fux648ux6a9ux633}
در نظریه مجموعه‌های کلاسیک \lr{(Naive Set Theory)، }فرض بر این بود که
برای هر ویژگی \(P(x)\)، مجموعه‌ای وجود دارد شامل تمام عناصری که در \(P\)
صدق می‌کنند: \(\{x \mid P(x)\}\). برتراند راسل در سال ۱۹۰۲ با تعریف ویژگی
\(P(x): x \notin x\) این فرض را به چالش کشید.
مجموعه \(R\) را به صورت زیر تعریف می‌کنیم:
\[R = \{ x \mid x \notin x \}\] \emph{(مجموعه تمام مجموعه‌هایی که عضو
خودشان نیستند)}.
\subsection{۲. تحلیل منطقی (تضاد
درونی)}\label{ux62aux62dux644ux6ccux644-ux645ux646ux637ux642ux6cc-ux62aux636ux627ux62f-ux62fux631ux648ux646ux6cc}
سوال اساسی این است: \textbf{آیا \(R \in R\)؟} برای پاسخ، دو حالت ممکن در
منطق دوارزشی را بررسی می‌کنیم:
\subsubsection{\texorpdfstring{حالت اول: فرض کنیم
\(R \in R\)}{حالت اول: فرض کنیم R \textbackslash in R}}\label{ux62dux627ux644ux62a-ux627ux648ux644-ux641ux631ux636-ux6a9ux646ux6ccux645-r-in-r}
\begin{enumerate}
\def\labelenumi{\arabic{enumi}.}
\tightlist
\item
  اگر \(R \in R\) باشد، طبق تعریف مجموعه \(R\)، باید ویژگی شرطی مجموعه
  (\(x \notin x\)) را داشته باشد.
\item
  بنابراین \(R \notin R\).
\item
  نتیجه: \((R \in R) \implies (R \notin R)\). (تناقض)
\end{enumerate}
\subsubsection{\texorpdfstring{حالت دوم: فرض کنیم
\(R \notin R\)}{حالت دوم: فرض کنیم R \textbackslash notin R}}\label{ux62dux627ux644ux62a-ux62fux648ux645-ux641ux631ux636-ux6a9ux646ux6ccux645-r-notin-r}
\begin{enumerate}
\def\labelenumi{\arabic{enumi}.}
\tightlist
\item
  اگر \(R \notin R\) باشد، پس \(R\) ویژگی لازم برای عضویت در مجموعه
  \(R\) (که همان عضو خود نبودن است) را دارد.
\item
  بنابراین \(R \in R\).
\item
  نتیجه: \((R \notin R) \implies (R \in R)\). (تناقض)
\end{enumerate}
\subsubsection{نتیجه
نهایی}\label{ux646ux62aux6ccux62cux647-ux646ux647ux627ux6ccux6cc}
گزاره \((R \in R) \iff (R \notin R)\) یک تناقض منطقی
\lr{(Contradiction) }است که نشان می‌دهد سیستم اصل موضوعی ما ایراد دارد.
\subsection{\texorpdfstring{۳. شبکه ارتباطی با سایر قضایا
\lr{(Analytic Map)}}{۳. شبکه ارتباطی با سایر قضایا }}\label{ux634ux628ux6a9ux647-ux627ux631ux62aux628ux627ux637ux6cc-ux628ux627-ux633ux627ux6ccux631-ux642ux636ux627ux6ccux627-analytic-map}
این پارادوکس نقطه عطفی است که نیاز به بازبینی در تعاریف پایه (فصل ۱ و ۲)
را ایجاد می‌کند:
\subsubsection{\texorpdfstring{۱. ارتباط با
\autoref{قضیه-۱۰---عدم-وجود-مجموعه-جهانی}}{۱. ارتباط با }}\label{ux627ux631ux62aux628ux627ux637-ux628ux627-ux642ux636ux6ccux647-ux6f1ux6f0---ux639ux62fux645-ux648ux62cux648ux62f-ux645ux62cux645ux648ux639ux647-ux62cux647ux627ux646ux6cc}
\begin{itemize}
\tightlist
\item
  \textbf{نتیجه مستقیم:} پارادوکس راسل ابزار اصلی اثبات
  \autoref{قضیه-۱۰---عدم-وجود-مجموعه-جهانی} است. آن قضیه نشان می‌دهد که
  برای رفع این پارادوکس، باید فرض «وجود مجموعه تمام مجموعه‌ها» را کنار
  بگذاریم.
\end{itemize}
\subsubsection{\texorpdfstring{۲. ارتباط با
\autoref{قضیه-۵---متمم-و-زیرمجموعه} (قانون عکس
نقیض)}{۲. ارتباط با  (قانون عکس نقیض)}}\label{ux627ux631ux62aux628ux627ux637-ux628ux627-ux642ux636ux6ccux647-ux6f5---ux645ux62aux645ux645-ux648-ux632ux6ccux631ux645ux62cux645ux648ux639ux647-ux642ux627ux646ux648ux646-ux639ux6a9ux633-ux646ux642ux6ccux636}
\begin{itemize}
\tightlist
\item
  \textbf{تحلیل ساختاری:} استدلال راسل شبیه به استفاده از قطری‌سازی
  کانتور و قانون عکس نقیض است. اگر نگاشتی وجود داشته باشد که ساختار را
  حفظ کند، با منفی کردن آن (متمم) به تناقض می‌رسیم.
\end{itemize}
\subsubsection{\texorpdfstring{۳. ارتباط با
\autoref{قضیه-۶---قواعد-استنتاج} (برهان
خلف)}{۳. ارتباط با  (برهان خلف)}}\label{ux627ux631ux62aux628ux627ux637-ux628ux627-ux642ux636ux6ccux647-ux6f6---ux642ux648ux627ux639ux62f-ux627ux633ux62aux646ux62aux627ux62c-ux628ux631ux647ux627ux646-ux62eux644ux641}
\begin{itemize}
\tightlist
\item
  \textbf{متدولوژی:} کل این پارادوکس یک مثال کلاسیک از رسیدن به
  \(p \land \sim p\) است که در منطق کلاسیک باطل است. این امر ریاضیدانان
  را مجبور کرد اصول موضوعی جدیدی (مانند \lr{ZFC) }تدوین کنند که در آن
  تشکیل مجموعه \(R\) غیرقانونی است.
\end{itemize}

\clearpage
% ---------------------------------------------------------------------
% Copyright (c) 2026 Arsalan Dalvand & Reyhaneh Darvishi.
% Licensed under CC BY-NC-SA 4.0.
% See LICENSE file for details.
% ---------------------------------------------------------------------

\section{تمرین ۱۹: رفتار مجموعه توانی با اشتراک و
اجتماع}\label{تمرین-۱۹---مجموعه-توانی-و-اعمال-روی-مجموعه‌ها}
\begin{tldr}{خلاصه سریع}
مجموعه توانی (\(\mathcal{P}\)) با \textbf{اشتراک} دوست است (تساوی دارد)،
اما با \textbf{اجتماع} سر ناسازگاری دارد (تساوی ندارد).
\begin{itemize}
\tightlist
\item
  الف) \(\mathcal{P}(A) \cap \mathcal{P}(B) = \mathcal{P}(A \cap B)\) ✅
\item
  ب) \(\mathcal{P}(A) \cup \mathcal{P}(B) \neq \mathcal{P}(A \cup B)\)
  ❌
\end{itemize}
\end{tldr}
\subsection{۱. حل قسمت (الف) -
اشتراک}\label{ux62dux644-ux642ux633ux645ux62a-ux627ux644ux641---ux627ux634ux62aux631ux627ux6a9}
\begin{info}{اثبات تساوی}
باید نشان دهیم \(X\) عضو سمت چپ است اگر و تنها اگر عضو سمت راست باشد.
\[X \in \mathcal{P}(A) \cap \mathcal{P}(B)\]
\(\equiv X \in \mathcal{P}(A) \land X \in \mathcal{P}(B)\) (تعریف
اشتراک) \(\equiv X \subseteq A \land X \subseteq B\) (تعریف مجموعه
توانی) \(\equiv X \subseteq (A \cap B)\) (اگر مجموعه‌ای زیرمجموعه دو
مجموعه باشد، زیرمجموعه اشتراک آن‌هاست)
\(\equiv X \in \mathcal{P}(A \cap B)\) (تعریف مجموعه توانی)
پس حکم \textbf{درست} است.
\end{info}
\subsection{۲. حل قسمت (ب) -
اجتماع}\label{ux62dux644-ux642ux633ux645ux62a-ux628---ux627ux62cux62aux645ux627ux639}
\begin{note}{مثال نقض (اثبات نادرستی)}
فرض کنید:
\begin{itemize}
\tightlist
\item
  \(A = \{1\}\)
\item
  \(B = \{2\}\)
\end{itemize}
\textbf{سمت چپ:}
\begin{itemize}
\tightlist
\item
  \(\mathcal{P}(A) = \{\emptyset, \{1\}\}\)
\item
  \(\mathcal{P}(B) = \{\emptyset, \{2\}\}\)
\item
  اجتماع آن‌ها: \(\{\emptyset, \{1\}, \{2\}\}\)
\end{itemize}
\textbf{سمت راست:}
\begin{itemize}
\tightlist
\item
  \(A \cup B = \{1, 2\}\)
\item
  \(\mathcal{P}(A \cup B) = \{\emptyset, \{1\}, \{2\}, \{1, 2\}\}\)
\end{itemize}
\textbf{نتیجه:} مجموعه \(\{1, 2\}\) در سمت راست هست اما در سمت چپ نیست.
پس حکم \textbf{نادرست} است.
\[\mathcal{P}(A) \cup \mathcal{P}(B) \subseteq \mathcal{P}(A \cup B)\]
(رابطه فقط به صورت زیرمجموعه برقرار است، نه تساوی).
\end{note}

\clearpage
% ---------------------------------------------------------------------
% Copyright (c) 2026 Arsalan Dalvand & Reyhaneh Darvishi.
% Licensed under CC BY-NC-SA 4.0.
% See LICENSE file for details.
% ---------------------------------------------------------------------

\section{تمرین ۲۰: بازنویسی مجموعه بر اساس
تفاضل}\label{تمرین-۲۰---اثبات-رابطه-متمم-و-تفاضل}
\begin{tldr}{خلاصه سریع}
اگر مجموعه‌ی بزرگ \(C\) به دو تکه جداگانه \(A\) و \(B\) تقسیم شده باشد
(افراز شده باشد)، آنگاه \(A\) دقیقاً همان «کل منهای \lr{B» }است. شرط‌ها:
\(A \cup B = C\) و \(A \cap B = \emptyset\).
\end{tldr}
\subsection{۱. صورت
مسأله}\label{ux635ux648ux631ux62a-ux645ux633ux623ux644ux647}
\begin{info}{سوال}
ثابت کنید اگر \(A \subseteq C\) و \(B \subseteq C\) و داشته باشیم
\(A \cup B = C\) و \(A \cap B = \emptyset\)، آنگاه: \[A = C - B\]
\end{info}
\subsection{۲. اثبات دو
طرفه}\label{ux627ux62bux628ux627ux62a-ux62fux648-ux637ux631ux641ux647}
\begin{info}{اثبات}
\textbf{مسیر رفت (\(A \subseteq C - B\)):}
\begin{enumerate}
\def\labelenumi{\arabic{enumi}.}
\tightlist
\item
  فرض کنید \(x \in A\).
\item
  چون \(A \subseteq C\)، پس \(x \in C\).
\item
  چون \(A \cap B = \emptyset\) (اشتراک ندارند)، و \(x\) در \(A\) است، پس
  قطعاً \(x \notin B\).
\item
  از (۲) و (۳) داریم: \(x \in C\) و \(x \notin B\).
\item
  طبق تعریف تفاضل: \(x \in C - B\).
\end{enumerate}
\textbf{مسیر برگشت (\(C - B \subseteq A\)):} 6. فرض کنید
\(x \in C - B\). 7. یعنی \(x \in C\) و \(x \notin B\). 8. طبق فرض مسأله
\(C = A \cup B\). پس چون \(x \in C\)، باید یا در \(A\) باشد یا در \(B\).
9. اما می‌دانیم \(x \notin B\) (از مرحله ۲). 10. پس تنها گزینه باقی‌مانده
این است که \(x \in A\).
\textbf{نتیجه:} چون هر دو طرف زیرمجموعه هم شدند، پس \(A = C - B\).
\(\blacksquare\)
\end{info}

\clearpage
\chapter{رابطه و تابع}
\clearpage
% ---------------------------------------------------------------------
% Copyright (c) 2026 Arsalan Dalvand & Reyhaneh Darvishi.
% Licensed under CC BY-NC-SA 4.0.
% See LICENSE file for details.
% ---------------------------------------------------------------------

\section{مفاهیم پایه: حاصلضرب دکارتی، رابطه و
تابع}\label{پیشنیاز---مفاهیم-پایه-رابطه-و-تابع}
\begin{tldr}{خلاصه سریع}
این یادداشت پیش‌نیازهای ضروری برای ورود به قضایای فصل ۳ (به‌ویژه قضایای ۱،
۱۵ و ۱۶) را پوشش می‌دهد. مفاهیم کلیدی شامل ساختار زوج مرتب، حاصلضرب
دکارتی، تعریف صوری تابع، ترکیب توابع و انواع خاص توابع (یک‌به‌یک و پوشا)
هستند.
\end{tldr}
\subsection{\texorpdfstring{۱. حاصلضرب دکارتی
\lr{(Cartesian Product)}}{۱. حاصلضرب دکارتی }}\label{ux62dux627ux635ux644ux636ux631ux628-ux62fux6a9ux627ux631ux62aux6cc-cartesian-product}
سنگ‌بنای تعریف رابطه و تابع، مفهوم زوج مرتب و حاصلضرب دکارتی است.
\subsubsection{الف) زوج
مرتب}\label{ux627ux644ux641-ux632ux648ux62c-ux645ux631ux62aux628}
برای هر دو شیء \(a\) و \(b\)، زوج مرتب \((a,b)\) شیئی است که در آن ترتیب
مؤلفه‌ها اهمیت دارد.
\begin{itemize}
\tightlist
\item
  \textbf{ویژگی اصلی:} \((a,b) = (c,d) \iff a=c \wedge b=d\).
\end{itemize}
\subsubsection{ب) تعریف حاصلضرب
دکارتی}\label{ux628-ux62aux639ux631ux6ccux641-ux62dux627ux635ux644ux636ux631ux628-ux62fux6a9ux627ux631ux62aux6cc}
برای دو مجموعه \(A\) و \(B\)، حاصلضرب دکارتی \(A \times B\) مجموعه‌ی تمام
زوج‌های مرتبی است که مؤلفه اول از \(A\) و مؤلفه دوم از \(B\) باشد.
\begin{tldr}{تعریف}
\[A \times B = \{ (a,b) \mid a \in A \wedge b \in B \}\]
\end{tldr}
\subsection{\texorpdfstring{۲. رابطه و تابع
\lr{(Relation and Function)}}{۲. رابطه و تابع }}\label{ux631ux627ux628ux637ux647-ux648-ux62aux627ux628ux639-relation-and-function}
\subsubsection{الف)
رابطه}\label{ux627ux644ux641-ux631ux627ux628ux637ux647}
هر زیرمجموعه از حاصلضرب دکارتی \(A \times B\) یک «رابطه» از \(A\) به
\(B\) نامیده می‌شود. اگر \(R \subseteq A \times B\)، گزاره
\((a,b) \in R\) را اغلب به صورت \(aRb\) می‌نویسیم.
\subsubsection{ب) تابع (به عنوان نوع خاصی از
رابطه)}\label{ux628-ux62aux627ux628ux639-ux628ux647-ux639ux646ux648ux627ux646-ux646ux648ux639-ux62eux627ux635ux6cc-ux627ux632-ux631ux627ux628ux637ux647}
تابع \(f: X \to Y\) رابطه‌ای است که در آن هر عضو دامنه (\(X\)) دقیقاً با
یک عضو از هم‌دامنه (\(Y\)) در ارتباط است.
\begin{tldr}{تعریف صوری}
\[f \subseteq X \times Y \text{ is a function } \iff \forall x \in X, \exists! y \in Y, (x,y) \in f\]
\end{tldr}
\subsection{\texorpdfstring{۳. ترکیب توابع
\lr{(Function Composition)}}{۳. ترکیب توابع }}\label{ux62aux631ux6a9ux6ccux628-ux62aux648ux627ux628ux639-function-composition}
عملیات ترکیب، قلب تپنده جبر توابع است.
\begin{tldr}{تعریف}
فرض کنید \(f: X \to Y\) و \(g: Y \to Z\) باشند. ترکیب \(g\) و \(f\) که
با \(g \circ f\) نشان داده می‌شود، تابعی است از \(X\) به \(Z\) که به صورت
زیر تعریف می‌شود:
\[(g \circ f)(x) = g(f(x)) \quad , \quad \forall x \in X\]
\lr{[cite\_start][cite: }570{]}
\end{tldr}
\textbf{نکته مهم:} در نمادگذاری \(g \circ f\)، ابتدا تابع سمت راست
(\(f\)) و سپس تابع سمت چپ (\(g\)) اثر می‌کند.
\subsection{۴. انواع خاص توابع و تابع
همانی}\label{ux627ux646ux648ux627ux639-ux62eux627ux635-ux62aux648ux627ux628ux639-ux648-ux62aux627ux628ux639-ux647ux645ux627ux646ux6cc}
برای درک قضایای مربوط به وارون‌پذیری (قضیه ۱۶)، تعاریف زیر حیاتی هستند:
\subsubsection{\texorpdfstring{الف) تابع همانی
\lr{(Identity Function)}}{الف) تابع همانی }}\label{ux627ux644ux641-ux62aux627ux628ux639-ux647ux645ux627ux646ux6cc-identity-function}
تابع \(I_X: X \to X\) که هر عضو را به خودش می‌نگارد.
\[I_X(x) = x \quad , \quad \forall x \in X\] این تابع نقش «عنصر خنثی» را
در عمل ترکیب ایفا می‌کند.
\subsubsection{\texorpdfstring{ب) تابع یک‌به‌یک \lr{(Injective }/
\lr{One-to-One)}}{ب) تابع یک‌به‌یک / }}\label{ux628-ux62aux627ux628ux639-ux6ccux6a9ux628ux647ux6ccux6a9-injective-one-to-one}
تابعی که اعضای متمایز دامنه را به اعضای متمایز هم‌دامنه می‌برد.
\[f(x_1) = f(x_2) \implies x_1 = x_2\]
\subsubsection{\texorpdfstring{ج) تابع پوشا \lr{(Surjective }/
\lr{Onto)}}{ج) تابع پوشا / }}\label{ux62c-ux62aux627ux628ux639-ux67eux648ux634ux627-surjective-onto}
تابعی که برد آن برابر با هم‌دامنه باشد (هر عضو \(Y\) تصویر حداقل یک عضو
\(X\) باشد). \[\forall y \in Y, \exists x \in X : f(x) = y\]
\subsection{\texorpdfstring{۵. شبکه ارتباطی با قضایای فصل
\lr{(Analytic Map)}}{۵. شبکه ارتباطی با قضایای فصل }}\label{ux634ux628ux6a9ux647-ux627ux631ux62aux628ux627ux637ux6cc-ux628ux627-ux642ux636ux627ux6ccux627ux6cc-ux641ux635ux644-analytic-map}
\begin{itemize}
\tightlist
\item
  \textbf{حاصلضرب دکارتی \(\leftarrow\)
  \autoref{قضیه-۱---پخش‌پذیری-حاصلضرب-دکارتی}:} درک تعریف \(A \times B\)
  پیش‌شرط اثبات خاصیت پخش‌پذیری آن روی اشتراک و اجتماع است. بدون دانستن
  اینکه شرط عضویت در \(A \times B\) عبارت است از
  \((x \in A \wedge y \in B)\)، نمی‌توان اثبات‌های قضیه ۱ و ۲ را دنبال
  کرد.
\item
  \textbf{ترکیب توابع \(\leftarrow\)
  \autoref{قضیه-۱۵---شرکت‌پذیری-ترکیب-توابع}:} قضیه ۱۵ ثابت می‌کند که
  \((h \circ g) \circ f = h \circ (g \circ f)\). این ویژگی مستقیماً از
  تعریف جبری \(g(f(x))\) ناشی می‌شود.
\item
  \textbf{توابع خاص و همانی \(\leftarrow\)
  \autoref{قضیه-۱۶---وارون‌های-یک‌طرفه}:} قضیه ۱۶ ارتباط عمیقی بین ترکیب
  توابع و یک‌به‌یک/پوشا بودن برقرار می‌کند. مثلاً اگر \(g \circ f = I_X\)
  باشد، \(f\) لزوماً یک‌به‌یک است. درک نقش \(I_X\) در اینجا کلیدی است.
\end{itemize}

\clearpage
% ---------------------------------------------------------------------
% Copyright (c) 2026 Arsalan Dalvand & Reyhaneh Darvishi.
% Licensed under CC BY-NC-SA 4.0.
% See LICENSE file for details.
% ---------------------------------------------------------------------

\section{قضیه ۱: پخش‌پذیری حاصلضرب دکارتی بر تقاطع و
اجتماع}\label{قضیه-۱---پخش‌پذیری-حاصلضرب-دکارتی}
\begin{tldr}{خلاصه سریع}
این قضیه بیان می‌کند که عملگر حاصلضرب دکارتی (\(\times\)) نسبت به
عملگرهای اشتراک (\(\cap\)) و اجتماع (\(\cup\)) خاصیت \textbf{پخش‌پذیری
\lr{(Distributivity)}} دارد. این رفتار، شباهت ساختاری جبر مجموعه‌ها را با
جبر اعداد (پخش ضرب روی جمع) تکمیل می‌کند.
\end{tldr}
\subsection{۱. متن ریاضی
قضیه}\label{ux645ux62aux646-ux631ux6ccux627ux636ux6cc-ux642ux636ux6ccux647}
فرض کنید \(A\)، \(B\) و \(C\) سه مجموعه دلبخواه باشند. همواره روابط زیر
برقرارند:
\begin{theorembox}{قضیه ۱}
\textbf{الف) پخش‌پذیری بر اشتراک:}
\[A \times (B \cap C) = (A \times B) \cap (A \times C)\] \textbf{ب)
پخش‌پذیری بر اجتماع:}
\[A \times (B \cup C) = (A \times B) \cup (A \times C)\]
\end{theorembox}
\subsection{\texorpdfstring{۲. اثبات صوری
\lr{(Formal Proof)}}{۲. اثبات صوری }}\label{ux627ux62bux628ux627ux62a-ux635ux648ux631ux6cc-formal-proof}
اثبات این قضیه بر پایه ترجمه تعاریف مجموعه‌ای به گزاره‌های منطقی و استفاده
از قوانین هم‌ارزی فصل ۱ بنا شده است.
\subsubsection{اثبات قسمت
(الف)}\label{ux627ux62bux628ux627ux62a-ux642ux633ux645ux62a-ux627ux644ux641}
برای اثبات برابری دو مجموعه، نشان می‌دهیم که گزاره‌نمای عضویت زوج مرتب
\((a, x)\) در هر دو طرف یکسان است.
\begin{info}{برهان}
\[(a, x) \in A \times (B \cap C)\] ۱. طبق تعریف حاصلضرب دکارتی و اشتراک:
\[\iff (a \in A) \wedge (x \in B \cap C)\]
\[\iff (a \in A) \wedge [(x \in B) \wedge (x \in C)]\]
۲. استفاده از \textbf{\autoref{قضیه-۲---هم‌ارزی‌های-منطقی-پایه}} برای
گزاره \((a \in A)\): \emph{(می‌دانیم \(p \equiv p \wedge p\))}
\[\iff [(a \in A) \wedge (a \in A)] \wedge (x \in B) \wedge (x \in C)\]
۳. استفاده از \textbf{\autoref{قضیه-۴---قوانین-شرکت-پذیری-و-پخش-پذیری}}
برای بازآرایی گزاره‌ها:
\[\iff [(a \in A) \wedge (x \in B)] \wedge [(a \in A) \wedge (x \in C)]\]
۴. طبق تعریف حاصلضرب دکارتی:
\[\iff [(a, x) \in A \times B] \wedge [(a, x) \in A \times C]\]
۵. طبق تعریف اشتراک: \[\iff (a, x) \in (A \times B) \cap (A \times C)\]
\textbf{نتیجه:} دو مجموعه با هم برابرند.
\end{info}
\subsection{\texorpdfstring{۳. شبکه ارتباطی با سایر قضایا
\lr{(Analytic Map)}}{۳. شبکه ارتباطی با سایر قضایا }}\label{ux634ux628ux6a9ux647-ux627ux631ux62aux628ux627ux637ux6cc-ux628ux627-ux633ux627ux6ccux631-ux642ux636ux627ux6ccux627-analytic-map}
این قضیه پیوند عمیقی بین ساختار دکارتی و اصول منطق گزاره‌ها برقرار می‌کند:
\subsubsection{\texorpdfstring{۱. ارتباط با
\autoref{قضیه-۴---قوانین-شرکت-پذیری-و-پخش-پذیری}}{۱. ارتباط با }}\label{ux627ux631ux62aux628ux627ux637-ux628ux627-ux642ux636ux6ccux647-ux6f4---ux642ux648ux627ux646ux6ccux646-ux634ux631ux6a9ux62a-ux67eux630ux6ccux631ux6cc-ux648-ux67eux62eux634-ux67eux630ux6ccux631ux6cc}
\begin{itemize}
\tightlist
\item
  \textbf{مبانی منطقی:} اگرچه در اثبات بالا (بخش الف) عمدتاً از خاصیت
  جابجایی و شرکت‌پذیری «و» استفاده شد، اما در اثبات بخش (ب) (پخش روی
  اجتماع)، مستقیماً از قانون پخش‌پذیری منطق
  (\(p \wedge (q \vee r) \equiv (p \wedge q) \vee (p \wedge r)\))
  استفاده می‌شود. این نشان می‌دهد که پخش‌پذیری در مجموعه‌ها، تصویرِ پخش‌پذیری
  در منطق است.
\end{itemize}
\subsubsection{\texorpdfstring{۲. ارتباط با
\autoref{قضیه-۴---جبر-مجموعه-ها}}{۲. ارتباط با }}\label{ux627ux631ux62aux628ux627ux637-ux628ux627-ux642ux636ux6ccux647-ux6f4---ux62cux628ux631-ux645ux62cux645ux648ux639ux647-ux647ux627}
\begin{itemize}
\tightlist
\item
  \textbf{توسعه ساختار جبری:} در فصل ۲، قوانین پخش‌پذیری را برای \(\cap\)
  و \(\cup\) دیدیم (\(A \cap (B \cup C)\)). قضیه فعلی، عملگر سوم
  (\(\times\)) را وارد این ساختار می‌کند و نشان می‌دهد که این عملگر جدید
  نیز با عملگرهای قبلی سازگار \lr{(Compatible) }است.
\end{itemize}
\subsubsection{\texorpdfstring{۳. ارتباط با
\autoref{قضیه-۲---پخش‌پذیری-حاصلضرب-دکارتی-بر-تفاضل}}{۳. ارتباط با }}\label{ux627ux631ux62aux628ux627ux637-ux628ux627-ux642ux636ux6ccux647-ux6f2---ux67eux62eux634ux67eux630ux6ccux631ux6cc-ux62dux627ux635ux644ux636ux631ux628-ux62fux6a9ux627ux631ux62aux6cc-ux628ux631-ux62aux641ux627ux636ux644}
\begin{itemize}
\tightlist
\item
  \textbf{تعمیم به تفاضل:} بلافاصله پس از این قضیه، خواهیم دید که
  حاصلضرب دکارتی روی تفاضل مجموعه‌ها نیز پخش می‌شود
  (\(A \times (B - C)\)). اثبات آن نیز از الگوی مشابهی پیروی می‌کند و
  نشان‌دهنده رفتار یکنواخت \(\times\) نسبت به تمام عملیات بولی است.
\end{itemize}

\clearpage
% ---------------------------------------------------------------------
% Copyright (c) 2026 Arsalan Dalvand & Reyhaneh Darvishi.
% Licensed under CC BY-NC-SA 4.0.
% See LICENSE file for details.
% ---------------------------------------------------------------------

\section{قضیه ۲: پخش‌پذیری حاصلضرب دکارتی بر
تفاضل}\label{قضیه-۲---پخش‌پذیری-حاصلضرب-دکارتی-بر-تفاضل}
\begin{tldr}{خلاصه سریع}
این قضیه نشان می‌دهد که عملگر حاصلضرب دکارتی (\(\times\)) نسبت به عملگر
تفاضل (\(-\)) نیز خاصیت پخش‌پذیری دارد. این ویژگی، رفتار خطی و توزیع‌پذیر
ضرب دکارتی را در تمامی عملیات اصلی جبر مجموعه‌ها (اشتراک، اجتماع و تفاضل)
تکمیل می‌کند.
\end{tldr}
\subsection{۱. متن ریاضی
قضیه}\label{ux645ux62aux646-ux631ux6ccux627ux636ux6cc-ux642ux636ux6ccux647}
فرض کنید \(A\)، \(B\) و \(C\) سه مجموعه دلبخواه باشند. همواره رابطه زیر
برقرار است:
\begin{theorembox}{قضیه ۲}
\[A \times (B - C) = (A \times B) - (A \times C)\]
\end{theorembox}
\subsection{\texorpdfstring{۲. اثبات صوری
\lr{(Formal Proof)}}{۲. اثبات صوری }}\label{ux627ux62bux628ux627ux62a-ux635ux648ux631ux6cc-formal-proof}
برای اثبات این تساوی، از روش تحلیل گزاره‌نمای عضویت و زنجیره هم‌ارزی‌های
منطقی استفاده می‌کنیم. نکته کلیدی در این اثبات، استفاده از قوانین منطق
گزاره‌ها برای مدیریت نقیض در مؤلفه دوم است.
\begin{info}{برهان}
باید نشان دهیم گزاره \((a, x) \in A \times (B - C)\) منطقاً هم‌ارز با
\((a, x) \in (A \times B) - (A \times C)\) است.
۱. \textbf{بسط تعریف سمت چپ:} \[(a, x) \in A \times (B - C)\] طبق تعریف
حاصلضرب دکارتی: \[\iff (a \in A) \wedge (x \in B - C)\] طبق تعریف تفاضل
مجموعه‌ها (\(x \in S - T \iff x \in S \wedge x \notin T\)):
\[\iff (a \in A) \wedge [(x \in B) \wedge (x \notin C)]\]
۲. \textbf{به‌کارگیری قانون خودتوانی \lr{(Idempotent Law):}} برای اینکه
بتوانیم گزاره \(a \in A\) را با هر دو بخش دیگر ترکیب کنیم، از هم‌ارزی
منطقی \(p \equiv p \wedge p\) استفاده می‌کنیم:
\[\iff [(a \in A) \wedge (a \in A)] \wedge (x \in B) \wedge (x \notin C)\]
۳. \textbf{استفاده از قوانین شرکت‌پذیری و جابجایی \lr{(Associativity }\&
\lr{Commutativity):}} گزاره‌ها را برای ساختن فرم حاصلضرب دکارتی بازآرایی
می‌کنیم:
\[\iff [(a \in A) \wedge (x \in B)] \wedge [(a \in A) \wedge (x \notin C)]\]
۴. \textbf{تحلیل منطقی بخش دوم (نکته ظریف):} بخش اول
\([(a \in A) \wedge (x \in B)]\) دقیقاً معادل \((a, x) \in A \times B\)
است. برای بخش دوم، باید نشان دهیم در حضور شرط اول (\(a \in A\))، گزاره
\(x \notin C\) هم‌ارز با \((a, x) \notin A \times C\) است.
\begin{itemize}
\tightlist
\item
  می‌دانیم
  \((a, x) \notin A \times C \iff \sim(a \in A \wedge x \in C) \iff a \notin A \vee x \notin C\).
\item
  چون در این زنجیره استدلال، \(a \in A\) مفروض است، پس گزاره
  \(a \notin A\) کاذب (\(F\)) می‌باشد.
\item
  بنابراین \(F \vee (x \notin C)\) منطقاً هم‌ارز با \(x \notin C\) است.
\end{itemize}
پس می‌توانیم بنویسیم:
\[\iff [(a, x) \in A \times B] \wedge [(a, x) \notin A \times C]\]
۵. \textbf{نتیجه‌گیری نهایی:} طبق تعریف تفاضل مجموعه‌ها:
\[\iff (a, x) \in (A \times B) - (A \times C)\]
\end{info}
\subsection{\texorpdfstring{۳. شبکه ارتباطی با سایر قضایا
\lr{(Analytic Map)}}{۳. شبکه ارتباطی با سایر قضایا }}\label{ux634ux628ux6a9ux647-ux627ux631ux62aux628ux627ux637ux6cc-ux628ux627-ux633ux627ux6ccux631-ux642ux636ux627ux6ccux627-analytic-map}
این قضیه پازل رفتار جبری حاصلضرب دکارتی را تکمیل می‌کند:
\subsubsection{\texorpdfstring{۱. ارتباط با
\autoref{قضیه-۱---پخش‌پذیری-حاصلضرب-دکارتی}}{۱. ارتباط با }}\label{ux627ux631ux62aux628ux627ux637-ux628ux627-ux642ux636ux6ccux647-ux6f1---ux67eux62eux634ux67eux630ux6ccux631ux6cc-ux62dux627ux635ux644ux636ux631ux628-ux62fux6a9ux627ux631ux62aux6cc}
\begin{itemize}
\tightlist
\item
  \textbf{تکمیل توزیع‌پذیری:} در قضیه ۱، پخش‌پذیری \(\times\) بر \(\cap\)
  و \(\cup\) اثبات شد. قضیه ۲ نشان می‌دهد که این رفتار برای تفاضل (\(-\))
  نیز صادق است. این یعنی حاصلضرب دکارتی با تمام عملیات بولی استاندارد
  سازگار است.
\end{itemize}
\subsubsection{\texorpdfstring{۲. ارتباط با
\autoref{قضیه-۵---متمم-و-زیرمجموعه}}{۲. ارتباط با }}\label{ux627ux631ux62aux628ux627ux637-ux628ux627-ux642ux636ux6ccux647-ux6f5-ux641ux635ux644-ux6f2}
\begin{itemize}
\tightlist
\item
  \textbf{ریشه تعریف تفاضل:} اثبات این قضیه (گام ۱) کاملاً وابسته به
  تعریف دقیق تفاضل است که در فصل ۲ بررسی شد (\(B - C = B \cap C'\)).
  بدون درک منطقی نقیض عضویت (\(x \notin C\))، گام ۴ اثبات قابل درک
  نخواهد بود.
\end{itemize}
\subsubsection{\texorpdfstring{۳. ارتباط با
\autoref{قضیه-۴---جبر-مجموعه-ها}}{۳. ارتباط با }}\label{ux627ux631ux62aux628ux627ux637-ux628ux627-ux642ux636ux6ccux647-ux6f4---ux62cux628ux631-ux645ux62cux645ux648ux639ux647-ux647ux627}
\begin{itemize}
\tightlist
\item
  \textbf{مبانی استنتاج:} استفاده از قانون خودتوانی (\(p \wedge p\)) و
  جابجایی در گام‌های ۲ و ۳، کاربرد مستقیم قوانین منطق گزاره‌ها در نظریه
  مجموعه‌هاست. این قضیه نمونه‌ای عالی از تبدیل یک مسئله مجموعه‌ای به مسئله
  منطقی و حل آن است.
\end{itemize}

\clearpage
% ---------------------------------------------------------------------
% Copyright (c) 2026 Arsalan Dalvand & Reyhaneh Darvishi.
% Licensed under CC BY-NC-SA 4.0.
% See LICENSE file for details.
% ---------------------------------------------------------------------

\section{تمرین ۱: نمایش هندسی مجموعه‌ها در صفحه
دکارتی}\label{تمرین-۱---نمایش-هندسی-حاصلضرب-دکارتی}
\begin{tldr}{خلاصه سریع}
در فضای \(R \times R\) (صفحه مختصات)، رابطه \(x=y\) یک خط، \(x>y\) یک
ناحیه (نیم‌صفحه) و \(|x-y| \le 1\) یک نوار مورب است.
\end{tldr}
\subsection{۱. صورت
سوال}\label{ux635ux648ux631ux62a-ux633ux648ux627ux644}
هر یک از مجموعه‌های زیر را با رسم نمودار در صفحه دکارتی به طور هندسی
نمایش دهید : الف) \(\{(x,y) \in R \times R \mid x=y\}\) ب)
\(\{(x,y) \in R \times R \mid x > y\}\) ج)
\(\{(x,y) \in R \times R \mid |x-y| \le 1\}\)
\subsection{۲. حل تشریحی و
تحلیل}\label{ux62dux644-ux62aux634ux631ux6ccux62dux6cc-ux648-ux62aux62dux644ux6ccux644}
\subsubsection{\texorpdfstring{الف) نمودار
\(x = y\)}{الف) نمودار x = y}}\label{ux627ux644ux641-ux646ux645ux648ux62fux627ux631-x-y}
این معادله بیانگر نقاطی است که طول و عرضشان برابر است.
\begin{itemize}
\tightlist
\item
  \textbf{تحلیل:} این همان نیمساز ربع اول و سوم است. خطی که از مبدا
  می‌گذرد و زاویه ۴۵ درجه با محور افق دارد.
\end{itemize}
\subsubsection{\texorpdfstring{ب) نمودار
\(x > y\)}{ب) نمودار x \textgreater{} y}}\label{ux628-ux646ux645ux648ux62fux627ux631-x-y}
این نابرابری بیانگر ناحیه‌ای است که طول نقاط از عرضشان بیشتر است.
\begin{itemize}
\tightlist
\item
  \textbf{تحلیل:} خط \(x=y\) صفحه را به دو نیمه تقسیم می‌کند.
  \begin{itemize}
  \tightlist
  \item
    ناحیه بالای خط: \(y > x\)
  \item
    ناحیه پایین خط: \(x > y\)
  \end{itemize}
\item
  \textbf{پاسخ:} تمام ناحیه سمت راست و پایین خط \(x=y\) (بدون خود خط،
  چون تساوی نداریم).
\end{itemize}
\subsubsection{\texorpdfstring{ج) نمودار
\(|x-y| \le 1\)}{ج) نمودار \textbar x-y\textbar{} \textbackslash le 1}}\label{ux62c-ux646ux645ux648ux62fux627ux631-x-y-le-1}
این نامساوی قدرمطلقی به معنای فاصله عمودی یا افقی بین \(x\) و \(y\) است.
\begin{itemize}
\tightlist
\item
  \textbf{باز کردن قدر مطلق:} \[|x - y| \le 1 \iff -1 \le x - y \le 1\]
\item
  \textbf{تبدیل به دو نامساوی خطی:}
  \begin{enumerate}
  \def\labelenumi{\arabic{enumi}.}
  \tightlist
  \item
    \(x - y \le 1 \Rightarrow y \ge x - 1\) (ناحیه بالای خط \(y=x-1\))
  \item
    \(x - y \ge -1 \Rightarrow y \le x + 1\) (ناحیه پایین خط \(y=x+1\))
  \end{enumerate}
\item
  \textbf{پاسخ:} ناحیه محدود بین دو خط موازی \(y=x+1\) و \(y=x-1\) (یک
  نوار مورب شامل خود خطوط) .
\end{itemize}

\clearpage
% ---------------------------------------------------------------------
% Copyright (c) 2026 Arsalan Dalvand & Reyhaneh Darvishi.
% Licensed under CC BY-NC-SA 4.0.
% See LICENSE file for details.
% ---------------------------------------------------------------------

\section{تمرین ۲: شرط جابجایی حاصلضرب
دکارتی}\label{تمرین-۲---جابجایی-در-حاصلضرب-دکارتی}
\subsection{۱. صورت
سوال}\label{ux635ux648ux631ux62a-ux633ux648ux627ux644}
مجموعه‌های \(A\) و \(B\) چه شرایطی دارا باشند تا تساوی
\(A \times B = B \times A\) راست باشد؟
\subsection{۲. استراتژی و
حل}\label{ux627ux633ux62aux631ux627ux62aux698ux6cc-ux648-ux62dux644}
در حالت کلی \(A \times B \neq B \times A\) است (چون زوج مرتب \((a,b)\)
با \((b,a)\) فرق دارد). این تساوی فقط در شرایط خاص برقرار است.
\begin{info}{تحلیل حالات}
تساوی برقرار است اگر و تنها اگر:
\begin{enumerate}
\def\labelenumi{\arabic{enumi}.}
\tightlist
\item
  \textbf{یکی از مجموعه‌ها تهی باشد:} اگر \(A = \emptyset\) یا
  \(B = \emptyset\) باشد، حاصلضرب دکارتی هر دو طرف \(\emptyset\) می‌شود و
  تساوی برقرار است.
\item
  \textbf{مجموعه‌ها برابر باشند:} اگر \(A = B\) باشد، بدیهی است که
  \(A \times A = A \times A\).
\end{enumerate}
\textbf{پاسخ نهایی:} \[A = \emptyset \lor B = \emptyset \lor A = B\]
\end{info}

\clearpage
% ---------------------------------------------------------------------
% Copyright (c) 2026 Arsalan Dalvand & Reyhaneh Darvishi.
% Licensed under CC BY-NC-SA 4.0.
% See LICENSE file for details.
% ---------------------------------------------------------------------

\section{تمرین ۳: اثبات پخش‌پذیری حاصلضرب دکارتی روی
اجتماع}\label{تمرین-۳---پخش‌پذیری-روی-اجتماع}
\subsection{۱. صورت
سوال}\label{ux635ux648ux631ux62a-ux633ux648ux627ux644}
قضیه ۱ (ب) را ثابت کنید:
\[A \times (B \cup C) = (A \times B) \cup (A \times C)\]
\subsection{۲. حل تشریحی (روش زنجیره
هم‌ارزی)}\label{ux62dux644-ux62aux634ux631ux6ccux62dux6cc-ux631ux648ux634-ux632ux646ux62cux6ccux631ux647-ux647ux645ux627ux631ux632ux6cc}
برای اثبات تساوی دو مجموعه، نشان می‌دهیم عضو بودن در سمت چپ معادل عضو
بودن در سمت راست است.
\begin{info}{مراحل اثبات}
\[(x,y) \in A \times (B \cup C)\] طبق تعریف حاصلضرب دکارتی:
\[\equiv (x \in A) \wedge (y \in B \cup C)\] طبق تعریف اجتماع:
\[\equiv (x \in A) \wedge (y \in B \lor y \in C)\] طبق قانون پخش‌پذیری
منطق (پخش \(\wedge\) روی \(\lor\)):
\[\equiv [(x \in A) \wedge (y \in B)] \lor [(x \in A) \wedge (y \in C)]\]
طبق تعریف حاصلضرب دکارتی برای هر کروشه:
\[\equiv [(x,y) \in A \times B] \lor [(x,y) \in A \times C]\] طبق تعریف
اجتماع: \[\equiv (x,y) \in (A \times B) \cup (A \times C)\]
\end{info}

\clearpage
% ---------------------------------------------------------------------
% Copyright (c) 2026 Arsalan Dalvand & Reyhaneh Darvishi.
% Licensed under CC BY-NC-SA 4.0.
% See LICENSE file for details.
% ---------------------------------------------------------------------

\section{تمرین ۴: حاصلضرب دکارتی چه زمانی تهی
است؟}\label{تمرین-۴---شرط-تهی-بودن-حاصلضرب}
\subsection{۱. صورت
سوال}\label{ux635ux648ux631ux62a-ux633ux648ux627ux644}
ثابت کنید:
\[A \times B = \emptyset \iff (A = \emptyset \lor B = \emptyset)\]
\subsection{۲. حل
تشریحی}\label{ux62dux644-ux62aux634ux631ux6ccux62dux6cc}
این اثبات دو طرفه است (\(\Rightarrow\) و \(\Leftarrow\)).
\begin{info}{اثبات}
\textbf{طرف اول (\(\Leftarrow\)):} اگر \(A = \emptyset\) یا
\(B = \emptyset\) باشد، هیچ زوج مرتبی \((x,y)\) نمی‌توان ساخت که
\(x \in A\) و \(y \in B\) باشد (چون یکی از آن‌ها عضو ندارد). پس
\(A \times B = \emptyset\).
\textbf{طرف دوم (\(\Rightarrow\)):} (از برهان خلف استفاده می‌کنیم). فرض
کنیم \(A \times B = \emptyset\) باشد اما حکم غلط باشد (یعنی هم
\(A \neq \emptyset\) و هم \(B \neq \emptyset\)). چون \(A\) ناتهی است، پس
عنصری مثل \(x\) دارد. چون \(B\) ناتهی است، پس عنصری مثل \(y\) دارد. پس
زوج \((x,y)\) وجود دارد که در \(A \times B\) باشد. این یعنی
\(A \times B \neq \emptyset\) که با فرض تناقض دارد. پس حتماً باید
\(A=\emptyset\) یا \(B=\emptyset\) باشد.
\end{info}

\clearpage
% ---------------------------------------------------------------------
% Copyright (c) 2026 Arsalan Dalvand & Reyhaneh Darvishi.
% Licensed under CC BY-NC-SA 4.0.
% See LICENSE file for details.
% ---------------------------------------------------------------------

\section{تمرین ۵: رابطه زیرمجموعه و حاصلضرب
دکارتی}\label{تمرین-۵---یکنوایی-حاصلضرب-دکارتی}
\subsection{۱. صورت
سوال}\label{ux635ux648ux631ux62a-ux633ux648ux627ux644}
ثابت کنید که اگر \(A, B, C\) مجموعه باشند و \(A \subseteq B\)، آنگاه
\(A \times C \subseteq B \times C\).
\subsection{۲. استراتژی و
حل}\label{ux627ux633ux62aux631ux627ux62aux698ux6cc-ux648-ux62dux644}
باید نشان دهیم هر عضوی که در سمت چپ است، در سمت راست هم هست.
\begin{info}{اثبات}
فرض کنیم \((x,y)\) یک عضو دلخواه از \(A \times C\) باشد.
\begin{enumerate}
\def\labelenumi{\arabic{enumi}.}
\tightlist
\item
  \((x,y) \in A \times C \implies (x \in A) \wedge (y \in C)\)
\item
  طبق فرض مسئله، می‌دانیم \(A \subseteq B\).این یعنی
  \((x \in A \implies x \in B)\).
\item
  پس می‌توانیم در گزاره مرحله ۱، به جای \(x \in A\) نتیجه آن یعنی
  \(x \in B\) را در نظر بگیریم (قیاس استثنایی):
  \[\implies (x \in B) \wedge (y \in C)\]
\item
  این تعریفِ \((x,y) \in B \times C\) است.
\item
  نتیجه: \(A \times C \subseteq B \times C\) .
\end{enumerate}
\end{info}

\clearpage
% ---------------------------------------------------------------------
% Copyright (c) 2026 Arsalan Dalvand & Reyhaneh Darvishi.
% Licensed under CC BY-NC-SA 4.0.
% See LICENSE file for details.
% ---------------------------------------------------------------------

\section{تمرین ۶ و ۷: شمارش اعضا و بازیابی
مجموعه}\label{تمرین-۶-و-۷---تعداد-اعضا-و-بازیابی-مجموعه}
\subsection{۱. صورت سوال تمرین
۶}\label{ux635ux648ux631ux62a-ux633ux648ux627ux644-ux62aux645ux631ux6ccux646-ux6f6}
\lr{[cite\_start]اگر }مجموعه \(A\) دارای \(m\) عنصر و مجموعه \(B\) دارای
\(n\) عنصر باشد، مجموعه \(A \times B\) چند عنصر (جفت مرتب) دارد؟
\lr{[cite: }676-677{]}
\subsubsection{حل تمرین
۶}\label{ux62dux644-ux62aux645ux631ux6ccux646-ux6f6}
طبق اصل ضرب، برای مولفه اول \(m\) انتخاب و برای مولفه دوم \(n\) انتخاب
داریم. \lr{[cite\_start]تعداد }اعضا برابر است با
\(m \times n\)\lr{[cite: }684{]}.
\begin{center}\rule{0.5\linewidth}{0.5pt}\end{center}
\subsection{۲. صورت سوال تمرین
۷}\label{ux635ux648ux631ux62a-ux633ux648ux627ux644-ux62aux645ux631ux6ccux646-ux6f7}
مجموعه \(A \times A\) نه (۹) عنصر دارد که \((-1,0)\) و \((0,1)\) دو عنصر
آن هستند. اعضای دیگر \(A \times A\) را بیابید.
\subsubsection{حل تمرین
۷}\label{ux62dux644-ux62aux645ux631ux6ccux646-ux6f7}
\begin{enumerate}
\def\labelenumi{\arabic{enumi}.}
\tightlist
\item
  \textbf{یافتن تعداد اعضای \lr{A:}} چون
  \(|A \times A| = |A| \cdot |A| = 9\)، پس \(|A|^2 = 9\) و در نتیجه
  \(A\) باید \textbf{۳ عنصر} داشته باشد.
\item
  \textbf{یافتن اعضای \lr{A:}} زوج‌های \((-1, 0)\) و \((0, 1)\) در
  \(A \times A\) هستند. در حاصلضرب دکارتی \(A \times A\)، هم مولفه اول و
  هم مولفه دوم باید از \(A\) باشند.
  \begin{itemize}
  \tightlist
  \item
    از \((-1, 0)\) می‌فهمیم: \(-1 \in A\) و \(0 \in A\).
  \item
    از \((0, 1)\) می‌فهمیم: \(0 \in A\) و \(1 \in A\).
  \item
    مجموعه اعضای یافت شده: \(\{-1, 0, 1\}\). چون \(A\) باید ۳ عضو داشته
    باشد، همین‌ها تمام اعضای \(A\) هستند. \[A = \{-1, 0, 1\}\]
  \end{itemize}
\item
  \textbf{تشکیل \(A \times A\):} تمام ترکیب‌های دوتایی از این ۳ عدد را
  می‌نویسیم:
  \[A \times A = \{ (-1,-1), (-1,0), (-1,1), (0,-1), (0,0), (0,1), (1,-1), (1,0), (1,1) \}\]
\end{enumerate}

\clearpage
% ---------------------------------------------------------------------
% Copyright (c) 2026 Arsalan Dalvand & Reyhaneh Darvishi.
% Licensed under CC BY-NC-SA 4.0.
% See LICENSE file for details.
% ---------------------------------------------------------------------

\section{تمرین ۸: بررسی گزاره‌ها با مثال
نقض}\label{تمرین-۸---مثال‌های-نقض}
\subsection{۱. صورت
سوال}\label{ux635ux648ux631ux62a-ux633ux648ux627ux644}
\lr{[cite\_start]درستی }یا نادرستی حکم‌های زیر را با آوردن مثال نقض بررسی
\lr{کنید[cite: }693{]}: الف)
\(A \times B \subseteq C \times D \iff A \subseteq C \land B \subseteq D\)
ب) \(\mathcal{P}(A \times B) = \mathcal{P}(A) \times \mathcal{P}(B)\)
\subsection{۲. حل
تشریحی}\label{ux62dux644-ux62aux634ux631ux6ccux62dux6cc}
\subsubsection{الف) بررسی شرط زیرمجموعه
بودن}\label{ux627ux644ux641-ux628ux631ux631ux633ux6cc-ux634ux631ux637-ux632ux6ccux631ux645ux62cux645ux648ux639ux647-ux628ux648ux62fux646}
این حکم \textbf{نادرست} است.
\begin{itemize}
\tightlist
\item
  \textbf{دلیل:} اگر یکی از مجموعه‌ها (مثلاً \(B\)) تهی باشد، حاصلضرب
  دکارتی تهی می‌شود و تهی زیرمجموعه هر چیزی است، حتی اگر شرط
  \(A \subseteq C\) برقرار نباشد.
\item
  \textbf{مثال نقض:} فرض کنید \(A=\{1, 2\}, B=\emptyset\) و
  \(C=\{1\}, D=\{a\}\). \[A \times B = \emptyset\]
  \[\emptyset \subseteq C \times D\] (این درست است) اما
  \(A \nsubseteq C\) (چون ۲ در \(A\) هست ولی در \(C\) نیست). پس شرط
  برقرار نیست.
\end{itemize}
\subsubsection{ب) مجموعه توانی
حاصلضرب}\label{ux628-ux645ux62cux645ux648ux639ux647-ux62aux648ux627ux646ux6cc-ux62dux627ux635ux644ux636ux631ux628}
این حکم \textbf{نادرست} است.
\begin{itemize}
\tightlist
\item
  \textbf{دلیل:} جنس اعضای دو طرف فرق می‌کند. \(\mathcal{P}(A \times B)\)
  شامل زیرمجموعه‌هایی از زوج‌مرتب‌هاست، اما
  \(\mathcal{P}(A) \times \mathcal{P}(B)\) شامل زوج‌مرتب‌هایی از
  مجموعه‌هاست. همچنین تعداد اعضایشان برابر نیست.
\item
  \textbf{تحلیل تعدادی:} اگر \(|A|=m, |B|=n\):
  \begin{itemize}
  \tightlist
  \item
    تعداد اعضای سمت راست: \(2^m \times 2^n = 2^{m+n}\)
  \item
    تعداد اعضای سمت چپ: \(2^{mn}\)
  \item
    معمولاً \(2^{mn} \neq 2^{m+n}\) (مثلاً \(m=n=2 \to 16 \neq 256\)) .
  \end{itemize}
\end{itemize}

\clearpage
% ---------------------------------------------------------------------
% Copyright (c) 2026 Arsalan Dalvand & Reyhaneh Darvishi.
% Licensed under CC BY-NC-SA 4.0.
% See LICENSE file for details.
% ---------------------------------------------------------------------

\section{تمرین ۹ و ۱۵: ویژگی اشتراک در حاصلضرب
دکارتی}\label{تمرین-۹-و-۱۵---اشتراک-حاصلضرب‌ها}
\subsection{۱. صورت سوال (تمرین
۹)}\label{ux635ux648ux631ux62a-ux633ux648ux627ux644-ux62aux645ux631ux6ccux646-ux6f9}
اگر \(A, B, C, D\) چهار مجموعه باشند، ثابت کنید:
\[(A \times C) \cap (B \times D) = (A \cap B) \times (C \cap D)\]
\subsection{۲. اثبات تمرین
۹}\label{ux627ux62bux628ux627ux62a-ux62aux645ux631ux6ccux646-ux6f9}
نشان می‌دهیم زوج \((x,y)\) اگر در یکی باشد، در دیگری هم هست.
\begin{info}{اثبات}
\[(x,y) \in (A \times C) \cap (B \times D)\]
\[\equiv [(x,y) \in A \times C] \wedge [(x,y) \in B \times D]\]
\[\equiv (x \in A \wedge y \in C) \wedge (x \in B \wedge y \in D)\] با
جابجایی پرانتزها (شرکت‌پذیری و جابجایی \(\wedge\)):
\[\equiv (x \in A \wedge x \in B) \wedge (y \in C \wedge y \in D)\]
\[\equiv (x \in A \cap B) \wedge (y \in C \cap D)\]
\[\equiv (x,y) \in (A \cap B) \times (C \cap D)\]
\end{info}
\begin{center}\rule{0.5\linewidth}{0.5pt}\end{center}
\subsection{۳. صورت سوال (تمرین
۱۵)}\label{ux635ux648ux631ux62a-ux633ux648ux627ux644-ux62aux645ux631ux6ccux646-ux6f1ux6f5}
ثابت کنید اگر \(A \cap B = \emptyset\)، آنگاه برای هر مجموعه \(C\) و
\(D\): \[(A \times C) \cap (B \times D) = \emptyset\]
\subsection{۴. حل تمرین
۱۵}\label{ux62dux644-ux62aux645ux631ux6ccux646-ux6f1ux6f5}
از نتیجه تمرین ۹ استفاده می‌کنیم.
\begin{info}{حل}
طبق تمرین ۹ داریم:
\[(A \times C) \cap (B \times D) = (A \cap B) \times (C \cap D)\] چون
فرض شده \(A \cap B = \emptyset\): \[= \emptyset \times (C \cap D)\]
حاصلضرب دکارتی هر مجموعه در تهی، برابر \textbf{تهی} است. \[= \emptyset\]
\end{info}

\clearpage
% ---------------------------------------------------------------------
% Copyright (c) 2026 Arsalan Dalvand & Reyhaneh Darvishi.
% Licensed under CC BY-NC-SA 4.0.
% See LICENSE file for details.
% ---------------------------------------------------------------------

\section{\texorpdfstring{تمرین ۱۰: حاصلضرب دکارتی
\lr{nتایی}}{تمرین ۱۰: حاصلضرب دکارتی }}\label{تمرین-۱۰---تعمیم-حاصلضرب-دکارتی}
\subsection{۱. صورت
سوال}\label{ux635ux648ux631ux62a-ux633ux648ux627ux644}
آیا می‌توانید تعریف حاصلضرب دکارتی را برای سه مجموعه
\(A_1 \times A_2 \times A_3\) و سپس برای \(n\) مجموعه تعمیم دهید؟
\subsection{۲. پاسخ}\label{ux67eux627ux633ux62e}
بله، به جای «زوج مرتب»، از «سه‌تایی مرتب» و \lr{«n-تایی }مرتب» استفاده
می‌کنیم.
\begin{tldr}{تعریف تعمیم یافته}
\textbf{برای ۳ مجموعه:}
\[A_1 \times A_2 \times A_3 = \{ (a_1, a_2, a_3) \mid a_1 \in A_1, a_2 \in A_2, a_3 \in A_3 \}\]
\textbf{برای \lr{n }مجموعه:}
\[A_1 \times A_2 \times \dots \times A_n = \{ (a_1, a_2, \dots, a_n) \mid a_i \in A_i, i=1,\dots,n \}\]
\end{tldr}

\clearpage
% ---------------------------------------------------------------------
% Copyright (c) 2026 Arsalan Dalvand & Reyhaneh Darvishi.
% Licensed under CC BY-NC-SA 4.0.
% See LICENSE file for details.
% ---------------------------------------------------------------------

\section{تمرین ۱۱، ۱۲ و ۱۶: قوانین حذف در تساوی
حاصلضرب‌ها}\label{تمرین-۱۱-و-۱۲-و-۱۶---قوانین-حذف-و-تساوی}
\subsection{۱. تمرین ۱۱}\label{ux62aux645ux631ux6ccux646-ux6f1ux6f1}
\textbf{سوال:} ثابت کنید اگر \(A \times A = B \times B\) آنگاه
\(A = B\). \textbf{اثبات:}
\[(x,y) \in A \times A \iff x \in A \wedge y \in A\]
\[(x,y) \in B \times B \iff x \in B \wedge y \in B\] چون دو طرف برابرند،
اگر عضوی مثل \(z\) در \(A\) باشد، زوج \((z,z)\) در \(A \times A\) است،
پس در \(B \times B\) هم هست، پس \(z \in B\). و برعکس. پس \(A=B\) .
\begin{center}\rule{0.5\linewidth}{0.5pt}\end{center}
\subsection{۲. تمرین ۱۲}\label{ux62aux645ux631ux6ccux646-ux6f1ux6f2}
\textbf{سوال:} ثابت کنید اگر \(A \times C = B \times C\) و
\(C \neq \emptyset\)، آنگاه \(A = B\). \textbf{اثبات:} چون
\(C \neq \emptyset\)، پس یک عضو \(y \in C\) وجود دارد. فرض کنیم
\(x \in A\). آنگاه \((x,y) \in A \times C\). طبق فرض تساوی،
\((x,y) \in B \times C\). پس \(x \in B\) (و \(y \in C\)). بنابراین
\(A \subseteq B\). به طور مشابه \(B \subseteq A\). پس \(A=B\).
\begin{center}\rule{0.5\linewidth}{0.5pt}\end{center}
\subsection{۳. تمرین ۱۶}\label{ux62aux645ux631ux6ccux646-ux6f1ux6f6}
\textbf{سوال:} فرض کنیم \(A, B, C, D\) ناتهی باشند. ثابت کنید
\(A \times B = C \times D \iff A=C \land B=D\).
\textbf{اثبات:} اگر \(A=C\) و \(B=D\) باشد، تساوی بدیهی است. برعکس، فرض
کنیم \(A \times B = C \times D\). برای هر \(a \in A\) و هر \(b \in B\)
(چون ناتهی هستند)، زوج \((a,b) \in A \times B\) است. پس
\((a,b) \in C \times D\). نتیجه می‌دهد \(a \in C\) (پس \(A \subseteq C\))
و \(b \in D\) (پس \(B \subseteq D\)). به همین ترتیب عکس آن ثابت می‌شود.
پس \(A=C\) و \(B=D\).

\clearpage
% ---------------------------------------------------------------------
% Copyright (c) 2026 Arsalan Dalvand & Reyhaneh Darvishi.
% Licensed under CC BY-NC-SA 4.0.
% See LICENSE file for details.
% ---------------------------------------------------------------------

\section{تمرین ۱۳ و ۱۴: بررسی توزیع‌پذیری‌های
نادرست}\label{تمرین-۱۳-و-۱۴---توزیع‌پذیری-اشتباه}
\begin{warning}{هشدار}
اجتماع نسبت به حاصلضرب دکارتی به شکل جبری ساده توزیع نمی‌شود.
\end{warning}
\subsection{۱. تمرین ۱۳}\label{ux62aux645ux631ux6ccux646-ux6f1ux6f3}
\textbf{سوال:} آیا
\(A \cup (B \times C) = (A \cup B) \times (A \cup C)\) درست است؟
\textbf{پاسخ:} \textbf{خیر}. \textbf{مثال نقض:}
\(A=\{a\}, B=\{b\}, C=\{c\}\).
\begin{itemize}
\tightlist
\item
  سمت چپ: \(\{a, (b,c)\}\) (یک عضو ساده و یک زوج مرتب).
\item
  سمت راست: \(\{a,b\} \times \{a,c\} = \{(a,a), (a,c), (b,a), (b,c)\}\).
\item
  این دو برابر نیستند.
\end{itemize}
\begin{center}\rule{0.5\linewidth}{0.5pt}\end{center}
\subsection{۲. تمرین ۱۴}\label{ux62aux645ux631ux6ccux646-ux6f1ux6f4}
\textbf{سوال:} آیا
\((A \times C) \cup (B \times D) = (A \cup B) \times (C \cup D)\) درست
است؟ \textbf{پاسخ:} \textbf{خیر}. \textbf{مثال نقض:}
\(A=\emptyset, B=\{b\}, C=\{c\}, D=\{d\}\).
\begin{itemize}
\tightlist
\item
  سمت چپ:
  \((\emptyset \times \{c\}) \cup (\{b\} \times \{d\}) = \emptyset \cup \{(b,d)\} = \{(b,d)\}\).
\item
  سمت راست:
  \((\emptyset \cup \{b\}) \times (\{c\} \cup \{d\}) = \{b\} \times \{c,d\} = \{(b,c), (b,d)\}\).
  *سمت راست یک عضو اضافه \((b,c)\) دارد .
\end{itemize}

\clearpage
% ---------------------------------------------------------------------
% Copyright (c) 2026 Arsalan Dalvand & Reyhaneh Darvishi.
% Licensed under CC BY-NC-SA 4.0.
% See LICENSE file for details.
% ---------------------------------------------------------------------

\section{تمرین ۱۷، ۱۸ و ۱۹: پخش‌پذیری حاصلضرب روی
تفاضل}\label{تمرین-۱۷-و-۱۸-و-۱۹---جبر-تفاضل-در-حاصلضرب}
\begin{tldr}{خلاصه}
این تمرین‌ها فرمول‌های تجزیه تفاضل حاصلضرب‌ها را ثابت می‌کنند. تمرین ۱۹ حالت
کلی تمرین ۱۷ و ۱۸ است.
\end{tldr}
\subsection{۱. تمرین ۱۹ (حالت
کلی)}\label{ux62aux645ux631ux6ccux646-ux6f1ux6f9-ux62dux627ux644ux62a-ux6a9ux644ux6cc}
\textbf{سوال:} ثابت کنید:
\[(A \times B) - (C \times D) = [(A - C) \times B] \cup [A \times (B - D)]\]
\textbf{اثبات:} \[(x,y) \in (A \times B) - (C \times D)\]
\[\equiv (x \in A \wedge y \in B) \wedge \sim(x \in C \wedge y \in D)\]
(نقیض ``و'' می‌شود ``یا'' - دمورگان):
\[\equiv (x \in A \wedge y \in B) \wedge (x \notin C \lor y \notin D)\]
(پخش کردن پرانتز اول روی پرانتز دوم):
\[\equiv [(x \in A \wedge y \in B) \wedge x \notin C] \lor [(x \in A \wedge y \in B) \wedge y \notin D]\]
(مرتب کردن جملات):
\[\equiv [x \in (A-C) \wedge y \in B] \lor [x \in A \wedge y \in (B-D)]\]
\[\equiv [(x,y) \in (A-C) \times B] \lor [(x,y) \in A \times (B-D)]\] که
همان اجتماع دو طرف است .
\subsection{۲. تمرین ۱۷ و
۱۸}\label{ux62aux645ux631ux6ccux646-ux6f1ux6f7-ux648-ux6f1ux6f8}
این دو تمرین حالت‌های خاص تمرین ۱۹ هستند:
\begin{itemize}
\tightlist
\item
  \textbf{تمرین ۱۷:} با قرار دادن \(B=B\) و \(D=C\) و \(C=C\) (کمی
  نامگذاری متفاوت است) فرمول برای \((A \times B) - (C \times C)\) اثبات
  می‌شود.
\item
  \textbf{تمرین ۱۸:} اثبات برای \((A \times A) - (B \times C)\) که دقیقاً
  با منطق بالا حل می‌شود.
\end{itemize}

\clearpage
% ---------------------------------------------------------------------
% Copyright (c) 2026 Arsalan Dalvand & Reyhaneh Darvishi.
% Licensed under CC BY-NC-SA 4.0.
% See LICENSE file for details.
% ---------------------------------------------------------------------

\section{تمرین ۲۰: تعریف دقیق زوج مرتب
(کوراتوسکی)}\label{تمرین-۲۰---تعریف-زوج-مرتب-کوراتوسکی}
\subsection{۱. صورت
سوال}\label{ux635ux648ux631ux62a-ux633ux648ux627ux644}
زوج مرتب \((x,y)\) را به صورت مجموعه \(\{\{x\}, \{x,y\}\}\) تعریف
می‌کنیم. ثابت کنید: \[(a,b) = (c,d) \iff a=c \land b=d\]
\subsection{۲. اثبات}\label{ux627ux62bux628ux627ux62a}
\subsubsection{\texorpdfstring{جهت اول
(\(\Leftarrow\))}{جهت اول (\textbackslash Leftarrow)}}\label{ux62cux647ux62a-ux627ux648ux644-leftarrow}
اگر \(a=c\) و \(b=d\) باشد، مجموعه‌های سازنده یکسان می‌شوند:
\[\{\{a\}, \{a,b\}\} = \{\{c\}, \{c,d\}\}\] پس \((a,b) = (c,d)\).
\subsubsection{\texorpdfstring{جهت دوم
(\(\Rightarrow\))}{جهت دوم (\textbackslash Rightarrow)}}\label{ux62cux647ux62a-ux62fux648ux645-rightarrow}
فرض کنیم \(\{\{a\}, \{a,b\}\} = \{\{c\}, \{c,d\}\}\). دو حالت داریم:
\textbf{حالت ۱: \(a=b\)} در این صورت
\((a,b) = \{\{a\}, \{a,a\}\} = \{\{a\}, \{a\}\} = \{\{a\}\}\) (مجموعه تک
عضوی). چون دو طرف تساوی برابرند، طرف راست هم باید تک‌عضوی باشد:
\(\{\{c\}, \{c,d\}\} = \{\{a\}\}\). این یعنی
\(\{c\} = \{c,d\} = \{a\}\). پس \(c=a\) و \(d=c\). در نتیجه \(a=b=c=d\)
و حکم ثابت است .
\textbf{حالت ۲: \(a \neq b\)} در این صورت طرف چپ دو عضو دارد: \(\{a\}\)
و \(\{a,b\}\). پس طرف راست هم باید دو عضو داشته باشد (پس \(c \neq d\)).
تنها راه برابری دو مجموعه دو عضوی این است که اعضا نظیر به نظیر برابر
باشند. عضو \(\{a\}\) (تک‌عضوی) باید با عضو تک‌عضوی طرف مقابل یعنی
\(\{c\}\) برابر باشد. پس \(\{a\} = \{c\} \implies a=c\). عضو دوم
\(\{a,b\}\) باید با \(\{c,d\}\) برابر باشد. چون \(a=c\)، پس
\(\{a,b\} = \{a,d\}\). چون \(a \neq b\)، پس \(b\) باید با عضو دیگر این
مجموعه یعنی \(d\) برابر باشد. پس \(b=d\).
\textbf{نتیجه:} در هر دو حالت \(a=c\) و \(b=d\) ثابت شد .

\clearpage
% ---------------------------------------------------------------------
% Copyright (c) 2026 Arsalan Dalvand & Reyhaneh Darvishi.
% Licensed under CC BY-NC-SA 4.0.
% See LICENSE file for details.
% ---------------------------------------------------------------------

\section{\texorpdfstring{تمرین ۱: خاصیت وارونِ وارون
(\(\mathcal{R}^{-1})^{-1} = \mathcal{R}\))}{تمرین ۱: خاصیت وارونِ وارون (\textbackslash mathcal\{R\}\^{}\{-1\})\^{}\{-1\} = \textbackslash mathcal\{R\})}}\label{تمرین-۱---وارونِ-وارون-رابطه}
\begin{tldr}{خلاصه سریع}
اگر جای مؤلفه‌های یک رابطه را دو بار عوض کنیم، به حالت اول برمی‌گردد. مثل
اینکه یک لباس را پشت‌رو کنید و دوباره پشت‌رو کنید؛ به حالت اصلی برمی‌گردد.
\end{tldr}
\subsection{۱. صورت
سوال}\label{ux635ux648ux631ux62a-ux633ux648ux627ux644}
فرض کنید \(\mathcal{R}\) رابطه‌ای از \(A\) به \(B\) است. ثابت کنید:
\[(\mathcal{R}^{-1})^{-1} = \mathcal{R}\]
\subsection{۲. اثبات
دقیق}\label{ux627ux62bux628ux627ux62a-ux62fux642ux6ccux642}
برای اثبات تساوی دو مجموعه، نشان می‌دهیم عضو بودن در یکی معادل عضو بودن
در دیگری است.
\begin{info}{مراحل اثبات}
\[(x,y) \in \mathcal{R}\] طبق تعریف وارون رابطه
(\((x,y) \in \mathcal{R} \iff (y,x) \in \mathcal{R}^{-1}\)):
\[\iff (y,x) \in \mathcal{R}^{-1}\] حالا دوباره تعریف وارون را روی
\(\mathcal{R}^{-1}\) اعمال می‌کنیم:
\[\iff (x,y) \in (\mathcal{R}^{-1})^{-1}\] پس دو مجموعه برابرند.
\end{info}

\clearpage
% ---------------------------------------------------------------------
% Copyright (c) 2026 Arsalan Dalvand & Reyhaneh Darvishi.
% Licensed under CC BY-NC-SA 4.0.
% See LICENSE file for details.
% ---------------------------------------------------------------------

\section{تمرین ۲: محاسبه دامنه و برد یک رابطه
خاص}\label{تمرین-۲---محاسبه-دامنه-و-برد}
\subsection{۱. صورت
سوال}\label{ux635ux648ux631ux62a-ux633ux648ux627ux644}
فرض کنید \(A=\{a,b,c\}\) و رابطه \(\mathcal{R}\) به صورت زیر تعریف شده
است: \[\mathcal{R} = \{(a,c), (c,b), (a,b)\}\] حوزه \lr{(Domain) }و
نگاره \lr{(Image/Range) }رابطه \(\mathcal{R}\) را بیابید.
\subsection{۲. حل
تشریحی}\label{ux62dux644-ux62aux634ux631ux6ccux62dux6cc}
\begin{itemize}
\tightlist
\item
  \textbf{دامنه (\(Dom\)):} مجموعه مؤلفه‌های اول زوج‌های مرتب.
\item
  \textbf{برد (\(Im\)):} مجموعه مؤلفه‌های دوم زوج‌های مرتب.
\end{itemize}
\begin{note}{محاسبه}
\begin{enumerate}
\def\labelenumi{\arabic{enumi}.}
\tightlist
\item
  مؤلفه‌های اول: \(a\) (از \((a,c)\))، \(c\) (از \((c,b)\))، \(a\) (از
  \((a,b)\)).
  \begin{itemize}
  \tightlist
  \item
    \[Dom(\mathcal{R}) = \{a, c\}\]
  \end{itemize}
\item
  مؤلفه‌های دوم: \(c\) (از \((a,c)\))، \(b\) (از \((c,b)\))، \(b\) (از
  \((a,b)\)).
  \begin{itemize}
  \tightlist
  \item
    \[Im(\mathcal{R}) = \{b, c\}\]
  \end{itemize}
\end{enumerate}
\end{note}

\clearpage
% ---------------------------------------------------------------------
% Copyright (c) 2026 Arsalan Dalvand & Reyhaneh Darvishi.
% Licensed under CC BY-NC-SA 4.0.
% See LICENSE file for details.
% ---------------------------------------------------------------------

\section{تمرین ۳: جابجایی دامنه و برد در وارون
رابطه}\label{تمرین-۳---رابطه-دامنه-و-برد-با-وارون}
\begin{tldr}{خلاصه سریع}
وقتی رابطه را وارون می‌کنیم، جای ورودی‌ها و خروجی‌ها عوض می‌شود. پس دامنه
تبدیل به برد و برد تبدیل به دامنه می‌شود.
\end{tldr}
\subsection{۱. صورت
سوال}\label{ux635ux648ux631ux62a-ux633ux648ux627ux644}
ثابت کنید برای هر رابطه \(\mathcal{R}\): الف)
\(Dom(\mathcal{R}^{-1}) = Im(\mathcal{R})\) ب)
\(Im(\mathcal{R}^{-1}) = Dom(\mathcal{R})\)
\lr{[cite\_start][cite: }830-832{]}
\subsection{۲. اثبات}\label{ux627ux62bux628ux627ux62a}
\begin{info}{اثبات قسمت (الف)}
\[y \in Im(\mathcal{R})\]
\[\iff \exists x \in A, (x,y) \in \mathcal{R}\] (تعریف برد)
\[\iff \exists x \in A, (y,x) \in \mathcal{R}^{-1}\] (تعریف وارون)
\[\iff y \in Dom(\mathcal{R}^{-1})\] (تعریف دامنه برای
\(\mathcal{R}^{-1}\))
\end{info}
\begin{info}{اثبات قسمت (ب)}
\[x \in Dom(\mathcal{R})\]
\[\iff \exists y \in B, (x,y) \in \mathcal{R}\]
\[\iff \exists y \in B, (y,x) \in \mathcal{R}^{-1}\]
\[\iff x \in Im(\mathcal{R}^{-1})\]
\end{info}

\clearpage
% ---------------------------------------------------------------------
% Copyright (c) 2026 Arsalan Dalvand & Reyhaneh Darvishi.
% Licensed under CC BY-NC-SA 4.0.
% See LICENSE file for details.
% ---------------------------------------------------------------------

\section{تمرین ۴: بررسی بازتابی، تقارنی و تعدی (مثال
خاص)}\label{تمرین-۴---بررسی-خواص-یک-رابطه}
\subsection{۱. صورت
سوال}\label{ux635ux648ux631ux62a-ux633ux648ux627ux644}
فرض کنید \(A=\{a,b,c\}\) و:
\[\mathcal{R} = \{(a,a), (b,b), (c,c), (a,b), (b,a), (c,a), (a,c)\}\]
ثابت کنید \(\mathcal{R}\) بازتابی و متقارن است، اما متعدی \textbf{نیست}.
\subsection{۲. تحلیل و
حل}\label{ux62aux62dux644ux6ccux644-ux648-ux62dux644}
\begin{itemize}
\tightlist
\item
  \textbf{بازتابی \lr{(Reflexive):}} آیا هر کس با خودش رابطه دارد؟
  \begin{itemize}
  \tightlist
  \item
    بله، \((a,a), (b,b), (c,c)\) همگی در \(\mathcal{R}\) هستند.
    \lr{[cite\_start]پس }\textbf{بازتابی است}.
  \end{itemize}
\item
  \textbf{متقارن \lr{(Symmetric):}} آیا هر رفت، برگشت دارد؟
  \begin{itemize}
  \tightlist
  \item
    \((a,b) \in \mathcal{R} \to (b,a) \in \mathcal{R}\) (هست)
  \item
    \((c,a) \in \mathcal{R} \to (a,c) \in \mathcal{R}\) (هست)
  \item
    سایر اعضا روی قطر اصلی‌اند و متقارن.
    \lr{[cite\_start]پس }\textbf{متقارن است}.
  \end{itemize}
\item
  \textbf{تعدی \lr{(Transitive):}} آیا \(x \to y\) و \(y \to z\) نتیجه
  می‌دهد \(x \to z\)؟
  \begin{itemize}
  \tightlist
  \item
    مشاهده می‌کنیم \((c,a) \in \mathcal{R}\) و \((a,b) \in \mathcal{R}\).
  \item
    برای تعدی باید \((c,b) \in \mathcal{R}\) باشد.
  \item
    اما \((c,b)\) در مجموعه \(\mathcal{R}\) \textbf{نیست}.
  \item
    پس \textbf{متعدی نیست}.
  \end{itemize}
\end{itemize}

\clearpage
% ---------------------------------------------------------------------
% Copyright (c) 2026 Arsalan Dalvand & Reyhaneh Darvishi.
% Licensed under CC BY-NC-SA 4.0.
% See LICENSE file for details.
% ---------------------------------------------------------------------

\section{تمرین ۵: ساختن رابطه بازتابی و متعدی اما
غیرمتقارن}\label{تمرین-۵---مثال-بازتابی-و-متعدی-غیرمتقارن}
\subsection{۱. صورت
سوال}\label{ux635ux648ux631ux62a-ux633ux648ux627ux644}
رابطه‌ای مثال بزنید که انعکاسی (بازتابی) و متعدی باشد، اما متقارن نباشد.
\subsection{۲. حل}\label{ux62dux644}
رابطه «کوچکتر مساوی» (\(\le\)) یا «زیرمجموعه بودن» (\(\subseteq\))
بهترین مثال‌های ذهنی هستند. روی مجموعه \(A=\{a,b\}\):
\[\mathcal{R} = \{(a,a), (b,b), (a,b)\}\]
\begin{itemize}
\tightlist
\item
  \textbf{انعکاسی:} \((a,a), (b,b)\) دارد.
\item
  \textbf{متعدی:} تنها ترکیب ممکن \((a,a)\wedge(a,b)\to(a,b)\) است که
  برقرار است.
\item
  \textbf{غیرمتقارن:} \((a,b)\) هست ولی \((b,a)\) نیست.
\end{itemize}

\clearpage
% ---------------------------------------------------------------------
% Copyright (c) 2026 Arsalan Dalvand & Reyhaneh Darvishi.
% Licensed under CC BY-NC-SA 4.0.
% See LICENSE file for details.
% ---------------------------------------------------------------------

\section{تمرین ۶: ساختن رابطه متقارن و متعدی اما
غیربازتابی}\label{تمرین-۶---مثال-متقارن-و-متعدی-غیربازتابی}
\subsection{۱. صورت
سوال}\label{ux635ux648ux631ux62a-ux633ux648ux627ux644}
رابطه‌ای مثال بزنید که متقارن و متعدی باشد، اما انعکاسی نباشد.
\subsection{۲. حل}\label{ux62dux644}
روی مجموعه \(A=\{a,b\}\):
\[\mathcal{R} = \{(a,a), (a,b), (b,a), (b,b)\}\] (این هم‌ارزی کامل است).
اما اگر بخواهیم انعکاسی \textbf{نباشد}، باید حداقل یک عضو روی قطر اصلی
نباشد. مثال کتاب: \[\mathcal{R} = \{(a,a), (a,b), (b,a)\}\]
\begin{itemize}
\tightlist
\item
  \textbf{متقارن:} \((a,b)\) و \((b,a)\) هست. اگر
  \(\mathcal{R} = \{(a,a), (a,b), (b,a)\}\) باشد:
  \((b,a) \in \mathcal{R}\) و \((a,b) \in \mathcal{R}\) \(\Rightarrow\)
  باید \((b,b) \in \mathcal{R}\) باشد. اگر \((b,b)\) نباشد، متعدی نیست.
  \textbf{اصلاح مثال:} بهترین مثال برای متقارن و متعدی ولی غیرانعکاسی،
  \textbf{رابطه تهی} روی یک مجموعه ناتهی است! یا رابطه‌ای مثل
  \(\mathcal{R}=\{(a,a)\}\) روی \(A=\{a,b\}\) (چون \((b,b)\) ندارد،
  انعکاسی نیست. ولی روی خودش متقارن و متعدی است). \emph{تذکر:} مثال کتاب
  در منبع 856 احتمالا دارای اشکال تایپی است یا فرض کرده \((b,b)\) وجود
  دارد. اما مثال \(\mathcal{R}=\{(a,a)\}\) کاملا صحیح است.
\end{itemize}

\clearpage
% ---------------------------------------------------------------------
% Copyright (c) 2026 Arsalan Dalvand & Reyhaneh Darvishi.
% Licensed under CC BY-NC-SA 4.0.
% See LICENSE file for details.
% ---------------------------------------------------------------------

\section{تمرین ۷: شرط‌های لازم و کافی برای خواص
رابطه}\label{تمرین-۷---ویژگی‌های-رابطه-با-وارون}
\subsection{۱. صورت
سوال}\label{ux635ux648ux631ux62a-ux633ux648ux627ux644}
ثابت کنید برای رابطه \(\mathcal{R}\) روی \(X\): الف) \(\mathcal{R}\)
انعکاسی است \(\iff \Delta_X \subseteq \mathcal{R}\). ب) \(\mathcal{R}\)
متقارن است \(\iff \mathcal{R} = \mathcal{R}^{-1}\). ت) \(\mathcal{R}\)
انعکاسی است \(\iff \mathcal{R}^{-1}\) انعکاسی است. ث) \(\mathcal{R}\)
متقارن است \(\iff \mathcal{R}^{-1}\) متقارن است. ج) \(\mathcal{R}\)
متعدی است \(\iff \mathcal{R}^{-1}\) متعدی است. چ) \(\mathcal{R}\) هم‌ارزی
است \(\iff \mathcal{R}^{-1}\) هم‌ارزی است.
\lr{[cite\_start][cite: }857-861{]}
\subsection{۲. اثبات‌های
منتخب}\label{ux627ux62bux628ux627ux62aux647ux627ux6cc-ux645ux646ux62aux62eux628}
\begin{info}{اثبات (الف) - انعکاسی}
انعکاسی یعنی \(\forall x, (x,x) \in \mathcal{R}\). مجموعه قطری
\(\Delta_X = \{(x,x) \mid x \in X\}\). پس شرط انعکاسی دقیقاً یعنی
\(\Delta_X \subseteq \mathcal{R}\).
\end{info}
\begin{info}{اثبات (ب) - متقارن}
متقارن یعنی \((x,y) \in \mathcal{R} \iff (y,x) \in \mathcal{R}\). سمت
راست (\((y,x) \in \mathcal{R}\)) معادل است با
\((x,y) \in \mathcal{R}^{-1}\). پس شرط متقارن یعنی
\((x,y) \in \mathcal{R} \iff (x,y) \in \mathcal{R}^{-1}\)، که یعنی
\(\mathcal{R} = \mathcal{R}^{-1}\).
\end{info}
\begin{info}{اثبات (ج) - متعدی بودن وارون}
فرض کنیم \(\mathcal{R}\) متعدی است. می‌خواهیم ثابت کنیم
\(\mathcal{R}^{-1}\) متعدی است. فرض: \((x,y) \in \mathcal{R}^{-1}\) و
\((y,z) \in \mathcal{R}^{-1}\). معادل است با: \((y,x) \in \mathcal{R}\)
و \((z,y) \in \mathcal{R}\). جابجا می‌کنیم: \((z,y) \in \mathcal{R}\) و
\((y,x) \in \mathcal{R}\). چون \(\mathcal{R}\) متعدی است
\(\Rightarrow (z,x) \in \mathcal{R}\). وارون می‌گیریم
\(\Rightarrow (x,z) \in \mathcal{R}^{-1}\). پس \(\mathcal{R}^{-1}\)
متعدی است.
\end{info}

\clearpage
% ---------------------------------------------------------------------
% Copyright (c) 2026 Arsalan Dalvand & Reyhaneh Darvishi.
% Licensed under CC BY-NC-SA 4.0.
% See LICENSE file for details.
% ---------------------------------------------------------------------

\section{تمرین ۸ و ۹: شمارش تعداد رابطه‌های
ممکن}\label{تمرین-۸-و-۹---شمارش-رابطه‌ها}
\subsection{۱. تمرین ۸ (روی یک
مجموعه)}\label{ux62aux645ux631ux6ccux646-ux6f8-ux631ux648ux6cc-ux6ccux6a9-ux645ux62cux645ux648ux639ux647}
\textbf{سوال:} چند رابطه روی یک مجموعه \(n=7\) عضوی وجود دارد؟
\textbf{حل:} رابطه روی \(A\) یعنی زیرمجموعه‌ای از \(A \times A\). تعداد
اعضای \(A \times A\) برابر است با \(n \times n = n^2\). تعداد کل
زیرمجموعه‌ها (مجموعه توانی) برابر است با \(2^{\text{تعداد اعضا}}\). پس
تعداد رابطه‌ها برابر است با: \[2^{n^2} = 2^{7^2} = 2^{49}\]
\subsection{۲. تمرین ۹ (بین دو
مجموعه)}\label{ux62aux645ux631ux6ccux646-ux6f9-ux628ux6ccux646-ux62fux648-ux645ux62cux645ux648ux639ux647}
\textbf{سوال:} چند رابطه از مجموعه \(m\) عضوی به مجموعه \(n\) عضوی وجود
دارد؟ \textbf{حل:} رابطه یعنی زیرمجموعه‌ای از \(A \times B\). تعداد اعضای
\(A \times B\) برابر \(mn\) است. تعداد رابطه‌ها برابر تعداد
زیرمجموعه‌هاست: \[2^{mn}\]

\clearpage
% ---------------------------------------------------------------------
% Copyright (c) 2026 Arsalan Dalvand & Reyhaneh Darvishi.
% Licensed under CC BY-NC-SA 4.0.
% See LICENSE file for details.
% ---------------------------------------------------------------------

\section{\texorpdfstring{تمرین ۱۰: فرمول تحدید رابطه
(\(\mathcal{R}|D\))}{تمرین ۱۰: فرمول تحدید رابطه (\textbackslash mathcal\{R\}\textbar D)}}\label{تمرین-۱۰---تحدید-رابطه-(Restriction)}
\begin{tldr}{خلاصه سریع}
تحدید رابطه \(\mathcal{R}\) به زیرمجموعه \(D\) یعنی تمام فلش‌هایی از
رابطه اصلی را نگه داریم که \textbf{شروعشان} از \(D\) باشد.
\end{tldr}
\subsection{۱. صورت
سوال}\label{ux635ux648ux631ux62a-ux633ux648ux627ux644}
فرض کنید \(\mathcal{R}\) رابطه‌ای از \(A\) به \(B\) است و
\(D \subseteq A\). تحدید رابطه را با
\(\mathcal{R}|D = \{(x,y) \in \mathcal{R} \mid x \in D\}\) تعریف می‌کنیم.
ثابت کنید:
\[\mathcal{R}|D = \mathcal{R} \cap (D \times Im(\mathcal{R}))\]
\subsection{۲. اثبات}\label{ux627ux62bux628ux627ux62a}
\[(x,y) \in \mathcal{R} \cap (D \times Im(\mathcal{R}))\]
\[\iff (x,y) \in \mathcal{R} \wedge (x,y) \in D \times Im(\mathcal{R})\]
\[\iff (x,y) \in \mathcal{R} \wedge (x \in D \wedge y \in Im(\mathcal{R}))\]
نکته: اگر \((x,y) \in \mathcal{R}\) باشد، شرط \(y \in Im(\mathcal{R})\)
به خودی خود برقرار است (چون \(y\) یک تصویر است). پس می‌توان آن را حذف
کرد. \[\iff (x,y) \in \mathcal{R} \wedge x \in D\] این دقیقاً تعریف
\(\mathcal{R}|D\) است.

\clearpage
% ---------------------------------------------------------------------
% Copyright (c) 2026 Arsalan Dalvand & Reyhaneh Darvishi.
% Licensed under CC BY-NC-SA 4.0.
% See LICENSE file for details.
% ---------------------------------------------------------------------

\section{تمرین ۱۱: رفتار تحدید با اجتماع و
اشتراک}\label{تمرین-۱۱---تحدید-روی-اجتماع-و-اشتراک}
\subsection{۱. صورت
سوال}\label{ux635ux648ux631ux62a-ux633ux648ux627ux644}
ثابت کنید اگر \(D, E \subseteq A\): الف)
\(\mathcal{R}|(D \cup E) = (\mathcal{R}|D) \cup (\mathcal{R}|E)\) ب)
\(\mathcal{R}|(D \cap E) \subseteq (\mathcal{R}|D) \cap (\mathcal{R}|E)\)
\subsection{۲. اثبات
(الف)}\label{ux627ux62bux628ux627ux62a-ux627ux644ux641}
\[(x,y) \in \mathcal{R}|(D \cup E)\]
\[\iff (x,y) \in \mathcal{R} \wedge x \in (D \cup E)\]
\[\iff (x,y) \in \mathcal{R} \wedge (x \in D \vee x \in E)\]
\[\iff [(x,y) \in \mathcal{R} \wedge x \in D] \vee [(x,y) \in \mathcal{R} \wedge x \in E]\]
(پخش‌پذیری) \[\iff (x,y) \in \mathcal{R}|D \cup \mathcal{R}|E\]
\subsection{۳. اثبات (ب) - (در واقع تساوی برقرار
است)}\label{ux627ux62bux628ux627ux62a-ux628---ux62fux631-ux648ux627ux642ux639-ux62aux633ux627ux648ux6cc-ux628ux631ux642ux631ux627ux631-ux627ux633ux62a}
کتاب شمول یکطرفه خواسته، ولی مراحل اثبات نشان می‌دهد تساوی است:
\[(x,y) \in \mathcal{R}|(D \cap E) \iff (x,y) \in \mathcal{R} \wedge x \in D \cap E\]
\[\iff [(x,y) \in \mathcal{R} \wedge x \in D] \wedge [(x,y) \in \mathcal{R} \wedge x \in E]\]
\[\iff (x,y) \in (\mathcal{R}|D) \cap (\mathcal{R}|E)\]

\clearpage
% ---------------------------------------------------------------------
% Copyright (c) 2026 Arsalan Dalvand & Reyhaneh Darvishi.
% Licensed under CC BY-NC-SA 4.0.
% See LICENSE file for details.
% ---------------------------------------------------------------------

\section{\texorpdfstring{تمرین ۱۲: نگاره مجموعه تحت رابطه
(\(\mathcal{R}(X)\))}{تمرین ۱۲: نگاره مجموعه تحت رابطه (\textbackslash mathcal\{R\}(X))}}\label{تمرین-۱۲---نگاره-یا-تصویر-مجموعه-R(X)}
\begin{tldr}{مفهوم}
\(\mathcal{R}(X)\) یعنی مجموعه‌ای از تمام \(y\)هایی که با حداقل یک \(x\)
از \(X\) رابطه دارند. (شبیه تصویر \(f(X)\) در توابع).
\end{tldr}
\subsection{۱. صورت
سوال}\label{ux635ux648ux631ux62a-ux633ux648ux627ux644}
تعریف می‌کنیم
\(\mathcal{R}(X) = \{y \in B \mid \exists x \in X, (x,y) \in \mathcal{R}\}\).
ثابت کنید: الف)
\(\mathcal{R}(D \cup E) = \mathcal{R}(D) \cup \mathcal{R}(E)\) ب)
\(\mathcal{R}(D \cap E) \subseteq \mathcal{R}(D) \cap \mathcal{R}(E)\)
\subsection{۲. اثبات
(الف)}\label{ux627ux62bux628ux627ux62a-ux627ux644ux641}
\[y \in \mathcal{R}(D \cup E) \iff \exists x \in (D \cup E), (x,y) \in \mathcal{R}\]
\[\iff \exists x, (x \in D \vee x \in E) \wedge (x,y) \in \mathcal{R}\]
\[\iff (\exists x \in D, (x,y) \in \mathcal{R}) \vee (\exists x \in E, (x,y) \in \mathcal{R})\]
(پخش وجودی روی فصل) \[\iff y \in \mathcal{R}(D) \cup \mathcal{R}(E)\]
\subsection{۳. مثال نقض برای تساوی در
(ب)}\label{ux645ux62bux627ux644-ux646ux642ux636-ux628ux631ux627ux6cc-ux62aux633ux627ux648ux6cc-ux62fux631-ux628}
چرا در اشتراک تساوی نیست؟ فرض کنید \(\mathcal{R}=\{(x,y), (z,y)\}\).
\(D=\{x\}, E=\{z\}\).
\(D \cap E = \emptyset \Rightarrow \mathcal{R}(D \cap E) = \emptyset\).
اما \(\mathcal{R}(D)=\{y\}\) و \(\mathcal{R}(E)=\{y\}\). اشتراکشان
\(\{y\}\) است که تهی نیست.

\clearpage
% ---------------------------------------------------------------------
% Copyright (c) 2026 Arsalan Dalvand & Reyhaneh Darvishi.
% Licensed under CC BY-NC-SA 4.0.
% See LICENSE file for details.
% ---------------------------------------------------------------------

\section{تمرین ۱۳: بیان دامنه و برد بر حسب
نگاره}\label{تمرین-۱۳---رابطه-دامنه-و-نگاره-با-وارون}
\subsection{۱. صورت
سوال}\label{ux635ux648ux631ux62a-ux633ux648ux627ux644}
با تعاریف تمرین ۱۲ ثابت کنید: الف)
\(Dom(\mathcal{R}) = \mathcal{R}^{-1}(B)\) ب)
\(Im(\mathcal{R}) = \mathcal{R}(A)\)
\subsection{۲. اثبات}\label{ux627ux62bux628ux627ux62a}
\begin{info}{قسمت (الف)}
\(x \in \mathcal{R}^{-1}(B)\) (طبق تعریف نگاره برای وارون)
\(\iff \exists y \in B, (y,x) \in \mathcal{R}^{-1}\)
\(\iff \exists y \in B, (x,y) \in \mathcal{R}\)
\(\iff x \in Dom(\mathcal{R})\)
\end{info}
\begin{info}{قسمت (ب)}
\(y \in \mathcal{R}(A)\) (نگاره کل دامنه)
\(\iff \exists x \in A, (x,y) \in \mathcal{R}\)
\(\iff y \in Im(\mathcal{R})\)
\end{info}

\clearpage
% ---------------------------------------------------------------------
% Copyright (c) 2026 Arsalan Dalvand & Reyhaneh Darvishi.
% Licensed under CC BY-NC-SA 4.0.
% See LICENSE file for details.
% ---------------------------------------------------------------------

\section{تمرین ۱۴: دامنه و نگاره اجتماع دو
رابطه}\label{تمرین-۱۴---دامنه-و-برد-اجتماع-روابط}
\subsection{۱. صورت
سوال}\label{ux635ux648ux631ux62a-ux633ux648ux627ux644}
اگر \(\mathcal{R}\) و \(\mathcal{S}\) دو رابطه باشند، ثابت کنید: الف)
\(Dom(\mathcal{R} \cup \mathcal{S}) = Dom(\mathcal{R}) \cup Dom(\mathcal{S})\)
ب)
\(Im(\mathcal{R} \cup \mathcal{S}) = Im(\mathcal{R}) \cup Im(\mathcal{S})\)
\subsection{۲. اثبات
(الف)}\label{ux627ux62bux628ux627ux62a-ux627ux644ux641}
\[x \in Dom(\mathcal{R} \cup \mathcal{S})\]
\[\iff \exists y, (x,y) \in (\mathcal{R} \cup \mathcal{S})\]
\[\iff \exists y, [(x,y) \in \mathcal{R} \vee (x,y) \in \mathcal{S}]\]
\[\iff (\exists y, (x,y) \in \mathcal{R}) \vee (\exists y, (x,y) \in \mathcal{S})\]
\[\iff x \in Dom(\mathcal{R}) \cup Dom(\mathcal{S})\]

\clearpage
% ---------------------------------------------------------------------
% Copyright (c) 2026 Arsalan Dalvand & Reyhaneh Darvishi.
% Licensed under CC BY-NC-SA 4.0.
% See LICENSE file for details.
% ---------------------------------------------------------------------

\section{\texorpdfstring{تمرین ۱۵: بستار متقارن
(\(\mathcal{R} \cup \mathcal{R}^{-1}\))}{تمرین ۱۵: بستار متقارن (\textbackslash mathcal\{R\} \textbackslash cup \textbackslash mathcal\{R\}\^{}\{-1\})}}\label{تمرین-۱۵---بستار-متقارن-(Symmetric-Closure)}
\begin{tldr}{مفهوم}
اگر یک رابطه متقارن نباشد، با اضافه کردن وارونش به خودش، آن را متقارن
می‌کنیم. این «کوچکترین» رابطه متقارنی است که شامل رابطه اصلی است.
\end{tldr}
\subsection{۱. صورت
سوال}\label{ux635ux648ux631ux62a-ux633ux648ux627ux644}
ثابت کنید \(\mathcal{T} = \mathcal{R} \cup \mathcal{R}^{-1}\): ۱. یک
رابطه متقارن است. ۲. اگر \(\mathcal{S}\) هر رابطه متقارنی باشد که
\(\mathcal{R} \subseteq \mathcal{S}\)، آنگاه
\(\mathcal{T} \subseteq \mathcal{S}\) (یعنی \(\mathcal{T}\) کوچکترین
است).
\subsection{۲. اثبات}\label{ux627ux62bux628ux627ux62a}
\textbf{۱. متقارن بودن:}
\[(\mathcal{R} \cup \mathcal{R}^{-1})^{-1} = \mathcal{R}^{-1} \cup (\mathcal{R}^{-1})^{-1} = \mathcal{R}^{-1} \cup \mathcal{R} = \mathcal{R} \cup \mathcal{R}^{-1}\]
چون وارونش با خودش برابر شد، پس متقارن است.
\textbf{۲. کوچکترین بودن:} فرض کنیم \(\mathcal{S}\) متقارن باشد
(\(\mathcal{S}=\mathcal{S}^{-1}\)) و
\(\mathcal{R} \subseteq \mathcal{S}\). چون
\(\mathcal{R} \subseteq \mathcal{S}\)، پس
\(\mathcal{R}^{-1} \subseteq \mathcal{S}^{-1} = \mathcal{S}\). حالا چون
هم \(\mathcal{R}\) و هم \(\mathcal{R}^{-1}\) زیرمجموعه \(\mathcal{S}\)
هستند، اجتماعشان هم هست:
\[\mathcal{R} \cup \mathcal{R}^{-1} \subseteq \mathcal{S}\]

\clearpage
% ---------------------------------------------------------------------
% Copyright (c) 2026 Arsalan Dalvand & Reyhaneh Darvishi.
% Licensed under CC BY-NC-SA 4.0.
% See LICENSE file for details.
% ---------------------------------------------------------------------

\section{\texorpdfstring{تمرین ۱۶: بزرگترین زیرمجموعه متقارن
(\(\mathcal{R} \cap \mathcal{R}^{-1}\))}{تمرین ۱۶: بزرگترین زیرمجموعه متقارن (\textbackslash mathcal\{R\} \textbackslash cap \textbackslash mathcal\{R\}\^{}\{-1\})}}\label{تمرین-۱۶---درون‌هسته-متقارن}
\subsection{۱. صورت
سوال}\label{ux635ux648ux631ux62a-ux633ux648ux627ux644}
ثابت کنید \(\mathcal{W} = \mathcal{R} \cap \mathcal{R}^{-1}\): ۱. متقارن
است. ۲. اگر \(\mathcal{S}\) رابطه متقارنی باشد که
\(\mathcal{S} \subseteq \mathcal{R}\)، آنگاه
\(\mathcal{S} \subseteq \mathcal{W}\) (یعنی \(\mathcal{W}\) بزرگترین
است).
\subsection{۲. اثبات}\label{ux627ux62bux628ux627ux62a}
\textbf{۱. متقارن بودن:}
\[(\mathcal{R} \cap \mathcal{R}^{-1})^{-1} = \mathcal{R}^{-1} \cap (\mathcal{R}^{-1})^{-1} = \mathcal{R}^{-1} \cap \mathcal{R} = \mathcal{W}\]
\textbf{۲. بزرگترین بودن:} فرض کنیم
\(\mathcal{S} \subseteq \mathcal{R}\) و \(\mathcal{S}\) متقارن باشد
(\(\mathcal{S}=\mathcal{S}^{-1}\)). چون
\(\mathcal{S} \subseteq \mathcal{R}\)، پس
\(\mathcal{S}^{-1} \subseteq \mathcal{R}^{-1}\). چون
\(\mathcal{S} = \mathcal{S}^{-1}\)، پس
\(\mathcal{S} \subseteq \mathcal{R}^{-1}\). حال \(\mathcal{S}\)
زیرمجموعه هر دو است، پس زیرمجموعه اشتراک آن‌هاست:
\[\mathcal{S} \subseteq \mathcal{R} \cap \mathcal{R}^{-1}\]

\clearpage
% ---------------------------------------------------------------------
% Copyright (c) 2026 Arsalan Dalvand & Reyhaneh Darvishi.
% Licensed under CC BY-NC-SA 4.0.
% See LICENSE file for details.
% ---------------------------------------------------------------------

\section{تمرین ۱۷: پخش وارون روی اجتماع و
اشتراک}\label{تمرین-۱۷---وارونِ-اجتماع-و-اشتراک}
\subsection{۱. صورت
سوال}\label{ux635ux648ux631ux62a-ux633ux648ux627ux644}
ثابت کنید: الف)
\((\mathcal{R} \cup \mathcal{S})^{-1} = \mathcal{R}^{-1} \cup \mathcal{S}^{-1}\)
ب)
\((\mathcal{R} \cap \mathcal{S})^{-1} = \mathcal{R}^{-1} \cap \mathcal{S}^{-1}\)
\subsection{۲. اثبات
(الف)}\label{ux627ux62bux628ux627ux62a-ux627ux644ux641}
\[(x,y) \in (\mathcal{R} \cup \mathcal{S})^{-1}\]
\[\iff (y,x) \in \mathcal{R} \cup \mathcal{S}\]
\[\iff (y,x) \in \mathcal{R} \vee (y,x) \in \mathcal{S}\]
\[\iff (x,y) \in \mathcal{R}^{-1} \vee (x,y) \in \mathcal{S}^{-1}\]
\[\iff (x,y) \in \mathcal{R}^{-1} \cup \mathcal{S}^{-1}\]

\clearpage
% ---------------------------------------------------------------------
% Copyright (c) 2026 Arsalan Dalvand & Reyhaneh Darvishi.
% Licensed under CC BY-NC-SA 4.0.
% See LICENSE file for details.
% ---------------------------------------------------------------------

\section{تمرین ۱۸: رابطه هم‌ارزی برای
کسرها}\label{تمرین-۱۸---ساخت-اعداد-گویا-(رابطه-هم‌ارزی)}
\begin{tldr}{کاربرد}
این تمرین نحوه ساخت اعداد گویا (\(\mathbb{Q}\)) از اعداد صحیح
(\(\mathbb{Z}\)) را نشان می‌دهد. زوج \((a,b)\) همان کسر \(\frac{a}{b}\)
است و شرط \(ad=bc\) همان تساوی کسرهاست (\(\frac{a}{b}=\frac{c}{d}\)).
\end{tldr}
\subsection{۱. صورت
سوال}\label{ux635ux648ux631ux62a-ux633ux648ux627ux644}
مجموعه \(X = Z \times (Z - \{0\})\) را در نظر بگیرید. رابطه \(\sim\) را
چنین تعریف می‌کنیم: \[(a,b) \sim (c,d) \iff ad = bc\] ثابت کنید \(\sim\)
یک رابطه هم‌ارزی است.
\subsection{۲. اثبات}\label{ux627ux62bux628ux627ux62a}
\textbf{۱. انعکاسی:} آیا \((a,b) \sim (a,b)\)؟ بله، چون \(ab = ba\) (ضرب
جابجایی است). \lr{[cite: }992-994{]}
\textbf{۲. متقارن:} اگر \((a,b) \sim (c,d)\)، آیا \((c,d) \sim (a,b)\)؟
\[ad = bc \Rightarrow cb = da \Rightarrow (c,d) \sim (a,b)\]
\textbf{۳. متعدی:} فرض: \((a,b) \sim (c,d)\) و \((c,d) \sim (e,f)\).
یعنی \(ad = bc\) و \(cf = de\). طرفین را در هم ضرب می‌کنیم:
\((ad)(cf) = (bc)(de)\). \(c\) و \(d\) (که مخالف صفر است) را ساده
می‌کنیم؟ بهتر است ضرب کنیم: از اولی \(d = \frac{bc}{a}\) (نه در اعداد
صحیح نمی‌شود). راه کتاب: \(adf = (ad)f = (bc)f = b(cf) = b(de) = bde\).
پس \(adf = bde\). چون \(d \neq 0\) (طبق تعریف دامنه)، می‌توانیم \(d\) را
از طرفین حذف کنیم (قانون حذف در اعداد صحیح). \[\Rightarrow af = be\] که
یعنی \((a,b) \sim (e,f)\).

\clearpage
% ---------------------------------------------------------------------
% Copyright (c) 2026 Arsalan Dalvand & Reyhaneh Darvishi.
% Licensed under CC BY-NC-SA 4.0.
% See LICENSE file for details.
% ---------------------------------------------------------------------

\section{قضیه ۳: ویژگی‌های بنیادی کلاس‌های
هم‌ارزی}\label{قضیه-۳---ویژگی‌های-بنیادی-کلاس-هم‌ارزی}
\begin{tldr}{خلاصه سریع}
این قضیه هندسهٔ فضایی را که یک «رابطه هم‌ارزی» روی یک مجموعه ایجاد می‌کند،
ترسیم می‌نماید. طبق این قضیه، کلاس‌های هم‌ارزی یا کاملاً بر هم منطبق‌اند و یا
کاملاً از هم جدا \lr{(Disjoint) }هستند. این ویژگی، زیربنای مفهوم «افراز»
و ساختارهای جبری مانند گروه‌های خارج‌قسمتی است.
\end{tldr}
\subsection{۱. تعاریف و مفاهیم
پیش‌نیاز}\label{ux62aux639ux627ux631ux6ccux641-ux648-ux645ux641ux627ux647ux6ccux645-ux67eux6ccux634ux646ux6ccux627ux632}
پیش از ورود به اثبات، لازم است تعاریف صوری «کلاس هم‌ارزی» و «مجموعه
خارج‌قسمتی» را که در این قضیه محوریت دارند، مرور کنیم.
\subsubsection{\texorpdfstring{الف) کلاس هم‌ارزی
\lr{(Equivalence Class)}}{الف) کلاس هم‌ارزی }}\label{ux627ux644ux641-ux6a9ux644ux627ux633-ux647ux645ux627ux631ux632ux6cc-equivalence-class}
فرض کنید \(\xi\) یک رابطه هم‌ارزی روی مجموعه \(X\) باشد. برای هر عنصر
\(x \in X\)، کلاس هم‌ارزی \(x\) (که با \(x/\xi\) یا \([x]_\xi\) نمایش
داده می‌شود)، مجموعه‌ی تمام عناصری از \(X\) است که با \(x\) هم‌ارز هستند:
\[x/\xi = \{ y \in X \mid y \xi x \}\] در اینجا \(x\) را
\textbf{نماینده} \lr{(Representative) }کلاس می‌نامیم.
\subsubsection{\texorpdfstring{ب) مجموعه خارج‌قسمتی
\lr{(Quotient Set)}}{ب) مجموعه خارج‌قسمتی }}\label{ux628-ux645ux62cux645ux648ux639ux647-ux62eux627ux631ux62cux642ux633ux645ux62aux6cc-quotient-set}
مجموعه تمام کلاس‌های هم‌ارزی متمایز حاصل از رابطه \(\xi\) را مجموعه
خارج‌قسمتی \(X\) نسبت به \(\xi\) می‌نامند و با \(X/\xi\) نمایش می‌دهند:
\[X/\xi = \{ x/\xi \mid x \in X \}\]
\begin{center}\rule{0.5\linewidth}{0.5pt}\end{center}
\subsection{۲. متن ریاضی
قضیه}\label{ux645ux62aux646-ux631ux6ccux627ux636ux6cc-ux642ux636ux6ccux647}
فرض کنید \(\xi\) یک رابطه هم‌ارزی روی مجموعه ناتهی \(X\) باشد. احکام زیر
برقرارند:
\begin{theorembox}{قضیه ۳}
\textbf{الف) اصل ناتهی بودن:}
\[\forall x \in X, \quad x/\xi \neq \emptyset\] \textbf{ب) شرط تلاقی
(اشتراک):} \[x/\xi \cap y/\xi \neq \emptyset \iff x \xi y\] \textbf{ج)
اصل انطباق (تساوی کلاس‌ها):} \[x/\xi = y/\xi \iff x \xi y\]
\end{theorembox}
\begin{center}\rule{0.5\linewidth}{0.5pt}\end{center}
\subsection{\texorpdfstring{۳. اثبات صوری
\lr{(Formal Proof)}}{۳. اثبات صوری }}\label{ux627ux62bux628ux627ux62a-ux635ux648ux631ux6cc-formal-proof}
اثبات این قضیه کاربرد مستقیم ویژگی‌های سه‌گانه رابطه هم‌ارزی (بازتابی،
تقارنی، تعدی) در نظریه مجموعه‌هاست.
\subsubsection{اثبات قسمت
(الف)}\label{ux627ux62bux628ux627ux62a-ux642ux633ux645ux62a-ux627ux644ux641}
\begin{info}{برهان}
۱. طبق تعریف، \(\xi\) یک رابطه هم‌ارزی است، بنابراین دارای ویژگی
\textbf{بازتابی \lr{(Reflexive)}} است: \[\forall x \in X, (x \xi x)\] ۲.
طبق تعریف کلاس هم‌ارزی (\(y \in x/\xi \iff y \xi x\))، چون \(x \xi x\)
برقرار است، نتیجه می‌شود: \[x \in x/\xi\] ۳. چون حداقل خود عنصر \(x\) در
کلاسش وجود دارد، پس این مجموعه هرگز تهی نیست.
\end{info}
\subsubsection{اثبات قسمت
(ب)}\label{ux627ux62bux628ux627ux62a-ux642ux633ux645ux62a-ux628}
اثبات دوطرفه (\(\iff\)) انجام می‌شود.
\begin{info}{برهان}
\textbf{جهت رفت (\(\Rightarrow\)):} فرض کنیم اشتراک تهی نیست
(\(x/\xi \cap y/\xi \neq \emptyset\)). ۱. وجود دارد عنصری مانند \(z\) که
در هر دو کلاس باشد (\(z \in x/\xi \wedge z \in y/\xi\)). ۲. طبق تعریف
کلاس‌ها: \((z \xi x) \wedge (z \xi y)\). ۳. از ویژگی \textbf{تقارنی} برای
جمله اول استفاده می‌کنیم (\(z \xi x \implies x \xi z\)). 4. اکنون داریم:
\((x \xi z) \wedge (z \xi y)\). ۵. طبق ویژگی \textbf{تعدی}، نتیجه می‌شود:
\(x \xi y\).
\textbf{جهت برگشت (\(\Leftarrow\)):} فرض کنیم \(x \xi y\). ۱. طبق ویژگی
بازتابی، \(x \xi x\)، پس \(x \in x/\xi\). ۲. چون \(x \xi y\) (فرض) و
رابطه متقارن است، پس \(y \xi x\). طبق تعریف کلاس \(x\)، یعنی
\(y \in x/\xi\). (و همچنین \(x \in y/\xi\)). ۳. بنابراین عنصر \(x\) (و
حتی \(y\)) در هر دو کلاس حضور دارد. ۴. پس اشتراک ناتهی است
(\(x \in x/\xi \cap y/\xi\)).
\end{info}
\subsubsection{اثبات قسمت
(ج)}\label{ux627ux62bux628ux627ux62a-ux642ux633ux645ux62a-ux62c}
\begin{info}{برهان}
\textbf{جهت رفت (\(\Rightarrow\)):} فرض کنیم \(x/\xi = y/\xi\). ۱. طبق
قسمت (الف)، کلاس‌ها ناتهی هستند، پس اشتراک آن‌ها (\(x/\xi \cap y/\xi\))
نیز ناتهی است (چون با خود مجموعه‌ها برابر است). ۲. طبق قسمت (ب)، ناتهی
بودن اشتراک ایجاب می‌کند که \(x \xi y\).
\textbf{جهت برگشت (\(\Leftarrow\)):} فرض کنیم \(x \xi y\). باید ثابت
کنیم \(x/\xi = y/\xi\). (از روش شمول دوطرفه
\(A \subseteq B \wedge B \subseteq A\) استفاده می‌کنیم).
\begin{itemize}
\item
  \textbf{گام ۱ (\(x/\xi \subseteq y/\xi\)):} فرض کنیم \(z\) عضو دلخواهی
  از \(x/\xi\) باشد. پس \(z \xi x\). از طرفی طبق فرض \(x \xi y\). بنابر
  ویژگی \textbf{تعدی}: \((z \xi x \wedge x \xi y \implies z \xi y)\). پس
  \(z \in y/\xi\).
\item
  \textbf{گام ۲ (\(y/\xi \subseteq x/\xi\)):} فرض کنیم \(w\) عضو دلخواهی
  از \(y/\xi\) باشد. پس \(w \xi y\). چون \(x \xi y\)، طبق \textbf{تقارن}
  \(y \xi x\). بنابر ویژگی \textbf{تعدی}:
  \((w \xi y \wedge y \xi x \implies w \xi x)\). پس \(w \in x/\xi\).
\end{itemize}
\textbf{نتیجه:} دو مجموعه زیرمجموعه یکدیگرند، پس \(x/\xi = y/\xi\).
\end{info}
\begin{center}\rule{0.5\linewidth}{0.5pt}\end{center}
\subsection{\texorpdfstring{۴. شبکه ارتباطی با سایر قضایا
\lr{(Analytic Map)}}{۴. شبکه ارتباطی با سایر قضایا }}\label{ux634ux628ux6a9ux647-ux627ux631ux62aux628ux627ux637ux6cc-ux628ux627-ux633ux627ux6ccux631-ux642ux636ux627ux6ccux627-analytic-map}
این قضیه پل ارتباطی بین ``جبر رابطه‌ها'' و ``ساختار افراز'' است:
\subsubsection{\texorpdfstring{۱. ارتباط با
\autoref{پیشنیاز---مفاهیم-پایه-رابطه-و-تابع}}{۱. ارتباط با }}\label{ux627ux631ux62aux628ux627ux637-ux628ux627-ux67eux6ccux634ux646ux6ccux627ux632---ux645ux641ux627ux647ux6ccux645-ux67eux627ux6ccux647-ux631ux627ux628ux637ux647-ux648-ux62aux627ux628ux639}
\begin{itemize}
\tightlist
\item
  اثبات این قضیه تماماً بر پایه ویژگی‌های تعریف شده در پیش‌نیاز (بازتابی،
  تقارنی، تعدی) استوار است. این قضیه نشان می‌دهد که چگونه انتزاعی‌ترین
  ویژگی‌های یک رابطه، منجر به نتایج ملموس مجموعه‌ای (تساوی یا جدایی کامل)
  می‌شود.
\end{itemize}
\subsubsection{\texorpdfstring{۲. ارتباط با
\autoref{قضیه-۴---افراز-ناشی-از-رابطه-هم‌ارزی}}{۲. ارتباط با }}\label{ux627ux631ux62aux628ux627ux637-ux628ux627-ux642ux636ux6ccux647-ux6f4---ux627ux641ux631ux627ux632-ux646ux627ux634ux6cc-ux627ux632-ux631ux627ux628ux637ux647-ux647ux645ux627ux631ux632ux6cc}
\begin{itemize}
\tightlist
\item
  این قضیه (بخش‌های ب و ج) ``لم اصلی'' برای اثبات قضیه ۴ است. قضیه ۳
  تضمین می‌کند که کلاس‌ها ``دو به دو جدا'' \lr{(Mutually Disjoint) }هستند.
  بدون قضیه ۳، نمی‌توان ثابت کرد که \(X/\xi\) تشکیل یک افراز می‌دهد.
\end{itemize}
\subsubsection{\texorpdfstring{۳. ارتباط با
\autoref{مفهوم-پارادوکس-راسل} (مفهوم
خوش‌تعریفی)}{۳. ارتباط با  (مفهوم خوش‌تعریفی)}}\label{ux627ux631ux62aux628ux627ux637-ux628ux627-ux645ux641ux647ux648ux645-ux67eux627ux631ux627ux62fux648ux6a9ux633-ux631ux627ux633ux644-ux645ux641ux647ux648ux645-ux62eux648ux634ux62aux639ux631ux6ccux641ux6cc}
\begin{itemize}
\tightlist
\item
  در مباحث پیشرفته‌تر، وقتی روی کلاس‌های هم‌ارزی عملیاتی تعریف می‌کنیم (مثل
  جمع در \(\mathbb{Z}_n\))، بخش (ج) این قضیه
  (\(x/\xi = y/\xi \iff x \xi y\)) ابزار اصلی برای بررسی ``خوش‌تعریفی''
  \lr{(Well-definedness) }آن عملیات است؛ یعنی نتیجه عملیات نباید وابسته
  به نماینده کلاس (\(x\) یا \(y\)) باشد.
\end{itemize}

\clearpage
% ---------------------------------------------------------------------
% Copyright (c) 2026 Arsalan Dalvand & Reyhaneh Darvishi.
% Licensed under CC BY-NC-SA 4.0.
% See LICENSE file for details.
% ---------------------------------------------------------------------

\section{قضیه ۴: تولید افراز توسط رابطه
هم‌ارزی}\label{قضیه-۴---افراز-ناشی-از-رابطه-هم‌ارزی}
\begin{tldr}{خلاصه سریع}
این قضیه بیان می‌کند که هر «رابطه هم‌ارزی» به طور طبیعی مجموعه را تکه‌تکه
(افراز) می‌کند. به عبارت دیگر، کلاس‌های هم‌ارزی همان قطعات پازلی هستند که
بدون هم‌پوشانی، کل مجموعه را می‌سازند.
\end{tldr}
\subsection{۱. تعاریف بنیادین
(پیش‌نیاز)}\label{ux62aux639ux627ux631ux6ccux641-ux628ux646ux6ccux627ux62fux6ccux646-ux67eux6ccux634ux646ux6ccux627ux632}
برای درک عمیق این قضیه، باید تعریف دقیق \textbf{افراز} را مرور کنیم.
\subsubsection{\texorpdfstring{تعریف افراز
\lr{(Partition)}}{تعریف افراز }}\label{ux62aux639ux631ux6ccux641-ux627ux641ux631ux627ux632-partition}
فرض کنید \(X\) یک مجموعه ناتهی باشد. خانواده‌ای از زیرمجموعه‌های \(X\)
مانند \(\mathcal{P} = \{A_\gamma\}_{\gamma \in \Gamma}\) را یک
\textbf{افراز} برای \(X\) می‌نامیم اگر سه شرط زیر برقرار باشد:
\begin{enumerate}
\def\labelenumi{\arabic{enumi}.}
\tightlist
\item
  \textbf{ناتهی بودن:} هیچ‌کدام از مجموعه‌ها خالی نباشند
  (\(\forall A \in \mathcal{P}, A \neq \emptyset\)).
\item
  \textbf{مجزا بودن \lr{(Disjointness):}} هیچ دو مجموعه متمایزی اشتراک
  نداشته باشند (\(A_i \neq A_j \implies A_i \cap A_j = \emptyset\)).
\item
  \textbf{پوشش کامل \lr{(Covering):}} اجتماع تمام مجموعه‌ها برابر با کل
  \(X\) باشد (\(\bigcup A_\gamma = X\)).
\end{enumerate}
\begin{center}\rule{0.5\linewidth}{0.5pt}\end{center}
\subsection{۲. متن ریاضی
قضیه}\label{ux645ux62aux646-ux631ux6ccux627ux636ux6cc-ux642ux636ux6ccux647}
فرض کنید \(\xi\) یک رابطه هم‌ارزی روی مجموعه ناتهی \(X\) باشد. مجموعه
خارج‌قسمتی \(X/\xi\) (شامل تمام کلاس‌های هم‌ارزی) یک \textbf{افراز} برای
\(X\) است.
\begin{theorembox}{قضیه ۴}
اگر \(\xi\) یک رابطه هم‌ارزی روی \(X\) باشد، آنگاه \(X/\xi\) یک افراز
برای \(X\) است.
\end{theorembox}
\begin{center}\rule{0.5\linewidth}{0.5pt}\end{center}
\subsection{\texorpdfstring{۳. اثبات صوری
\lr{(Formal Proof)}}{۳. اثبات صوری }}\label{ux627ux62bux628ux627ux62a-ux635ux648ux631ux6cc-formal-proof}
برای اثبات اینکه \(X/\xi\) یک افراز است، باید نشان دهیم سه شرط تعریف
افراز (بخش ۱) را دارد. ما از
\textbf{\autoref{قضیه-۳---ویژگی‌های-بنیادی-کلاس-هم‌ارزی}} به عنوان ابزار
اصلی استفاده می‌کنیم.
\begin{info}{برهان}
\textbf{گام ۱: بررسی شرط ناتهی بودن} طبق
\autoref{قضیه-۳---ویژگی‌های-بنیادی-کلاس-هم‌ارزی}، برای هر \(x \in X\)،
کلاس \(x/\xi\) یک زیرمجموعه ناتهی از \(X\) است (حداقل شامل خود \(x\)
است).
\textbf{گام ۲: بررسی شرط مجزا بودن} باید ثابت کنیم:
\(x/\xi \neq y/\xi \implies (x/\xi) \cap (y/\xi) = \emptyset\). از برهان
خلف استفاده می‌کنیم (یا عکس نقیض قضیه ۳-ب): طبق
\autoref{قضیه-۳---ویژگی‌های-بنیادی-کلاس-هم‌ارزی}، اگر اشتراک دو کلاس تهی
نباشد (\(x/\xi \cap y/\xi \neq \emptyset\))، آنگاه \(x \xi y\). طبق
\autoref{قضیه-۳---ویژگی‌های-بنیادی-کلاس-هم‌ارزی}، اگر \(x \xi y\)، آنگاه
\(x/\xi = y/\xi\). بنابراین، اگر دو کلاس برابر نباشند، محال است اشتراک
داشته باشند.
\textbf{گام ۳: بررسی شرط پوشش} باید ثابت کنیم
\(\bigcup_{x \in X} x/\xi = X\).
\begin{itemize}
\tightlist
\item
  (شمول \(\subseteq\)): واضح است که هر کلاس زیرمجموعه \(X\) است، پس
  اجتماعشان هم زیرمجموعه \(X\) است.
\item
  (شمول \(\supseteq\)): هر عضو دلخواه \(y \in X\) را در نظر بگیرید.
  می‌دانیم هر عضو متعلق به کلاس خودش است (\(y \in y/\xi\)). پس \(y\) در
  اجتماع کلاس‌ها حضور دارد.
\end{itemize}
\textbf{نتیجه:} هر سه شرط برقرار است، پس \(X/\xi\) یک افراز است.
\end{info}
\begin{center}\rule{0.5\linewidth}{0.5pt}\end{center}
\subsection{\texorpdfstring{۴. شبکه ارتباطی با سایر قضایا
\lr{(Analytic Map)}}{۴. شبکه ارتباطی با سایر قضایا }}\label{ux634ux628ux6a9ux647-ux627ux631ux62aux628ux627ux637ux6cc-ux628ux627-ux633ux627ux6ccux631-ux642ux636ux627ux6ccux627-analytic-map}
\subsubsection{\texorpdfstring{۱. وابستگی به
\autoref{قضیه-۳---ویژگی‌های-بنیادی-کلاس-هم‌ارزی}}{۱. وابستگی به }}\label{ux648ux627ux628ux633ux62aux6afux6cc-ux628ux647-ux642ux636ux6ccux647-ux6f3---ux648ux6ccux698ux6afux6ccux647ux627ux6cc-ux628ux646ux6ccux627ux62fux6cc-ux6a9ux644ux627ux633-ux647ux645ux627ux631ux632ux6cc}
\begin{itemize}
\tightlist
\item
  این قضیه بدون قضیه ۳ قابل اثبات نیست. قضیه ۳ ``لم'' \lr{(Lemma) }یا
  ابزار تکنیکی است که تضمین می‌کند کلاس‌ها یا ``یکی هستند'' یا ``کاملاً
  جدا''. این خاصیت دوگانه \lr{(Dichotomy) }جوهره‌ی اصلی افراز است.
\end{itemize}
\subsubsection{\texorpdfstring{۲. هم‌ارزی با
\autoref{قضیه-۵---رابطه-هم‌ارزی-ناشی-از-افراز}}{۲. هم‌ارزی با }}\label{ux647ux645ux627ux631ux632ux6cc-ux628ux627-ux642ux636ux6ccux647-ux6f5---ux631ux627ux628ux637ux647-ux647ux645ux627ux631ux632ux6cc-ux646ux627ux634ux6cc-ux627ux632-ux627ux641ux631ux627ux632}
\begin{itemize}
\tightlist
\item
  قضیه ۴ مسیر ``رابطه \(\to\) افراز'' را طی می‌کند. قضیه ۵ مسیر عکس آن،
  یعنی ``افراز \(\to\) رابطه'' را طی می‌کند. این دو قضیه با هم نشان
  می‌دهند که مفاهیم ``رابطه هم‌ارزی'' و ``افراز'' از نظر منطقی
  \textbf{هم‌ارز} \lr{(Equivalent) }هستند (دو روی یک سکه).
\end{itemize}
\subsubsection{۳. کاربرد در جبر (فصل‌های
آینده)}\label{ux6a9ux627ux631ux628ux631ux62f-ux62fux631-ux62cux628ux631-ux641ux635ux644ux647ux627ux6cc-ux622ux6ccux646ux62fux647}
\begin{itemize}
\tightlist
\item
  در جبر مجرد، این قضیه زیربنای ساختن \(Z_n\) (اعداد صحیح به پیمانه
  \(n\)) است. رابطه ``همنهشتی'' اعداد را افراز می‌کند و ما با این قطعات
  افراز (کلاس‌ها) به عنوان اعداد جدید کار می‌کنیم.
\end{itemize}

\clearpage
% ---------------------------------------------------------------------
% Copyright (c) 2026 Arsalan Dalvand & Reyhaneh Darvishi.
% Licensed under CC BY-NC-SA 4.0.
% See LICENSE file for details.
% ---------------------------------------------------------------------

\section{قضیه ۵: تولید رابطه هم‌ارزی توسط
افراز}\label{قضیه-۵---رابطه-هم‌ارزی-ناشی-از-افراز}
\begin{tldr}{خلاصه سریع}
این قضیه عکس قضیه ۴ است: اگر شما مجموعه‌ای را تکه‌تکه (افراز) کنید، به طور
خودکار یک رابطه هم‌ارزی ساخته‌اید. تعریف رابطه این است: «دو نفر با هم
رابطه دارند، اگر و تنها اگر در یک تکه باشند».
\end{tldr}
\subsection{۱. تعریف رابطه ناشی از
افراز}\label{ux62aux639ux631ux6ccux641-ux631ux627ux628ux637ux647-ux646ux627ux634ux6cc-ux627ux632-ux627ux641ux631ux627ux632}
فرض کنید \(\mathcal{P}\) یک افراز برای مجموعه \(X\) باشد. رابطه هم‌ارزی
متناظر با آن (که با \(X/\mathcal{P}\) نشان داده می‌شود) چنین تعریف می‌شود:
\[x (X/\mathcal{P}) y \iff \exists A \in \mathcal{P} : (x \in A \wedge y \in A)\]
\emph{(ترجمه: \(x\) و \(y\) با هم رابطه دارند اگر هم‌گروهی باشند).}
\begin{center}\rule{0.5\linewidth}{0.5pt}\end{center}
\subsection{۲. متن ریاضی
قضیه}\label{ux645ux62aux646-ux631ux6ccux627ux636ux6cc-ux642ux636ux6ccux647}
\begin{theorembox}{قضیه ۵}
اگر \(\mathcal{P}\) یک افراز برای مجموعه ناتهی \(X\) باشد، آنگاه:
\textbf{الف)} رابطه \(X/\mathcal{P}\) یک \textbf{رابطه هم‌ارزی} روی \(X\)
است. \textbf{ب)} کلاس‌های هم‌ارزی حاصل از این رابطه، دقیقاً همان مجموعه‌های
افراز هستند (\(X/(X/\mathcal{P}) = \mathcal{P}\)).
\end{theorembox}
\begin{center}\rule{0.5\linewidth}{0.5pt}\end{center}
\subsection{\texorpdfstring{۳. اثبات صوری
\lr{(Formal Proof)}}{۳. اثبات صوری }}\label{ux627ux62bux628ux627ux62a-ux635ux648ux631ux6cc-formal-proof}
\subsubsection{اثبات قسمت (الف): هم‌ارزی
بودن}\label{ux627ux62bux628ux627ux62a-ux642ux633ux645ux62a-ux627ux644ux641-ux647ux645ux627ux631ux632ux6cc-ux628ux648ux62fux646}
باید سه ویژگی بازتابی، تقارنی و تعدی را برای رابطه تعریف شده چک کنیم.
\begin{info}{برهان}
\textbf{۱. بازتابی \lr{(Reflexive):}} هر \(x \in X\)، طبق تعریف افراز
(پوشش کامل)، حتماً متعلق به یکی از مجموعه‌های افراز (مثلاً \(A\)) است. پس
\(x, x \in A\) و در نتیجه \(x \sim x\).
\textbf{۲. تقارنی \lr{(Symmetric):}} اگر \(x \sim y\)، یعنی مجموعه‌ای مثل
\(A\) هست که \(x, y \in A\). واضح است که \(y, x \in A\) نیز برقرار است.
پس \(y \sim x\).
\textbf{۳. تعدی \lr{(Transitive):}} فرض کنیم \(x \sim y\) و
\(y \sim z\).
\begin{itemize}
\tightlist
\item
  \(x \sim y \implies \exists A \in \mathcal{P}, (x, y \in A)\)
\item
  \(y \sim z \implies \exists B \in \mathcal{P}, (y, z \in B)\)
\item
  اکنون \(y\) هم در \(A\) است و هم در \(B\). پس \(y \in A \cap B\).
\item
  یعنی اشتراک \(A\) و \(B\) تهی نیست (\(A \cap B \neq \emptyset\)).
\item
  طبق تعریف افراز (شرط مجزا بودن)، اگر دو مجموعه اشتراک داشته باشند،
  باید \textbf{یکی} باشند. پس \(A = B\).
\item
  نتیجه: \(x, z\) هر دو در \(A\) هستند، پس \(x \sim z\).
\end{itemize}
\end{info}
\subsubsection{اثبات قسمت (ب): بازگشت به افراز
اولیه}\label{ux627ux62bux628ux627ux62a-ux642ux633ux645ux62a-ux628-ux628ux627ux632ux6afux634ux62a-ux628ux647-ux627ux641ux631ux627ux632-ux627ux648ux644ux6ccux647}
\begin{info}{برهان}
می‌خواهیم نشان دهیم کلاس‌های هم‌ارزی تولید شده، همان مجموعه‌های
\(A \in \mathcal{P}\) هستند. فرض کنیم \(x \in X\) باشد و \(A\) آن
مجموعه‌ای از افراز باشد که شامل \(x\) است (\(x \in A\)). طبق تعریف رابطه،
تمام هم‌گروهی‌های \(x\) (یعنی اعضای کلاس \([x]\)) دقیقاً همان اعضای \(A\)
هستند. پس \([x] = A\). بنابراین مجموعه تمام کلاس‌ها، همان مجموعه
\(\mathcal{P}\) است.
\end{info}
\begin{center}\rule{0.5\linewidth}{0.5pt}\end{center}
\subsection{\texorpdfstring{۴. شبکه ارتباطی با سایر قضایا
\lr{(Analytic Map)}}{۴. شبکه ارتباطی با سایر قضایا }}\label{ux634ux628ux6a9ux647-ux627ux631ux62aux628ux627ux637ux6cc-ux628ux627-ux633ux627ux6ccux631-ux642ux636ux627ux6ccux627-analytic-map}
\subsubsection{\texorpdfstring{۱. چرخه کامل با
\autoref{قضیه-۴---افراز-ناشی-از-رابطه-هم‌ارزی}}{۱. چرخه کامل با }}\label{ux686ux631ux62eux647-ux6a9ux627ux645ux644-ux628ux627-ux642ux636ux6ccux647-ux6f4---ux627ux641ux631ux627ux632-ux646ux627ux634ux6cc-ux627ux632-ux631ux627ux628ux637ux647-ux647ux645ux627ux631ux632ux6cc}
\begin{itemize}
\tightlist
\item
  ترکیب قضیه ۴ و ۵ یک چرخه بسته \lr{(Bijection) }بین ``مجموعه تمام
  رابطه‌های هم‌ارزی روی \(X\)'' و ``مجموعه تمام افرازهای \(X\)'' ایجاد
  می‌کند.
  \begin{itemize}
  \tightlist
  \item
    از رابطه به افراز: \(\xi \to X/\xi\)
  \item
    از افراز به رابطه: \(\mathcal{P} \to X/\mathcal{P}\)
  \item
    قضیه ۵-ب تضمین می‌کند که اگر یک دور کامل بزنیم، به جای اول برمی‌گردیم:
    \(X/(X/\mathcal{P}) = \mathcal{P}\).
  \end{itemize}
\end{itemize}
\subsubsection{\texorpdfstring{۲. مثال کاربردی:
\autoref{تمرین-۵---هم‌نهشتی-اعداد-صحیح}}{۲. مثال کاربردی: }}\label{ux645ux62bux627ux644-ux6a9ux627ux631ux628ux631ux62fux6cc-ux62aux645ux631ux6ccux646-ux6f5---ux647ux645ux646ux647ux634ux62aux6cc-ux627ux639ux62fux627ux62f-ux635ux62dux6ccux62d}
\begin{itemize}
\tightlist
\item
  در تمرین ۵ فصل ۳، دیدیم که رابطه همنهشتی به پیمانه \(n\)، مجموعه اعداد
  صحیح را به \(n\) کلاس افراز می‌کند. این مثال عملی دقیقاً مصداق همین دو
  قضیه است. دسته‌بندی اعداد (افراز) \(\iff\) رابطه همنهشتی.
\end{itemize}

\clearpage
% ---------------------------------------------------------------------
% Copyright (c) 2026 Arsalan Dalvand & Reyhaneh Darvishi.
% Licensed under CC BY-NC-SA 4.0.
% See LICENSE file for details.
% ---------------------------------------------------------------------

\section{تمرین ۱: بررسی افراز و رابطه
نظیر}\label{تمرین-۱---بررسی-افراز-و-رابطه-نظیر}
\subsection{۱. صورت
سوال}\label{ux635ux648ux631ux62a-ux633ux648ux627ux644}
\begin{info}{فرض کنید \(X=\{a,b,c,d,e\}\)، \(A=\{a,b\}\) و \(B=\{c,d,e\}\). خانواده مجموعه‌های \(\{A,B\}\) را در نظر بگیرید.}
\textbf{الف)} تحقیق کنید که آیا \(\{A,B\}\) یک افراز برای \(X\) است؟
\textbf{ب)} رابطه هم‌ارزی حاصل از این افراز (که با \(X/\{A,B\}\) نمایش
داده می‌شود) را بنویسید .
\end{info}
\subsection{۲. استراتژی
حل}\label{ux627ux633ux62aux631ux627ux62aux698ux6cc-ux62dux644}
برای حل قسمت (الف) باید سه شرط تعریف افراز را چک کنیم:
\begin{enumerate}
\def\labelenumi{\arabic{enumi}.}
\tightlist
\item
  مجموعه‌ها تهی نباشند.
\item
  اشتراک هر دو مجموعه متمایز، تهی باشد (دو به دو جدا).
\item
  اجتماع کل مجموعه‌ها برابر با مجموعه اصلی \(X\) شود.
\end{enumerate}
برای حل قسمت (ب)، از این اصل استفاده می‌کنیم که هر افراز، یک
\textbf{\autoref{قضیه-۵---رابطه-هم‌ارزی-ناشی-از-افراز}} تولید می‌کند که در
آن: \[x \sim y \iff \text{x و y در یک خانه از افراز باشند}\] بنابراین
رابطه کل برابر است با اتحادِ حاصل‌ضرب دکارتی هر مجموعه در خودش:
\[R = (A \times A) \cup (B \times B)\]
\subsection{۳. حل
تشریحی}\label{ux62dux644-ux62aux634ux631ux6ccux62dux6cc}
\begin{info}{پاسخ}
\textbf{قسمت (الف): بررسی شرایط افراز} ۱. \textbf{شرط ناتهی بودن:} واضح
است که \(A \neq \emptyset\) و \(B \neq \emptyset\). ۲. \textbf{شرط جدا
بودن:} \(A\) و \(B\) هیچ عضو مشترکی ندارند، پس \(A \cap B = \emptyset\).
۳. \textbf{شرط پوشش:} اجتماع آنها را حساب می‌کنیم:
\[A \cup B = \{a,b\} \cup \{c,d,e\} = \{a,b,c,d,e\} = X\] چون هر سه شرط
برقرار است، پس \(\{A,B\}\) یک \textbf{افراز} برای \(X\) است.
\textbf{قسمت (ب): یافتن رابطه هم‌ارزی} طبق قضیه متناظر بین افراز و رابطه
هم‌ارزی، رابطه \(R\) برابر است با اجتماع روابط روی هر قطعه افراز:
\[X/\{A,B\} = (A \times A) \cup (B \times B)\]
محاسبه \(A \times A\) (زوج‌های ساخته شده از \(a,b\)):
\[\{(a,a), (a,b), (b,a), (b,b)\}\]
محاسبه \(B \times B\) (زوج‌های ساخته شده از \(c,d,e\)):
\[\{(c,c), (c,d), (c,e), (d,c), (d,d), (d,e), (e,c), (e,d), (e,e)\}\]
\textbf{رابطه نهایی:} اجتماع دو مجموعه بالا خواهد بود.
\end{info}

\clearpage
% ---------------------------------------------------------------------
% Copyright (c) 2026 Arsalan Dalvand & Reyhaneh Darvishi.
% Licensed under CC BY-NC-SA 4.0.
% See LICENSE file for details.
% ---------------------------------------------------------------------

\section{تمرین ۲: استخراج افراز از رابطه
هم‌ارزی}\label{تمرین-۲---استخراج-افراز-از-رابطه}
\subsection{۱. صورت
سوال}\label{ux635ux648ux631ux62a-ux633ux648ux627ux644}
\begin{info}{فرض کنید \(X=\{a,b,c,d\}\) و رابطه \(\mathcal{R}\) به صورت زیر داده شده است:}
\[\mathcal{R}=\{(a,b),(b,a),(a,a),(b,b),(c,d),(d,c),(c,c),(d,d)\}\]
\textbf{الف)} تحقیق کنید که \(\mathcal{R}\) یک رابطه هم‌ارزی روی \(X\)
است. \textbf{ب)} افراز \(X/\mathcal{R}\) را که از \(\mathcal{R}\) پدید
آمده است، بیابید.
\end{info}
\subsection{۲. استراتژی
حل}\label{ux627ux633ux62aux631ux627ux62aux698ux6cc-ux62dux644}
برای (الف) باید سه ویژگی رابطه هم ارزی را روی زوج‌مرتب‌ها چک کنیم:
\begin{enumerate}
\def\labelenumi{\arabic{enumi}.}
\tightlist
\item
  \textbf{بازتابی:} آیا \((x,x)\) برای همه اعضا هست؟
\item
  \textbf{تقارنی:} آیا هر جا \((x,y)\) هست، \((y,x)\) هم هست؟
\item
  \textbf{تعدی:} آیا زنجیره‌های \((x,y)\) و \((y,z)\) به \((x,z)\) ختم
  می‌شوند؟
\end{enumerate}
برای (ب) باید \textbf{\autoref{قضیه-۳---ویژگی‌های-بنیادی-کلاس-هم‌ارزی}} هر
عضو را پیدا کنیم (\([x] = \{y \mid x\mathcal{R}y\}\)). مجموعه این
کلاس‌های متمایز، افراز مطلوب است.
\subsection{۳. حل
تشریحی}\label{ux62dux644-ux62aux634ux631ux6ccux62dux6cc}
\begin{info}{پاسخ}
\textbf{قسمت (الف): بررسی ویژگی‌ها} ۱. \textbf{بازتابی:} اعضای \(X\)
عبارتند از \(a,b,c,d\). در رابطه \(\mathcal{R}\) زوج‌های
\((a,a), (b,b), (c,c), (d,d)\) وجود دارند. \(\checkmark\) ۲.
\textbf{تقارنی:}
\begin{itemize}
\tightlist
\item
  برای \((a,b)\)، معکوسش \((b,a)\) وجود دارد.
\item
  برای \((c,d)\)، معکوسش \((d,c)\) وجود دارد.
\item
  زوج‌های قطری (مثل \(a,a\)) متقارن خودشان هستند. \(\checkmark\) ۳.
  \textbf{تعدی:}
\item
  ترکیب \((a,b)\) و \((b,a)\) می‌دهد \((a,a)\) که موجود است.
\item
  ترکیب \((c,d)\) و \((d,c)\) می‌دهد \((c,c)\) که موجود است.
\item
  سایر ترکیب‌ها هم بررسی می‌شوند و برقرارند. \(\checkmark\) پس
  \(\mathcal{R}\) یک رابطه هم‌ارزی است .
\end{itemize}
\textbf{قسمت (ب): یافتن افراز (کلاس‌های هم‌ارزی)} باید ببینیم هر عضو با چه
کسانی رابطه دارد:
\begin{itemize}
\tightlist
\item
  کلاس \(a\): تمام عناصری که با \(a\) در رابطه‌اند
  \(\to [a]_\mathcal{R} = \{a, b\}\)
\item
  کلاس \(b\): تمام عناصری که با \(b\) در رابطه‌اند
  \(\to [b]_\mathcal{R} = \{a, b\}\) (تکراری)
\item
  کلاس \(c\): تمام عناصری که با \(c\) در رابطه‌اند
  \(\to [c]_\mathcal{R} = \{c, d\}\)
\item
  کلاس \(d\): تمام عناصری که با \(d\) در رابطه‌اند
  \(\to [d]_\mathcal{R} = \{c, d\}\) (تکراری)
\end{itemize}
بنابراین افراز نهایی شامل کلاس‌های متمایز است:
\[X/\mathcal{R} = \{ \{a,b\}, \{c,d\} \}\]
\end{info}

\clearpage
% ---------------------------------------------------------------------
% Copyright (c) 2026 Arsalan Dalvand & Reyhaneh Darvishi.
% Licensed under CC BY-NC-SA 4.0.
% See LICENSE file for details.
% ---------------------------------------------------------------------

\section{تمرین ۳: بازگشت از افراز به رابطه
(تناظر)}\label{تمرین-۳---تناظر-یک‌به‌یک-افراز-و-رابطه}
\subsection{۱. صورت
سوال}\label{ux635ux648ux631ux62a-ux633ux648ux627ux644}
\begin{info}{فرض کنید افراز \(X/\mathcal{P}\) همان افراز به‌دست آمده در مسئله ۲ باشد (یعنی \(\{ \{a,b\}, \{c,d\} \}\)).}
رابطه هم‌ارزی متناظر با این افراز را روی \(X\) بیابید.
\end{info}
\subsection{۲. استراتژی
حل}\label{ux627ux633ux62aux631ux627ux62aux698ux6cc-ux62dux644}
این تمرین نکته ظریفی دارد: در تمرین ۲، ما از رابطه \(\mathcal{R}\) به
افراز رسیدیم. حالا می‌خواهد از افراز دوباره رابطه را بسازیم. طبق
\textbf{قضیه اصلی رابطه‌های هم‌ارزی}، تناظر بین افرازها و روابط هم‌ارزی
یک‌به‌یک است. یعنی اگر از یک رابطه افراز بگیریم و از آن افراز دوباره رابطه
بسازیم، باید به همان رابطه اولیه برسیم.
\subsection{۳. حل
تشریحی}\label{ux62dux644-ux62aux634ux631ux6ccux62dux6cc}
\begin{info}{پاسخ}
افراز داده شده \(\mathcal{P} = \{ \{a,b\}, \{c,d\} \}\) است. رابطه
متناظر با این افراز (\(R_\mathcal{P}\)) شامل تمام زوج‌هایی است که
مولفه‌هایشان در ``یک بسته'' قرار دارند.
۱. \textbf{بسته اول \(\{a,b\}\):} حاصل‌ضرب دکارتی این بسته در خودش:
\[\{(a,a), (a,b), (b,a), (b,b)\}\]
۲. \textbf{بسته دوم \(\{c,d\}\):} حاصل‌ضرب دکارتی این بسته در خودش:
\[\{(c,c), (c,d), (d,c), (d,d)\}\]
\textbf{نتیجه:} رابطه نهایی اجتماع این دو مجموعه است:
\[\{(a,a), (b,b), (a,b), (b,a), (c,c), (d,d), (c,d), (d,c)\}\] مشاهده
می‌کنیم که این دقیقاً همان رابطه \(\mathcal{R}\) در تمرین ۲ است.
\end{info}

\clearpage
% ---------------------------------------------------------------------
% Copyright (c) 2026 Arsalan Dalvand & Reyhaneh Darvishi.
% Licensed under CC BY-NC-SA 4.0.
% See LICENSE file for details.
% ---------------------------------------------------------------------

\section{تمرین ۴: افراز سه قسمتی و تعیین
کلاس‌ها}\label{تمرین-۴---افراز-سه-قسمتی}
\subsection{۱. صورت
سوال}\label{ux635ux648ux631ux62a-ux633ux648ux627ux644}
\begin{info}{فرض کنید \(X=\{a,b,c,d,e\}\) و \(\mathcal{P}=\{\{a,b\},\{c\},\{d,e\}\}\) باشد.}
\textbf{الف)} نشان دهید که \(\mathcal{P}\) یک افراز \(X\) است.
\textbf{ب)} رابطه هم‌ارزی \(X/\mathcal{P}\) را به صورت مجموعه جفت‌های مرتب
مشخص کنید. \textbf{پ)} مجموعه خارج‌قسمتی \(X/\mathcal{P}\) را
\(\mathcal{E}\) بنامید و کلاس‌های \(a/\mathcal{E}\)، \(c/\mathcal{E}\)،
\(d/\mathcal{E}\) و \(e/\mathcal{E}\) را صریحاً مشخص کنید.
\end{info}
\subsection{۲. استراتژی
حل}\label{ux627ux633ux62aux631ux627ux62aux698ux6cc-ux62dux644}
\begin{itemize}
\tightlist
\item
  \textbf{الف:} بررسی سه شرط افراز (ناتهی بودن، جدا بودن، پوشش کامل).
\item
  \textbf{ب:} تشکیل رابطه با ضرب دکارتی هر مجموعه در خودش (مشابه تمرین
  ۱).
\item
  \textbf{پ:} تعیین کلاس هم‌ارزی برای هر عضو (یعنی آن عضو متعلق به کدام
  زیرمجموعه از افراز است).
\end{itemize}
\subsection{۳. حل
تشریحی}\label{ux62dux644-ux62aux634ux631ux6ccux62dux6cc}
\begin{info}{پاسخ}
\textbf{قسمت (الف):} فرض کنیم \(A_1=\{a,b\}\)، \(A_2=\{c\}\) و
\(A_3=\{d,e\}\). ۱. هر سه مجموعه ناتهی هستند. ۲. اشتراک آنها دو به دو
تهی است (\(A_1 \cap A_2 = \emptyset\) و \ldots). ۳. اجتماع آنها کل \(X\)
را می‌سازد: \(\{a,b\} \cup \{c\} \cup \{d,e\} = X\). پس \(\mathcal{P}\)
یک افراز است.
\textbf{قسمت (ب): رابطه هم‌ارزی} رابطه برابر است با
\((A_1 \times A_1) \cup (A_2 \times A_2) \cup (A_3 \times A_3)\):
\[R = \{(a,a),(b,b),(a,b),(b,a)\} \cup \{(c,c)\} \cup \{(d,d),(e,e),(d,e),(e,d)\}\]
مجموعاً ۹ زوج مرتب خواهد داشت.
\textbf{قسمت (پ): کلاس‌های هم‌ارزی} کلاس هم‌ارزی \(x\) (نماد
\(x/\mathcal{E}\)) همان زیرمجموعه‌ای از افراز است که \(x\) در آن قرار
دارد:
\begin{itemize}
\tightlist
\item
  \(a\) در بسته \(\{a,b\}\) است \(\Rightarrow a/\mathcal{E} = \{a,b\}\)
\item
  \(b\) هم در همان بسته است \(\Rightarrow b/\mathcal{E} = \{a,b\}\)
\item
  \(c\) در بسته \(\{c\}\) است \(\Rightarrow c/\mathcal{E} = \{c\}\)
\item
  \(d\) در بسته \(\{d,e\}\) است \(\Rightarrow d/\mathcal{E} = \{d,e\}\)
\item
  \(e\) در بسته \(\{d,e\}\) است \(\Rightarrow e/\mathcal{E} = \{d,e\}\)
\end{itemize}
\end{info}

\clearpage
% ---------------------------------------------------------------------
% Copyright (c) 2026 Arsalan Dalvand & Reyhaneh Darvishi.
% Licensed under CC BY-NC-SA 4.0.
% See LICENSE file for details.
% ---------------------------------------------------------------------

\section{تمرین ۵: رابطه هم‌نهشتی (پیمانه
۵)}\label{تمرین-۵---هم‌نهشتی-اعداد-صحیح}
\subsection{۱. صورت
سوال}\label{ux635ux648ux631ux62a-ux633ux648ux627ux644}
\begin{info}{فرض کنید \(X=\mathbb{Z}\) (مجموعه اعداد صحیح) باشد. رابطه \(\mathcal{E}\) را روی \(\mathbb{Z}\) به صورت زیر تعریف می‌کنیم:}
\[x \mathcal{E} y \iff x - y = 5k \quad (k \in \mathbb{Z})\]
\textbf{الف)} ثابت کنید \(\mathcal{E}\) یک رابطه هم‌ارزی است. \textbf{ب)}
افراز \(\mathbb{Z}/\mathcal{E}\) را پیدا کنید.
\lr{[cite\_start]}\textbf{ج)} تحقیق کنید که رابطه حاصل از این افراز،
همان رابطه اولیه است \lr{[cite: }115-117{]}.
\end{info}
\subsection{۲. استراتژی
حل}\label{ux627ux633ux62aux631ux627ux62aux698ux6cc-ux62dux644}
این همان مفهوم \textbf{«هم‌نهشتی به پیمانه ۵»} است
(\(x \equiv y \pmod 5\)).
\begin{itemize}
\tightlist
\item
  \textbf{الف:} باید بازتابی (\(x-x=5k\))، تقارنی (اگر \(x-y=5k\) آنگاه
  \(y-x=5(-k)\)) و تعدی (\(x-y=5k, y-z=5k' \to x-z=5(k+k')\)) را اثبات
  کنیم.
\item
  \textbf{ب:} کلاس‌های هم‌ارزی را پیدا کنیم. هر عدد صحیح با باقی‌مانده
  تقسیمش بر ۵ هم‌ارز است. باقی‌مانده‌های ممکن \(\{0, 1, 2, 3, 4\}\) هستند.
\end{itemize}
\subsection{۳. حل
تشریحی}\label{ux62dux644-ux62aux634ux631ux6ccux62dux6cc}
\begin{info}{پاسخ}
\textbf{قسمت (الف): اثبات هم‌ارزی} ۱. \textbf{بازتابی:} برای هر
\(x \in \mathbb{Z}\)، داریم \(x - x = 0 = 5(0)\). چون
\(0 \in \mathbb{Z}\)، پس \(x \mathcal{E} x\). ۲. \textbf{تقارنی:} اگر
\(x \mathcal{E} y\)، یعنی \(x - y = 5k\).
\[y - x = -(x - y) = -(5k) = 5(-k)\] چون \(-k\) صحیح است، پس
\(y \mathcal{E} x\). ۳. \textbf{تعدی:} اگر \(x \mathcal{E} y\) و
\(y \mathcal{E} z\): \[x - y = 5k\] \[y - z = 5k'\] با جمع طرفین:
\((x - y) + (y - z) = 5k + 5k' \Rightarrow x - z = 5(k+k')\). پس
\(x \mathcal{E} z\). بنابراین \(\mathcal{E}\) یک رابطه هم‌ارزی است.
\textbf{قسمت (ب): افراز (کلاس‌های هم‌نهشتی)} این رابطه کل اعداد صحیح را به
۵ دسته تقسیم می‌کند (بر اساس باقی‌مانده تقسیم بر ۵):
\begin{itemize}
\tightlist
\item
  \(Z_0 = \{\dots, -5, 0, 5, 10, \dots\} = [0]\)
\item
  \(Z_1 = \{\dots, -4, 1, 6, 11, \dots\} = [1]\)
\item
  \(Z_2 = \{\dots, -3, 2, 7, 12, \dots\} = [2]\)
\item
  \(Z_3 = \{\dots, -2, 3, 8, 13, \dots\} = [3]\)
\item
  \(Z_4 = \{\dots, -1, 4, 9, 14, \dots\} = [4]\)
\end{itemize}
افراز نهایی: \(\mathbb{Z}/\mathcal{E} = \{Z_0, Z_1, Z_2, Z_3, Z_4\}\).
\textbf{قسمت (ج):} طبق قضیه تناظر (و آنچه در تمرین ۳ دیدیم)، چون این
افراز دقیقاً از دسته‌بندی اعداد بر اساس تفاضل مضرب ۵ به دست آمده، رابطه
حاصل از آن نیز همان \(x-y=5k\) خواهد بود.
\end{info}

\clearpage
% ---------------------------------------------------------------------
% Copyright (c) 2026 Arsalan Dalvand & Reyhaneh Darvishi.
% Licensed under CC BY-NC-SA 4.0.
% See LICENSE file for details.
% ---------------------------------------------------------------------

\section{مفهوم تابع و توسعه هم‌دامنه (قضیه
۶)}\label{مفهوم-تابع-و-قضیه-۶---تعریف-و-دامنه}
\begin{tldr}{خلاصه سریع}
تابع، نوع خاصی از «رابطه» است که رفتار قطعی دارد: هر ورودی دقیقاً یک
خروجی تولید می‌کند. قضیه ۶ بیان می‌کند که بزرگ کردن مجموعه مقصد (هم‌دامنه)،
آسیبی به ساختار تابع نمی‌زند.
\end{tldr}
\subsection{\texorpdfstring{۱. تعریف صوری تابع
\lr{(Function Definition)}}{۱. تعریف صوری تابع }}\label{ux62aux639ux631ux6ccux641-ux635ux648ux631ux6cc-ux62aux627ux628ux639-function-definition}
در نظریه مجموعه‌ها، تابع یک مفهوم اولیه نیست، بلکه بر اساس مفهوم «رابطه»
و «زوج مرتب» تعریف می‌شود.
فرض کنید \(X\) و \(Y\) دو مجموعه باشند. یک تابع \(f\) از \(X\) به \(Y\)
(نماد: \(f: X \to Y\)) زیرمجموعه‌ای از حاصلضرب دکارتی \(X \times Y\) است
که دو شرط زیر را ارضا کند:
\begin{tldr}{تعریف تابع}
۱. \textbf{دامنه کامل \lr{(Existence):}} به ازای هر عضو در دامنه، حداقل
یک تصویر وجود داشته باشد.
\[\forall x \in X, \exists y \in Y : (x,y) \in f\] ۲. \textbf{یکتایی
\lr{(Uniqueness):}} هر عضو دامنه، حداکثر یک تصویر داشته باشد.
\[(x, y) \in f \wedge (x, z) \in f \implies y = z\]
\end{tldr}
اگر \((x,y) \in f\) باشد، معمولاً می‌نویسیم \(y = f(x)\).
\begin{center}\rule{0.5\linewidth}{0.5pt}\end{center}
\subsection{\texorpdfstring{۲. قضیه ۶: توسعه هم‌دامنه
\lr{(Codomain Extension)}}{۲. قضیه ۶: توسعه هم‌دامنه }}\label{ux642ux636ux6ccux647-ux6f6-ux62aux648ux633ux639ux647-ux647ux645ux62fux627ux645ux646ux647-codomain-extension}
این قضیه به ما اجازه می‌دهد که «ظرف مقصد» را بدون تغییر ضابطه تابع،
بزرگ‌تر کنیم.
\subsubsection{متن ریاضی
قضیه}\label{ux645ux62aux646-ux631ux6ccux627ux636ux6cc-ux642ux636ux6ccux647}
فرض کنید \(f: X \to Y\) یک تابع باشد و \(Y \subseteq W\).
\begin{theorembox}{قضیه ۶}
اگر \(Y \subseteq W\) باشد، آنگاه \(f\) یک تابع از \(X\) به \(W\) نیز
محسوب می‌شود (\(f: X \to W\)).
\end{theorembox}
\subsubsection{اثبات
صوری}\label{ux627ux62bux628ux627ux62a-ux635ux648ux631ux6cc}
برای اثبات اینکه \(f\) تابعی از \(X\) به \(W\) است، باید سه شرط را بررسی
کنیم: زیرمجموعه بودن، دامنه کامل و یکتایی.
\begin{info}{برهان}
\textbf{۱. شرط زیرمجموعه بودن:} می‌دانیم \(f \subseteq X \times Y\). چون
طبق فرض \(Y \subseteq W\)، طبق ویژگی‌های حاصلضرب دکارتی داریم
\(X \times Y \subseteq X \times W\). بنابراین با استفاده از خاصیت تعدی
زیرمجموعه‌ها (\autoref{قضیه-۲---تعدی-در-مجموعه-ها}):
\[f \subseteq X \times W\]
\textbf{۲. شرط دامنه کامل:} چون \(f: X \to Y\) تابع است، برای هر
\(x \in X\)، یک \(y \in Y\) هست که \((x,y) \in f\). چون
\(Y \subseteq W\)، پس این \(y\) در \(W\) هم هست (\(y \in W\)). پس شرط
وجود تصویر در \(W\) برقرار است.
\textbf{۳. شرط یکتایی:} ویژگی یکتایی تابع (\(y=z\)) وابسته به عناصر زوج
مرتب است و به مجموعه هم‌دامنه بستگی ندارد. چون \(f\) ذاتاً تابع است، این
ویژگی همچنان برقرار باقی می‌ماند.
\textbf{نتیجه:} \(f: X \to W\) یک تابع است.
\end{info}
\begin{center}\rule{0.5\linewidth}{0.5pt}\end{center}
\subsection{\texorpdfstring{۳. شبکه ارتباطی با سایر قضایا
\lr{(Analytic Map)}}{۳. شبکه ارتباطی با سایر قضایا }}\label{ux634ux628ux6a9ux647-ux627ux631ux62aux628ux627ux637ux6cc-ux628ux627-ux633ux627ux6ccux631-ux642ux636ux627ux6ccux627-analytic-map}
\subsubsection{\texorpdfstring{۱. ارتباط با
\autoref{پیشنیاز---مفاهیم-پایه-رابطه-و-تابع}}{۱. ارتباط با }}\label{ux627ux631ux62aux628ux627ux637-ux628ux627-ux67eux6ccux634ux646ux6ccux627ux632---ux645ux641ux627ux647ux6ccux645-ux67eux627ux6ccux647-ux631ux627ux628ux637ux647-ux648-ux62aux627ux628ux639}
\begin{itemize}
\tightlist
\item
  \textbf{رابطه والد-فرزندی:} تعریف تابع مستقیماً روی تعریف «رابطه»
  (زیرمجموعه حاصلضرب دکارتی) بنا شده است. قضیه ۶ نشان می‌دهد که تابع
  بودن، بیشتر به رفتار مولفه اول (دامنه) وابسته است تا محدودیت مولفه دوم
  (هم‌دامنه).
\end{itemize}
\subsubsection{\texorpdfstring{۲. ارتباط با
\autoref{قضیه-۱---پخش‌پذیری-حاصلضرب-دکارتی}}{۲. ارتباط با }}\label{ux627ux631ux62aux628ux627ux637-ux628ux627-ux642ux636ux6ccux647-ux6f1---ux67eux62eux634ux67eux630ux6ccux631ux6cc-ux62dux627ux635ux644ux636ux631ux628-ux62fux6a9ux627ux631ux62aux6cc}
\begin{itemize}
\tightlist
\item
  \textbf{ابزار اثبات:} در گام اول اثبات قضیه ۶، از اصلی استفاده کردیم
  که اگر \(B \subseteq C\) آنگاه \(A \times B \subseteq A \times C\).
  این نتیجه مستقیم تعاریف فصل ۳ درباره حاصلضرب دکارتی است.
\end{itemize}

\clearpage
% ---------------------------------------------------------------------
% Copyright (c) 2026 Arsalan Dalvand & Reyhaneh Darvishi.
% Licensed under CC BY-NC-SA 4.0.
% See LICENSE file for details.
% ---------------------------------------------------------------------

\section{قضیه ۷: شرط تساوی دو
تابع}\label{قضیه-۷---شرط-تساوی-توابع}
\begin{tldr}{خلاصه سریع}
دو تابع زمانی با هم «برابر» هستند که دقیقاً مجموعه زوج‌های مرتب یکسانی
باشند. این قضیه نشان می‌دهد که این شرط مجموعه‌ای، معادل است با اینکه خروجی
آن‌ها به ازای تک‌تک ورودی‌ها یکسان باشد.
\end{tldr}
\subsection{۱. متن ریاضی
قضیه}\label{ux645ux62aux646-ux631ux6ccux627ux636ux6cc-ux642ux636ux6ccux647}
فرض کنید \(f\) و \(g\) دو تابع از مجموعه \(X\) به مجموعه \(Y\) باشند
(\(f,g: X \to Y\)).
\begin{theorembox}{قضیه ۷}
دو تابع \(f\) و \(g\) برابرند (\(f=g\)) اگر و تنها اگر:
\[\forall x \in X, \quad f(x) = g(x)\]
\end{theorembox}
\subsection{\texorpdfstring{۲. اثبات صوری
\lr{(Formal Proof)}}{۲. اثبات صوری }}\label{ux627ux62bux628ux627ux62a-ux635ux648ux631ux6cc-formal-proof}
چون توابع در اصل «مجموعه» (از زوج‌های مرتب) هستند، اثبات برابری آن‌ها باید
بر اساس اصول نظریه مجموعه‌ها
(\(A=B \iff A \subseteq B \land B \subseteq A\)) باشد.
\begin{info}{برهان}
\textbf{جهت رفت (\(\Rightarrow\)):} فرض کنیم \(f = g\) (به عنوان دو
مجموعه برابرند). پس هر زوج مرتبی که در \(f\) باشد، در \(g\) هم هست. اگر
\((x, y) \in f\)، آنگاه \(y = f(x)\). چون \(f=g\)، پس \((x, y) \in g\) و
در نتیجه \(y = g(x)\). بنابراین \(f(x) = g(x)\).
\textbf{جهت برگشت (\(\Leftarrow\)):} فرض کنیم برای هر \(x \in X\) داشته
باشیم \(f(x) = g(x)\). باید ثابت کنیم مجموعه \(f\) زیرمجموعه \(g\) است و
برعکس.
۱. فرض کنید \((x, y)\) یک عضو دلخواه از \(f\) باشد. ۲. طبق تعریف تابع،
این یعنی \(y = f(x)\). ۳. طبق فرض خلف، می‌دانیم \(f(x) = g(x)\). ۴. پس
\(y = g(x)\). ۵. طبق تعریف تابع \(g\)، این یعنی \((x, y) \in g\). ۶.
نتیجه: \(f \subseteq g\).
(به طریق کاملاً مشابه ثابت می‌شود \(g \subseteq f\)). چون دو مجموعه
زیرمجموعه هم هستند، پس \(f = g\).
\end{info}
\subsection{\texorpdfstring{۳. شبکه ارتباطی با سایر قضایا
\lr{(Analytic Map)}}{۳. شبکه ارتباطی با سایر قضایا }}\label{ux634ux628ux6a9ux647-ux627ux631ux62aux628ux627ux637ux6cc-ux628ux627-ux633ux627ux6ccux631-ux642ux636ux627ux6ccux627-analytic-map}
\subsubsection{\texorpdfstring{۱. ارتباط با
\autoref{قضیه-۳---ویژگی‌های-بنیادی-کلاس-هم‌ارزی} (قیاس
ساختاری)}{۱. ارتباط با  (قیاس ساختاری)}}\label{ux627ux631ux62aux628ux627ux637-ux628ux627-ux642ux636ux6ccux647-ux6f3---ux648ux6ccux698ux6afux6ccux647ux627ux6cc-ux628ux646ux6ccux627ux62fux6cc-ux6a9ux644ux627ux633-ux647ux645ux627ux631ux632ux6cc-ux642ux6ccux627ux633-ux633ux627ux62eux62aux627ux631ux6cc}
\begin{itemize}
\tightlist
\item
  \textbf{یکتایی نمایش:} همان‌طور که در کلاس‌های هم‌ارزی دیدیم
  \(x/\xi = y/\xi \iff x \xi y\)، اینجا هم می‌بینیم که تساوی دو شیء
  ساختاری (تابع)، به تساوی مولفه‌های سازنده‌شان (مقادیر خروجی) تقلیل
  می‌یابد.
\end{itemize}
\subsubsection{\texorpdfstring{۲. کاربرد در
\autoref{تمرین-۱۵---تابع-بازتابی}}{۲. کاربرد در }}\label{ux6a9ux627ux631ux628ux631ux62f-ux62fux631-ux62aux645ux631ux6ccux646-ux6f1ux6f5---ux62aux627ux628ux639-ux628ux627ux632ux62aux627ux628ux6cc}
\begin{itemize}
\tightlist
\item
  \textbf{ابزار حل:} در تمرین ۱۵، برای اثبات اینکه \(f\) همان تابع همانی
  \(I_X\) است، دقیقاً از همین قضیه ۷ استفاده می‌کنیم. آنجا نشان می‌دهیم
  \(\forall x, f(x) = I_X(x)\) و سپس طبق قضیه ۷ نتیجه می‌گیریم
  \(f = I_X\). بدون قضیه ۷، استدلال در مورد برابری توابع ناقص می‌ماند.
\end{itemize}

\clearpage
% ---------------------------------------------------------------------
% Copyright (c) 2026 Arsalan Dalvand & Reyhaneh Darvishi.
% Licensed under CC BY-NC-SA 4.0.
% See LICENSE file for details.
% ---------------------------------------------------------------------

\section{\texorpdfstring{قضیه ۸: لم چسباندن
\lr{(Pasting Lemma)}}{قضیه ۸: لم چسباندن }}\label{قضیه-۸---لم-چسباندن-(اجتماع-توابع)}
\begin{tldr}{خلاصه سریع}
این قضیه شرط لازم و کافی برای «اجتماع دو تابع» را بیان می‌کند. اگر دو
تابع \(f\) و \(g\) داشته باشیم، اجتماع آن‌ها (\(f \cup g\)) تنها در صورتی
یک تابع خواهد بود که در دامنه مشترکشان، خروجی‌های یکسانی تولید کنند
(توافق داشته باشند).
\end{tldr}
\subsection{۱. متن ریاضی
قضیه}\label{ux645ux62aux646-ux631ux6ccux627ux636ux6cc-ux642ux636ux6ccux647}
فرض کنید \(f\) و \(g\) دو تابع باشند. رابطه \(h = f \cup g\) یک تابع است
اگر و تنها اگر \(f\) و \(g\) روی اشتراک دامنه‌هایشان هم‌مقدار باشند
(سازگار باشند).
\begin{theorembox}{قضیه ۸}
\[f \cup g \text{ is a function} \iff \forall x \in \text{dom}(f) \cap \text{dom}(g), f(x) = g(x)\]
در این صورت، دامنه تابع حاصل برابر است با:
\[\text{dom}(f \cup g) = \text{dom}(f) \cup \text{dom}(g)\]
\end{theorembox}
\subsection{\texorpdfstring{۲. اثبات صوری
\lr{(Formal Proof)}}{۲. اثبات صوری }}\label{ux627ux62bux628ux627ux62a-ux635ux648ux631ux6cc-formal-proof}
برای اثبات تابع بودن یک رابطه، باید نشان دهیم که به ازای هر ورودی، خروجی
\textbf{یکتا} است (خوش‌تعریفی).
\begin{info}{برهان}
\textbf{فرض:} شرط سازگاری برقرار است:
\(\forall x \in \text{dom}(f) \cap \text{dom}(g) \implies f(x) = g(x)\).
\textbf{حکم:} \(h = f \cup g\) یک تابع است.
۱. فرض کنید \(x\) عضوی از دامنه \(h\) باشد و دو زوج مرتب \((x, y_1)\) و
\((x, y_2)\) در \(h\) وجود داشته باشند. ۲. طبق تعریف اجتماع مجموعه‌ها:
\[[(x, y_1) \in f \lor (x, y_1) \in g] \wedge [(x, y_2) \in f \lor (x, y_2) \in g]\]
۳. حال حالات ممکن برای \(x\) را بررسی می‌کنیم:
\begin{itemize}
\tightlist
\item
  \textbf{حالت اول (\(x \in \text{dom}(f) \setminus \text{dom}(g)\)):}
  در این صورت زوج‌ها نمی‌توانند در \(g\) باشند، پس الزاماً
  \((x, y_1) \in f\) و \((x, y_2) \in f\). چون \(f\) تابع است
  \(\implies y_1 = y_2\).
\item
  \textbf{حالت دوم (\(x \in \text{dom}(g) \setminus \text{dom}(f)\)):}
  مشابه حالت قبل، هر دو زوج در \(g\) هستند. چون \(g\) تابع است
  \(\implies y_1 = y_2\).
\item
  \textbf{حالت سوم (\(x \in \text{dom}(f) \cap \text{dom}(g)\)):} در
  اینجا ممکن است یک زوج از \(f\) و دیگری از \(g\) آمده باشد. مثلاً
  \((x, y_1) \in f\) و \((x, y_2) \in g\). این یعنی \(y_1 = f(x)\) و
  \(y_2 = g(x)\). اما طبق \textbf{فرض سازگاری قضیه}، می‌دانیم در اشتراک
  دامنه‌ها \(f(x) = g(x)\). بنابراین \(y_1 = y_2\).
\end{itemize}
\textbf{نتیجه:} در تمام حالات، خروجی یکتاست. پس \(f \cup g\) تابع است.
\end{info}
\subsection{\texorpdfstring{۳. شبکه ارتباطی با سایر قضایا
\lr{(Analytic Map)}}{۳. شبکه ارتباطی با سایر قضایا }}\label{ux634ux628ux6a9ux647-ux627ux631ux62aux628ux627ux637ux6cc-ux628ux627-ux633ux627ux6ccux631-ux642ux636ux627ux6ccux627-analytic-map}
این قضیه ابزاری قدرتمند برای ساخت توابع پیچیده از توابع ساده‌تر است:
\subsubsection{\texorpdfstring{۱. ارتباط با
\autoref{تمرین-۱۳---تحدید-تابع} (عملیات
معکوس)}{۱. ارتباط با  (عملیات معکوس)}}\label{ux627ux631ux62aux628ux627ux637-ux628ux627-ux62aux645ux631ux6ccux646-ux6f1ux6f3---ux62aux62dux62fux6ccux62f-ux62aux627ux628ux639-ux639ux645ux644ux6ccux627ux62a-ux645ux639ux6a9ux648ux633}
\begin{itemize}
\tightlist
\item
  \textbf{رابطه دوطرفه:} در تمرین ۱۳ دیدیم که اگر یک تابع بزرگ را به
  زیرمجموعه‌ای از دامنه‌اش محدود کنیم \lr{(Restriction)، }حاصل همیشه تابع
  است (\(f|_A\)). قضیه ۸ \lr{(Pasting) }عکس آن عمل را انجام می‌دهد: تلاش
  می‌کند توابع کوچک (تحدیدها) را به هم بچسباند تا تابع اصلی (توسیع) را
  بسازد. در واقع اگر \(h = f \cup g\) تابع باشد، آنگاه
  \(f = h|_{\text{dom}(f)}\) و \(g = h|_{\text{dom}(g)}\).
\end{itemize}
\subsubsection{\texorpdfstring{۲. کاربرد در
\autoref{قضیه-۷---شرط-تساوی-توابع}}{۲. کاربرد در }}\label{ux6a9ux627ux631ux628ux631ux62f-ux62fux631-ux642ux636ux6ccux647-ux6f7---ux634ux631ux637-ux62aux633ux627ux648ux6cc-ux62aux648ux627ux628ux639}
\begin{itemize}
\tightlist
\item
  \textbf{سازگاری:} شرط \(f(x)=g(x)\) در قضیه ۸، یادآور شرط تساوی توابع
  در قضیه ۷ است. با این تفاوت که در قضیه ۷ دامنه‌ها برابر بودند، اما در
  اینجا دامنه‌ها متفاوت‌اند و تساوی فقط در بخش مشترک (\(A \cap B\)) بررسی
  می‌شود.
\end{itemize}
\subsubsection{۳. پیش‌نیاز توپولوژی و
آنالیز}\label{ux67eux6ccux634ux646ux6ccux627ux632-ux62aux648ux67eux648ux644ux648ux698ux6cc-ux648-ux622ux646ux627ux644ux6ccux632}
\begin{itemize}
\tightlist
\item
  \textbf{توابع چندضابطه‌ای:} تعریف توابع چندضابطه‌ای
  \lr{(Piecewise functions) }که در حساب دیفرانسیل می‌بینید (مانند قدر
  مطلق)، دقیقاً کاربرد عملی همین قضیه است. ما دو تابع \(y=x\) و \(y=-x\)
  را در نقطه \(x=0\) (اشتراک دامنه‌ها) چک می‌کنیم. چون هر دو در صفر برابر
  صفرند، می‌توانیم آن‌ها را بچسبانیم.
\end{itemize}

\clearpage
% ---------------------------------------------------------------------
% Copyright (c) 2026 Arsalan Dalvand & Reyhaneh Darvishi.
% Licensed under CC BY-NC-SA 4.0.
% See LICENSE file for details.
% ---------------------------------------------------------------------

\section{توابع خاص: تابع همانی و تابع
ثابت}\label{توابع-خاص---همانی-و-ثابت}
\begin{tldr}{خلاصه سریع}
در دنیای توابع، دو بازیگر بسیار مهم وجود دارند: ۱. \textbf{تابع همانی
(\(I_X\)):} آینه‌ای که هر کس را به خودش نشان می‌دهد. ۲. \textbf{تابع ثابت
(\(C_c\)):} سیاه‌چاله‌ای که همه ورودی‌ها را به یک نقطه واحد می‌فرستد.
\end{tldr}
\subsection{\texorpdfstring{۱. تابع همانی
\lr{(Identity Function)}}{۱. تابع همانی }}\label{ux62aux627ux628ux639-ux647ux645ux627ux646ux6cc-identity-function}
\subsubsection{الف) درک
شهودی}\label{ux627ux644ux641-ux62fux631ux6a9-ux634ux647ux648ux62fux6cc}
این «بی‌اثرترین» تابع ممکن است. مثل ضرب در عدد ۱ یا جمع با عدد ۰. هر چه
به آن بدهید، همان را پس می‌گیرید.
\subsubsection{ب) تعریف
ریاضی}\label{ux628-ux62aux639ux631ux6ccux641-ux631ux6ccux627ux636ux6cc}
فرض کنید \(X\) یک مجموعه باشد. تابع همانی روی \(X\) را با \(I_X\) (یا
\(1_X\)) نمایش می‌دهیم.
\begin{tldr}{تعریف}
تابع \(I_X: X \to X\) به صورت زیر تعریف می‌شود:
\[\forall x \in X, \quad I_X(x) = x\] یا به زبان زوج مرتب:
\[I_X = \{ (x, x) \mid x \in X \}\] (دقت کنید که این همان \textbf{رابطه
قطری} \(\Delta_X\) است).
\end{tldr}
\subsubsection{ج) ویژگی مهم (عنصر
خنثی)}\label{ux62c-ux648ux6ccux698ux6afux6cc-ux645ux647ux645-ux639ux646ux635ux631-ux62eux646ux62bux6cc}
تابع همانی نقش «عنصر خنثی» در عمل ترکیب توابع را دارد. برای هر تابع
\(f: X \to Y\): \[f \circ I_X = f \quad \text{و} \quad I_Y \circ f = f\]
\begin{center}\rule{0.5\linewidth}{0.5pt}\end{center}
\subsection{\texorpdfstring{۲. تابع ثابت
\lr{(Constant Function)}}{۲. تابع ثابت }}\label{ux62aux627ux628ux639-ux62bux627ux628ux62a-constant-function}
\subsubsection{الف) درک
شهودی}\label{ux627ux644ux641-ux62fux631ux6a9-ux634ux647ux648ux62fux6cc-1}
این تابع تمام تنوع ورودی‌ها را نادیده می‌گیرد و همه را مجبور می‌کند به یک
خروجی خاص تبدیل شوند. برد \lr{(Range) }این تابع همیشه تک‌عضوی است.
\lr{[Image of constant function graph horizontal line]}
\subsubsection{ب) تعریف
ریاضی}\label{ux628-ux62aux639ux631ux6ccux641-ux631ux6ccux627ux636ux6cc-1}
فرض کنید \(f: X \to Y\) یک تابع باشد. \(f\) را تابع ثابت می‌نامیم اگر
عنصری مانند \(c \in Y\) وجود داشته باشد که:
\begin{tldr}{تعریف}
\[\exists c \in Y, \forall x \in X, \quad f(x) = c\]
\end{tldr}
\subsubsection{ج) مثال}\label{ux62c-ux645ux62bux627ux644}
اگر \(X = \mathbb{R}\) و \(f(x) = 5\)، این یک تابع ثابت است. نمودار آن
خطی موازی محور افقی است.
\begin{center}\rule{0.5\linewidth}{0.5pt}\end{center}
\subsection{\texorpdfstring{۳. شبکه ارتباطی با سایر قضایا
\lr{(Analytic Map)}}{۳. شبکه ارتباطی با سایر قضایا }}\label{ux634ux628ux6a9ux647-ux627ux631ux62aux628ux627ux637ux6cc-ux628ux627-ux633ux627ux6ccux631-ux642ux636ux627ux6ccux627-analytic-map}
\subsubsection{\texorpdfstring{۱. ارتباط با
\autoref{تمرین-۱۵---تابع-بازتابی}}{۱. ارتباط با }}\label{ux627ux631ux62aux628ux627ux637-ux628ux627-ux62aux645ux631ux6ccux646-ux6f1ux6f5---ux62aux627ux628ux639-ux628ux627ux632ux62aux627ux628ux6cc}
\begin{itemize}
\tightlist
\item
  \textbf{تابع همانی:} در تمرین ۱۵ ثابت کردیم که اگر تابعی خاصیت بازتابی
  داشته باشد (یعنی \((x,x) \in f\))، آن تابع لزوماً همان \textbf{تابع
  همانی} (\(I_X\)) است. \(I_X\) تنها تابعی است که تماماً روی قطر اصلی
  قرار دارد.
\end{itemize}
\subsubsection{\texorpdfstring{۲. ارتباط با
\autoref{قضیه-۱۶---وارون‌های-یک‌طرفه}}{۲. ارتباط با }}\label{ux627ux631ux62aux628ux627ux637-ux628ux627-ux642ux636ux6ccux647-ux6f1ux6f6---ux648ux627ux631ux648ux646ux647ux627ux6cc-ux6ccux6a9ux637ux631ux641ux647}
\begin{itemize}
\tightlist
\item
  \textbf{نقش کلیدی \(I_X\):} در قضایای مربوط به وارون‌پذیری (که در ادامه
  می‌خوانیم)، شرط اینکه \(g\) وارون چپ \(f\) باشد، این است که ترکیب آن‌ها
  تابع همانی شود (\(g \circ f = I_X\)). \lr{[cite\_start]بدون }تعریف
  دقیق \(I_X\)، تعریف وارون تابع ممکن \lr{نیست[cite: }93{]}.
\end{itemize}
\subsubsection{\texorpdfstring{۳. ارتباط با
\autoref{مفهوم-تابع-و-قضیه-۶---تعریف-و-دامنه}}{۳. ارتباط با }}\label{ux627ux631ux62aux628ux627ux637-ux628ux627-ux645ux641ux647ux648ux645-ux62aux627ux628ux639-ux648-ux642ux636ux6ccux647-ux6f6---ux62aux639ux631ux6ccux641-ux648-ux62fux627ux645ux646ux647}
\begin{itemize}
\tightlist
\item
  \textbf{تفاوت در برد:}
  \begin{itemize}
  \tightlist
  \item
    در تابع همانی، برد دقیقاً برابر با دامنه است (\(R_f = D_f = X\)).
  \item
    در تابع ثابت، برد کوچک‌ترین اندازه ممکن را دارد (تک‌عضوی)، در حالی که
    دامنه می‌تواند بسیار بزرگ (حتی نامتناهی) باشد.
  \end{itemize}
\end{itemize}

\clearpage
% ---------------------------------------------------------------------
% Copyright (c) 2026 Arsalan Dalvand & Reyhaneh Darvishi.
% Licensed under CC BY-NC-SA 4.0.
% See LICENSE file for details.
% ---------------------------------------------------------------------

\section{\texorpdfstring{تمرین ۱۳: تحدید تابع
\lr{(Restriction)}}{تمرین ۱۳: تحدید تابع }}\label{تمرین-۱۳---تحدید-تابع}
\subsection{۱. صورت
سوال}\label{ux635ux648ux631ux62a-ux633ux648ux627ux644}
\begin{info}{فرض کنید \(f: X \to Y\) یک تابع باشد و \(A\) زیرمجموعه‌ای ناتهی از \(X\) باشد (\(A \subseteq X, A \neq \emptyset\)). ثابت کنید که \textbf{تحدید تابع} \(f\) به مجموعه \(A\) که با نماد \(f|_A\) نمایش داده می‌شود، خود یک تابع است.}
\end{info}
\subsection{۲. استراتژی
حل}\label{ux627ux633ux62aux631ux627ux62aux698ux6cc-ux62dux644}
برای اثبات اینکه یک رابطه، «تابع» است، باید دو شرط اساسی را بررسی کنیم:
\begin{enumerate}
\def\labelenumi{\arabic{enumi}.}
\tightlist
\item
  \textbf{دامنه:} آیا رابطه برای تمام اعضای دامنه تعریف شده است؟
\item
  \textbf{یکتایی:} آیا هر عضو دامنه دقیقاً به یک عضو از هم‌دامنه نگاشته
  می‌شود؟
\end{enumerate}
در اینجا، رابطه \(f|_A\) به صورت زیر تعریف می‌شود:
\[f|_A = \{ (x,y) \in f \mid x \in A \}\] ما باید نشان دهیم که این رابطه
یک تابع از \(A\) به \(Y\) است.
\subsection{۳. حل
تشریحی}\label{ux62dux644-ux62aux634ux631ux6ccux62dux6cc}
\begin{info}{اثبات}
\textbf{گام ۱: بررسی خوش‌تعریفی (وجود تصویر)} فرض کنیم \(x\) یک عضو
دلخواه در \(A\) باشد (\(x \in A\)). از آنجا که \(A \subseteq X\) است، پس
\(x \in X\) می‌باشد. چون \(f\) یک تابع با دامنه \(X\) است، پس حتماً عضوی
مانند \(y \in Y\) وجود دارد که \((x,y) \in f\). چون \(x \in A\) و
\((x,y) \in f\)، طبق تعریف تحدید، زوج مرتب \((x,y)\) در \(f|_A\) نیز
وجود دارد. پس هر عضو \(A\) دارای تصویر است.
\textbf{گام ۲: بررسی یکتایی تصویر} فرض کنیم برای یک \(x \in A\)، دو
تصویر \(y_1\) و \(y_2\) وجود داشته باشد:
\[(x, y_1) \in f|_A \quad \text{و} \quad (x, y_2) \in f|_A\] طبق تعریف
زیرمجموعه، این یعنی:
\[(x, y_1) \in f \quad \text{و} \quad (x, y_2) \in f\] اما \(f\) تابع
است و ویژگی اصلی تابع، یکتایی تصویر است. بنابراین الزاماً \(y_1 = y_2\).
\textbf{نتیجه:} چون هر دو شرط برقرار است، \(f|_A: A \to Y\) یک تابع است.
\end{info}

\clearpage
% ---------------------------------------------------------------------
% Copyright (c) 2026 Arsalan Dalvand & Reyhaneh Darvishi.
% Licensed under CC BY-NC-SA 4.0.
% See LICENSE file for details.
% ---------------------------------------------------------------------

\section{تمرین ۱۴: زیرمجموعه یک
تابع}\label{تمرین-۱۴---زیرمجموعه-تابع}
\subsection{۱. صورت
سوال}\label{ux635ux648ux631ux62a-ux633ux648ux627ux644}
\begin{info}{تابع \(f: X \to Y\) مفروض است. ثابت کنید که هر زیرمجموعه از \(f\) (مانند \(g \subseteq f\)) نیز یک تابع است.}
\end{info}
\subsection{۲. استراتژی
حل}\label{ux627ux633ux62aux631ux627ux62aux698ux6cc-ux62dux644}
یک تابع، مجموعه‌ای از زوج‌های مرتب است. هر زیرمجموعه‌ای از آن، یک «رابطه»
است. برای اینکه این رابطه جدید (\(g\)) تابع باشد، باید شرط
\textbf{یکتایی} را حفظ کند. دامنه تابع جدید (\(D_g\)) زیرمجموعه‌ای از
دامنه اصلی (\(X\)) خواهد بود.
\subsection{۳. حل
تشریحی}\label{ux62dux644-ux62aux634ux631ux6ccux62dux6cc}
\begin{info}{اثبات}
فرض کنیم \(g \subseteq f\) باشد. دامنه \(g\) را \(D_g\) می‌نامیم. برای
اثبات تابع بودن \(g\)، باید نشان دهیم برای هر \(x \in D_g\)، تصویر یگانه
است.
۱. فرض کنیم \((x, y_1) \in g\) و \((x, y_2) \in g\). ۲. چون
\(g \subseteq f\) است، پس تمام اعضای \(g\) در \(f\) نیز هستند:
\[(x, y_1) \in f \quad \text{و} \quad (x, y_2) \in f\] ۳. اما می‌دانیم که
\(f\) یک تابع است. طبق تعریف تابع، یک عنصر نمی‌تواند به دو عنصر متمایز
نگاشته شود. ۴. بنابراین، الزاماً \(y_1 = y_2\).
\textbf{نتیجه:} رابطه \(g\) شرط تک‌مقداری بودن را دارد، پس \(g\) یک تابع
از \(D_g\) به \(Y\) است.
\end{info}

\clearpage
% ---------------------------------------------------------------------
% Copyright (c) 2026 Arsalan Dalvand & Reyhaneh Darvishi.
% Licensed under CC BY-NC-SA 4.0.
% See LICENSE file for details.
% ---------------------------------------------------------------------

\section{تمرین ۱۵: تابع بازتابی و تابع
همانی}\label{تمرین-۱۵---تابع-بازتابی}
\subsection{۱. صورت
سوال}\label{ux635ux648ux631ux62a-ux633ux648ux627ux644}
\begin{info}{فرض کنید \(f: X \to X\) یک تابع از \(X\) به \(X\) باشد و هم‌زمان یک \textbf{رابطه بازتابی} \lr{(Reflexive Relation) }روی \(X\) محسوب شود. ثابت کنید در این صورت \(f\) همان تابع همانی \(I_X\) است.}
\end{info}
\subsection{۲. استراتژی
حل}\label{ux627ux633ux62aux631ux627ux62aux698ux6cc-ux62dux644}
ما با دو ویژگی طرف هستیم که باید آن‌ها را ترکیب کنیم:
\begin{enumerate}
\def\labelenumi{\arabic{enumi}.}
\tightlist
\item
  \textbf{رابطه بازتابی:} یعنی هر عضو با خودش رابطه دارد
  (\(\forall x, (x,x) \in f\)).
\item
  \textbf{تابع:} یعنی هر عضو دقیقاً با \textbf{یک} نفر رابطه دارد. ترکیب
  این دو ما را مجبور می‌کند که آن «یک نفر»، همان «خودش» باشد.
\end{enumerate}
\subsection{۳. حل
تشریحی}\label{ux62dux644-ux62aux634ux631ux6ccux62dux6cc}
\begin{info}{اثبات}
۱. \textbf{استفاده از ویژگی بازتابی:} چون \(f\) یک رابطه بازتابی روی
\(X\) است، طبق تعریف باید شامل تمام زوج‌های قطری باشد:
\[\forall x \in X, \quad (x, x) \in f\]
۲. \textbf{استفاده از ویژگی تابع بودن:} چون \(f\) تابع است، به ازای هر
ورودی \(x\)، خروجی باید \textbf{یکتا} باشد. فرض کنیم \((x, y) \in f\)
باشد.
۳. \textbf{نتیجه‌گیری:} از گام (۱) می‌دانیم که \((x, x)\) حتماً در \(f\)
هست. از گام (۲) می‌دانیم که \(x\) نمی‌تواند به بیش از یک چیز وصل شود. پس
تنها انتخابی که برای \(y\) باقی می‌ماند، خودِ \(x\) است.
\[y = x \implies f(x) = x\]
۴. \textbf{تطبیق با تعریف تابع همانی:} تابعی که در آن برای همه اعضا
\(f(x)=x\) باشد، تابع همانی (\(I_X\)) نامیده می‌شود.
\[\therefore f = I_X\]
\end{info}

\clearpage
% ---------------------------------------------------------------------
% Copyright (c) 2026 Arsalan Dalvand & Reyhaneh Darvishi.
% Licensed under CC BY-NC-SA 4.0.
% See LICENSE file for details.
% ---------------------------------------------------------------------

\section{تمرین ۱۶: تابع
متقارن}\label{تمرین-۱۶---تابع-متقارن}
\subsection{۱. صورت
سوال}\label{ux635ux648ux631ux62a-ux633ux648ux627ux644}
\begin{info}{فرض کنید \(X\) بازه بسته واحد \([0, 1]\) باشد. یک تابع \(f: X \to X\) بیابید که یک \textbf{رابطه متقارن} روی \(X\) باشد.}
\end{info}
\subsection{۲. استراتژی
حل}\label{ux627ux633ux62aux631ux627ux62aux698ux6cc-ux62dux644}
بیایید تحلیل کنیم که «تابع متقارن» چه ویژگی ریاضی‌ای دارد:
\begin{itemize}
\tightlist
\item
  \textbf{رابطه:} \(f\) مجموعه‌ای از زوج‌های مرتب \((x, y)\) است که
  \(y = f(x)\).
\item
  \textbf{متقارن بودن:} اگر \((x, y) \in f\) باشد، باید \((y, x) \in f\)
  نیز باشد.
  \begin{itemize}
  \tightlist
  \item
    یعنی اگر \(y = f(x)\)، آنگاه باید \(x = f(y)\).
  \item
    با جایگذاری \(y\): \(x = f(f(x))\).
  \end{itemize}
\end{itemize}
بنابراین ما دنبال تابعی هستیم که \textbf{وارون خودش} باشد
(\(f = f^{-1}\)) یا به عبارتی \(f(f(x)) = x\). نمودار چنین تابعی باید
نسبت به خط \(y=x\) متقارن باشد.
\subsection{۳. حل
تشریحی}\label{ux62dux644-ux62aux634ux631ux6ccux62dux6cc}
ما باید تابعی \(f: [0,1] \to [0,1]\) مثال بزنیم که خاصیت \(f(f(x))=x\)
را داشته باشد.
\begin{note}{مثال ۱: تابع همانی}
ساده‌ترین پاسخ، تابع همانی است: \[f(x) = x\] بررسی تقارن: اگر
\((x, y) \in f \implies y = x \implies x = y \implies (y, x) \in f\).
(این پاسخ صحیح است، اما معمولاً مثال‌های غیربدیهی مد نظر هستند).
\end{note}
\begin{note}{مثال ۲: تابع مکمل (پیشنهادی)}
یک مثال غیربدیهی و هندسی: \[f(x) = 1 - x\]
\textbf{بررسی شرایط:}
\begin{enumerate}
\def\labelenumi{\arabic{enumi}.}
\tightlist
\item
  \textbf{دامنه و برد:} اگر \(x \in [0, 1]\)، آنگاه \(1-x\) نیز در
  \([0, 1]\) است. پس \(f: X \to X\) برقرار است.
\item
  \textbf{تقارن:} فرض کنیم \((a, b) \in f\).
  \[b = 1 - a \implies a = 1 - b \implies (b, a) \in f\]
\end{enumerate}
\textbf{نتیجه:} تابع \(f(x) = 1-x\) یک تابع متقارن روی بازه \([0, 1]\)
است.
\end{note}

\clearpage
% ---------------------------------------------------------------------
% Copyright (c) 2026 Arsalan Dalvand & Reyhaneh Darvishi.
% Licensed under CC BY-NC-SA 4.0.
% See LICENSE file for details.
% ---------------------------------------------------------------------

\section{مفاهیم تصویر و تصویر وارون + قضیه
۹}\label{قضیه-۹---تصویر-و-تصویر-وارون-مجموعه}
\begin{tldr}{خلاصه سریع}
این یادداشت ابتدا دو عملگر بنیادی روی مجموعه‌ها تحت اثر تابع (\(f(A)\) و
\(f^{-1}(B)\)) را تعریف می‌کند. سپس در قضیه ۹ ثابت می‌کنیم که عملگر «تصویر
مستقیم» ساختار اجتماع را حفظ می‌کند (\(=\))، اما در مورد اشتراک ضعیف‌تر
عمل کرده و فقط زیرمجموعه بودن (\(\subseteq\)) را تضمین می‌کند.
\end{tldr}
\subsection{۱. تعاریف بنیادین (تصویر و
وارون)}\label{ux62aux639ux627ux631ux6ccux641-ux628ux646ux6ccux627ux62fux6ccux646-ux62aux635ux648ux6ccux631-ux648-ux648ux627ux631ux648ux646}
پیش از بیان قضیه، لازم است تعاریف دقیق «تصویر مستقیم» \lr{(Image) }و
«تصویر وارون» \lr{(Inverse Image }/ \lr{Preimage) }را تثبیت کنیم. فرض
کنید \(f: X \to Y\) یک تابع باشد.
\subsubsection{\texorpdfstring{الف) تصویر مستقیم
\lr{(Direct Image)}}{الف) تصویر مستقیم }}\label{ux627ux644ux641-ux62aux635ux648ux6ccux631-ux645ux633ux62aux642ux6ccux645-direct-image}
اگر \(A \subseteq X\) باشد، تصویر \(A\) تحت تابع \(f\)، زیرمجموعه‌ای از
\(Y\) است که شامل خروجی‌های متناظر با اعضای \(A\) می‌باشد.
\begin{tldr}{تعریف ۱}
\[f(A) = \{ y \in Y \mid \exists x \in A, y = f(x) \}\] معادل مجموعه‌ای:
\(f(A) = \{ f(x) \mid x \in A \}\)
\end{tldr}
\subsubsection{\texorpdfstring{ب) تصویر وارون
\lr{(Inverse Image)}}{ب) تصویر وارون }}\label{ux628-ux62aux635ux648ux6ccux631-ux648ux627ux631ux648ux646-inverse-image}
اگر \(B \subseteq Y\) باشد، تصویر وارون \(B\) تحت تابع \(f\)،
زیرمجموعه‌ای از \(X\) است که تمام اعضای آن توسط \(f\) به درون \(B\) پرتاب
می‌شوند.
\begin{tldr}{تعریف ۲}
\[f^{-1}(B) = \{ x \in X \mid f(x) \in B \}\]
\end{tldr}
\begin{warning}{هشدار مهم}
نماد \(f^{-1}\) در اینجا به معنای «تابع وارون» نیست. تصویر وارون برای
\textbf{هر تابعی} (حتی اگر یک‌به‌یک یا پوشا نباشد) روی مجموعه‌ها تعریف
می‌شود.
\end{warning}
\begin{center}\rule{0.5\linewidth}{0.5pt}\end{center}
\subsection{۲. متن ریاضی قضیه
۹}\label{ux645ux62aux646-ux631ux6ccux627ux636ux6cc-ux642ux636ux6ccux647-ux6f9}
فرض کنید \(f: X \to Y\) یک تابع باشد و \(A, B \subseteq X\). ویژگی‌های
جبری تصویر مستقیم به شرح زیر است:
\begin{theorembox}{قضیه ۹}
\textbf{الف) حفظ شمول \lr{(Monotonicity):}}
\[A \subseteq B \implies f(A) \subseteq f(B)\] \textbf{ب) پخش‌پذیری روی
اجتماع:} \[f(A \cup B) = f(A) \cup f(B)\] \textbf{ج) رفتار روی اشتراک:}
\[f(A \cap B) \subseteq f(A) \cap f(B)\] \emph{(نکته: تساوی در قسمت ج
لزوماً برقرار نیست).}
\end{theorembox}
\begin{center}\rule{0.5\linewidth}{0.5pt}\end{center}
\subsection{\texorpdfstring{۳. اثبات صوری
\lr{(Formal Proof)}}{۳. اثبات صوری }}\label{ux627ux62bux628ux627ux62a-ux635ux648ux631ux6cc-formal-proof}
\subsubsection{اثبات قسمت (الف): حفظ
شمول}\label{ux627ux62bux628ux627ux62a-ux642ux633ux645ux62a-ux627ux644ux641-ux62dux641ux638-ux634ux645ux648ux644}
\begin{info}{برهان}
۱. فرض کنیم \(y \in f(A)\). ۲. طبق تعریف تصویر، یعنی \(\exists x \in A\)
به طوری که \(y = f(x)\). ۳. چون طبق فرض \(A \subseteq B\)، پس این \(x\)
متعلق به \(B\) نیز هست (\(x \in B\)). ۴. اکنون ما عنصری در \(B\) داریم
(\(x\)) که تصویرش \(y\) است. ۵. پس طبق تعریف، \(y \in f(B)\). ۶.
\lr{[cite\_start]نتیجه: }\(f(A) \subseteq f(B)\). \lr{[cite: }515-516{]}
\end{info}
\subsubsection{اثبات قسمت (ب):
اجتماع}\label{ux627ux62bux628ux627ux62a-ux642ux633ux645ux62a-ux628-ux627ux62cux62aux645ux627ux639}
باید تساوی دو مجموعه را با شمول دوطرفه ثابت کنیم.
\begin{info}{برهان}
\textbf{جهت اول (\(f(A \cup B) \subseteq f(A) \cup f(B)\)):} ۱. فرض کنیم
\(y \in f(A \cup B)\). ۲. یعنی \(\exists x \in A \cup B\) که
\(y = f(x)\). ۳. چون \(x \in A \cup B\)، پس (\(x \in A\)) یا
(\(x \in B\)). ۴. اگر \(x \in A \implies y \in f(A)\). ۵. اگر
\(x \in B \implies y \in f(B)\). ۶. در هر حال \(y \in f(A) \cup f(B)\).
\textbf{جهت دوم (\(f(A) \cup f(B) \subseteq f(A \cup B)\)):} ۱. می‌دانیم
\(A \subseteq A \cup B\) و \(B \subseteq A \cup B\). ۲. طبق قسمت (الف)
همین قضیه (حفظ شمول):
\[f(A) \subseteq f(A \cup B) \quad \text{و} \quad f(B) \subseteq f(A \cup B)\]
۳. اجتماع دو زیرمجموعه از یک مجموعه، باز هم زیرمجموعه آن است:
\[f(A) \cup f(B) \subseteq f(A \cup B)\]
\textbf{نتیجه:} تساوی برقرار است.
\end{info}
\subsubsection{اثبات قسمت (ج):
اشتراک}\label{ux627ux62bux628ux627ux62a-ux642ux633ux645ux62a-ux62c-ux627ux634ux62aux631ux627ux6a9}
\begin{info}{برهان}
۱. می‌دانیم \(A \cap B \subseteq A\) و \(A \cap B \subseteq B\). ۲. طبق
قسمت (الف) (حفظ شمول):
\[f(A \cap B) \subseteq f(A) \quad \text{و} \quad f(A \cap B) \subseteq f(B)\]
۳. چون مجموعه سمت چپ زیرمجموعه هر دو مجموعه سمت راست است، پس زیرمجموعه
اشتراک آن‌ها نیز می‌باشد: \[f(A \cap B) \subseteq f(A) \cap f(B)\]
\end{info}
\begin{center}\rule{0.5\linewidth}{0.5pt}\end{center}
\subsection{\texorpdfstring{۴. شبکه ارتباطی با سایر قضایا
\lr{(Analytic Map)}}{۴. شبکه ارتباطی با سایر قضایا }}\label{ux634ux628ux6a9ux647-ux627ux631ux62aux628ux627ux637ux6cc-ux628ux627-ux633ux627ux6ccux631-ux642ux636ux627ux6ccux627-analytic-map}
\subsubsection{\texorpdfstring{۱. ارتباط با
\autoref{قضیه-۱---پخش‌پذیری-حاصلضرب-دکارتی}}{۱. ارتباط با }}\label{ux627ux631ux62aux628ux627ux637-ux628ux627-ux642ux636ux6ccux647-ux6f1---ux67eux62eux634ux67eux630ux6ccux631ux6cc-ux62dux627ux635ux644ux636ux631ux628-ux62fux6a9ux627ux631ux62aux6cc}
\begin{itemize}
\tightlist
\item
  \textbf{چرا اشتراک برابر نمی‌شود؟} در اثبات اجتماع (بخش ب)، ما از
  هم‌ارزی
  \((\exists x, P(x) \lor Q(x)) \iff (\exists x, P(x)) \lor (\exists x, Q(x))\)
  استفاده کردیم. اما سور وجودی (\(\exists\)) روی اشتراک (\(\land\)) پخش
  نمی‌شود. به همین دلیل \(f(A \cap B) \neq f(A) \cap f(B)\).
\item
  \textbf{مثال نقض:} تابع \(f(x)=x^2\) را در نظر بگیرید. \(A=\{-2\}\) و
  \(B=\{2\}\).
  \begin{itemize}
  \tightlist
  \item
    \(A \cap B = \emptyset \implies f(A \cap B) = \emptyset\).
  \item
    \(f(A) = \{4\}\) و \(f(B) = \{4\} \implies f(A) \cap f(B) = \{4\}\).
  \item
    واضح است که \(\emptyset \neq \{4\}\).
  \end{itemize}
\end{itemize}
\subsubsection{\texorpdfstring{۲. ارتباط با
\autoref{قضیه-۵---متمم-و-زیرمجموعه} (تصویر
وارون)}{۲. ارتباط با  (تصویر وارون)}}\label{ux627ux631ux62aux628ux627ux637-ux628ux627-ux642ux636ux6ccux647-ux6f5---ux645ux62aux645ux645-ux648-ux632ux6ccux631ux645ux62cux645ux648ux639ux647-ux62aux635ux648ux6ccux631-ux648ux627ux631ux648ux646}
\begin{itemize}
\tightlist
\item
  \textbf{رفتار بهتر وارون:} در قضایای بعدی خواهیم دید که \(f^{-1}\)
  (تصویر وارون) برخلاف \(f\) (تصویر مستقیم)، رفتار بسیار منظمی دارد و با
  تمام عملیات (اجتماع، اشتراک و تفاضل) سازگار است
  (\(f^{-1}(A \cap B) = f^{-1}(A) \cap f^{-1}(B)\)). این تفاوت بنیادی
  بین «تابع» و «رابطه» را نشان می‌دهد.
\end{itemize}
\subsubsection{\texorpdfstring{۳. ارتباط با
\autoref{قضیه-۸---لم-چسباندن-(اجتماع-توابع)}}{۳. ارتباط با }}\label{ux627ux631ux62aux628ux627ux637-ux628ux627-ux642ux636ux6ccux647-ux6f8---ux644ux645-ux686ux633ux628ux627ux646ux62fux646-ux627ux62cux62aux645ux627ux639-ux62aux648ux627ux628ux639}
\begin{itemize}
\tightlist
\item
  \textbf{ساختار اجتماع:} قضیه ۹-ب نشان می‌دهد که تصویرِ اجتماع دامنه‌ها،
  برابر با اجتماعِ تصویرهاست. این مفهوم با لم چسباندن سازگار است؛ اگر
  توابع را روی دامنه‌های جداگانه تعریف کنیم، برد نهایی اجتماع بردهای
  آنهاست.
\end{itemize}

\clearpage
% ---------------------------------------------------------------------
% Copyright (c) 2026 Arsalan Dalvand & Reyhaneh Darvishi.
% Licensed under CC BY-NC-SA 4.0.
% See LICENSE file for details.
% ---------------------------------------------------------------------

\section{قضیه ۱۰: رفتار تصویر تابع با اجتماع و اشتراک
تعمیم‌یافته}\label{قضیه-۱۰---تعمیم-تصویر-اجتماع-و-اشتراک}
\begin{tldr}{خلاصه سریع}
این قضیه، \textbf{\autoref{قضیه-۹---تصویر-و-تصویر-وارون-مجموعه}} را برای
خانواده‌های نامتناهی تعمیم می‌دهد. نتیجه مهم این است که عملگر تصویر
(\(f\)) با عملگر اجتماع (\(\bigcup\)) کاملاً جابجا می‌شود (هم‌ریختی)، اما
در برابر اشتراک (\(\bigcap\)) ضعیف عمل کرده و تنها یک رابطه زیرمجموعه‌ای
را حفظ می‌کند.
\end{tldr}
\subsection{۱. متن ریاضی
قضیه}\label{ux645ux62aux646-ux631ux6ccux627ux636ux6cc-ux642ux636ux6ccux647}
فرض کنید \(f: X \to Y\) یک تابع باشد و
\(\{A_\gamma\}_{\gamma \in \Gamma}\) خانواده‌ای از زیرمجموعه‌های دامنه
\(X\) باشد.
\begin{theorembox}{قضیه ۱۰}
\textbf{الف) حفظ دقیق اجتماع:}
\[f\left(\bigcup_{\gamma \in \Gamma} A_\gamma\right) = \bigcup_{\gamma \in \Gamma} f(A_\gamma)\]
\textbf{ب) حفظ شمول در اشتراک:}
\[f\left(\bigcap_{\gamma \in \Gamma} A_\gamma\right) \subseteq \bigcap_{\gamma \in \Gamma} f(A_\gamma)\]
\end{theorembox}
\subsection{\texorpdfstring{۲. اثبات صوری
\lr{(Formal Proof)}}{۲. اثبات صوری }}\label{ux627ux62bux628ux627ux62a-ux635ux648ux631ux6cc-formal-proof}
\subsubsection{اثبات قسمت (الف): تساوی
اجتماع}\label{ux627ux62bux628ux627ux62a-ux642ux633ux645ux62a-ux627ux644ux641-ux62aux633ux627ux648ux6cc-ux627ux62cux62aux645ux627ux639}
برای اثبات تساوی، نشان می‌دهیم گزاره‌نمای عضویت در طرفین هم‌ارز است. نکته
کلیدی، \textbf{جابه‎جایی دو سور وجودی} است.
\begin{info}{برهان}
\[y \in f\left(\bigcup_{\gamma \in \Gamma} A_\gamma\right)\] ۱. طبق
تعریف تصویر مستقیم:
\[\iff \exists x \left( x \in \bigcup_{\gamma \in \Gamma} A_\gamma \wedge f(x) = y \right)\]
۲. طبق تعریف اجتماع تعمیم‌یافته (سور وجودی):
\[\iff \exists x \left( [\exists \gamma \in \Gamma, x \in A_\gamma] \wedge f(x) = y \right)\]
۳. طبق قوانین منطق مرتبه اول، می‌توانیم ترتیب دو سور وجودی را عوض کنیم و
گزاره مستقل (\(f(x)=y\)) را به داخل ببریم:
\[\iff \exists \gamma \in \Gamma \left( \exists x [x \in A_\gamma \wedge f(x) = y] \right)\]
۴. عبارت داخل پرانتز دقیقاً تعریف \(y \in f(A_\gamma)\) است:
\[\iff \exists \gamma \in \Gamma \left( y \in f(A_\gamma) \right)\]
۵. طبق تعریف اجتماع تعمیم‌یافته:
\[\iff y \in \bigcup_{\gamma \in \Gamma} f(A_\gamma)\]
\textbf{نتیجه:} دو مجموعه با هم برابرند.
\end{info}
\subsubsection{اثبات قسمت (ب): شمول
اشتراک}\label{ux627ux62bux628ux627ux62a-ux642ux633ux645ux62a-ux628-ux634ux645ux648ux644-ux627ux634ux62aux631ux627ux6a9}
در اینجا تساوی برقرار نیست زیرا سور وجودی (در تعریف تصویر) و سور عمومی
(در تعریف اشتراک) با هم جابجا نمی‌شوند.
\begin{info}{برهان}
۱. فرض کنید \(y\) عضو دلخواهی از سمت چپ باشد:
\[y \in f\left(\bigcap_{\gamma \in \Gamma} A_\gamma\right)\]
۲. طبق تعریف تصویر، عنصری مانند \(x_0\) وجود دارد که:
\[x_0 \in \bigcap_{\gamma \in \Gamma} A_\gamma \quad \text{و} \quad f(x_0) = y\]
۳. طبق تعریف اشتراک، \(x_0\) در تمام مجموعه‌ها حضور دارد:
\[\forall \gamma \in \Gamma, x_0 \in A_\gamma\]
۴. چون \(x_0 \in A_\gamma\) و \(f(x_0) = y\)، پس برای هر \(\gamma\)،
عنصر \(y\) تصویری از عضوی در \(A_\gamma\) است. بنابراین:
\[\forall \gamma \in \Gamma, y \in f(A_\gamma)\]
۵. طبق تعریف اشتراک تعمیم‌یافته:
\[y \in \bigcap_{\gamma \in \Gamma} f(A_\gamma)\]
\textbf{نتیجه:} \(LHS \subseteq RHS\).
\end{info}
\subsection{\texorpdfstring{۳. شبکه ارتباطی با سایر قضایا
\lr{(Analytic Map)}}{۳. شبکه ارتباطی با سایر قضایا }}\label{ux634ux628ux6a9ux647-ux627ux631ux62aux628ux627ux637ux6cc-ux628ux627-ux633ux627ux6ccux631-ux642ux636ux627ux6ccux627-analytic-map}
\subsubsection{\texorpdfstring{۱. تعمیم
\autoref{قضیه-۹---تصویر-و-تصویر-وارون-مجموعه}}{۱. تعمیم }}\label{ux62aux639ux645ux6ccux645-ux642ux636ux6ccux647-ux6f9---ux62aux635ux648ux6ccux631-ux648-ux62aux635ux648ux6ccux631-ux648ux627ux631ux648ux646-ux645ux62cux645ux648ux639ux647}
\begin{itemize}
\tightlist
\item
  \textbf{گذر از متناهی به نامتناهی:} قضیه ۹ بیان می‌کرد که
  \(f(A \cup B) = f(A) \cup f(B)\). قضیه ۱۰ نشان می‌دهد که این ویژگی جبری
  حتی اگر تعداد مجموعه‌ها بی‌نهایت باشد (خانواده ایندکس‌دار) همچنان برقرار
  است. این ویژگی در توپولوژی برای تعریف پیوستگی با مجموعه‌های باز بسیار
  حیاتی است.
\end{itemize}
\subsubsection{\texorpdfstring{۲. ارتباط با
\autoref{قضیه-۱---پخش‌پذیری-حاصلضرب-دکارتی} (ریشه منطقی عدم
تساوی)}{۲. ارتباط با  (ریشه منطقی عدم تساوی)}}\label{ux627ux631ux62aux628ux627ux637-ux628ux627-ux642ux636ux6ccux647-ux6f1---ux67eux62eux634ux67eux630ux6ccux631ux6cc-ux62dux627ux635ux644ux636ux631ux628-ux62fux6a9ux627ux631ux62aux6cc-ux631ux6ccux634ux647-ux645ux646ux637ux642ux6cc-ux639ux62fux645-ux62aux633ux627ux648ux6cc}
\begin{itemize}
\tightlist
\item
  \textbf{چرا اشتراک مساوی نیست؟} در منطق گزاره‌ها، می‌دانیم که
  \(\exists x (P(x) \wedge Q(x))\) لزوماً هم‌ارز با
  \((\exists x P(x)) \wedge (\exists x Q(x))\) نیست.
  \begin{itemize}
  \tightlist
  \item
    مثال نقض کلاسیک: \(A_1 = \{1\}, A_2 = \{2\}\) و تابع ثابت
    \(f(x) = c\).
  \item
    اشتراک دامنه‌ها تهی است (\(f(\emptyset) = \emptyset\)).
  \item
    اما اشتراک تصاویر \(\{c\} \cap \{c\} = \{c\}\) است.
  \item
    این شکاف منطقی تنها زمانی پر می‌شود که تابع \textbf{یک‌به‌یک} باشد (که
    در قضایای بعدی بررسی می‌شود).
  \end{itemize}
\end{itemize}
\subsubsection{\texorpdfstring{۳. ارتباط با
\autoref{قضیه-۸---لم-چسباندن-(اجتماع-توابع)}}{۳. ارتباط با }}\label{ux627ux631ux62aux628ux627ux637-ux628ux627-ux642ux636ux6ccux647-ux6f8---ux644ux645-ux686ux633ux628ux627ux646ux62fux646-ux627ux62cux62aux645ux627ux639-ux62aux648ux627ux628ux639}
\begin{itemize}
\tightlist
\item
  \textbf{سازگاری با چسباندن:} بخش (الف) این قضیه تضمین می‌کند که اگر
  توابعی را روی دامنه‌های \(A_\gamma\) تعریف کنیم، تصویر نهایی اجتماعِ
  دامنه‌ها، دقیقاً برابر با اجتماعِ بردهای جزئی است. این پایه منطقی برای
  تحلیل توابع چندضابطه‌ای روی دامنه‌های پیچیده است.
\end{itemize}

\clearpage
% ---------------------------------------------------------------------
% Copyright (c) 2026 Arsalan Dalvand & Reyhaneh Darvishi.
% Licensed under CC BY-NC-SA 4.0.
% See LICENSE file for details.
% ---------------------------------------------------------------------

\section{قضیه ۱۱: رفتار جبری تصویر وارون (هم‌ریختی
کامل)}\label{قضیه-۱۱---رفتار-جبری-تصویر-وارون}
\begin{tldr}{خلاصه سریع}
برخلاف «تصویر مستقیم» (\(f\)) که در برخورد با اشتراک دچار ضعف می‌شود،
«تصویر وارون» (\(f^{-1}\)) یک عملگر ایده‌آل است. این عملگر ساختار جبری
اجتماع و اشتراک را دقیقاً حفظ می‌کند و با هر دو عملگر جابجا می‌شود.
\end{tldr}
\subsection{۱. متن ریاضی
قضیه}\label{ux645ux62aux646-ux631ux6ccux627ux636ux6cc-ux642ux636ux6ccux647}
فرض کنید \(f: X \to Y\) یک تابع باشد و
\(\{B_\gamma\}_{\gamma \in \Gamma}\) خانواده‌ای از زیرمجموعه‌های هم‌دامنه
\(Y\) باشد.
\begin{theorembox}{قضیه ۱۱}
\textbf{الف) پخش‌پذیری کامل بر اجتماع:}
\[f^{-1}\left(\bigcup_{\gamma \in \Gamma} B_\gamma\right) = \bigcup_{\gamma \in \Gamma} f^{-1}(B_\gamma)\]
\textbf{ب) پخش‌پذیری کامل بر اشتراک:}
\[f^{-1}\left(\bigcap_{\gamma \in \Gamma} B_\gamma\right) = \bigcap_{\gamma \in \Gamma} f^{-1}(B_\gamma)\]
\end{theorembox}
\subsection{\texorpdfstring{۲. اثبات صوری
\lr{(Formal Proof)}}{۲. اثبات صوری }}\label{ux627ux62bux628ux627ux62a-ux635ux648ux631ux6cc-formal-proof}
اثبات این قضیه بر پایه «تعریف تصویر وارون» و «قوانین منطق مرتبه اول»
استوار است. زیبایی این اثبات در این است که تصویر وارون، گزاره‌های عضویت
را بدون تغییر ساختار منطقی منتقل می‌کند.
\subsubsection{اثبات قسمت (الف):
اجتماع}\label{ux627ux62bux628ux627ux62a-ux642ux633ux645ux62a-ux627ux644ux641-ux627ux62cux62aux645ux627ux639}
\begin{info}{برهان}
\[x \in f^{-1}\left(\bigcup_{\gamma \in \Gamma} B_\gamma\right)\] ۱. طبق
تعریف تصویر وارون (\(x \in f^{-1}(S) \iff f(x) \in S\)):
\[\iff f(x) \in \bigcup_{\gamma \in \Gamma} B_\gamma\]
۲. طبق تعریف اجتماع تعمیم‌یافته:
\[\iff \exists \gamma \in \Gamma, (f(x) \in B_\gamma)\]
۳. طبق تعریف تصویر وارون (بازگشت به دامنه):
\[\iff \exists \gamma \in \Gamma, (x \in f^{-1}(B_\gamma))\]
۴. طبق تعریف اجتماع تعمیم‌یافته:
\[\iff x \in \bigcup_{\gamma \in \Gamma} f^{-1}(B_\gamma)\]
\textbf{نتیجه:} تساوی برقرار است.
\end{info}
\subsubsection{اثبات قسمت (ب):
اشتراک}\label{ux627ux62bux628ux627ux62a-ux642ux633ux645ux62a-ux628-ux627ux634ux62aux631ux627ux6a9}
\begin{info}{برهان}
\[x \in f^{-1}\left(\bigcap_{\gamma \in \Gamma} B_\gamma\right)\] ۱. طبق
تعریف تصویر وارون:
\[\iff f(x) \in \bigcap_{\gamma \in \Gamma} B_\gamma\]
۲. طبق تعریف اشتراک تعمیم‌یافته:
\[\iff \forall \gamma \in \Gamma, (f(x) \in B_\gamma)\]
۳. طبق تعریف تصویر وارون:
\[\iff \forall \gamma \in \Gamma, (x \in f^{-1}(B_\gamma))\]
۴. طبق تعریف اشتراک تعمیم‌یافته:
\[\iff x \in \bigcap_{\gamma \in \Gamma} f^{-1}(B_\gamma)\]
\textbf{نتیجه:} تساوی برقرار است.
\end{info}
\subsection{\texorpdfstring{۳. شبکه ارتباطی با سایر قضایا
\lr{(Analytic Map)}}{۳. شبکه ارتباطی با سایر قضایا }}\label{ux634ux628ux6a9ux647-ux627ux631ux62aux628ux627ux637ux6cc-ux628ux627-ux633ux627ux6ccux631-ux642ux636ux627ux6ccux627-analytic-map}
\subsubsection{\texorpdfstring{۱. مقایسه با
\autoref{قضیه-۱۰---تعمیم-تصویر-اجتماع-و-اشتراک}}{۱. مقایسه با }}\label{ux645ux642ux627ux6ccux633ux647-ux628ux627-ux642ux636ux6ccux647-ux6f1ux6f0---ux62aux639ux645ux6ccux645-ux62aux635ux648ux6ccux631-ux627ux62cux62aux645ux627ux639-ux648-ux627ux634ux62aux631ux627ux6a9}
\begin{itemize}
\tightlist
\item
  \textbf{تفاوت بنیادین:} در قضیه ۱۰ دیدیم که
  \(f(\cap A_\gamma) \subseteq \cap f(A_\gamma)\) و تساوی لزوماً برقرار
  نیست. اما قضیه ۱۱ نشان می‌دهد که \(f^{-1}\) هیچ‌گونه اطلاعاتی را در
  فرایند اشتراک‌گیری از دست نمی‌دهد.
\item
  \textbf{علت منطقی:} دلیل این تفاوت در ساختار سورهاست. تعریف \(f(A)\)
  شامل یک سور وجودی (\(\exists x\)) است که با سور عمومی (\(\forall\)) در
  اشتراک سازگار نیست. اما تعریف \(f^{-1}(B)\) هیچ سور مخفی‌ای ندارد و
  صرفاً یک ترجمه شرطی (\(P(f(x))\)) است، لذا با همه سورها جابجا می‌شود.
\end{itemize}
\subsubsection{\texorpdfstring{۲. ارتباط با
\autoref{قضیه-۹---تصویر-و-تصویر-وارون-مجموعه}}{۲. ارتباط با }}\label{ux627ux631ux62aux628ux627ux637-ux628ux627-ux642ux636ux6ccux647-ux6f9---ux62aux635ux648ux6ccux631-ux648-ux62aux635ux648ux6ccux631-ux648ux627ux631ux648ux646-ux645ux62cux645ux648ux639ux647}
\begin{itemize}
\tightlist
\item
  \textbf{تعمیم:} قضیه ۱۱ تعمیم مستقیم قضیه ۹ (اگر آن را برای وارون
  بنویسیم) به حالت نامتناهی است. این نشان می‌دهد ویژگی‌های توپولوژیکی
  وارون تابع (مانند پیوستگی که با باز بودن وارون مجموعه‌های باز تعریف
  می‌شود) بسیار پایدارتر از تصویر مستقیم هستند.
\end{itemize}

\clearpage
% ---------------------------------------------------------------------
% Copyright (c) 2026 Arsalan Dalvand & Reyhaneh Darvishi.
% Licensed under CC BY-NC-SA 4.0.
% See LICENSE file for details.
% ---------------------------------------------------------------------

\section{قضیه ۱۲: پخش‌پذیری تصویر وارون بر
تفاضل}\label{قضیه-۱۲---پخش‌پذیری-تصویر-وارون-بر-تفاضل}
\begin{tldr}{خلاصه سریع}
این قضیه پازل «خوش‌رفتاری» عملگر تصویر وارون (\(f^{-1}\)) را تکمیل می‌کند.
همانطور که تصویر وارون با اجتماع و اشتراک جابجا می‌شد، با عملگر تفاضل
(\(-\)) نیز جابجا می‌شود. این یعنی عملیات بولی در دامنه و هم‌دامنه تحت
\(f^{-1}\) ایزومورف هستند.
\end{tldr}
\subsection{۱. متن ریاضی
قضیه}\label{ux645ux62aux646-ux631ux6ccux627ux636ux6cc-ux642ux636ux6ccux647}
فرض کنید \(f: X \to Y\) یک تابع باشد و \(B, C\) دو زیرمجموعه از \(Y\)
باشند.
\begin{theorembox}{قضیه ۱۲}
\[f^{-1}(B - C) = f^{-1}(B) - f^{-1}(C)\]
\end{theorembox}
\subsection{\texorpdfstring{۲. اثبات صوری
\lr{(Formal Proof)}}{۲. اثبات صوری }}\label{ux627ux62bux628ux627ux62a-ux635ux648ux631ux6cc-formal-proof}
برای اثبات، از هم‌ارزی‌های منطقی استفاده می‌کنیم. نکته کلیدی، رفتار نقیض
(\(x \notin A\)) در تعریف تصویر وارون است.
\begin{info}{برهان}
\[x \in f^{-1}(B - C)\] ۱. طبق تعریف تصویر وارون:
\[\iff f(x) \in (B - C)\]
۲. طبق تعریف تفاضل مجموعه‌ها: \[\iff f(x) \in B \wedge f(x) \notin C\]
۳. تحلیل گزاره دوم (\(f(x) \notin C\)): این گزاره دقیقاً نقیض
\(f(x) \in C\) است. \[f(x) \in C \iff x \in f^{-1}(C)\] پس طبق قانون عکس
نقیض در عضویت: \[f(x) \notin C \iff x \notin f^{-1}(C)\]
۴. جایگذاری در عبارت اصلی:
\[\iff x \in f^{-1}(B) \wedge x \notin f^{-1}(C)\]
۵. طبق تعریف تفاضل مجموعه‌ها: \[\iff x \in f^{-1}(B) - f^{-1}(C)\]
\textbf{نتیجه:} دو مجموعه با هم برابرند.
\end{info}
\subsection{\texorpdfstring{۳. شبکه ارتباطی با سایر قضایا
\lr{(Analytic Map)}}{۳. شبکه ارتباطی با سایر قضایا }}\label{ux634ux628ux6a9ux647-ux627ux631ux62aux628ux627ux637ux6cc-ux628ux627-ux633ux627ux6ccux631-ux642ux636ux627ux6ccux627-analytic-map}
\subsubsection{\texorpdfstring{۱. تضاد با تصویر مستقیم
\lr{(Direct Image)}}{۱. تضاد با تصویر مستقیم }}\label{ux62aux636ux627ux62f-ux628ux627-ux62aux635ux648ux6ccux631-ux645ux633ux62aux642ux6ccux645-direct-image}
\begin{itemize}
\tightlist
\item
  \textbf{عدم برقراری برای \(f\):} برای تصویر مستقیم، رابطه
  \(f(A - B) = f(A) - f(B)\) \textbf{غلط} است.
  \begin{itemize}
  \tightlist
  \item
    \emph{مثال نقض:} \(f\) یک تابع ثابت روی \(X=\{1,2\}\) باشد
    (\(f(1)=f(2)=c\)). اگر \(A=\{1,2\}\) و \(B=\{2\}\)، آنگاه
    \(A-B=\{1\}\) و \(f(A-B)=\{c\}\). اما \(f(A)=\{c\}\) و
    \(f(B)=\{c\}\)، پس \(f(A)-f(B)=\emptyset\).
  \item
    قضیه ۱۲ نشان می‌دهد که \(f^{-1}\) از این مشکل مبراست.
  \end{itemize}
\end{itemize}
\subsubsection{\texorpdfstring{۲. ارتباط با
\autoref{قضیه-۲---پخش‌پذیری-حاصلضرب-دکارتی-بر-تفاضل}}{۲. ارتباط با }}\label{ux627ux631ux62aux628ux627ux637-ux628ux627-ux642ux636ux6ccux647-ux6f2---ux67eux62eux634ux67eux630ux6ccux631ux6cc-ux62dux627ux635ux644ux636ux631ux628-ux62fux6a9ux627ux631ux62aux6cc-ux628ux631-ux62aux641ux627ux636ux644}
\begin{itemize}
\tightlist
\item
  \textbf{تشابه ساختاری:} در فصل ۳ دیدیم که
  \(A \times (B - C) = (A \times B) - (A \times C)\). قضیه ۱۲ نشان می‌دهد
  که \(f^{-1}\) نیز مانند حاصلضرب دکارتی، خاصیت پخش‌پذیری بر تفاضل دارد.
  این رفتار خطی نسبت به عملیات مجموعه‌ای، ویژگی عملگرهای «خوش‌تعریف» در
  جبر مجموعه‌هاست.
\end{itemize}
\subsubsection{\texorpdfstring{۳. ارتباط با
\autoref{قضیه-۵---متمم-و-زیرمجموعه}
(متمم‌گیری)}{۳. ارتباط با  (متمم‌گیری)}}\label{ux627ux631ux62aux628ux627ux637-ux628ux627-ux642ux636ux6ccux647-ux6f5-ux641ux635ux644-ux6f2-ux645ux62aux645ux645ux6afux6ccux631ux6cc}
\begin{itemize}
\tightlist
\item
  \textbf{نتیجه فرعی:} اگر در قضیه ۱۲، مجموعه \(B\) را برابر با مجموعه
  مرجع \(Y\) بگیریم (\(B=Y\))، به قانون متمم می‌رسیم:
  \[f^{-1}(Y - C) = f^{-1}(Y) - f^{-1}(C) \implies f^{-1}(C') = (f^{-1}(C))'\]
  این یعنی تصویر وارونِ متمم، برابر است با متممِ تصویر وارون.
\end{itemize}

\clearpage
% ---------------------------------------------------------------------
% Copyright (c) 2026 Arsalan Dalvand & Reyhaneh Darvishi.
% Licensed under CC BY-NC-SA 4.0.
% See LICENSE file for details.
% ---------------------------------------------------------------------

\section{قضیه ۱۳: حفظ اشتراک در توابع
یک‌به‌یک}\label{قضیه-۱۳---حفظ-اشتراک-در-توابع-یک‌به‌یک}
\begin{tldr}{خلاصه سریع}
در حالت کلی، تصویر تابع با اشتراک جابجا نمی‌شود
(\(f(A \cap B) \neq f(A) \cap f(B)\)). این قضیه شرط لازم و کافی برای
برقراری این تساوی را بیان می‌کند: تابع باید \textbf{یک‌به‌یک
\lr{(Injective)}} باشد. در این صورت، ساختار اشتراک دقیقاً در هم‌دامنه حفظ
می‌شود.
\end{tldr}
\subsection{۱. متن ریاضی
قضیه}\label{ux645ux62aux646-ux631ux6ccux627ux636ux6cc-ux642ux636ux6ccux647}
فرض کنید \(f: X \to Y\) یک تابع باشد و
\(\{A_\gamma\}_{\gamma \in \Gamma}\) خانواده‌ای از زیرمجموعه‌های دامنه
\(X\) باشد. اگر \(f\) \textbf{یک‌به‌یک} باشد، آنگاه:
\begin{theorembox}{قضیه ۱۳}
\[f\left(\bigcap_{\gamma \in \Gamma} A_\gamma\right) = \bigcap_{\gamma \in \Gamma} f(A_\gamma)\]
\end{theorembox}
\subsection{\texorpdfstring{۲. اثبات صوری
\lr{(Formal Proof)}}{۲. اثبات صوری }}\label{ux627ux62bux628ux627ux62a-ux635ux648ux631ux6cc-formal-proof}
برای اثبات تساوی دو مجموعه، از روش شمول دوطرفه استفاده می‌کنیم.
\subsubsection{\texorpdfstring{بخش اول: شمول مستقیم
(\(\subseteq\))}{بخش اول: شمول مستقیم (\textbackslash subseteq)}}\label{ux628ux62eux634-ux627ux648ux644-ux634ux645ux648ux644-ux645ux633ux62aux642ux6ccux645-subseteq}
این بخش برای \textbf{هر تابعی} (چه یک‌به‌یک باشد چه نباشد) صادق است و در
\textbf{\autoref{قضیه-۱۰---تعمیم-تصویر-اجتماع-و-اشتراک}} اثبات شده است:
\[f\left(\bigcap_{\gamma \in \Gamma} A_\gamma\right) \subseteq \bigcap_{\gamma \in \Gamma} f(A_\gamma)\]
\subsubsection{\texorpdfstring{بخش دوم: شمول معکوس
(\(\supseteq\))}{بخش دوم: شمول معکوس (\textbackslash supseteq)}}\label{ux628ux62eux634-ux62fux648ux645-ux634ux645ux648ux644-ux645ux639ux6a9ux648ux633-supseteq}
این بخش نیازمند فرض \textbf{یک‌به‌یک} بودن تابع است.
\begin{info}{برهان}
۱. فرض کنیم \(y\) عضو دلخواهی از سمت راست تساوی باشد:
\[y \in \bigcap_{\gamma \in \Gamma} f(A_\gamma)\]
۲. طبق تعریف اشتراک تعمیم‌یافته:
\[\forall \gamma \in \Gamma, \quad y \in f(A_\gamma)\]
۳. طبق تعریف تصویر مجموعه، برای هر اندیس \(\gamma\)، باید عنصری در
\(A_\gamma\) وجود داشته باشد که تصویرش \(y\) شود. فرض کنیم برای هر
\(\gamma\)، این عنصر \(x_\gamma\) باشد:
\[\forall \gamma \in \Gamma, \exists x_\gamma \in A_\gamma : f(x_\gamma) = y\]
۴. \textbf{کاربرد فرض یک‌به‌یک:} اکنون مجموعه‌ای از پیش‌نگاره‌ها داریم
\(\{x_\gamma\}_{\gamma \in \Gamma}\) که همگی توسط \(f\) به \(y\) نگاشته
می‌شوند. چون \(f\) تابع است و مقدار \(y\) ثابت است، و مهم‌تر از آن چون
\(f\) \textbf{یک‌به‌یک} است، تمام این پیش‌نگاره‌ها باید یکسان باشند (تابع
یک‌به‌یک نمی‌تواند دو ورودی متمایز را به یک خروجی ببرد).
\[f(x_\alpha) = y \wedge f(x_\beta) = y \implies f(x_\alpha) = f(x_\beta) \xrightarrow{\text{1-1}} x_\alpha = x_\beta\]
بنابراین یک عنصر یکتا (مثلاً \(x_0\)) وجود دارد که:
\[x_0 = x_\gamma, \quad \forall \gamma \in \Gamma\]
۵. چون \(x_0\) همان \(x_\gamma\) است و \(x_\gamma \in A_\gamma\)، پس:
\[\forall \gamma \in \Gamma, \quad x_0 \in A_\gamma\]
۶. طبق تعریف اشتراک، \(x_0\) در اشتراک خانواده است:
\[x_0 \in \bigcap_{\gamma \in \Gamma} A_\gamma\]
۷. چون \(f(x_0) = y\)، پس:
\[y \in f\left(\bigcap_{\gamma \in \Gamma} A_\gamma\right)\]
\textbf{نتیجه:} \(\text{RHS} \subseteq \text{LHS}\).
\end{info}
\subsection{\texorpdfstring{۳. شبکه ارتباطی با سایر قضایا
\lr{(Analytic Map)}}{۳. شبکه ارتباطی با سایر قضایا }}\label{ux634ux628ux6a9ux647-ux627ux631ux62aux628ux627ux637ux6cc-ux628ux627-ux633ux627ux6ccux631-ux642ux636ux627ux6ccux627-analytic-map}
\subsubsection{\texorpdfstring{۱. تکمیل
\autoref{قضیه-۱۰---تعمیم-تصویر-اجتماع-و-اشتراک}}{۱. تکمیل }}\label{ux62aux6a9ux645ux6ccux644-ux642ux636ux6ccux647-ux6f1ux6f0---ux62aux639ux645ux6ccux645-ux62aux635ux648ux6ccux631-ux627ux62cux62aux645ux627ux639-ux648-ux627ux634ux62aux631ux627ux6a9}
\begin{itemize}
\tightlist
\item
  \textbf{رفع نقص:} قضیه ۱۰ بیان می‌کرد که
  \(f(\cap A) \subseteq \cap f(A)\). قضیه ۱۳ متمم آن است و نشان می‌دهد که
  ``یک‌به‌یک بودن'' شرط کافی برای تبدیل این زیرمجموعه به تساوی است. در
  واقع، ناتوانی تابع معمولی در حفظ اشتراک، ناشی از ``تلاقی''
  \lr{(Collision) }ورودی‌های متمایز در یک خروجی مشترک است که در توابع
  یک‌به‌یک رخ نمی‌دهد.
\end{itemize}
\subsubsection{\texorpdfstring{۲. ارتباط با
\autoref{انواع-توابع---یک‌به‌یک-پوشا-و-دوسویی}}{۲. ارتباط با }}\label{ux627ux631ux62aux628ux627ux637-ux628ux627-ux627ux646ux648ux627ux639-ux62aux648ux627ux628ux639---ux6ccux6a9ux628ux647ux6ccux6a9-ux67eux648ux634ux627-ux648-ux62fux648ux633ux648ux6ccux6cc}
\begin{itemize}
\tightlist
\item
  \textbf{تعریف کاربردی:} اثبات این قضیه یکی از مهم‌ترین کاربردهای تعریف
  صوری یک‌به‌یک بودن (\(f(x_1)=f(x_2) \implies x_1=x_2\)) در نظریه
  مجموعه‌هاست. در گام ۴ اثبات، دقیقاً از همین استلزام منطقی برای یکی کردن
  تمام \(x_\gamma\)ها استفاده شد.
\end{itemize}
\subsubsection{\texorpdfstring{۳. ارتباط با
\autoref{قضیه-۱۱---رفتار-جبری-تصویر-وارون}}{۳. ارتباط با }}\label{ux627ux631ux62aux628ux627ux637-ux628ux627-ux642ux636ux6ccux647-ux6f1ux6f1---ux631ux641ux62aux627ux631-ux62cux628ux631ux6cc-ux62aux635ux648ux6ccux631-ux648ux627ux631ux648ux646}
\begin{itemize}
\tightlist
\item
  \textbf{تقارن جبری:} تصویر وارون (\(f^{-1}\)) همیشه اشتراک را حفظ
  می‌کند (\(f^{-1}(\cap B) = \cap f^{-1}(B)\)). قضیه ۱۳ نشان می‌دهد که اگر
  \(f\) یک‌به‌یک باشد، تصویر مستقیم (\(f\)) نیز رفتاری مشابه تصویر وارون
  پیدا می‌کند و ``هم‌ریختی'' \lr{(Homomorphism) }کامل نسبت به عملیات
  مجموعه‌ای برقرار می‌شود.
\end{itemize}

\clearpage
% ---------------------------------------------------------------------
% Copyright (c) 2026 Arsalan Dalvand & Reyhaneh Darvishi.
% Licensed under CC BY-NC-SA 4.0.
% See LICENSE file for details.
% ---------------------------------------------------------------------

\section{\texorpdfstring{قضیه: رابطه مجموعه \lr{A }و نگاره وارونِ نگاره
آن}{قضیه: رابطه مجموعه و نگاره وارونِ نگاره آن}}\label{تمرین-۹-(الف)---زیرمجموعه-بودن-A-در-وارون-تصویر-A}
\begin{tldr}{خلاصه سریع}
اگر یک مجموعه را با تابع \(f\) بفرستیم به مقصد و دوباره با \(f^{-1}\)
برگردانیم، لزوماً به همان مجموعه اولیه نمی‌رسیم؛ بلکه ممکن است مجموعه‌ای
\textbf{بزرگتر} شود که مجموعه اصلی ما زیرمجموعه آن است.
\[A \subseteq f^{-1}(f(A))\]
\end{tldr}
\subsection{۱. درک
شهودی}\label{ux62fux631ux6a9-ux634ux647ux648ux62fux6cc}
فرض کن \(A\) مجموعه‌ای از دانش‌آموزان یک کلاس خاص در مدرسه است و تابع
\(f\) به هر دانش‌آموز، «شماره پلاک منزل» او را نسبت می‌دهد.
\begin{enumerate}
\def\labelenumi{\arabic{enumi}.}
\tightlist
\item
  \(f(A)\): می‌شود مجموعه پلاک‌های این دانش‌آموزان.
\item
  \(f^{-1}(f(A))\): یعنی تمام کسانی در کل شهر (دامنه \(X\)) که پلاکشان
  جزو لیست بالا باشد.
\end{enumerate}
ممکن است در شهر، افراد دیگری (خارج از کلاس \(A\)) هم باشند که پلاکشان
مشابه یکی از بچه‌های کلاس باشد (اگر تابع یک‌به‌یک نباشد). پس وقتی
برمی‌گردیم، جمعیت ما برابر یا بزرگتر از کلاس \(A\) خواهد بود.
\subsection{۲. صورت
ریاضی}\label{ux635ux648ux631ux62a-ux631ux6ccux627ux636ux6cc}
\begin{theorembox}{قضیه}
فرض کنید \(f:X \rightarrow Y\) یک تابع باشد و \(A \subseteq X\). در این
صورت: \[A \subseteq f^{-1}(f(A))\]
\end{theorembox}
\subsection{۳. اثبات
دقیق}\label{ux627ux62bux628ux627ux62a-ux62fux642ux6ccux642}
\begin{info}{اثبات}
می‌خواهیم نشان دهیم هر عضوی که در \(A\) است، در طرف راست هم هست.
\begin{enumerate}
\def\labelenumi{\arabic{enumi}.}
\tightlist
\item
  فرض کنید \(x\) عضوی دلخواه از \(A\) باشد (\(x \in A\)).
\item
  طبق تعریف تابع، تصویر این عضو یعنی \(f(x)\) باید در مجموعه تصاویر
  \(A\) باشد: \[f(x) \in f(A)\]
\item
  حال به تعریف \textbf{نگاره وارون} دقت کنید:
  \(f^{-1}(S) = \{x \in X | f(x) \in S\}\).
\item
  چون \(f(x)\) متعلق به مجموعه \(f(A)\) است، پس خودِ \(x\) باید متعلق به
  وارونِ آن مجموعه باشد: \[x \in f^{-1}(f(A))\]
\item
  \textbf{نتیجه:} چون \(x\) نماینده هر عضو دلخواه \(A\) بود، پس:
  \[A \subseteq f^{-1}(f(A))\]
\end{enumerate}
\end{info}
\begin{info}{نکته تکمیلی}
تساوی \(A = f^{-1}(f(A))\) تنها زمانی برقرار است که تابع \(f\)
\textbf{یک‌به‌یک \lr{(Injective)}} باشد.
\end{info}

\clearpage
% ---------------------------------------------------------------------
% Copyright (c) 2026 Arsalan Dalvand & Reyhaneh Darvishi.
% Licensed under CC BY-NC-SA 4.0.
% See LICENSE file for details.
% ---------------------------------------------------------------------

\section{\texorpdfstring{قضیه: رابطه نگاره‌یِ نگاره وارون \lr{B }با خود
\lr{B}}{قضیه: رابطه نگاره‌یِ نگاره وارون با خود }}\label{تمرین-۹-(ب)---زیرمجموعه-بودن-تصویر-وارون-B-در-B}
\begin{tldr}{خلاصه سریع}
اگر از مقصد (\(Y\))، مجموعه‌ای را با \(f^{-1}\) به مبدا بکشیم و دوباره با
\(f\) به مقصد پرتاب کنیم، حتماً درونِ مجموعه اولیه می‌افتیم.
\[f(f^{-1}(B)) \subseteq B\]
\end{tldr}
\subsection{۱. درک
شهودی}\label{ux62fux631ux6a9-ux634ux647ux648ux62fux6cc}
فرض کن \(B\) لیست «نمرات قابل قبول» (مثلاً ۱۰ تا ۲۰) است.
\begin{enumerate}
\def\labelenumi{\arabic{enumi}.}
\tightlist
\item
  \(f^{-1}(B)\): لیست تمام دانشجویانی است که نمره‌شان بین ۱۰ تا ۲۰ شده.
\item
  \(f(f^{-1}(B))\): حالا نمراتِ همین دانشجویانِ انتخاب شده را دوباره نگاه
  می‌کنیم.
\end{enumerate}
مشخص است که نمرات این افراد، قطعاً جزوی از همان بازه ۱۰ تا ۲۰ (\(B\))
است. اما ممکن است برخی نمرات در \(B\) باشد که هیچکس نگرفته باشد، پس شاید
کل \(B\) پوشش داده نشود (مگر تابع پوشا باشد).
\subsection{۲. صورت
ریاضی}\label{ux635ux648ux631ux62a-ux631ux6ccux627ux636ux6cc}
\begin{theorembox}{قضیه}
فرض کنید \(f:X \rightarrow Y\) و \(B \subseteq Y\). در این صورت:
\[f(f^{-1}(B)) \subseteq B\]
\end{theorembox}
\subsection{۳. اثبات
دقیق}\label{ux627ux62bux628ux627ux62a-ux62fux642ux6ccux642}
\begin{info}{اثبات}
برای اثبات \(S_1 \subseteq S_2\)، باید نشان دهیم اگر \(y \in S_1\) آنگاه
\(y \in S_2\).
\begin{enumerate}
\def\labelenumi{\arabic{enumi}.}
\tightlist
\item
  فرض کنید \(y \in f(f^{-1}(B))\) باشد.
\item
  طبق تعریف تصویر تابع، یعنی باید یک \(x\) در دامنه (در اینجا دامنه ما
  مجموعه \(f^{-1}(B)\) است) وجود داشته باشد که:
  \[y = f(x) \quad \text{for some } x \in f^{-1}(B)\]
\item
  حالا عبارت \(x \in f^{-1}(B)\) را باز می‌کنیم. طبق تعریف وارون، این
  یعنی: \[f(x) \in B\]
\item
  از طرفی در مرحله (۲) گفتیم \(y = f(x)\).
\item
  پس با جایگذاری داریم: \[y \in B\]
\item
  \textbf{نتیجه:} تمام اعضای سمت چپ، در سمت راست هستند.
  \[f(f^{-1}(B)) \subseteq B\]
\end{enumerate}
\end{info}
\begin{info}{نکته تکمیلی}
تساوی \(f(f^{-1}(B)) = B\) تنها زمانی برقرار است که تابع \(f\)
\textbf{پوشا \lr{(Surjective)}} باشد (یا حداقل مجموعه \(B\) زیرمجموعه‌ای
از برد تابع باشد).
\end{info}

\clearpage
% ---------------------------------------------------------------------
% Copyright (c) 2026 Arsalan Dalvand & Reyhaneh Darvishi.
% Licensed under CC BY-NC-SA 4.0.
% See LICENSE file for details.
% ---------------------------------------------------------------------

\section{قضیه: رفتار نگاره وارون نسبت به متمم
(تفاضل)}\label{تمرین-۱۰---تساوی-وارون-متمم-B-با-متمم-وارون-B}
\begin{tldr}{خلاصه سریع}
نگاره وارون (برخلافِ خودِ نگاره) رفتار بسیار خوش‌رفتاری دارد و عملگرهای
مجموعه (اجتماع، اشتراک و تفاضل) را حفظ می‌کند. این قضیه برای تفاضل (متمم)
است. \[f^{-1}(Y - B) = X - f^{-1}(B)\]
\end{tldr}
\subsection{۱. صورت ریاضی (تمرین
۱۰)}\label{ux635ux648ux631ux62a-ux631ux6ccux627ux636ux6cc-ux62aux645ux631ux6ccux646-ux6f1ux6f0}
\begin{theorembox}{قضیه}
فرض کنید \(f:X \rightarrow Y\) و \(B \subseteq Y\). ثابت کنید وارونِ متممِ
\lr{B }برابر است با متممِ وارونِ
\lr{B: }\[f^{-1}(Y - B) = f^{-1}(Y) - f^{-1}(B)\] \emph{(نکته: می‌دانیم
\(f^{-1}(Y) = X\) است)}
\end{theorembox}
\subsection{۲. اثبات
جبری}\label{ux627ux62bux628ux627ux62a-ux62cux628ux631ux6cc}
\begin{info}{اثبات}
نشان می‌دهیم یک عضو دلخواه \(x\) اگر در سمت چپ باشد، در سمت راست هم هست و
بالعکس.
\[x \in f^{-1}(Y - B)\]
\(\equiv\) طبق تعریف وارون: \[f(x) \in (Y - B)\]
\(\equiv\) طبق تعریف تفاضل مجموعه‌ها: \[f(x) \in Y \land f(x) \notin B\]
\(\equiv\) گزاره \(f(x) \in Y\) برای هر \(x \in X\) همواره صادق است (چون
\(Y\) هم‌دامنه است). پس می‌توانیم آن را به زبان وارون بنویسیم
(\(x \in f^{-1}(Y)\)). همچنین \(f(x) \notin B\) معادل است با
\(x \notin f^{-1}(B)\): \[x \in f^{-1}(Y) \land x \notin f^{-1}(B)\]
\(\equiv\) طبق تعریف تفاضل: \[x \in f^{-1}(Y) - f^{-1}(B)\]
\(\blacksquare\)
\end{info}
\begin{note}{یادآوری مهم}
چرا \(f^{-1}(Y) = X\)؟ چون دامنه تابع \(f\) برابر با \(X\) است، پس هر
\(x \in X\) قطعاً تصویری در \(Y\) دارد (\(f(x) \in Y\)). بنابراین تمام
\(x\)ها در تعریف وارونِ \(Y\) صدق می‌کنند.
\end{note}

\clearpage
% ---------------------------------------------------------------------
% Copyright (c) 2026 Arsalan Dalvand & Reyhaneh Darvishi.
% Licensed under CC BY-NC-SA 4.0.
% See LICENSE file for details.
% ---------------------------------------------------------------------

\section{مثال نقض: عدم تساوی تصویر تفاضل با تفاضل
تصاویر}\label{تمرین-۱۱---مثال-نقض-تصویر-تفاضل-مجموعه‌ها}
\begin{tldr}{خلاصه سریع}
برخلاف «نگاره وارون»، خودِ «نگاره» (\(f\)) تفاضل را حفظ نمی‌کند.
\[f(A - B) \neq f(A) - f(B)\] معمولاً \(f(A - B)\) بزرگتر یا مساوی طرف
راست است، اما تساوی کلی برقرار نیست.
\end{tldr}
\subsection{۱. صورت سوال (تمرین
۱۱)}\label{ux635ux648ux631ux62a-ux633ux648ux627ux644-ux62aux645ux631ux6ccux646-ux6f1ux6f1}
\begin{info}{سوال}
تابعی مثال بزنید که نشان دهد رابطه \(f(A - B) = f(A) - f(B)\) نادرست
است.
\end{info}
\subsection{۲. استراتژی حل (ساخت مثال
نقض)}\label{ux627ux633ux62aux631ux627ux62aux698ux6cc-ux62dux644-ux633ux627ux62eux62a-ux645ux62bux627ux644-ux646ux642ux636}
برای ساخت مثال نقض، باید حالتی را پیدا کنیم که دو عضو متفاوت، تصویر
یکسان داشته باشند (تابع یک‌به‌یک نباشد). اگر عضوی در \(A-B\) باشد اما
تصویرش با تصویر عضوی در \(B\) یکی شود، در سمت راست معادله (تفاضل) حذف
می‌شود، اما در سمت چپ باقی می‌ماند.
\subsection{۳. حل
تشریحی}\label{ux62dux644-ux62aux634ux631ux6ccux62dux6cc}
\begin{note}{مثال نقض}
\textbf{فرض‌ها:}
\begin{itemize}
\tightlist
\item
  مجموعه مبدا: \(X = \{1, 2, 3\}\)
\item
  مجموعه مقصد: \(Y = \{a, b\}\)
\item
  زیرمجموعه‌ها: \(A = \{1, 2\}\) و \(B = \{2\}\)
\item
  تعریف تابع \(f\):
  \begin{itemize}
  \tightlist
  \item
    \(f(1) = a\)
  \item
    \(f(2) = a\)
  \item
    \(f(3) = b\)
  \end{itemize}
\end{itemize}
\textbf{محاسبه سمت چپ \lr{(LHS):}}
\begin{enumerate}
\def\labelenumi{\arabic{enumi}.}
\tightlist
\item
  ابتدا \(A - B\) را حساب می‌کنیم: \[A - B = \{1, 2\} - \{2\} = \{1\}\]
\item
  حالا تصویر آن را می‌گیریم: \[f(A - B) = f(\{1\}) = \{a\}\]
\end{enumerate}
\textbf{محاسبه سمت راست \lr{(RHS):}}
\begin{enumerate}
\def\labelenumi{\arabic{enumi}.}
\tightlist
\item
  تصویر \(A\): \[f(A) = \{f(1), f(2)\} = \{a, a\} = \{a\}\]
\item
  تصویر \(B\): \[f(B) = \{f(2)\} = \{a\}\]
\item
  تفاضل تصاویر: \[f(A) - f(B) = \{a\} - \{a\} = \emptyset\]
\end{enumerate}
\textbf{نتیجه‌گیری:}
\[\{a\} \neq \emptyset \implies f(A - B) \neq f(A) - f(B)\]
\end{note}
\begin{warning}{تحلیل علت}
مشکل اینجا بود که \(1\) و \(2\) هر دو به \(a\) می‌روند.
\begin{itemize}
\tightlist
\item
  عضو \(1\) در \(A-B\) هست، پس \(a\) در تصویرِ چپ هست.
\item
  اما چون \(2\) در \(B\) هست و تصویرش هم \(a\) است، باعث می‌شود \(a\) از
  تصویرِ \(A\) (در سمت راست) حذف شود.
\end{itemize}
\end{warning}

\clearpage
% ---------------------------------------------------------------------
% Copyright (c) 2026 Arsalan Dalvand & Reyhaneh Darvishi.
% Licensed under CC BY-NC-SA 4.0.
% See LICENSE file for details.
% ---------------------------------------------------------------------

\section{قضیه: تصویرِ اشتراک یک مجموعه با وارون یک مجموعه
دیگر}\label{تمرین-تکمیلی---تساوی-تصویر-اشتراک-A-و-وارون-B}
\begin{tldr}{خلاصه سریع}
این فرمول نشان می‌دهد که عملگر تصویر \(f\) نسبت به اشتراک با یک «نگاره
وارون» چگونه رفتار می‌کند. می‌توان \(B\) را از داخل پرانتزِ تصویر بیرون
کشید. \[f(A \cap f^{-1}(B)) = f(A) \cap B\]
\end{tldr}
\subsection{۱. صورت
ریاضی}\label{ux635ux648ux631ux62a-ux631ux6ccux627ux636ux6cc}
\begin{theorembox}{قضیه}
فرض کنید \(f:X \rightarrow Y\)، \(A \subseteq X\) و \(B \subseteq Y\).
داریم: \[f(A \cap f^{-1}(B)) = f(A) \cap B\]
\end{theorembox}
\subsection{۲. اثبات
دوطرفه}\label{ux627ux62bux628ux627ux62a-ux62fux648ux637ux631ux641ux647}
\begin{info}{اثبات}
از خواص منطقی سورها و تعاریف استفاده می‌کنیم تا تساوی را مستقیم نشان
دهیم.
\[y \in f(A \cap f^{-1}(B))\]
\(\equiv\) طبق تعریف تصویر، یعنی \(x\)ای وجود دارد که در مجموعه
\(A \cap f^{-1}(B)\) است و تصویرش \(y\) است:
\[\exists x \left[ x \in (A \cap f^{-1}(B)) \land f(x) = y \right]\]
\(\equiv\) تعریف اشتراک را باز می‌کنیم:
\[\exists x \left[ (x \in A \land x \in f^{-1}(B)) \land f(x) = y \right]\]
\(\equiv\) تعریف وارون (\(x \in f^{-1}(B) \iff f(x) \in B\)) را اعمال
می‌کنیم:
\[\exists x \left[ x \in A \land f(x) \in B \land f(x) = y \right]\]
\(\equiv\) در اینجا نکته کلیدی این است که \(f(x)=y\). پس شرط
\(f(x) \in B\) معادل است با \(y \in B\). چون \(y\) به \(x\) وابسته نیست
(در این بخش از گزاره)، می‌توانیم آن را از زیر سور وجودی بیرون بیاوریم یا
جدا کنیم:
\[\left( \exists x [x \in A \land f(x)=y] \right) \land y \in B\]
\(\equiv\) قسمت داخل پرانتز دقیقاً تعریف \(y \in f(A)\) است:
\[y \in f(A) \land y \in B\]
\(\equiv\) تعریف اشتراک: \[y \in f(A) \cap B\]
\(\blacksquare\)
\end{info}
\subsection{۳. حالت خاص
(نتیجه)}\label{ux62dux627ux644ux62a-ux62eux627ux635-ux646ux62aux6ccux62cux647}
اگر در فرمول بالا به جای \(A\)، کل مجموعه مرجع \(X\) را قرار دهیم:
\[f(X \cap f^{-1}(B)) = f(X) \cap B\] چون \(f^{-1}(B) \subseteq X\) است،
پس اشتراکشان خودِ \(f^{-1}(B)\) می‌شود: \[f(f^{-1}(B)) = f(X) \cap B\]
\emph{(این فرمول نشان می‌دهد تصویرِ وارونِ \lr{B }دقیقاً برابر است با اشتراکِ
برد تابع با \lr{B).}}

\clearpage
% ---------------------------------------------------------------------
% Copyright (c) 2026 Arsalan Dalvand & Reyhaneh Darvishi.
% Licensed under CC BY-NC-SA 4.0.
% See LICENSE file for details.
% ---------------------------------------------------------------------

\section{تمرین ۱۲: چه زمانی زیرمجموعه‌ها تبدیل به تساوی
می‌شوند؟}\label{تمرین-۱۲---شرط-تساوی-در-روابط-نگاره-و-وارون}
\begin{tldr}{خلاصه سریع}
در حالت کلی \(A \subseteq f^{-1}(f(A))\) و \(f(f^{-1}(B)) \subseteq B\)
است.
\begin{itemize}
\tightlist
\item
  اگر تابع \textbf{یک‌به‌یک} باشد \(\iff\) رابطه اول تبدیل به تساوی می‌شود.
\item
  اگر تابع \textbf{پوشا} باشد \(\iff\) رابطه دوم تبدیل به تساوی می‌شود.
\end{itemize}
\end{tldr}
\subsection{۱. صورت
تمرین}\label{ux635ux648ux631ux62a-ux62aux645ux631ux6ccux646}
\begin{info}{سوال}
فرض کنید \(f:X \rightarrow Y\). ثابت کنید: الف) اگر \(f\) یک‌به‌یک باشد،
آنگاه \(f^{-1}(f(A)) = A\). ب) اگر \(f\) پوشا باشد، آنگاه
\(f(f^{-1}(B)) = B\).
\end{info}
\subsection{۲. اثبات قسمت (الف) - شرط
یک‌به‌یک}\label{ux627ux62bux628ux627ux62a-ux642ux633ux645ux62a-ux627ux644ux641---ux634ux631ux637-ux6ccux6a9ux628ux647ux6ccux6a9}
\begin{info}{اثبات}
می‌دانیم همواره \(A \subseteq f^{-1}(f(A))\) برقرار است (تمرین ۹). پس فقط
باید ثابت کنیم \(f^{-1}(f(A)) \subseteq A\).
\begin{enumerate}
\def\labelenumi{\arabic{enumi}.}
\tightlist
\item
  فرض کنید \(x \in f^{-1}(f(A))\).
\item
  طبق تعریف وارون، یعنی \(f(x) \in f(A)\).
\item
  این یعنی عنصری مانند \(x' \in A\) وجود دارد که \(f(x) = f(x')\).
\item
  \textbf{نکته کلیدی:} چون \(f\) \textbf{یک‌به‌یک} است، از
  \(f(x) = f(x')\) نتیجه می‌گیریم \(x = x'\).
\item
  چون \(x' \in A\) بود، پس \(x \in A\).
\item
  نتیجه: \(f^{-1}(f(A)) \subseteq A\).
\end{enumerate}
\(\therefore A = f^{-1}(f(A))\)
\end{info}
\subsection{۳. اثبات قسمت (ب) - شرط
پوشا}\label{ux627ux62bux628ux627ux62a-ux642ux633ux645ux62a-ux628---ux634ux631ux637-ux67eux648ux634ux627}
\begin{info}{اثبات}
می‌دانیم همواره \(f(f^{-1}(B)) \subseteq B\) برقرار است (تمرین ۹). پس فقط
باید ثابت کنیم \(B \subseteq f(f^{-1}(B))\).
\begin{enumerate}
\def\labelenumi{\arabic{enumi}.}
\tightlist
\item
  فرض کنید \(y \in B\).
\item
  \textbf{نکته کلیدی:} چون \(f\) \textbf{پوشا} است (\(f(X)=Y\)) و
  \(B \subseteq Y\)، پس حتماً یک \(x \in X\) وجود دارد که \(y = f(x)\).
\item
  چون \(f(x) = y\) و \(y \in B\)، پس \(f(x) \in B\).
\item
  طبق تعریف وارون، این یعنی \(x \in f^{-1}(B)\).
\item
  حالا تصویر این \(x\) را می‌گیریم: \(f(x) \in f(f^{-1}(B))\).
\item
  چون \(f(x) = y\)، پس \(y \in f(f^{-1}(B))\).
\end{enumerate}
\(\therefore f(f^{-1}(B)) = B\)
\end{info}

\clearpage
% ---------------------------------------------------------------------
% Copyright (c) 2026 Arsalan Dalvand & Reyhaneh Darvishi.
% Licensed under CC BY-NC-SA 4.0.
% See LICENSE file for details.
% ---------------------------------------------------------------------

\section{تمرین ۱۴: تصویرِ تفاضل در توابع
یک‌به‌یک}\label{تمرین-۱۴---توزیع‌پذیری-تصویر-روی-تفاضل}
\begin{tldr}{خلاصه سریع}
در تمرین ۱۱ (فایل قبلی) دیدیم که \(f(A-B) \neq f(A)-f(B)\) در حالت کلی.
اما اگر تابع \textbf{یک‌به‌یک} باشد، این تساوی برقرار می‌شود. یک‌به‌یک بودن
مانع از آن می‌شود که عناصر خارج از \(B\)، روی تصاویر عناصرِ داخل \(B\)
بیفتند.
\end{tldr}
\subsection{۱. صورت
تمرین}\label{ux635ux648ux631ux62a-ux62aux645ux631ux6ccux646}
\begin{info}{سوال}
فرض کنید \(f:X \rightarrow Y\) یک‌به‌یک باشد و \(B \subseteq X\). ثابت
کنید: \[f(X - B) = f(X) - f(B)\]
\end{info}
\subsection{۲. اثبات}\label{ux627ux62bux628ux627ux62a}
\begin{info}{اثبات}
باید نشان دهیم \(y\) عضو سمت چپ است اگر و تنها اگر عضو سمت راست باشد.
\textbf{مسیر رفت (\(\subseteq\)):} همیشه برقرار است (حتی اگر یک‌به‌یک
نباشد). اگر \(y \in f(X-B)\)، یعنی \(y=f(x)\) که \(x \notin B\). اگر
\(y \in f(B)\) باشد، یعنی \(y=f(x')\) که \(x' \in B\). چون \(f\) یک‌به‌یک
است، \(x=x'\) که تناقض است. پس \(y \notin f(B)\).
\textbf{مسیر برگشت (\(\supseteq\)):}
\begin{enumerate}
\def\labelenumi{\arabic{enumi}.}
\tightlist
\item
  فرض کنید \(y \in f(X) - f(B)\).
\item
  یعنی \(y \in f(X)\) \textbf{و} \(y \notin f(B)\).
\item
  از \(y \in f(X)\) نتیجه می‌شود \(\exists x \in X\) که \(y = f(x)\).
\item
  آیا ممکن است این \(x\) در \(B\) باشد؟
  \begin{itemize}
  \tightlist
  \item
    اگر \(x \in B\) باشد
    \(\Rightarrow f(x) \in f(B) \Rightarrow y \in f(B)\).
  \item
    این با فرض (۲) در تناقض است.
  \end{itemize}
\item
  پس قطعاً \(x \notin B\) است. یعنی \(x \in X - B\).
\item
  بنابراین \(y = f(x) \in f(X - B)\).
\end{enumerate}
\(\blacksquare\)
\end{info}

\clearpage
% ---------------------------------------------------------------------
% Copyright (c) 2026 Arsalan Dalvand & Reyhaneh Darvishi.
% Licensed under CC BY-NC-SA 4.0.
% See LICENSE file for details.
% ---------------------------------------------------------------------

\section{تمرین ۱۵: تعمیم قانون
تفاضل}\label{تمرین-۱۵---تعمیم-تصویر-تفاضل}
\begin{tldr}{خلاصه سریع}
این تمرین همان تمرین ۱۴ است، با این تفاوت که به جای کل دامنه \(X\)، برای
هر زیرمجموعه دلخواه \(A\) بیان شده است.
\end{tldr}
\subsection{۱. صورت
تمرین}\label{ux635ux648ux631ux62a-ux62aux645ux631ux6ccux646}
\begin{info}{سوال}
اگر \(f\) یک‌به‌یک باشد و \(A, B \subseteq X\)، ثابت کنید:
\[f(A - B) = f(A) - f(B)\]
\end{info}
\subsection{۲. حل}\label{ux62dux644}
\begin{info}{استراتژی}
استدلال دقیقاً مشابه تمرین ۱۴ است.
اگر \(y \in f(A) - f(B)\):
\begin{enumerate}
\def\labelenumi{\arabic{enumi}.}
\tightlist
\item
  پس \(y = f(a)\) برای یک \(a \in A\).
\item
  و \(y \notin f(B)\).
\item
  اگر \(a \in B\) باشد، آنگاه \(f(a) \in f(B)\) که تناقض است (چون
  \(y \notin f(B)\)).
  \begin{itemize}
  \tightlist
  \item
    \textbf{نکته:} اینجا نیاز مبرم به \textbf{یک‌به‌یک} بودن داریم تا
    مطمئن شویم تصویر \(a\) با تصویر هیچ عضو دیگری از \(B\) یکی نمی‌شود.
  \end{itemize}
\item
  پس \(a \in A\) و \(a \notin B\)، یعنی \(a \in A - B\).
\item
  در نتیجه \(y \in f(A - B)\).
\end{enumerate}
\end{info}

\clearpage
% ---------------------------------------------------------------------
% Copyright (c) 2026 Arsalan Dalvand & Reyhaneh Darvishi.
% Licensed under CC BY-NC-SA 4.0.
% See LICENSE file for details.
% ---------------------------------------------------------------------

\section{\texorpdfstring{تمرین ۱۶: توابع خود-وارون
\lr{(Involution)}}{تمرین ۱۶: توابع خود-وارون }}\label{تمرین-۱۸---توابع-خود-وارون-و-تقارن}
\begin{tldr}{خلاصه سریع}
اگر تابعی خاصیت \(f(f(x)) = x\) را داشته باشد (مثل تابع \(f(x) = -x\) یا
\(f(x) = 1/x\))، نمودار آن نسبت به نیمساز ربع اول و سوم (\(y=x\)) متقارن
است. در زبان روابط، این یعنی رابطه \(f\) یک \textbf{رابطه متقارن} است.
\end{tldr}
\subsection{۱. صورت
تمرین}\label{ux635ux648ux631ux62a-ux62aux645ux631ux6ccux646}
\begin{info}{سوال}
فرض کنید \(f: X \rightarrow X\) تابعی باشد که برای هر \(x \in X\)، داشته
باشیم \(f(f(x)) = x\). ثابت کنید \(f\) (به عنوان یک رابطه) متقارن است.
\end{info}
\subsection{۲. اثبات}\label{ux627ux62bux628ux627ux62a}
\begin{info}{اثبات}
یک رابطه \(R\) متقارن است اگر: \((x, y) \in R \implies (y, x) \in R\).
در اینجا رابطه ما تابع \(f\) است، پس زوج مرتب‌ها به صورت \((x, f(x))\)
هستند.
\begin{enumerate}
\def\labelenumi{\arabic{enumi}.}
\tightlist
\item
  فرض کنید \((x, y) \in f\). این یعنی \(y = f(x)\).
\item
  می‌خواهیم ثابت کنیم \((y, x) \in f\). یعنی باید نشان دهیم \(x = f(y)\).
\item
  از فرض مسئله استفاده می‌کنیم: \(f(f(x)) = x\).
\item
  در رابطه بالا به جای \(f(x)\)، مقدار \(y\) (از مرحله ۱) را قرار
  می‌دهیم: \[f(y) = x\]
\item
  رابطه \(x = f(y)\) دقیقاً به این معنی است که زوج مرتب \((y, x)\) متعلق
  به تابع \(f\) است.
\end{enumerate}
پس \(f\) یک رابطه متقارن است. \(\blacksquare\)
\end{info}

\clearpage
% ---------------------------------------------------------------------
% Copyright (c) 2026 Arsalan Dalvand & Reyhaneh Darvishi.
% Licensed under CC BY-NC-SA 4.0.
% See LICENSE file for details.
% ---------------------------------------------------------------------

\section{انواع توابع: یک‌به‌یک، پوشا و
دوسویی}\label{انواع-توابع---یک‌به‌یک-پوشا-و-دوسویی}
\begin{tldr}{خلاصه سریع}
توابع بر اساس رفتار «نگاشت» اعضای دامنه به هم‌دامنه به سه دسته اصلی تقسیم
می‌شوند: ۱. \textbf{یک‌به‌یک \lr{(Injective):}} هیچ دو ورودی، خروجی یکسان
ندارند (تلاقی ممنوع). ۲. \textbf{پوشا \lr{(Surjective):}} تمام اعضای
مقصد، حداقل یک بار هدف قرار گرفته‌اند (هم‌دامنه = برد). ۳. \textbf{دوسویی
\lr{(Bijective):}} ترکیبی از هر دو؛ یک تناظر کامل و بی‌نقص بین دو مجموعه.
\end{tldr}
\begin{center}\rule{0.5\linewidth}{0.5pt}\end{center}
\subsection{\texorpdfstring{۱. تابع یک‌به‌یک \lr{(Injective }/
\lr{One-to-One)}}{۱. تابع یک‌به‌یک / }}\label{ux62aux627ux628ux639-ux6ccux6a9ux628ux647ux6ccux6a9-injective-one-to-one}
\subsubsection{الف) تعریف
ریاضی}\label{ux627ux644ux641-ux62aux639ux631ux6ccux641-ux631ux6ccux627ux636ux6cc}
تابع \(f: X \to Y\) را \textbf{یک‌به‌یک} گوییم هرگاه تصاویرِ عناصر متمایز،
متمایز باشند.
\begin{tldr}{تعریف ۱۰}
\[f \text{ is injective} \iff \forall x_1, x_2 \in X, \quad f(x_1) = f(x_2) \implies x_1 = x_2\]
\end{tldr}
\subsubsection{ب) تحلیل
منطقی}\label{ux628-ux62aux62dux644ux6ccux644-ux645ux646ux637ux642ux6cc}
این تعریف بر اساس \textbf{قانون عکس نقیض} (فصل ۱) معادل گزاره زیر است:
\[x_1 \neq x_2 \implies f(x_1) \neq f(x_2)\] یعنی تابع \(f\) هرگز دو عضو
مختلف دامنه را به یک نقطه در هم‌دامنه «فشرده» نمی‌کند
\lr{(Information Lossless).}
\subsubsection{ج) مثال}\label{ux62c-ux645ux62bux627ux644}
\begin{itemize}
\tightlist
\item
  تابع \(f(x) = 2x\) روی اعداد حقیقی یک‌به‌یک است.
\item
  تابع \(f(x) = x^2\) روی اعداد حقیقی یک‌به‌یک \textbf{نیست} (چون
  \(f(2) = f(-2) = 4\)).
\end{itemize}
\begin{center}\rule{0.5\linewidth}{0.5pt}\end{center}
\subsection{\texorpdfstring{۲. تابع پوشا \lr{(Surjective }/
\lr{Onto)}}{۲. تابع پوشا / }}\label{ux62aux627ux628ux639-ux67eux648ux634ux627-surjective-onto}
\subsubsection{الف) تعریف
ریاضی}\label{ux627ux644ux641-ux62aux639ux631ux6ccux641-ux631ux6ccux627ux636ux6cc-1}
تابع \(f: X \to Y\) را \textbf{پوشا} گوییم هرگاه برد تابع (\(Im(f)\))
دقیقاً برابر با هم‌دامنه (\(Y\)) باشد.
\begin{tldr}{تعریف ۱۱}
\[f \text{ is surjective} \iff \forall y \in Y, \exists x \in X : f(x) = y\]
معادل مجموعه‌ای: \[f(X) = Y\]
\end{tldr}
\subsubsection{ب) تحلیل
ساختاری}\label{ux628-ux62aux62dux644ux6ccux644-ux633ux627ux62eux62aux627ux631ux6cc}
در تابع پوشا، هیچ عنصری در \(Y\) «بی‌نصیب» نمی‌ماند. اگر \(f\) را به عنوان
یک تیراندازی از \(X\) به \(Y\) در نظر بگیریم، پوشا بودن یعنی تمام اهداف
در \(Y\) تیر خورده‌اند.
\begin{center}\rule{0.5\linewidth}{0.5pt}\end{center}
\subsection{\texorpdfstring{۳. تابع دوسویی
\lr{(Bijective)}}{۳. تابع دوسویی }}\label{ux62aux627ux628ux639-ux62fux648ux633ux648ux6ccux6cc-bijective}
\subsubsection{الف) تعریف
ریاضی}\label{ux627ux644ux641-ux62aux639ux631ux6ccux641-ux631ux6ccux627ux636ux6cc-2}
تابع \(f: X \to Y\) را \textbf{دوسویی} (یا تناظر یک‌به‌یک) گوییم اگر
هم‌زمان یک‌به‌یک و پوشا باشد.
\begin{tldr}{تعریف ۱۲}
\[f \text{ is bijective} \iff (f \text{ is injective}) \wedge (f \text{ is surjective})\]
\end{tldr}
\subsubsection{ب) اهمیت
(وارون‌پذیری)}\label{ux628-ux627ux647ux645ux6ccux62a-ux648ux627ux631ux648ux646ux67eux630ux6ccux631ux6cc}
توابع دوسویی مهم‌ترین کلاس توابع در جبر هستند زیرا ساختار دو مجموعه را
کاملاً به هم منتقل می‌کنند (ایزومورفیسم). تنها توابع دوسویی هستند که
\textbf{وارون‌پذیر} می‌باشند.
\begin{center}\rule{0.5\linewidth}{0.5pt}\end{center}
\subsection{\texorpdfstring{۴. شبکه ارتباطی با سایر قضایا
\lr{(Analytic Map)}}{۴. شبکه ارتباطی با سایر قضایا }}\label{ux634ux628ux6a9ux647-ux627ux631ux62aux628ux627ux637ux6cc-ux628ux627-ux633ux627ux6ccux631-ux642ux636ux627ux6ccux627-analytic-map}
\subsubsection{\texorpdfstring{۱. ارتباط با
\autoref{قضیه-۹---تصویر-و-تصویر-وارون-مجموعه} (رفتار با
برد)}{۱. ارتباط با  (رفتار با برد)}}\label{ux627ux631ux62aux628ux627ux637-ux628ux627-ux642ux636ux6ccux647-ux6f9---ux62aux635ux648ux6ccux631-ux648-ux62aux635ux648ux6ccux631-ux648ux627ux631ux648ux646-ux645ux62cux645ux648ux639ux647-ux631ux641ux62aux627ux631-ux628ux627-ux628ux631ux62f}
\begin{itemize}
\tightlist
\item
  \textbf{شرط پوشا بودن:} در قضیه ۹ دیدیم که \(f(X) \subseteq Y\) همواره
  برقرار است. تعریف پوشا بودن (تعریف ۱۱) این شمول را به تساوی
  \(f(X) = Y\) ارتقا می‌دهد.
\end{itemize}
\subsubsection{\texorpdfstring{۲. ارتباط با
\autoref{قضیه-۱۰---تعمیم-تصویر-اجتماع-و-اشتراک} (رفتار با
اشتراک)}{۲. ارتباط با  (رفتار با اشتراک)}}\label{ux627ux631ux62aux628ux627ux637-ux628ux627-ux642ux636ux6ccux647-ux6f1ux6f0---ux62aux639ux645ux6ccux645-ux62aux635ux648ux6ccux631-ux627ux62cux62aux645ux627ux639-ux648-ux627ux634ux62aux631ux627ux6a9-ux631ux641ux62aux627ux631-ux628ux627-ux627ux634ux62aux631ux627ux6a9}
\begin{itemize}
\tightlist
\item
  \textbf{یک‌به‌یک بودن و اشتراک:} در قضیه ۱۰ دیدیم که
  \(f(A \cap B) \subseteq f(A) \cap f(B)\) و تساوی لزوماً برقرار نیست.
  اما اگر تابع \textbf{یک‌به‌یک} باشد، تساوی برقرار می‌شود:
  \[f \text{ is 1-1} \implies f(A \cap B) = f(A) \cap f(B)\] این نتیجه
  در \textbf{\autoref{قضیه-۱۳---حفظ-اشتراک-در-توابع-یک‌به‌یک}} (که بعداً
  بررسی می‌کنیم) اثبات می‌شود.
\end{itemize}
\subsubsection{\texorpdfstring{۳. ارتباط با
\autoref{قضیه-۱۶---وارون‌های-یک‌طرفه}}{۳. ارتباط با }}\label{ux627ux631ux62aux628ux627ux637-ux628ux627-ux642ux636ux6ccux647-ux6f1ux6f6---ux648ux627ux631ux648ux646ux647ux627ux6cc-ux6ccux6a9ux637ux631ux641ux647}
\begin{itemize}
\tightlist
\item
  \textbf{وارون پذیری:}
  \begin{itemize}
  \tightlist
  \item
    اگر \(g \circ f = I_X\) (وارون چپ)، آنگاه \(f\) \textbf{یک‌به‌یک} است.
  \item
    اگر \(f \circ h = I_Y\) (وارون راست)، آنگاه \(f\) \textbf{پوشا} است.
  \item
    اگر هر دو برقرار باشند، \(f\) \textbf{دوسویی} است.
  \end{itemize}
\end{itemize}

\clearpage
% ---------------------------------------------------------------------
% Copyright (c) 2026 Arsalan Dalvand & Reyhaneh Darvishi.
% Licensed under CC BY-NC-SA 4.0.
% See LICENSE file for details.
% ---------------------------------------------------------------------

\section{\texorpdfstring{مفهوم ترکیب توابع
\lr{(Function Composition)}}{مفهوم ترکیب توابع }}\label{مفهوم-ترکیب-توابع}
\begin{tldr}{خلاصه سریع}
ترکیب توابع یعنی اتصال سریالی دو ماشین پردازشگر. خروجی ماشین اول،
مستقیماً به عنوان ورودی وارد ماشین دوم می‌شود. اگر \(f\) کارش «شستن» و
\(g\) کارش «خشک کردن» باشد، \(g \circ f\) ماشینی است که لباس چرک را
می‌گیرد و لباس شسته و خشک شده تحویل می‌دهد.
\end{tldr}
\subsection{۱. درک شهودی: آنالیز
ماشین‌ها}\label{ux62fux631ux6a9-ux634ux647ux648ux62fux6cc-ux622ux646ux627ux644ux6ccux632-ux645ux627ux634ux6ccux646ux647ux627}
بهترین راه برای درک ترکیب توابع، تصور آن‌ها به عنوان \textbf{ماشین} است:
فرض کنید دو ماشین داریم:
\begin{enumerate}
\def\labelenumi{\arabic{enumi}.}
\tightlist
\item
  \textbf{ماشین \(f\) (لباسشویی):} لباس چرک (\(x\)) را می‌گیرد و لباس خیس
  و تمیز (\(f(x)\)) تحویل می‌دهد.
  \begin{itemize}
  \tightlist
  \item
    \(f: X \to Y\)
  \end{itemize}
\item
  \textbf{ماشین \(g\) (خشک‌کن):} لباس خیس (\(y\)) را می‌گیرد و لباس خشک و
  تمیز (\(g(y)\)) تحویل می‌دهد.
  \begin{itemize}
  \tightlist
  \item
    \(g: Y \to Z\)
  \end{itemize}
\end{enumerate}
اگر این دو ماشین را به هم وصل کنیم (خروجی اولی به ورودی دومی)، یک ماشین
جدید و پیشرفته \(h\) ساخته‌ایم که لباس چرک می‌گیرد و لباس آماده پوشیدن
تحویل می‌دهد. این ماشین جدید را \textbf{ترکیب \(g\) با \(f\)} می‌نامیم و
با نماد \textbf{\(g \circ f\)} نشان می‌دهیم.
\begin{warning}{نکته مهم در نمادگذاری}
در نماد \(g \circ f\)، با اینکه \(g\) سمت چپ نوشته شده، اما در عمل
\textbf{دومین} تابعی است که اجرا می‌شود.
\[(\underbrace{g}_{\text{دوم}} \circ \underbrace{f}_{\text{اول}})(x) = g(f(x))\]
چون \(x\) ابتدا وارد \(f\) می‌شود.
\end{warning}
\begin{center}\rule{0.5\linewidth}{0.5pt}\end{center}
\subsection{\texorpdfstring{۲. تعریف ریاضی
\lr{(Definition }13)}{۲. تعریف ریاضی 13)}}\label{ux62aux639ux631ux6ccux641-ux631ux6ccux627ux636ux6cc-definition-13}
فرض کنید \(f: X \to Y\) و \(g: Y \to Z\) دو تابع باشند (دقت کنید که
هم‌دامنه اولی باید با دامنه دومی سازگار باشد).
\begin{tldr}{تعریف ترکیب}
تابع \(g \circ f: X \to Z\) به صورت زیر تعریف می‌شود:
\[\forall x \in X, \quad (g \circ f)(x) = g(f(x))\]
\end{tldr}
\subsubsection{تعریف مجموعه‌ای
(دقیق)}\label{ux62aux639ux631ux6ccux641-ux645ux62cux645ux648ux639ux647ux627ux6cc-ux62fux642ux6ccux642}
از آنجا که تابع مجموعه‌ای از زوج‌های مرتب است، تعریف دقیق مجموعه‌ای
\(g \circ f\) چنین است:
\[g \circ f = \{ (x, z) \in X \times Z \mid \exists y \in Y : (x, y) \in f \wedge (y, z) \in g \}\]
\emph{(ترجمه: زوج \((x,z)\) در ترکیب است اگر واسطه‌ای مثل \(y\) وجود
داشته باشد که \(x\) را به \(y\) (توسط \(f\)) و \(y\) را به \(z\) (توسط
\(g\)) وصل کند).}
\begin{center}\rule{0.5\linewidth}{0.5pt}\end{center}
\subsection{\texorpdfstring{۳. شبکه ارتباطی با سایر قضایا
\lr{(Analytic Map)}}{۳. شبکه ارتباطی با سایر قضایا }}\label{ux634ux628ux6a9ux647-ux627ux631ux62aux628ux627ux637ux6cc-ux628ux627-ux633ux627ux6ccux631-ux642ux636ux627ux6ccux627-analytic-map}
\subsubsection{\texorpdfstring{۱. ارتباط با
\autoref{مفهوم-تابع-و-قضیه-۶---تعریف-و-دامنه}}{۱. ارتباط با }}\label{ux627ux631ux62aux628ux627ux637-ux628ux627-ux645ux641ux647ux648ux645-ux62aux627ux628ux639-ux648-ux642ux636ux6ccux647-ux6f6---ux62aux639ux631ux6ccux641-ux648-ux62fux627ux645ux646ux647}
\begin{itemize}
\tightlist
\item
  \textbf{شرط وجود:} برای اینکه \(g \circ f\) قابل تعریف باشد، خروجی‌های
  \(f\) (یعنی \(Im(f)\)) باید حتماً زیرمجموعه‌ای از ورودی‌های مجاز \(g\)
  (یعنی \(Dom(g)\)) باشند. اگر \(Range(f) \not\subseteq Dom(g)\)، ماشین
  دوم ممکن است ورودی نامعتبر دریافت کند و خراب شود.
\end{itemize}
\subsubsection{\texorpdfstring{۲. پیش‌نیاز
\autoref{قضیه-۱۵---شرکت‌پذیری-ترکیب-توابع}}{۲. پیش‌نیاز }}\label{ux67eux6ccux634ux646ux6ccux627ux632-ux642ux636ux6ccux647-ux6f1ux6f5---ux634ux631ux6a9ux62aux67eux630ux6ccux631ux6cc-ux62aux631ux6a9ux6ccux628-ux62aux648ux627ux628ux639}
\begin{itemize}
\tightlist
\item
  \textbf{مقدمه:} در قضایای بعدی خواهیم دید که ترکیب توابع خاصیت جابجایی
  ندارد (\(f \circ g \neq g \circ f\))، اما خاصیت شرکت‌پذیری دارد
  (\(h \circ (g \circ f) = (h \circ g) \circ f\)).
\end{itemize}
\subsubsection{\texorpdfstring{۳. ارتباط با
\autoref{قضیه-۱۴---وجود-و-ویژگی‌های-تابع-وارون}}{۳. ارتباط با }}\label{ux627ux631ux62aux628ux627ux637-ux628ux627-ux642ux636ux6ccux647-ux6f1ux6f4---ux648ux62cux648ux62f-ux648-ux648ux6ccux698ux6afux6ccux647ux627ux6cc-ux62aux627ux628ux639-ux648ux627ux631ux648ux646}
\begin{itemize}
\tightlist
\item
  \textbf{خنثی‌سازی:} مفهوم «تابع وارون» دقیقاً بر عکس کردنِ عملیات ترکیب
  است. اگر \(f\) لباسی را بشوید، \(f^{-1}\) باید بتواند آن را دوباره چرک
  کند! رابطه آن‌ها چنین است:
  \[f^{-1} \circ f = I_X \quad (\text{تابع همانی})\]
\end{itemize}

\clearpage
% ---------------------------------------------------------------------
% Copyright (c) 2026 Arsalan Dalvand & Reyhaneh Darvishi.
% Licensed under CC BY-NC-SA 4.0.
% See LICENSE file for details.
% ---------------------------------------------------------------------

\section{قضیه ۱۴: شرط وجود و ویژگی‌های تابع
وارون}\label{قضیه-۱۴---وجود-و-ویژگی‌های-تابع-وارون}
\begin{tldr}{خلاصه سریع}
هر تابعی لزوماً وارون‌پذیر نیست. این قضیه شرط لازم و کافی برای اینکه
«رابطه وارون»ِ یک تابع، خودش یک «تابع» باشد را بیان می‌کند: تابع اصلی باید
\textbf{دوسویی} \lr{(Bijective) }باشد. علاوه بر این، وارون یک تابع
دوسویی، خودش نیز دوسویی است.
\end{tldr}
\subsection{۱. پیش‌زمینه: تعریف رابطه
وارون}\label{ux67eux6ccux634ux632ux645ux6ccux646ux647-ux62aux639ux631ux6ccux641-ux631ux627ux628ux637ux647-ux648ux627ux631ux648ux646}
پیش از بیان قضیه، یادآوری می‌کنیم که برای هر تابع \(f: X \to Y\)، رابطه
وارون \(f^{-1}\) به صورت زیر تعریف می‌شود:
\[f^{-1} = \{ (y, x) \in Y \times X \mid (x, y) \in f \}\] این رابطه
همواره وجود دارد، اما لزوماً «تابع» نیست.
\subsection{۲. متن ریاضی
قضیه}\label{ux645ux62aux646-ux631ux6ccux627ux636ux6cc-ux642ux636ux6ccux647}
فرض کنید \(f: X \to Y\) یک تابع باشد.
\begin{theorembox}{قضیه ۱۴}
اگر \(f\) یک تابع \textbf{دوسویی} (یک‌به‌یک و پوشا) باشد، آنگاه
\(f^{-1}: Y \to X\) نیز یک تابع \textbf{دوسویی} است.
\end{theorembox}
\subsection{\texorpdfstring{۳. اثبات صوری
\lr{(Formal Proof)}}{۳. اثبات صوری }}\label{ux627ux62bux628ux627ux62a-ux635ux648ux631ux6cc-formal-proof}
این اثبات سه بخش دارد: ۱. اثبات تابع بودن \(f^{-1}\)، ۲. اثبات یک‌به‌یک
بودن \(f^{-1}\)، ۳. اثبات پوشا بودن \(f^{-1}\).
\subsubsection{\texorpdfstring{گام اول: اثبات تابع بودن
\(f^{-1}\)}{گام اول: اثبات تابع بودن f\^{}\{-1\}}}\label{ux6afux627ux645-ux627ux648ux644-ux627ux62bux628ux627ux62a-ux62aux627ux628ux639-ux628ux648ux62fux646-f-1}
برای اینکه \(f^{-1}\) تابع باشد، باید دو شرط (دامنه کامل) و (یکتایی
مقدار) را داشته باشد.
\begin{info}{برهان}
\textbf{۱. بررسی دامنه \lr{(Existence):}} چون \(f\) \textbf{پوشا}
\lr{(Surjective) }است، برد آن برابر با هم‌دامنه‌اش است (\(Im(f) = Y\)).
طبق ویژگی‌های رابطه وارون، \(Dom(f^{-1}) = Im(f)\). بنابراین
\(Dom(f^{-1}) = Y\). پس \(f^{-1}\) روی تمام اعضای \(Y\) تعریف شده است.
\textbf{۲. بررسی یکتایی \lr{(Uniqueness/Well-definedness):}} فرض کنید
\(y \in Y\) به دو مقدار \(x_1\) و \(x_2\) نگاشته شود:
\[(y, x_1) \in f^{-1} \quad \text{و} \quad (y, x_2) \in f^{-1}\] طبق
تعریف رابطه وارون:
\[(x_1, y) \in f \quad \text{و} \quad (x_2, y) \in f\] یعنی
\(f(x_1) = y\) و \(f(x_2) = y\). چون \(f\) \textbf{یک‌به‌یک}
\lr{(Injective) }است، از تساوی تصاویر نتیجه می‌شود که پیش‌نگاره‌ها برابرند:
\[f(x_1) = f(x_2) \implies x_1 = x_2\]
\textbf{نتیجه:} \(f^{-1}: Y \to X\) یک تابع است.
\end{info}
\subsubsection{\texorpdfstring{گام دوم: اثبات یک‌به‌یک بودن
\(f^{-1}\)}{گام دوم: اثبات یک‌به‌یک بودن f\^{}\{-1\}}}\label{ux6afux627ux645-ux62fux648ux645-ux627ux62bux628ux627ux62a-ux6ccux6a9ux628ux647ux6ccux6a9-ux628ux648ux62fux646-f-1}
\begin{info}{برهان}
فرض کنید \(y_1, y_2 \in Y\) و \(f^{-1}(y_1) = f^{-1}(y_2)\). فرض کنیم
مقدار این تصویر مشترک \(x\) باشد (\(x \in X\)).
\[f^{-1}(y_1) = x \implies f(x) = y_1\]
\[f^{-1}(y_2) = x \implies f(x) = y_2\] چون \(f\) تابع است (تک‌مقداری)،
خروجی \(x\) یکتاست. \[\implies y_1 = y_2\] پس \(f^{-1}\) یک‌به‌یک است.
\end{info}
\subsubsection{\texorpdfstring{گام سوم: اثبات پوشا بودن
\(f^{-1}\)}{گام سوم: اثبات پوشا بودن f\^{}\{-1\}}}\label{ux6afux627ux645-ux633ux648ux645-ux627ux62bux628ux627ux62a-ux67eux648ux634ux627-ux628ux648ux62fux646-f-1}
\begin{info}{برهان}
برد تابع وارون برابر است با دامنه تابع اصلی: \[Im(f^{-1}) = Dom(f) = X\]
چون برد \(f^{-1}\) برابر با هم‌دامنه آن (\(X\)) است، پس \(f^{-1}\)
پوشاست.
\end{info}
\subsection{\texorpdfstring{۴. شبکه ارتباطی با سایر قضایا
\lr{(Analytic Map)}}{۴. شبکه ارتباطی با سایر قضایا }}\label{ux634ux628ux6a9ux647-ux627ux631ux62aux628ux627ux637ux6cc-ux628ux627-ux633ux627ux6ccux631-ux642ux636ux627ux6ccux627-analytic-map}
\subsubsection{\texorpdfstring{۱. ارتباط با
\autoref{انواع-توابع---یک‌به‌یک-پوشا-و-دوسویی}}{۱. ارتباط با }}\label{ux627ux631ux62aux628ux627ux637-ux628ux627-ux627ux646ux648ux627ux639-ux62aux648ux627ux628ux639---ux6ccux6a9ux628ux647ux6ccux6a9-ux67eux648ux634ux627-ux648-ux62fux648ux633ux648ux6ccux6cc}
\begin{itemize}
\tightlist
\item
  \textbf{وابستگی مطلق:} اثبات قسمت اول (تابع بودن وارون) دقیقاً نشان
  می‌دهد که چرا تعاریف «یک‌به‌یک» و «پوشا» مهم هستند.
  \begin{itemize}
  \tightlist
  \item
    اگر \(f\) پوشا نباشد \(\implies\) دامنه \(f^{-1}\) ناقص می‌شود (تابع
    نیست).
  \item
    اگر \(f\) یک‌به‌یک نباشد \(\implies\) خروجی \(f^{-1}\) چندمقداری می‌شود
    (تابع نیست).
  \end{itemize}
\end{itemize}
\subsubsection{\texorpdfstring{۲. ارتباط با
\autoref{قضیه-۱۶---وارون‌های-یک‌طرفه}}{۲. ارتباط با }}\label{ux627ux631ux62aux628ux627ux637-ux628ux627-ux642ux636ux6ccux647-ux6f1ux6f6---ux648ux627ux631ux648ux646ux647ux627ux6cc-ux6ccux6a9ux637ux631ux641ux647}
\begin{itemize}
\tightlist
\item
  \textbf{تعمیم:} قضیه ۱۴ حالت خاص و کامل‌تری از قضیه ۱۶ است.
  \begin{itemize}
  \tightlist
  \item
    در قضیه ۱۶ خواهیم دید که اگر \(g \circ f = I_X\)، آنگاه \(f\) فقط
    یک‌به‌یک است (وارون چپ).
  \item
    در قضیه ۱۶ خواهیم دید که اگر \(f \circ h = I_Y\)، آنگاه \(f\) فقط
    پوشا است (وارون راست).
  \item
    قضیه ۱۴ می‌گوید اگر \(f\) دوسویی باشد، \(f^{-1}\) هم وارون چپ است و
    هم وارون راست.
  \end{itemize}
\end{itemize}
\subsubsection{۳. تقارن هندسی
(بازتاب)}\label{ux62aux642ux627ux631ux646-ux647ux646ux62fux633ux6cc-ux628ux627ux632ux62aux627ux628}
\begin{itemize}
\tightlist
\item
  \textbf{نمودار:} اگر نمودار \(f\) را داشته باشیم، نمودار \(f^{-1}\)
  قرینه آن نسبت به نیمساز ربع اول و سوم (\(y=x\)) است. این تقارن هندسی
  نتیجه مستقیم تبدیل زوج \((x,y)\) به \((y,x)\) است که در تعریف رابطه
  وارون آمد.
\end{itemize}

\clearpage
% ---------------------------------------------------------------------
% Copyright (c) 2026 Arsalan Dalvand & Reyhaneh Darvishi.
% Licensed under CC BY-NC-SA 4.0.
% See LICENSE file for details.
% ---------------------------------------------------------------------

\section{\texorpdfstring{قضیه ۱۵: شرکت‌پذیری ترکیب توابع
\lr{(Associativity of Composition)}}{قضیه ۱۵: شرکت‌پذیری ترکیب توابع }}\label{قضیه-۱۵---شرکت‌پذیری-ترکیب-توابع}
\begin{tldr}{خلاصه سریع}
این قضیه بیان می‌کند که در زنجیره‌ای از توابع متوالی، اولویت ترکیب (محل
قرارگیری پرانتزها) اهمیتی ندارد. اگرچه ترکیب توابع خاصیت «جابجایی»
ندارد، اما همواره از خاصیت «شرکت‌پذیری» برخوردار است. این ویژگی، مجموعه
توابع را به یک «نیم‌گروه» \lr{(Semigroup) }تبدیل می‌کند.
\end{tldr}
\subsection{۱. متن ریاضی
قضیه}\label{ux645ux62aux646-ux631ux6ccux627ux636ux6cc-ux642ux636ux6ccux647}
فرض کنید سه تابع با دامنه‌ها و هم‌دامنه‌های متوالی به صورت زیر تعریف شده
باشند: \[f: X \to Y, \quad g: Y \to Z, \quad h: Z \to W\]
\begin{theorembox}{قضیه ۱۵}
ترکیب این توابع شرکت‌پذیر است، یعنی:
\[(h \circ g) \circ f = h \circ (g \circ f)\]
\end{theorembox}
\subsection{\texorpdfstring{۲. اثبات صوری
\lr{(Formal Proof)}}{۲. اثبات صوری }}\label{ux627ux62bux628ux627ux62a-ux635ux648ux631ux6cc-formal-proof}
برای اثبات تساوی دو تابع، طبق
\textbf{\autoref{قضیه-۷---شرط-تساوی-توابع}}، باید نشان دهیم که:
\begin{enumerate}
\def\labelenumi{\arabic{enumi}.}
\tightlist
\item
  دامنه‌ها و هم‌دامنه‌های دو طرف یکسان هستند.
\item
  به ازای هر ورودی یکسان، خروجی‌های یکسانی تولید می‌کنند.
\end{enumerate}
\begin{info}{برهان}
\textbf{گام ۱: بررسی خوش‌تعریفی و دامنه}
\begin{itemize}
\tightlist
\item
  تابع سمت چپ: ابتدا \(g \circ f\) تابعی از \(X\) به \(Z\) است. سپس
  ترکیب آن با \(h\)، تابعی از \(X\) به \(W\) می‌سازد.
\item
  تابع سمت راست: ابتدا \(h \circ g\) تابعی از \(Y\) به \(W\) است. سپس
  ترکیب آن با \(f\)، تابعی از \(X\) به \(W\) می‌سازد.
\item
  بنابراین هر دو تابع دارای دامنه \(X\) و هم‌دامنه \(W\) هستند.
\end{itemize}
\textbf{گام ۲: بررسی تساوی مقداری \lr{(Point-wise Equality)}} فرض کنیم
\(x\) عنصری دلخواه از \(X\) باشد (\(x \in X\)).
\begin{itemize}
\item
  \textbf{محاسبه سمت چپ:} \[((h \circ g) \circ f)(x)\] طبق تعریف ترکیب
  (\((\alpha \circ \beta)(x) = \alpha(\beta(x))\))، تابع بیرونی
  \((h \circ g)\) و درونی \(f\) است: \[= (h \circ g)(f(x))\] اکنون روی
  ترکیب \((h \circ g)\) اعمال تعریف می‌کنیم (ورودی آن \(f(x)\) است):
  \[= h(g(f(x)))\]
\item
  \textbf{محاسبه سمت راست:} \[(h \circ (g \circ f))(x)\] تابع بیرونی
  \(h\) و درونی \((g \circ f)\) است: \[= h((g \circ f)(x))\] اکنون داخل
  پرانتز را باز می‌کنیم: \[= h(g(f(x)))\]
\end{itemize}
\textbf{نتیجه:} چون برای هر \(x \in X\)، خروجی‌ها یکسان هستند
(\(h(g(f(x)))\))، طبق قضیه ۷، دو تابع برابرند.
\end{info}
\subsection{\texorpdfstring{۳. شبکه ارتباطی با سایر قضایا
\lr{(Analytic Map)}}{۳. شبکه ارتباطی با سایر قضایا }}\label{ux634ux628ux6a9ux647-ux627ux631ux62aux628ux627ux637ux6cc-ux628ux627-ux633ux627ux6ccux631-ux642ux636ux627ux6ccux627-analytic-map}
\subsubsection{\texorpdfstring{۱. وابستگی به
\autoref{قضیه-۷---شرط-تساوی-توابع}}{۱. وابستگی به }}\label{ux648ux627ux628ux633ux62aux6afux6cc-ux628ux647-ux642ux636ux6ccux647-ux6f7---ux634ux631ux637-ux62aux633ux627ux648ux6cc-ux62aux648ux627ux628ux639}
\begin{itemize}
\tightlist
\item
  \textbf{اصل موضوعی:} اثبات قضیه ۱۵ تماماً بر پایه قضیه ۷ استوار است.
  بدون قضیه ۷، رسیدن به \(h(g(f(x)))\) در هر دو طرف، لزوماً به معنای یکی
  بودن ``اشیاء'' تابع نیست. قضیه ۷ پل عبور از ``تساوی مقادیر'' به
  ``تساوی توابع'' است.
\end{itemize}
\subsubsection{۲. تضاد با خاصیت
جابجایی}\label{ux62aux636ux627ux62f-ux628ux627-ux62eux627ux635ux6ccux62a-ux62cux627ux628ux62cux627ux6ccux6cc}
\begin{itemize}
\tightlist
\item
  \textbf{هشدار جبری:} بسیار مهم است که شرکت‌پذیری را با جابجایی اشتباه
  نگیریم.
  \begin{itemize}
  \tightlist
  \item
    \textbf{شرکت‌پذیری (صحیح):}
    \((h \circ g) \circ f = h \circ (g \circ f)\)
  \item
    \textbf{جابجایی (غلط):} \(g \circ f \neq f \circ g\) (مگر در حالات
    خاص). این نشان می‌دهد که جبر توابع شبیه ضرب ماتریس‌هاست (شرکت‌پذیر اما
    غیرجابجایی).
  \end{itemize}
\end{itemize}
\subsubsection{\texorpdfstring{۳. ارتباط با
\autoref{مفهوم-ترکیب-توابع}}{۳. ارتباط با }}\label{ux627ux631ux62aux628ux627ux637-ux628ux627-ux645ux641ux647ux648ux645-ux62aux631ux6a9ux6ccux628-ux62aux648ux627ux628ux639}
\begin{itemize}
\tightlist
\item
  \textbf{تعریف بازگشتی:} اثبات بالا نشان می‌دهد که نماد
  \(h \circ g \circ f\) (بدون پرانتز) یک نماد خوش‌تعریف و بدون ابهام است.
  این نتیجه مستقیم تعریف بازگشتی ترکیب است که در یادداشت ``مفهوم ترکیب
  توابع'' معرفی شد.
\end{itemize}

\clearpage
% ---------------------------------------------------------------------
% Copyright (c) 2026 Arsalan Dalvand & Reyhaneh Darvishi.
% Licensed under CC BY-NC-SA 4.0.
% See LICENSE file for details.
% ---------------------------------------------------------------------

\section{قضیه ۱۶: وارون‌های یک‌طرفه (چپ و
راست)}\label{قضیه-۱۶---وارون‌های-یک‌طرفه}
\begin{tldr}{خلاصه سریع}
این قضیه ارتباط عمیق بین «ساختار جبری» (ترکیب توابع) و «ویژگی‌های نگاشتی»
(یک‌به‌یک و پوشا بودن) را نشان می‌دهد.
\begin{itemize}
\tightlist
\item
  اگر بتوان اثر تابع را از \textbf{چپ} خنثی کرد \(\implies\) تابع
  \textbf{یک‌به‌یک} است.
\item
  اگر بتوان اثر تابع را از \textbf{راست} خنثی کرد \(\implies\) تابع
  \textbf{پوشا} است.
\end{itemize}
\end{tldr}
\subsection{۱. متن ریاضی
قضیه}\label{ux645ux62aux646-ux631ux6ccux627ux636ux6cc-ux642ux636ux6ccux647}
فرض کنید \(f: X \to Y\) یک تابع باشد.
\begin{theorembox}{قضیه ۱۶}
\textbf{الف) شرط یک‌به‌یک بودن (وارون چپ):} اگر تابعی مانند \(g: Y \to X\)
وجود داشته باشد که \(g \circ f = I_X\)، آنگاه \(f\) \textbf{یک‌به‌یک}
\lr{(Injective) }است.
\textbf{ب) شرط پوشا بودن (وارون راست):} اگر تابعی مانند \(h: Y \to X\)
وجود داشته باشد که \(f \circ h = I_Y\)، آنگاه \(f\) \textbf{پوشا}
\lr{(Surjective) }است.
\end{theorembox}
\emph{(تذکر: \(I_X\) و \(I_Y\) توابع همانی روی دامنه‌های مربوطه هستند).}
\subsection{\texorpdfstring{۲. اثبات صوری
\lr{(Formal Proof)}}{۲. اثبات صوری }}\label{ux627ux62bux628ux627ux62a-ux635ux648ux631ux6cc-formal-proof}
\subsubsection{\texorpdfstring{اثبات قسمت (الف): وارون چپ \(\implies\)
یک‌به‌یک}{اثبات قسمت (الف): وارون چپ \textbackslash implies یک‌به‌یک}}\label{ux627ux62bux628ux627ux62a-ux642ux633ux645ux62a-ux627ux644ux641-ux648ux627ux631ux648ux646-ux686ux67e-implies-ux6ccux6a9ux628ux647ux6ccux6a9}
برای اثبات یک‌به‌یک بودن، باید نشان دهیم
\(f(x_1) = f(x_2) \implies x_1 = x_2\).
\begin{info}{برهان}
۱. فرض کنید \(x_1, x_2 \in X\) باشند و تصاویرشان برابر باشد:
\[f(x_1) = f(x_2)\] ۲. روی طرفین تساوی، تابع \(g\) را اعمال می‌کنیم (چون
\(g\) تابع است، خروجی‌های یکسان برای ورودی‌های یکسان دارد):
\[g(f(x_1)) = g(f(x_2))\] ۳. طبق تعریف ترکیب توابع:
\[(g \circ f)(x_1) = (g \circ f)(x_2)\] ۴. طبق فرض قضیه
(\(g \circ f = I_X\)): \[I_X(x_1) = I_X(x_2)\] ۵. طبق تعریف تابع همانی
(\(I_X(x) = x\)): \[x_1 = x_2\]
\textbf{نتیجه:} تابع \(f\) یک‌به‌یک است.
\end{info}
\subsubsection{\texorpdfstring{اثبات قسمت (ب): وارون راست \(\implies\)
پوشا}{اثبات قسمت (ب): وارون راست \textbackslash implies پوشا}}\label{ux627ux62bux628ux627ux62a-ux642ux633ux645ux62a-ux628-ux648ux627ux631ux648ux646-ux631ux627ux633ux62a-implies-ux67eux648ux634ux627}
برای اثبات پوشا بودن، باید نشان دهیم برای هر \(y \in Y\)، پیش‌نگاره‌ای
وجود دارد.
\begin{info}{برهان}
۱. فرض کنید \(y\) عضوی دلخواه از هم‌دامنه \(Y\) باشد (\(y \in Y\)). ۲. ما
به دنبال یک \(x \in X\) هستیم که \(f(x) = y\). ۳. عنصر \(x\) را به صورت
\(x = h(y)\) تعریف می‌کنیم. (چون \(h: Y \to X\) است، پس \(x \in X\)
می‌باشد). ۴. حال مقدار تابع \(f\) را در این نقطه محاسبه می‌کنیم:
\[f(x) = f(h(y))\] ۵. طبق تعریف ترکیب توابع: \[= (f \circ h)(y)\] ۶. طبق
فرض قضیه (\(f \circ h = I_Y\)): \[= I_Y(y)\] ۷. طبق تعریف تابع همانی:
\[= y\]
\textbf{نتیجه:} برای هر \(y\)، یک \(x\) (همان \(h(y)\)) یافت شد که
\(f(x)=y\). پس \(f\) پوشا است.
\end{info}
\subsection{\texorpdfstring{۳. شبکه ارتباطی با سایر قضایا
\lr{(Analytic Map)}}{۳. شبکه ارتباطی با سایر قضایا }}\label{ux634ux628ux6a9ux647-ux627ux631ux62aux628ux627ux637ux6cc-ux628ux627-ux633ux627ux6ccux631-ux642ux636ux627ux6ccux627-analytic-map}
\subsubsection{\texorpdfstring{۱. تکمیل
\autoref{قضیه-۱۴---وجود-و-ویژگی‌های-تابع-وارون}}{۱. تکمیل }}\label{ux62aux6a9ux645ux6ccux644-ux642ux636ux6ccux647-ux6f1ux6f4---ux648ux62cux648ux62f-ux648-ux648ux6ccux698ux6afux6ccux647ux627ux6cc-ux62aux627ux628ux639-ux648ux627ux631ux648ux646}
\begin{itemize}
\tightlist
\item
  \textbf{تحلیل ساختاری:} قضیه ۱۴ بیان می‌کرد که اگر \(f\)
  \textbf{دوسویی} باشد، وارون‌پذیر است. قضیه ۱۶ این شرط را تجزیه می‌کند:
  \begin{itemize}
  \tightlist
  \item
    بخش «یک‌به‌یک» بودن \(f\) ناشی از وجود وارون چپ است.
  \item
    بخش «پوشا» بودن \(f\) ناشی از وجود وارون راست است.
  \item
    اگر \(f\) هم وارون چپ داشته باشد و هم راست (و این دو برابر باشند)،
    آنگاه وارون‌پذیر کامل (\(f^{-1}\)) است.
  \end{itemize}
\end{itemize}
\subsubsection{\texorpdfstring{۲. ارتباط با
\autoref{انواع-توابع---یک‌به‌یک-پوشا-و-دوسویی}}{۲. ارتباط با }}\label{ux627ux631ux62aux628ux627ux637-ux628ux627-ux627ux646ux648ux627ux639-ux62aux648ux627ux628ux639---ux6ccux6a9ux628ux647ux6ccux6a9-ux67eux648ux634ux627-ux648-ux62fux648ux633ux648ux6ccux6cc}
\begin{itemize}
\tightlist
\item
  \textbf{آزمون جبری:} این قضیه یک روش ``عملیاتی'' برای تست یک‌به‌یک یا
  پوشا بودن ارائه می‌دهد. به جای چک کردن تک‌تک اعضا (تعریف اصلی)، کافی است
  سعی کنیم تابعی بسازیم که اثر \(f\) را خنثی کند. این روش در معادلات
  دیفرانسیل و جبر خطی (معکوس ماتریس) بسیار کاربرد دارد.
\end{itemize}
\subsubsection{\texorpdfstring{۳. ارتباط با
\autoref{توابع-خاص---همانی-و-ثابت}}{۳. ارتباط با }}\label{ux627ux631ux62aux628ux627ux637-ux628ux627-ux62aux648ux627ux628ux639-ux62eux627ux635---ux647ux645ux627ux646ux6cc-ux648-ux62bux627ux628ux62a}
\begin{itemize}
\tightlist
\item
  \textbf{نقش عنصر خنثی:} بدون تعریف دقیق تابع همانی (\(I_X\)) و درک نقش
  آن به عنوان ``عنصر خنثی'' در ترکیب توابع، بیان این قضیه ممکن نیست.
  تفاوت \(I_X\) و \(I_Y\) در صورت قضیه بسیار حیاتی است (چون دامنه‌ها
  متفاوت‌اند).
\end{itemize}

\clearpage
% ---------------------------------------------------------------------
% Copyright (c) 2026 Arsalan Dalvand & Reyhaneh Darvishi.
% Licensed under CC BY-NC-SA 4.0.
% See LICENSE file for details.
% ---------------------------------------------------------------------

\section{\texorpdfstring{تمرین ۱۹: خاصیت شرکت‌پذیری
\lr{(Associativity)}}{تمرین ۱۹: خاصیت شرکت‌پذیری }}\label{تمرین-۱۹---خاصیت-شرکت‌پذیری-ترکیب-توابع}
\begin{tldr}{خلاصه سریع}
در ترکیب توابع، ترتیب پرانتزگذاری مهم نیست (به شرطی که ترتیب توابع
\(f, g, h\) عوض نشود). \[(h \circ g) \circ f = h \circ (g \circ f)\]
\end{tldr}
\subsection{۱. صورت
تمرین}\label{ux635ux648ux631ux62a-ux62aux645ux631ux6ccux646}
\begin{info}{سوال}
با فرض توابع زیر:
\begin{itemize}
\tightlist
\item
  \(f(x) = 2x^2 + 5\)
\item
  \(g(x) = \cos x\)
\item
  \(h(x) = x^2 - 1\)
\end{itemize}
موارد زیر را محاسبه کنید و نشان دهید نتیجه نهایی یکی است: الف)
\((h \circ g) \circ f\) ب) \(h \circ (g \circ f)\)
\end{info}
\subsection{۲. حل
تشریحی}\label{ux62dux644-ux62aux634ux631ux6ccux62dux6cc}
\begin{note}{الف) محاسبه \((h \circ g) \circ f\)}
ابتدا ترکیب داخلی \((h \circ g)\) را حساب می‌کنیم، سپس \(f\) را وارد
می‌کنیم:
\begin{enumerate}
\def\labelenumi{\arabic{enumi}.}
\tightlist
\item
  \textbf{مرحله اول (\(h \circ g\)):}
  \[(h \circ g)(x) = h(g(x)) = h(\cos x) = (\cos x)^2 - 1 = \cos^2 x - 1\]
\item
  \textbf{مرحله دوم (ترکیب با \(f\)):}
  \[[(h \circ g) \circ f](x) = (h \circ g)(f(x))\] به جای \(x\) در رابطه
  بالا، عبارت \(f(x)\) یعنی \((2x^2+5)\) را می‌گذاریم:
  \[= \cos^2(2x^2+5) - 1\]
\end{enumerate}
\end{note}
\begin{note}{ب) محاسبه \(h \circ (g \circ f)\)}
ابتدا \((g \circ f)\) را حساب می‌کنیم، سپس آن را درون \(h\) می‌اندازیم:
\begin{enumerate}
\def\labelenumi{\arabic{enumi}.}
\tightlist
\item
  \textbf{مرحله اول (\(g \circ f\)):}
  \[(g \circ f)(x) = g(f(x)) = g(2x^2+5) = \cos(2x^2+5)\]
\item
  \textbf{مرحله دوم (ترکیب با \(h\)):}
  \[[h \circ (g \circ f)](x) = h((g \circ f)(x))\] خروجی مرحله قبل را
  درون \(h(x)=x^2-1\) می‌گذاریم:
  \[= (\cos(2x^2+5))^2 - 1 = \cos^2(2x^2+5) - 1\]
\end{enumerate}
\end{note}
\subsection{۳.
نتیجه‌گیری}\label{ux646ux62aux6ccux62cux647ux6afux6ccux631ux6cc}
همانطور که می‌بینید، خروجی هر دو حالت یکسان شد. این یک قانون کلی در
ریاضیات است: \textbf{عمل ترکیب توابع، خاصیت شرکت‌پذیری دارد.}

\clearpage
% ---------------------------------------------------------------------
% Copyright (c) 2026 Arsalan Dalvand & Reyhaneh Darvishi.
% Licensed under CC BY-NC-SA 4.0.
% See LICENSE file for details.
% ---------------------------------------------------------------------

\section{تمرین ۲۰: تابع همانی به عنوان عنصر
خنثی}\label{تمرین-۲۰---نقش-عنصر-خنثی-در-ترکیب-توابع}
\begin{tldr}{خلاصه سریع}
تابع همانی (\(I(x)=x\)) در دنیای توابع، نقش عدد «۱» در ضرب را دارد.
ترکیب هر تابع با تابع همانی، خودِ آن تابع را نتیجه می‌دهد.
\[f \circ I_X = f = I_Y \circ f\]
\end{tldr}
\subsection{۱. صورت
تمرین}\label{ux635ux648ux631ux62a-ux62aux645ux631ux6ccux646}
\begin{info}{سوال}
فرض کنید \(f: X \rightarrow Y\) یک تابع باشد. ثابت کنید ترکیب \(f\) با
توابع همانیِ دامنه (\(I_X\)) و هم‌دامنه (\(I_Y\))، خود تابع \(f\) می‌شود.
\end{info}
\subsection{۲. درک
شهودی}\label{ux62fux631ux6a9-ux634ux647ux648ux62fux6cc}
\begin{itemize}
\tightlist
\item
  \(f \circ I_X\): یعنی اول «هیچ کاری روی ورودی نکن» (\(I_X\))، بعد
  \(f\) را اعمال کن. نتیجه: همان \(f\) اعمال شده.
\item
  \(I_Y \circ f\): یعنی اول \(f\) را اعمال کن، بعد روی خروجی «هیچ کاری
  نکن» (\(I_Y\)). نتیجه: همان خروجی \(f\).
\end{itemize}
\subsection{۳. اثبات
دقیق}\label{ux627ux62bux628ux627ux62a-ux62fux642ux6ccux642}
\begin{info}{اثبات}
برای اثبات برابری دو تابع، باید نشان دهیم برای هر \(x\)، خروجی‌ها یکسان
است.
\textbf{بخش اول (\(f \circ I_X = f\)):} برای هر \(x \in X\):
\[(f \circ I_X)(x) = f(I_X(x))\] چون \(I_X(x) = x\) (تعریف تابع همانی):
\[= f(x)\] \(\therefore f \circ I_X = f\)
\textbf{بخش دوم (\(I_Y \circ f = f\)):} برای هر \(x \in X\)، می‌دانیم
\(y = f(x) \in Y\): \[(I_Y \circ f)(x) = I_Y(f(x))\] چون برای هر عضو در
\(Y\) (مثل \(f(x)\))، تابع همانی آن را تغییر نمی‌دهد: \[= f(x)\]
\(\therefore I_Y \circ f = f\)
\end{info}

\clearpage
% ---------------------------------------------------------------------
% Copyright (c) 2026 Arsalan Dalvand & Reyhaneh Darvishi.
% Licensed under CC BY-NC-SA 4.0.
% See LICENSE file for details.
% ---------------------------------------------------------------------

\section{تمرین ۲۱: خاصیت اصلی تابع
وارون}\label{تمرین-۲۱---ترکیب-تابع-با-وارون-خود}
\begin{tldr}{خلاصه سریع}
اگر تابعی شما را از خانه به مدرسه ببرد (\(f\))، وارون آن شما را از مدرسه
به خانه برمی‌گرداند (\(f^{-1}\)). ترکیب این دو یعنی «رفتن و برگشتن» که
معادل «تکان نخوردن» (تابع همانی) است.
\[f^{-1} \circ f = I_X \quad , \quad f \circ f^{-1} = I_Y\]
\end{tldr}
\subsection{۱. صورت
تمرین}\label{ux635ux648ux631ux62a-ux62aux645ux631ux6ccux646}
\begin{info}{سوال}
اگر \(f: X \rightarrow Y\) یک تابع دوسویی \lr{(Bijective) }باشد و
\(f^{-1}: Y \rightarrow X\) وارون آن باشد، ثابت کنید ترکیب آن‌ها تابع
همانی می‌شود.
\end{info}
\subsection{۲. اثبات}\label{ux627ux62bux628ux627ux62a}
\begin{info}{اثبات جبری}
\textbf{الف) اثبات \(f^{-1} \circ f = I_X\):} برای هر \(x \in X\)، فرض
کنید \(y = f(x)\). طبق تعریف وارون، داریم \(x = f^{-1}(y)\).
\[(f^{-1} \circ f)(x) = f^{-1}(f(x))\] جایگذاری \(f(x)\) با \(y\):
\[= f^{-1}(y)\] جایگذاری \(f^{-1}(y)\) با \(x\): \[= x\] چون ورودی \(x\)
تبدیل به خروجی \(x\) شد، این همان تابع همانی \(I_X\) است.
\textbf{ب) اثبات \(f \circ f^{-1} = I_Y\):} برای هر \(y \in Y\)، فرض
کنید \(x = f^{-1}(y)\). پس \(y = f(x)\).
\[(f \circ f^{-1})(y) = f(f^{-1}(y))\] \[= f(x)\] \[= y\] پس این تابع
روی مجموعه \(Y\) همان همانی (\(I_Y\)) است.
\end{info}

\clearpage
% ---------------------------------------------------------------------
% Copyright (c) 2026 Arsalan Dalvand & Reyhaneh Darvishi.
% Licensed under CC BY-NC-SA 4.0.
% See LICENSE file for details.
% ---------------------------------------------------------------------

\section{تمرین ۲۲: یگانگی وارون (چپ و
راست)}\label{تمرین-۲۲---یگانگی-تابع-وارون}
\begin{tldr}{خلاصه سریع}
اگر تابعی هم «وارون چپ» (\(g\)) داشته باشد و هم «وارون راست» (\(h\))،
آنگاه تابع اصلی حتماً دوسویی است و آن دو وارون با هم برابرند
(\(g=h=f^{-1}\)). \emph{این قضیه ثابت می‌کند که یک تابع نمی‌تواند دو وارون
متفاوت داشته باشد.}
\end{tldr}
\subsection{۱. صورت
تمرین}\label{ux635ux648ux631ux62a-ux62aux645ux631ux6ccux646}
\begin{info}{سوال}
فرض کنید \(f: X \rightarrow Y\). اگر توابع \(g: Y \rightarrow X\) و
\(h: Y \rightarrow X\) طوری باشند که:
\begin{enumerate}
\def\labelenumi{\arabic{enumi}.}
\tightlist
\item
  \(g \circ f = I_X\) (وارون چپ)
\item
  \(f \circ h = I_Y\) (وارون راست)
\end{enumerate}
ثابت کنید \(f\) دوسویی است و \(g = h = f^{-1}\).
\end{info}
\subsection{۲. اثبات (بسیار زیبا و
جبری)}\label{ux627ux62bux628ux627ux62a-ux628ux633ux6ccux627ux631-ux632ux6ccux628ux627-ux648-ux62cux628ux631ux6cc}
\begin{info}{اثبات برابری \(g\) و \(h\)}
از خاصیت شرکت‌پذیری و خاصیت همانی استفاده می‌کنیم.
\[g = g \circ I_Y\] (چون \(I_Y\) خنثی است)
\[= g \circ (f \circ h)\] (جایگذاری \(I_Y\) با فرض \(f \circ h\))
\[= (g \circ f) \circ h\] (استفاده از خاصیت شرکت‌پذیری - پرانتز را جابجا
کردیم)
\[= I_X \circ h\] (جایگذاری \(g \circ f\) با فرض مسئله)
\[= h\] (چون \(I_X\) خنثی است)
\textbf{نتیجه:} \(g = h\). حال چون این تابع هم چپ و هم راست وارون \(f\)
است، پس \(f\) وارون‌پذیر (دوسویی) است و \(g=h=f^{-1}\).
\end{info}
\subsection{۳. اثبات یک‌به‌یک و پوشا بودن (روش
دوم)}\label{ux627ux62bux628ux627ux62a-ux6ccux6a9ux628ux647ux6ccux6a9-ux648-ux67eux648ux634ux627-ux628ux648ux62fux646-ux631ux648ux634-ux62fux648ux645}
\begin{itemize}
\tightlist
\item
  \textbf{یک‌به‌یک:} اگر \(f(x_1)=f(x_2)\)، با اعمال \(g\) به طرفین داریم
  \(g(f(x_1))=g(f(x_2)) \Rightarrow I(x_1)=I(x_2) \Rightarrow x_1=x_2\).
\item
  \textbf{پوشا:} برای هر \(y\)، اگر قرار دهیم \(x=h(y)\)، آنگاه
  \(f(x)=f(h(y))=I_Y(y)=y\). پس برای هر \(y\)، یک \(x\) وجود دارد.
\end{itemize}

\clearpage
% ---------------------------------------------------------------------
% Copyright (c) 2026 Arsalan Dalvand & Reyhaneh Darvishi.
% Licensed under CC BY-NC-SA 4.0.
% See LICENSE file for details.
% ---------------------------------------------------------------------

\section{تمرین ۲۳: انتقال خواص در ترکیب
توابع}\label{تمرین-۲۳---حفظ-خواص-یک‌به‌یک-و-پوشا-در-ترکیب}
\begin{tldr}{خلاصه سریع}
\begin{itemize}
\tightlist
\item
  ترکیبِ توابع \textbf{یک‌به‌یک}، حتماً \textbf{یک‌به‌یک} است.
\item
  ترکیبِ توابع \textbf{پوشا}، حتماً \textbf{پوشا} است.
\item
  (و در نتیجه: ترکیب توابع دوسویی، دوسویی است).
\end{itemize}
\end{tldr}
\subsection{۱. صورت
تمرین}\label{ux635ux648ux631ux62a-ux62aux645ux631ux6ccux646}
\begin{info}{سوال}
فرض کنید \(f: X \rightarrow Y\) و \(g: Y \rightarrow Z\). الف) اگر
\(f, g\) یک‌به‌یک باشند، ثابت کنید \(g \circ f\) یک‌به‌یک است. ب) اگر
\(f, g\) پوشا باشند، ثابت کنید \(g \circ f\) پوشا است.
\end{info}
\subsection{۲. اثبات (الف) -
یک‌به‌یک}\label{ux627ux62bux628ux627ux62a-ux627ux644ux641---ux6ccux6a9ux628ux647ux6ccux6a9}
\begin{info}{اثبات}
فرض کنید \((g \circ f)(x_1) = (g \circ f)(x_2)\). باید ثابت کنیم
\(x_1 = x_2\).
\begin{enumerate}
\def\labelenumi{\arabic{enumi}.}
\tightlist
\item
  طبق تعریف: \(g(f(x_1)) = g(f(x_2))\).
\item
  چون \(g\) \textbf{یک‌به‌یک} است، می‌توانیم \(g\) را از طرفین برداریم:
  \[f(x_1) = f(x_2)\]
\item
  حالا چون \(f\) \textbf{یک‌به‌یک} است، می‌توانیم \(f\) را برداریم:
  \[x_1 = x_2\] \(\blacksquare\)
\end{enumerate}
\end{info}
\subsection{۳. اثبات (ب) -
پوشا}\label{ux627ux62bux628ux627ux62a-ux628---ux67eux648ux634ux627}
\begin{info}{اثبات}
باید ثابت کنیم بردِ تابع ترکیبی، کل مجموعه \(Z\) است
(\((g \circ f)(X) = Z\)).
\begin{enumerate}
\def\labelenumi{\arabic{enumi}.}
\tightlist
\item
  چون \(g\) \textbf{پوشا} است، برد آن کل \(Z\) است (\(g(Y) = Z\)).
\item
  چون \(f\) \textbf{پوشا} است، برد آن کل \(Y\) است (\(f(X) = Y\)).
\item
  حال برد تابع مرکب را حساب می‌کنیم: \[(g \circ f)(X) = g(f(X))\]
\item
  جایگذاری رابطه (۲): \[= g(Y)\]
\item
  جایگذاری رابطه (۱): \[= Z\] \(\blacksquare\)
\end{enumerate}
\end{info}

\end{document}