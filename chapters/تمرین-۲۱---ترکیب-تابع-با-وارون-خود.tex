% ---------------------------------------------------------------------
% Copyright (c) 2026 Arsalan Dalvand & Reyhaneh Darvishi.
% Licensed under CC BY-NC-SA 4.0.
% See LICENSE file for details.
% ---------------------------------------------------------------------

\section{تمرین ۲۱: خاصیت اصلی تابع
وارون}\label{تمرین-۲۱---ترکیب-تابع-با-وارون-خود}
\begin{tldr}{خلاصه سریع}
اگر تابعی شما را از خانه به مدرسه ببرد (\(f\))، وارون آن شما را از مدرسه
به خانه برمی‌گرداند (\(f^{-1}\)). ترکیب این دو یعنی «رفتن و برگشتن» که
معادل «تکان نخوردن» (تابع همانی) است.
\[f^{-1} \circ f = I_X \quad , \quad f \circ f^{-1} = I_Y\]
\end{tldr}
\subsection{۱. صورت
تمرین}\label{ux635ux648ux631ux62a-ux62aux645ux631ux6ccux646}
\begin{info}{سوال}
اگر \(f: X \rightarrow Y\) یک تابع دوسویی \lr{(Bijective) }باشد و
\(f^{-1}: Y \rightarrow X\) وارون آن باشد، ثابت کنید ترکیب آن‌ها تابع
همانی می‌شود.
\end{info}
\subsection{۲. اثبات}\label{ux627ux62bux628ux627ux62a}
\begin{info}{اثبات جبری}
\textbf{الف) اثبات \(f^{-1} \circ f = I_X\):} برای هر \(x \in X\)، فرض
کنید \(y = f(x)\). طبق تعریف وارون، داریم \(x = f^{-1}(y)\).
\[(f^{-1} \circ f)(x) = f^{-1}(f(x))\] جایگذاری \(f(x)\) با \(y\):
\[= f^{-1}(y)\] جایگذاری \(f^{-1}(y)\) با \(x\): \[= x\] چون ورودی \(x\)
تبدیل به خروجی \(x\) شد، این همان تابع همانی \(I_X\) است.
\textbf{ب) اثبات \(f \circ f^{-1} = I_Y\):} برای هر \(y \in Y\)، فرض
کنید \(x = f^{-1}(y)\). پس \(y = f(x)\).
\[(f \circ f^{-1})(y) = f(f^{-1}(y))\] \[= f(x)\] \[= y\] پس این تابع
روی مجموعه \(Y\) همان همانی (\(I_Y\)) است.
\end{info}
