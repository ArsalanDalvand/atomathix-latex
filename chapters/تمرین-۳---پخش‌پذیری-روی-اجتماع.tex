% ---------------------------------------------------------------------
% Copyright (c) 2026 Arsalan Dalvand & Reyhaneh Darvishi.
% Licensed under CC BY-NC-SA 4.0.
% See LICENSE file for details.
% ---------------------------------------------------------------------

\section{تمرین ۳: اثبات پخش‌پذیری حاصلضرب دکارتی روی
اجتماع}\label{تمرین-۳---پخش‌پذیری-روی-اجتماع}
\subsection{۱. صورت
سوال}\label{ux635ux648ux631ux62a-ux633ux648ux627ux644}
قضیه ۱ (ب) را ثابت کنید:
\[A \times (B \cup C) = (A \times B) \cup (A \times C)\]
\subsection{۲. حل تشریحی (روش زنجیره
هم‌ارزی)}\label{ux62dux644-ux62aux634ux631ux6ccux62dux6cc-ux631ux648ux634-ux632ux646ux62cux6ccux631ux647-ux647ux645ux627ux631ux632ux6cc}
برای اثبات تساوی دو مجموعه، نشان می‌دهیم عضو بودن در سمت چپ معادل عضو
بودن در سمت راست است.
\begin{info}{مراحل اثبات}
\[(x,y) \in A \times (B \cup C)\] طبق تعریف حاصلضرب دکارتی:
\[\equiv (x \in A) \wedge (y \in B \cup C)\] طبق تعریف اجتماع:
\[\equiv (x \in A) \wedge (y \in B \lor y \in C)\] طبق قانون پخش‌پذیری
منطق (پخش \(\wedge\) روی \(\lor\)):
\[\equiv [(x \in A) \wedge (y \in B)] \lor [(x \in A) \wedge (y \in C)]\]
طبق تعریف حاصلضرب دکارتی برای هر کروشه:
\[\equiv [(x,y) \in A \times B] \lor [(x,y) \in A \times C]\] طبق تعریف
اجتماع: \[\equiv (x,y) \in (A \times B) \cup (A \times C)\]
\end{info}
