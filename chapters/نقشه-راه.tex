% ---------------------------------------------------------------------
% Copyright (c) 2026 Arsalan Dalvand & Reyhaneh Darvishi.
% Licensed under CC BY-NC-SA 4.0.
% See LICENSE file for details.
% ---------------------------------------------------------------------

\section{🗺️ نقشه راه جامع: مبانی
ریاضیات}\label{نقشه-راه}
\begin{tldr}{درباره این پایگاه دانش}
این فایل، دروازه ورودی \lr{(Map of Content) }به سیستم یادداشت‌های اتمیک
\textbf{مبانی ریاضیات} است. مسیر یادگیری به گونه‌ای طراحی شده که از منطق
گزاره‌ها آغاز، به نظریه مجموعه‌ها گسترش و در نهایت به ساختارهای عمیق رابطه
و تابع ختم می‌شود. برای تسلط کامل، پیشنهاد می‌شود مسیر را \textbf{از بالا
به پایین} طی کنید.
\end{tldr}
\begin{center}\rule{0.5\linewidth}{0.5pt}\end{center}
\section{🏛️ فصل اول: منطق و روش‌های
اثبات}\label{ux641ux635ux644-ux627ux648ux644-ux645ux646ux637ux642-ux648-ux631ux648ux634ux647ux627ux6cc-ux627ux62bux628ux627ux62a}
\begin{info}{هدف فصل}
درک الفبای ریاضیات، آشنایی با گزاره‌ها، سورها و روش‌های استنتاج صحیح.
\end{info}
\subsection{🔹 قضایا و مفاهیم
پایه}\label{ux642ux636ux627ux6ccux627-ux648-ux645ux641ux627ux647ux6ccux645-ux67eux627ux6ccux647}
\begin{itemize}
\tightlist
\item
  \autoref{قضیه-۱---قوانین-جمع-و-اختصار} (ورود و خروج اطلاعات)
\item
  \autoref{قضیه-۲---هم‌ارزی‌های-منطقی-پایه} (قوانین جابجایی، عکس نقیض)
\item
  \autoref{قضیه-۳---قوانین-دمورگان} (رفتار نفی با پرانتزها)
\item
  \autoref{قضیه-۴---قوانین-شرکت-پذیری-و-پخش-پذیری} (معماری جملات مرکب)
\item
  \autoref{قضیه-۵---قیاس-ها} (مدیریت دو راهی‌ها)
\item
  \autoref{قضیه-۶---قواعد-استنتاج} (موتور محرک اثبات‌ها:
  \lr{Modus Ponens/Tollens)}
\item
  \autoref{قضیه-۷---قوانین-تناقض} (رفتار راستگو و تناقض)
\item
  \autoref{قواعد-تسویر} (سورهای وجودی و عمومی)
\item
  \autoref{مفهوم-استقرا} (اثبات روی اعداد طبیعی)
\item
  \autoref{قضیه-۸---ترکیب}
\item
  \autoref{قضیه-۹---دو-جمله-ای} (کاربرد استقرا در جبر)
\end{itemize}
\subsection{🛠️ کارگاه حل مسئله (فصل
۱)}\label{ux6a9ux627ux631ux6afux627ux647-ux62dux644-ux645ux633ux626ux644ux647-ux641ux635ux644-ux6f1}
\begin{note}{تمارین منتخب}
\textbf{بخش اول (منطق پایه):}
\begin{itemize}
\tightlist
\item
  \autoref{تمرین-۵---قانون-توزیع‌پذیری-در-منطق}
\item
  \autoref{تمرین-۶---اثبات-استلزام-شرطی}
\item
  \autoref{تمرین-۷---بازنویسی-ترکیب-دوشرطی}
\end{itemize}
\textbf{بخش دوم (استقرا و جبر):}
\begin{itemize}
\tightlist
\item
  \autoref{تمرین-۱۷---فرمول-مجموع-مکعبات}
\item
  \autoref{تمرین-۱۸---رابطه-مجموع-اعداد-و-مجموع-مکعبات}
\end{itemize}
\end{note}
\begin{center}\rule{0.5\linewidth}{0.5pt}\end{center}
\section{📦 فصل دوم: نظریه
مجموعه‌ها}\label{ux641ux635ux644-ux62fux648ux645-ux646ux638ux631ux6ccux647-ux645ux62cux645ux648ux639ux647ux647ux627}
\begin{info}{هدف فصل}
گذار از منطق به ساختارهای مجموعه‌ای، بررسی بی‌نهایت‌ها و پارادوکس‌های
کلاسیک.
\end{info}
\subsection{🔹 مفاهیم بنیادی
(پیش‌نیاز)}\label{ux645ux641ux627ux647ux6ccux645-ux628ux646ux6ccux627ux62fux6cc-ux67eux6ccux634ux646ux6ccux627ux632}
\begin{itemize}
\tightlist
\item
  \autoref{پیشنیاز---مفاهیم-بنیادین-مجموعه‌ها} (زیرمجموعه، تهی،
  خانواده‌ها)
\item
  \autoref{پیشنیاز---تعریف-مجموعه-توانی} (مجموعه همه زیرمجموعه‌ها)
\end{itemize}
\subsection{🔹 قضایای
اصلی}\label{ux642ux636ux627ux6ccux627ux6cc-ux627ux635ux644ux6cc}
\begin{itemize}
\tightlist
\item
  \autoref{قضیه-۱---شمول-تهی} (تهی زیرمجموعه همه است)
\item
  \autoref{قضیه-۲---تعدی-در-مجموعه-ها} (خاصیت زنجیره‌ای)
\item
  \autoref{قضیه-۳---تعداد-اعضای-مجموعه-توانی} (رابطه نمایی \(2^n\))
\item
  \autoref{قضیه-۴---جبر-مجموعه-ها} (ایزومورفیسم با منطق)
\item
  \autoref{قضیه-۵---متمم-و-زیرمجموعه} (ارتباط عکس نقیض با متمم)
\item
  \autoref{قضیه-۶---دمورگان-در-مجموعه-ها} (حالت متناهی)
\item
  \autoref{قضیه-۷.---تعمیم-اشتراک-و-اجتماع} (رفتار حدی روی تهی)
\item
  \autoref{قضیه-۸---تعمیم-دمورگان} (برای خانواده‌های نامتناهی)
\item
  \autoref{قضیه-۹---تعمیم-پخش-پذیری} (پخش اشتراک بر اجتماع نامتناهی)
\end{itemize}
\subsection{🧠 مبانی و
پارادوکس‌ها}\label{ux645ux628ux627ux646ux6cc-ux648-ux67eux627ux631ux627ux62fux648ux6a9ux633ux647ux627}
\begin{itemize}
\tightlist
\item
  \autoref{قضیه-۱۰---عدم-وجود-مجموعه-جهانی}
\item
  \autoref{مفهوم-پارادوکس-راسل}
\end{itemize}
\subsection{🛠️ کارگاه حل مسئله (فصل
۲)}\label{ux6a9ux627ux631ux6afux627ux647-ux62dux644-ux645ux633ux626ux644ux647-ux641ux635ux644-ux6f2}
\begin{note}{تمارین منتخب}
\begin{itemize}
\tightlist
\item
  \autoref{تمرین-۱۹---مجموعه-توانی-و-اعمال-روی-مجموعه‌ها}
\item
  \autoref{تمرین-۲۰---اثبات-رابطه-متمم-و-تفاضل}
\end{itemize}
\end{note}
\begin{center}\rule{0.5\linewidth}{0.5pt}\end{center}
\section{🔗 فصل سوم: رابطه و
تابع}\label{ux641ux635ux644-ux633ux648ux645-ux631ux627ux628ux637ux647-ux648-ux62aux627ux628ux639}
\begin{info}{هدف فصل}
قلب تپنده ریاضیات مدرن. بررسی چگونگی ارتباط اشیاء، دسته‌بندی آن‌ها (افراز)
و تبدیل آن‌ها (تابع).
\end{info}
\subsection{۱. حاصلضرب دکارتی و
روابط}\label{ux62dux627ux635ux644ux636ux631ux628-ux62fux6a9ux627ux631ux62aux6cc-ux648-ux631ux648ux627ux628ux637}
\paragraph{🔹 قضایا}\label{ux642ux636ux627ux6ccux627}
\begin{itemize}
\tightlist
\item
  \autoref{پیشنیاز---مفاهیم-پایه-رابطه-و-تابع} (زوج مرتب، تعریف رابطه)
\item
  \autoref{قضیه-۱---پخش‌پذیری-حاصلضرب-دکارتی} (پخش‌پذیری روی \(\cap\) و
  \(\cup\))
\item
  \autoref{قضیه-۲---پخش‌پذیری-حاصلضرب-دکارتی-بر-تفاضل} (پخش‌پذیری روی
  تفاضل)
\end{itemize}
\subsection{🛠️ تمارین حاصلضرب دکارتی (حل تمارین سری
۱)}\label{ux62aux645ux627ux631ux6ccux646-ux62dux627ux635ux644ux636ux631ux628-ux62fux6a9ux627ux631ux62aux6cc-ux62dux644-ux62aux645ux627ux631ux6ccux646-ux633ux631ux6cc-ux6f1}
\begin{note}{لیست تمارین}
\begin{itemize}
\tightlist
\item
  \autoref{تمرین-۱---نمایش-هندسی-حاصلضرب-دکارتی}
\item
  \autoref{تمرین-۲---جابجایی-در-حاصلضرب-دکارتی}
\item
  \autoref{تمرین-۳---پخش‌پذیری-روی-اجتماع}
\item
  \autoref{تمرین-۴---شرط-تهی-بودن-حاصلضرب}
\item
  \autoref{تمرین-۵---یکنوایی-حاصلضرب-دکارتی}
\item
  \autoref{تمرین-۶-و-۷---تعداد-اعضا-و-بازیابی-مجموعه}
\item
  \autoref{تمرین-۸---مثال‌های-نقض}
\item
  \autoref{تمرین-۹-و-۱۵---اشتراک-حاصلضرب‌ها}
\item
  \autoref{تمرین-۱۰---تعمیم-حاصلضرب-دکارتی}
\item
  \autoref{تمرین-۱۱-و-۱۲-و-۱۶---قوانین-حذف-و-تساوی}
\item
  \autoref{تمرین-۱۳-و-۱۴---توزیع‌پذیری-اشتباه}
\item
  \autoref{تمرین-۱۷-و-۱۸-و-۱۹---جبر-تفاضل-در-حاصلضرب}
\item
  \autoref{تمرین-۲۰---تعریف-زوج-مرتب-کوراتوسکی}
\end{itemize}
\end{note}
\subsection{🛠️ تمارین ویژگی‌های روابط (حل تمارین سری
۲)}\label{ux62aux645ux627ux631ux6ccux646-ux648ux6ccux698ux6afux6ccux647ux627ux6cc-ux631ux648ux627ux628ux637-ux62dux644-ux62aux645ux627ux631ux6ccux646-ux633ux631ux6cc-ux6f2}
\begin{note}{لیست تمارین}
\begin{itemize}
\tightlist
\item
  \autoref{تمرین-۱---وارونِ-وارون-رابطه}
\item
  \autoref{تمرین-۲---محاسبه-دامنه-و-برد}
\item
  \autoref{تمرین-۳---رابطه-دامنه-و-برد-با-وارون}
\item
  \autoref{تمرین-۴---بررسی-خواص-یک-رابطه}
\item
  \autoref{تمرین-۵---مثال-بازتابی-و-متعدی-غیرمتقارن}
\item
  \autoref{تمرین-۶---مثال-متقارن-و-متعدی-غیربازتابی}
\item
  \autoref{تمرین-۷---ویژگی‌های-رابطه-با-وارون}
\item
  \autoref{تمرین-۸-و-۹---شمارش-رابطه‌ها}
\item
  \autoref{تمرین-۱۰---تحدید-رابطه-(Restriction)}
\item
  \autoref{تمرین-۱۱---تحدید-روی-اجتماع-و-اشتراک}
\item
  \autoref{تمرین-۱۲---نگاره-یا-تصویر-مجموعه-R(X)}
\item
  \autoref{تمرین-۱۳---رابطه-دامنه-و-نگاره-با-وارون}
\item
  \autoref{تمرین-۱۴---دامنه-و-برد-اجتماع-روابط}
\item
  \autoref{تمرین-۱۵---بستار-متقارن-(Symmetric-Closure)}
\item
  \autoref{تمرین-۱۶---درون‌هسته-متقارن}
\item
  \autoref{تمرین-۱۷---وارونِ-اجتماع-و-اشتراک}
\item
  \autoref{تمرین-۱۸---ساخت-اعداد-گویا-(رابطه-هم‌ارزی)}
\end{itemize}
\end{note}
\begin{center}\rule{0.5\linewidth}{0.5pt}\end{center}
\subsection{۲. هم‌ارزی و
افراز}\label{ux647ux645ux627ux631ux632ux6cc-ux648-ux627ux641ux631ux627ux632}
\paragraph{🔹 قضایا}\label{ux642ux636ux627ux6ccux627-1}
\begin{itemize}
\tightlist
\item
  \autoref{قضیه-۳---ویژگی‌های-بنیادی-کلاس-هم‌ارزی} (اصل انطباق یا جدایی)
\item
  \autoref{قضیه-۴---افراز-ناشی-از-رابطه-هم‌ارزی} (تبدیل رابطه به افراز)
\item
  \autoref{قضیه-۵---رابطه-هم‌ارزی-ناشی-از-افراز} (تبدیل افراز به رابطه)
\end{itemize}
\subsection{🛠️ تمارین هم‌ارزی (حل تمارین سری
۳)}\label{ux62aux645ux627ux631ux6ccux646-ux647ux645ux627ux631ux632ux6cc-ux62dux644-ux62aux645ux627ux631ux6ccux646-ux633ux631ux6cc-ux6f3}
\begin{note}{لیست تمارین}
\begin{itemize}
\tightlist
\item
  \autoref{تمرین-۱---بررسی-افراز-و-رابطه-نظیر}
\item
  \autoref{تمرین-۲---استخراج-افراز-از-رابطه}
\item
  \autoref{تمرین-۳---تناظر-یک‌به‌یک-افراز-و-رابطه}
\item
  \autoref{تمرین-۴---افراز-سه-قسمتی}
\item
  \autoref{تمرین-۵---هم‌نهشتی-اعداد-صحیح}
\end{itemize}
\end{note}
\begin{center}\rule{0.5\linewidth}{0.5pt}\end{center}
\subsection{\texorpdfstring{۳. تابع
\lr{(Function)}}{۳. تابع }}\label{ux62aux627ux628ux639-function}
\paragraph{🔹 مفاهیم و قضایای
پایه}\label{ux645ux641ux627ux647ux6ccux645-ux648-ux642ux636ux627ux6ccux627ux6cc-ux67eux627ux6ccux647}
\begin{itemize}
\tightlist
\item
  \autoref{مفهوم-تابع-و-قضیه-۶---تعریف-و-دامنه} (توسعه هم‌دامنه)
\item
  \autoref{قضیه-۷---شرط-تساوی-توابع} (کی دو تابع برابرند؟)
\item
  \autoref{قضیه-۸---لم-چسباندن-(اجتماع-توابع)} (چسباندن توابع به هم)
\item
  \autoref{توابع-خاص---همانی-و-ثابت} (ابزارهای پایه)
\end{itemize}
\subsection{🛠️ تمارین پایه تابع (حل تمارین سری
۴)}\label{ux62aux645ux627ux631ux6ccux646-ux67eux627ux6ccux647-ux62aux627ux628ux639-ux62dux644-ux62aux645ux627ux631ux6ccux646-ux633ux631ux6cc-ux6f4}
\begin{note}{لیست تمارین}
\begin{itemize}
\tightlist
\item
  \autoref{تمرین-۱۳---تحدید-تابع}
\item
  \autoref{تمرین-۱۴---زیرمجموعه-تابع}
\item
  \autoref{تمرین-۱۵---تابع-بازتابی}
\item
  \autoref{تمرین-۱۶---تابع-متقارن}
\end{itemize}
\end{note}
\begin{center}\rule{0.5\linewidth}{0.5pt}\end{center}
\subsection{\texorpdfstring{۴. تصویر و تصویر وارون \lr{(Image }\&
\lr{Inverse Image)}}{۴. تصویر و تصویر وارون \& }}\label{ux62aux635ux648ux6ccux631-ux648-ux62aux635ux648ux6ccux631-ux648ux627ux631ux648ux646-image-inverse-image}
\paragraph{🔹 قضایا}\label{ux642ux636ux627ux6ccux627-2}
\begin{itemize}
\tightlist
\item
  \autoref{قضیه-۹---تصویر-و-تصویر-وارون-مجموعه} (تعریف \(f(A)\) و
  \(f^{-1}(B)\))
\item
  \autoref{قضیه-۱۰---تعمیم-تصویر-اجتماع-و-اشتراک} (ضعف تصویر در اشتراک)
\item
  \autoref{قضیه-۱۱---رفتار-جبری-تصویر-وارون} (رفتار ایده‌آل تصویر وارون)
\item
  \autoref{قضیه-۱۲---پخش‌پذیری-تصویر-وارون-بر-تفاضل} (سازگاری کامل وارون)
\item
  \autoref{قضیه-۱۳---حفظ-اشتراک-در-توابع-یک‌به‌یک} (شرط رفع ضعف تصویر)
\end{itemize}
\subsection{🛠️ تمارین تصویر مجموعه (سری
۵)}\label{ux62aux645ux627ux631ux6ccux646-ux62aux635ux648ux6ccux631-ux645ux62cux645ux648ux639ux647-ux633ux631ux6cc-ux6f5}
\begin{note}{لیست تمارین}
\begin{itemize}
\tightlist
\item
  \autoref{تمرین-۹-(الف)---زیرمجموعه-بودن-A-در-وارون-تصویر-A}
\item
  \autoref{تمرین-۹-(ب)---زیرمجموعه-بودن-تصویر-وارون-B-در-B}
\item
  \autoref{تمرین-۱۰---تساوی-وارون-متمم-B-با-متمم-وارون-B}
\item
  \autoref{تمرین-۱۱---مثال-نقض-تصویر-تفاضل-مجموعه‌ها}
\item
  \autoref{تمرین-تکمیلی---تساوی-تصویر-اشتراک-A-و-وارون-B}
\end{itemize}
\end{note}
\subsection{🛠️ تمارین تصویر مجموعه (سری
۶)}\label{ux62aux645ux627ux631ux6ccux646-ux62aux635ux648ux6ccux631-ux645ux62cux645ux648ux639ux647-ux633ux631ux6cc-ux6f6}
\begin{note}{لیست تمارین}
\begin{itemize}
\tightlist
\item
  \autoref{تمرین-۱۲---شرط-تساوی-در-روابط-نگاره-و-وارون}
\item
  \autoref{تمرین-۱۴---توزیع‌پذیری-تصویر-روی-تفاضل}
\item
  \autoref{تمرین-۱۵---تعمیم-تصویر-تفاضل}
\item
  \autoref{تمرین-۱۸---توابع-خود-وارون-و-تقارن}
\end{itemize}
\end{note}
\begin{center}\rule{0.5\linewidth}{0.5pt}\end{center}
\subsection{۵. جبر توابع (ترکیب و
وارون)}\label{ux62cux628ux631-ux62aux648ux627ux628ux639-ux62aux631ux6a9ux6ccux628-ux648-ux648ux627ux631ux648ux646}
\subsubsection{🔹 قضایا}\label{ux642ux636ux627ux6ccux627-3}
\begin{itemize}
\tightlist
\item
  \autoref{انواع-توابع---یک‌به‌یک-پوشا-و-دوسویی} (دسته‌بندی توابع)
\item
  \autoref{مفهوم-ترکیب-توابع} (ماشین‌های سری)
\item
  \autoref{قضیه-۱۴---وجود-و-ویژگی‌های-تابع-وارون} (شرط دوسویی بودن)
\item
  \autoref{قضیه-۱۵---شرکت‌پذیری-ترکیب-توابع} (معماری نیم‌گروهی)
\item
  \autoref{قضیه-۱۶---وارون‌های-یک‌طرفه} (وارون‌های چپ و راست)
\end{itemize}
\subsection{🛠️ تمارین جبر توابع (سری
۷)}\label{ux62aux645ux627ux631ux6ccux646-ux62cux628ux631-ux62aux648ux627ux628ux639-ux633ux631ux6cc-ux6f7}
\begin{note}{لیست تمارین}
\begin{itemize}
\tightlist
\item
  \autoref{تمرین-۱۹---خاصیت-شرکت‌پذیری-ترکیب-توابع}
\item
  \autoref{تمرین-۲۰---نقش-عنصر-خنثی-در-ترکیب-توابع}
\item
  \autoref{تمرین-۲۱---ترکیب-تابع-با-وارون-خود}
\item
  \autoref{تمرین-۲۲---یگانگی-تابع-وارون}
\item
  \autoref{تمرین-۲۳---حفظ-خواص-یک‌به‌یک-و-پوشا-در-ترکیب}
\end{itemize}
\end{note}
\begin{center}\rule{0.5\linewidth}{0.5pt}\end{center}
\begin{info}{پایان مسیر}
شما اکنون به تمامی سرفصل‌های این دوره مسلط هستید. برای مرور، می‌توانید از
تگ‌های موجود در فایل‌ها استفاده کنید.
\end{info}
