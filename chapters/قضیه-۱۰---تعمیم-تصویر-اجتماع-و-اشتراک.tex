% ---------------------------------------------------------------------
% Copyright (c) 2026 Arsalan Dalvand & Reyhaneh Darvishi.
% Licensed under CC BY-NC-SA 4.0.
% See LICENSE file for details.
% ---------------------------------------------------------------------

\section{قضیه ۱۰: رفتار تصویر تابع با اجتماع و اشتراک
تعمیم‌یافته}\label{قضیه-۱۰---تعمیم-تصویر-اجتماع-و-اشتراک}
\begin{tldr}{خلاصه سریع}
این قضیه، \textbf{\autoref{قضیه-۹---تصویر-و-تصویر-وارون-مجموعه}} را برای
خانواده‌های نامتناهی تعمیم می‌دهد. نتیجه مهم این است که عملگر تصویر
(\(f\)) با عملگر اجتماع (\(\bigcup\)) کاملاً جابجا می‌شود (هم‌ریختی)، اما
در برابر اشتراک (\(\bigcap\)) ضعیف عمل کرده و تنها یک رابطه زیرمجموعه‌ای
را حفظ می‌کند.
\end{tldr}
\subsection{۱. متن ریاضی
قضیه}\label{ux645ux62aux646-ux631ux6ccux627ux636ux6cc-ux642ux636ux6ccux647}
فرض کنید \(f: X \to Y\) یک تابع باشد و
\(\{A_\gamma\}_{\gamma \in \Gamma}\) خانواده‌ای از زیرمجموعه‌های دامنه
\(X\) باشد.
\begin{theorembox}{قضیه ۱۰}
\textbf{الف) حفظ دقیق اجتماع:}
\[f\left(\bigcup_{\gamma \in \Gamma} A_\gamma\right) = \bigcup_{\gamma \in \Gamma} f(A_\gamma)\]
\textbf{ب) حفظ شمول در اشتراک:}
\[f\left(\bigcap_{\gamma \in \Gamma} A_\gamma\right) \subseteq \bigcap_{\gamma \in \Gamma} f(A_\gamma)\]
\end{theorembox}
\subsection{\texorpdfstring{۲. اثبات صوری
\lr{(Formal Proof)}}{۲. اثبات صوری }}\label{ux627ux62bux628ux627ux62a-ux635ux648ux631ux6cc-formal-proof}
\subsubsection{اثبات قسمت (الف): تساوی
اجتماع}\label{ux627ux62bux628ux627ux62a-ux642ux633ux645ux62a-ux627ux644ux641-ux62aux633ux627ux648ux6cc-ux627ux62cux62aux645ux627ux639}
برای اثبات تساوی، نشان می‌دهیم گزاره‌نمای عضویت در طرفین هم‌ارز است. نکته
کلیدی، \textbf{جابه‎جایی دو سور وجودی} است.
\begin{info}{برهان}
\[y \in f\left(\bigcup_{\gamma \in \Gamma} A_\gamma\right)\] ۱. طبق
تعریف تصویر مستقیم:
\[\iff \exists x \left( x \in \bigcup_{\gamma \in \Gamma} A_\gamma \wedge f(x) = y \right)\]
۲. طبق تعریف اجتماع تعمیم‌یافته (سور وجودی):
\[\iff \exists x \left( [\exists \gamma \in \Gamma, x \in A_\gamma] \wedge f(x) = y \right)\]
۳. طبق قوانین منطق مرتبه اول، می‌توانیم ترتیب دو سور وجودی را عوض کنیم و
گزاره مستقل (\(f(x)=y\)) را به داخل ببریم:
\[\iff \exists \gamma \in \Gamma \left( \exists x [x \in A_\gamma \wedge f(x) = y] \right)\]
۴. عبارت داخل پرانتز دقیقاً تعریف \(y \in f(A_\gamma)\) است:
\[\iff \exists \gamma \in \Gamma \left( y \in f(A_\gamma) \right)\]
۵. طبق تعریف اجتماع تعمیم‌یافته:
\[\iff y \in \bigcup_{\gamma \in \Gamma} f(A_\gamma)\]
\textbf{نتیجه:} دو مجموعه با هم برابرند.
\end{info}
\subsubsection{اثبات قسمت (ب): شمول
اشتراک}\label{ux627ux62bux628ux627ux62a-ux642ux633ux645ux62a-ux628-ux634ux645ux648ux644-ux627ux634ux62aux631ux627ux6a9}
در اینجا تساوی برقرار نیست زیرا سور وجودی (در تعریف تصویر) و سور عمومی
(در تعریف اشتراک) با هم جابجا نمی‌شوند.
\begin{info}{برهان}
۱. فرض کنید \(y\) عضو دلخواهی از سمت چپ باشد:
\[y \in f\left(\bigcap_{\gamma \in \Gamma} A_\gamma\right)\]
۲. طبق تعریف تصویر، عنصری مانند \(x_0\) وجود دارد که:
\[x_0 \in \bigcap_{\gamma \in \Gamma} A_\gamma \quad \text{و} \quad f(x_0) = y\]
۳. طبق تعریف اشتراک، \(x_0\) در تمام مجموعه‌ها حضور دارد:
\[\forall \gamma \in \Gamma, x_0 \in A_\gamma\]
۴. چون \(x_0 \in A_\gamma\) و \(f(x_0) = y\)، پس برای هر \(\gamma\)،
عنصر \(y\) تصویری از عضوی در \(A_\gamma\) است. بنابراین:
\[\forall \gamma \in \Gamma, y \in f(A_\gamma)\]
۵. طبق تعریف اشتراک تعمیم‌یافته:
\[y \in \bigcap_{\gamma \in \Gamma} f(A_\gamma)\]
\textbf{نتیجه:} \(LHS \subseteq RHS\).
\end{info}
\subsection{\texorpdfstring{۳. شبکه ارتباطی با سایر قضایا
\lr{(Analytic Map)}}{۳. شبکه ارتباطی با سایر قضایا }}\label{ux634ux628ux6a9ux647-ux627ux631ux62aux628ux627ux637ux6cc-ux628ux627-ux633ux627ux6ccux631-ux642ux636ux627ux6ccux627-analytic-map}
\subsubsection{\texorpdfstring{۱. تعمیم
\autoref{قضیه-۹---تصویر-و-تصویر-وارون-مجموعه}}{۱. تعمیم }}\label{ux62aux639ux645ux6ccux645-ux642ux636ux6ccux647-ux6f9---ux62aux635ux648ux6ccux631-ux648-ux62aux635ux648ux6ccux631-ux648ux627ux631ux648ux646-ux645ux62cux645ux648ux639ux647}
\begin{itemize}
\tightlist
\item
  \textbf{گذر از متناهی به نامتناهی:} قضیه ۹ بیان می‌کرد که
  \(f(A \cup B) = f(A) \cup f(B)\). قضیه ۱۰ نشان می‌دهد که این ویژگی جبری
  حتی اگر تعداد مجموعه‌ها بی‌نهایت باشد (خانواده ایندکس‌دار) همچنان برقرار
  است. این ویژگی در توپولوژی برای تعریف پیوستگی با مجموعه‌های باز بسیار
  حیاتی است.
\end{itemize}
\subsubsection{\texorpdfstring{۲. ارتباط با
\autoref{قضیه-۱---پخش‌پذیری-حاصلضرب-دکارتی} (ریشه منطقی عدم
تساوی)}{۲. ارتباط با  (ریشه منطقی عدم تساوی)}}\label{ux627ux631ux62aux628ux627ux637-ux628ux627-ux642ux636ux6ccux647-ux6f1---ux67eux62eux634ux67eux630ux6ccux631ux6cc-ux62dux627ux635ux644ux636ux631ux628-ux62fux6a9ux627ux631ux62aux6cc-ux631ux6ccux634ux647-ux645ux646ux637ux642ux6cc-ux639ux62fux645-ux62aux633ux627ux648ux6cc}
\begin{itemize}
\tightlist
\item
  \textbf{چرا اشتراک مساوی نیست؟} در منطق گزاره‌ها، می‌دانیم که
  \(\exists x (P(x) \wedge Q(x))\) لزوماً هم‌ارز با
  \((\exists x P(x)) \wedge (\exists x Q(x))\) نیست.
  \begin{itemize}
  \tightlist
  \item
    مثال نقض کلاسیک: \(A_1 = \{1\}, A_2 = \{2\}\) و تابع ثابت
    \(f(x) = c\).
  \item
    اشتراک دامنه‌ها تهی است (\(f(\emptyset) = \emptyset\)).
  \item
    اما اشتراک تصاویر \(\{c\} \cap \{c\} = \{c\}\) است.
  \item
    این شکاف منطقی تنها زمانی پر می‌شود که تابع \textbf{یک‌به‌یک} باشد (که
    در قضایای بعدی بررسی می‌شود).
  \end{itemize}
\end{itemize}
\subsubsection{\texorpdfstring{۳. ارتباط با
\autoref{قضیه-۸---لم-چسباندن-(اجتماع-توابع)}}{۳. ارتباط با }}\label{ux627ux631ux62aux628ux627ux637-ux628ux627-ux642ux636ux6ccux647-ux6f8---ux644ux645-ux686ux633ux628ux627ux646ux62fux646-ux627ux62cux62aux645ux627ux639-ux62aux648ux627ux628ux639}
\begin{itemize}
\tightlist
\item
  \textbf{سازگاری با چسباندن:} بخش (الف) این قضیه تضمین می‌کند که اگر
  توابعی را روی دامنه‌های \(A_\gamma\) تعریف کنیم، تصویر نهایی اجتماعِ
  دامنه‌ها، دقیقاً برابر با اجتماعِ بردهای جزئی است. این پایه منطقی برای
  تحلیل توابع چندضابطه‌ای روی دامنه‌های پیچیده است.
\end{itemize}
