% ---------------------------------------------------------------------
% Copyright (c) 2026 Arsalan Dalvand & Reyhaneh Darvishi.
% Licensed under CC BY-NC-SA 4.0.
% See LICENSE file for details.
% ---------------------------------------------------------------------

\section{تمرین ۴: افراز سه قسمتی و تعیین
کلاس‌ها}\label{تمرین-۴---افراز-سه-قسمتی}
\subsection{۱. صورت
سوال}\label{ux635ux648ux631ux62a-ux633ux648ux627ux644}
\begin{info}{فرض کنید \(X=\{a,b,c,d,e\}\) و \(\mathcal{P}=\{\{a,b\},\{c\},\{d,e\}\}\) باشد.}
\textbf{الف)} نشان دهید که \(\mathcal{P}\) یک افراز \(X\) است.
\textbf{ب)} رابطه هم‌ارزی \(X/\mathcal{P}\) را به صورت مجموعه جفت‌های مرتب
مشخص کنید. \textbf{پ)} مجموعه خارج‌قسمتی \(X/\mathcal{P}\) را
\(\mathcal{E}\) بنامید و کلاس‌های \(a/\mathcal{E}\)، \(c/\mathcal{E}\)،
\(d/\mathcal{E}\) و \(e/\mathcal{E}\) را صریحاً مشخص کنید.
\end{info}
\subsection{۲. استراتژی
حل}\label{ux627ux633ux62aux631ux627ux62aux698ux6cc-ux62dux644}
\begin{itemize}
\tightlist
\item
  \textbf{الف:} بررسی سه شرط افراز (ناتهی بودن، جدا بودن، پوشش کامل).
\item
  \textbf{ب:} تشکیل رابطه با ضرب دکارتی هر مجموعه در خودش (مشابه تمرین
  ۱).
\item
  \textbf{پ:} تعیین کلاس هم‌ارزی برای هر عضو (یعنی آن عضو متعلق به کدام
  زیرمجموعه از افراز است).
\end{itemize}
\subsection{۳. حل
تشریحی}\label{ux62dux644-ux62aux634ux631ux6ccux62dux6cc}
\begin{info}{پاسخ}
\textbf{قسمت (الف):} فرض کنیم \(A_1=\{a,b\}\)، \(A_2=\{c\}\) و
\(A_3=\{d,e\}\). ۱. هر سه مجموعه ناتهی هستند. ۲. اشتراک آنها دو به دو
تهی است (\(A_1 \cap A_2 = \emptyset\) و \ldots). ۳. اجتماع آنها کل \(X\)
را می‌سازد: \(\{a,b\} \cup \{c\} \cup \{d,e\} = X\). پس \(\mathcal{P}\)
یک افراز است.
\textbf{قسمت (ب): رابطه هم‌ارزی} رابطه برابر است با
\((A_1 \times A_1) \cup (A_2 \times A_2) \cup (A_3 \times A_3)\):
\[R = \{(a,a),(b,b),(a,b),(b,a)\} \cup \{(c,c)\} \cup \{(d,d),(e,e),(d,e),(e,d)\}\]
مجموعاً ۹ زوج مرتب خواهد داشت.
\textbf{قسمت (پ): کلاس‌های هم‌ارزی} کلاس هم‌ارزی \(x\) (نماد
\(x/\mathcal{E}\)) همان زیرمجموعه‌ای از افراز است که \(x\) در آن قرار
دارد:
\begin{itemize}
\tightlist
\item
  \(a\) در بسته \(\{a,b\}\) است \(\Rightarrow a/\mathcal{E} = \{a,b\}\)
\item
  \(b\) هم در همان بسته است \(\Rightarrow b/\mathcal{E} = \{a,b\}\)
\item
  \(c\) در بسته \(\{c\}\) است \(\Rightarrow c/\mathcal{E} = \{c\}\)
\item
  \(d\) در بسته \(\{d,e\}\) است \(\Rightarrow d/\mathcal{E} = \{d,e\}\)
\item
  \(e\) در بسته \(\{d,e\}\) است \(\Rightarrow e/\mathcal{E} = \{d,e\}\)
\end{itemize}
\end{info}
