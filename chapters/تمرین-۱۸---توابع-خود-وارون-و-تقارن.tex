% ---------------------------------------------------------------------
% Copyright (c) 2026 Arsalan Dalvand & Reyhaneh Darvishi.
% Licensed under CC BY-NC-SA 4.0.
% See LICENSE file for details.
% ---------------------------------------------------------------------

\section{\texorpdfstring{تمرین ۱۶: توابع خود-وارون
\lr{(Involution)}}{تمرین ۱۶: توابع خود-وارون }}\label{تمرین-۱۸---توابع-خود-وارون-و-تقارن}
\begin{tldr}{خلاصه سریع}
اگر تابعی خاصیت \(f(f(x)) = x\) را داشته باشد (مثل تابع \(f(x) = -x\) یا
\(f(x) = 1/x\))، نمودار آن نسبت به نیمساز ربع اول و سوم (\(y=x\)) متقارن
است. در زبان روابط، این یعنی رابطه \(f\) یک \textbf{رابطه متقارن} است.
\end{tldr}
\subsection{۱. صورت
تمرین}\label{ux635ux648ux631ux62a-ux62aux645ux631ux6ccux646}
\begin{info}{سوال}
فرض کنید \(f: X \rightarrow X\) تابعی باشد که برای هر \(x \in X\)، داشته
باشیم \(f(f(x)) = x\). ثابت کنید \(f\) (به عنوان یک رابطه) متقارن است.
\end{info}
\subsection{۲. اثبات}\label{ux627ux62bux628ux627ux62a}
\begin{info}{اثبات}
یک رابطه \(R\) متقارن است اگر: \((x, y) \in R \implies (y, x) \in R\).
در اینجا رابطه ما تابع \(f\) است، پس زوج مرتب‌ها به صورت \((x, f(x))\)
هستند.
\begin{enumerate}
\def\labelenumi{\arabic{enumi}.}
\tightlist
\item
  فرض کنید \((x, y) \in f\). این یعنی \(y = f(x)\).
\item
  می‌خواهیم ثابت کنیم \((y, x) \in f\). یعنی باید نشان دهیم \(x = f(y)\).
\item
  از فرض مسئله استفاده می‌کنیم: \(f(f(x)) = x\).
\item
  در رابطه بالا به جای \(f(x)\)، مقدار \(y\) (از مرحله ۱) را قرار
  می‌دهیم: \[f(y) = x\]
\item
  رابطه \(x = f(y)\) دقیقاً به این معنی است که زوج مرتب \((y, x)\) متعلق
  به تابع \(f\) است.
\end{enumerate}
پس \(f\) یک رابطه متقارن است. \(\blacksquare\)
\end{info}
