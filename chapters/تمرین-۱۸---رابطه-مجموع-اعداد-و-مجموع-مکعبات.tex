% ---------------------------------------------------------------------
% Copyright (c) 2026 Arsalan Dalvand & Reyhaneh Darvishi.
% Licensed under CC BY-NC-SA 4.0.
% See LICENSE file for details.
% ---------------------------------------------------------------------

\section{تمرین ۱۸: قضیه نیکوماخوس (رابطه توان ۲ و
۳)}\label{تمرین-۱۸---رابطه-مجموع-اعداد-و-مجموع-مکعبات}
\begin{tldr}{خلاصه سریع}
یک حقیقت بسیار زیبا در ریاضیات: «مربعِ مجموعِ اعداد» برابر است با «مجموعِ
مکعبِ اعداد». \[(1 + 2 + \dots + n)^2 = 1^3 + 2^3 + \dots + n^3\]
\end{tldr}
\subsection{۱. صورت
تمرین}\label{ux635ux648ux631ux62a-ux62aux645ux631ux6ccux646}
\begin{info}{سوال}
ثابت کنید برای تمام اعداد طبیعی \(n\):
\[(1 + 2 + 3 + \dots + n)^2 = 1^3 + 2^3 + 3^3 + \dots + n^3\]
\end{info}
\subsection{۲. حل
تشریحی}\label{ux62dux644-ux62aux634ux631ux6ccux62dux6cc}
\begin{info}{استراتژی حل}
به جای اثبات مستقیم با استقراء (که ممکن است طولانی شود)، از نتایج
تمرین‌های قبلی استفاده می‌کنیم. ما فرمولِ داخل پرانتز (مجموع اعداد) و فرمولِ
سمت راست (مجموع مکعبات) را جداگانه می‌دانیم. کافیست نشان دهیم این دو با
هم سازگارند.
\end{info}
\begin{info}{اثبات جبری}
\begin{enumerate}
\def\labelenumi{\arabic{enumi}.}
\item
  \textbf{محاسبه سمت چپ \lr{(LHS):}} می‌دانیم مجموع اعداد طبیعی (تصاعد
  حسابی) برابر است با \(\frac{n(n+1)}{2}\). (تمرین ۴ کتاب) پس سمت چپ
  می‌شود:
  \[\text{LHS} = \left( \sum_{i=1}^n i \right)^2 = \left( \frac{n(n+1)}{2} \right)^2\]
  با توان رساندن صورت و مخرج: \[= \frac{n^2(n+1)^2}{4}\]
\item
  \textbf{محاسبه سمت راست \lr{(RHS):}} طبق تمرین ۱۷ (فایل قبلی)، ثابت
  کردیم که مجموع مکعبات برابر است با:
  \[\text{RHS} = \sum_{i=1}^n i^3 = \frac{n^2(n+1)^2}{4}\]
\item
  \textbf{نتیجه‌گیری:} چون \(\text{LHS} = \text{RHS}\)، پس تساوی برقرار
  است. \[(1 + \dots + n)^2 = 1^3 + \dots + n^3\] \(\blacksquare\)
\end{enumerate}
\end{info}
