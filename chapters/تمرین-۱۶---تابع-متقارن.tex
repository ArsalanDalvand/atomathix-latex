% ---------------------------------------------------------------------
% Copyright (c) 2026 Arsalan Dalvand & Reyhaneh Darvishi.
% Licensed under CC BY-NC-SA 4.0.
% See LICENSE file for details.
% ---------------------------------------------------------------------

\section{تمرین ۱۶: تابع
متقارن}\label{تمرین-۱۶---تابع-متقارن}
\subsection{۱. صورت
سوال}\label{ux635ux648ux631ux62a-ux633ux648ux627ux644}
\begin{info}{فرض کنید \(X\) بازه بسته واحد \([0, 1]\) باشد. یک تابع \(f: X \to X\) بیابید که یک \textbf{رابطه متقارن} روی \(X\) باشد.}
\end{info}
\subsection{۲. استراتژی
حل}\label{ux627ux633ux62aux631ux627ux62aux698ux6cc-ux62dux644}
بیایید تحلیل کنیم که «تابع متقارن» چه ویژگی ریاضی‌ای دارد:
\begin{itemize}
\tightlist
\item
  \textbf{رابطه:} \(f\) مجموعه‌ای از زوج‌های مرتب \((x, y)\) است که
  \(y = f(x)\).
\item
  \textbf{متقارن بودن:} اگر \((x, y) \in f\) باشد، باید \((y, x) \in f\)
  نیز باشد.
  \begin{itemize}
  \tightlist
  \item
    یعنی اگر \(y = f(x)\)، آنگاه باید \(x = f(y)\).
  \item
    با جایگذاری \(y\): \(x = f(f(x))\).
  \end{itemize}
\end{itemize}
بنابراین ما دنبال تابعی هستیم که \textbf{وارون خودش} باشد
(\(f = f^{-1}\)) یا به عبارتی \(f(f(x)) = x\). نمودار چنین تابعی باید
نسبت به خط \(y=x\) متقارن باشد.
\subsection{۳. حل
تشریحی}\label{ux62dux644-ux62aux634ux631ux6ccux62dux6cc}
ما باید تابعی \(f: [0,1] \to [0,1]\) مثال بزنیم که خاصیت \(f(f(x))=x\)
را داشته باشد.
\begin{note}{مثال ۱: تابع همانی}
ساده‌ترین پاسخ، تابع همانی است: \[f(x) = x\] بررسی تقارن: اگر
\((x, y) \in f \implies y = x \implies x = y \implies (y, x) \in f\).
(این پاسخ صحیح است، اما معمولاً مثال‌های غیربدیهی مد نظر هستند).
\end{note}
\begin{note}{مثال ۲: تابع مکمل (پیشنهادی)}
یک مثال غیربدیهی و هندسی: \[f(x) = 1 - x\]
\textbf{بررسی شرایط:}
\begin{enumerate}
\def\labelenumi{\arabic{enumi}.}
\tightlist
\item
  \textbf{دامنه و برد:} اگر \(x \in [0, 1]\)، آنگاه \(1-x\) نیز در
  \([0, 1]\) است. پس \(f: X \to X\) برقرار است.
\item
  \textbf{تقارن:} فرض کنیم \((a, b) \in f\).
  \[b = 1 - a \implies a = 1 - b \implies (b, a) \in f\]
\end{enumerate}
\textbf{نتیجه:} تابع \(f(x) = 1-x\) یک تابع متقارن روی بازه \([0, 1]\)
است.
\end{note}
