% ---------------------------------------------------------------------
% Copyright (c) 2026 Arsalan Dalvand & Reyhaneh Darvishi.
% Licensed under CC BY-NC-SA 4.0.
% See LICENSE file for details.
% ---------------------------------------------------------------------

\section{\texorpdfstring{مفهوم استقرای ریاضی
\lr{(Mathematical Induction)}}{مفهوم استقرای ریاضی }}\label{مفهوم-استقرا}
\begin{tldr}{خلاصه سریع}
استقرای ریاضی یک تکنیک قدرتمند برای اثبات قضایایی است که برای «تمام
اعداد طبیعی» بیان می‌شوند. این روش شبیه «ریختن دومینو» است: ۱. ضربه به
اولی (پایه)، ۲. اطمینان از اینکه افتادن هر مهره باعث افتادن مهره بعدی
می‌شود (گام استقرا).
\end{tldr}
\subsection{۱. درک شهودی (اثر
دومینو)}\label{ux62fux631ux6a9-ux634ux647ux648ux62fux6cc-ux627ux62bux631-ux62fux648ux645ux6ccux646ux648}
فرض کنید صفی بی‌نهایت از دومینوها چیده‌ایم. برای اینکه مطمئن شویم
\textbf{همه} دومینوها می‌ریزند، فقط باید دو چیز را ثابت کنیم:
\begin{enumerate}
\def\labelenumi{\arabic{enumi}.}
\tightlist
\item
  \textbf{شرط شروع:} دومینو اول می‌افتد.
\item
  \textbf{مکانیزم انتقال:} هر دومینویی که بیفتد، حتماً دومینو بعدی‌اش را
  می‌اندازد. اگر این دو برقرار باشند، کل صف تا بی‌نهایت خواهد ریخت.
\end{enumerate}
\subsection{۲. متن ریاضی اصل
استقرا}\label{ux645ux62aux646-ux631ux6ccux627ux636ux6cc-ux627ux635ux644-ux627ux633ux62aux642ux631ux627}
اگر \(P(n)\) یک حکم مربوط به عدد طبیعی \(n\) باشد، چنانچه دو شرط زیر
برقرار باشند:
\begin{enumerate}
\def\labelenumi{\arabic{enumi}.}
\tightlist
\item
  \textbf{پایه استقرا:} \(P(1)\) راست باشد.
\item
  \textbf{گام استقرا:} برای هر عدد طبیعی \(k\)، اگر \(P(k)\) راست باشد،
  آنگاه \(P(k+1)\) نیز راست باشد (\(P(k) \Rightarrow P(k+1)\)).
\end{enumerate}
آنگاه \(P(n)\) برای \textbf{هر عدد طبیعی} \(n\) راست است.
\subsection{۳. مثال آموزشی (جمع
اعداد)}\label{ux645ux62bux627ux644-ux622ux645ux648ux632ux634ux6cc-ux62cux645ux639-ux627ux639ux62fux627ux62f}
\textbf{حکم:} ثابت کنید \(1 + 2 + \dots + n = \frac{n(n+1)}{2}\).
\begin{info}{مراحل اثبات}
\textbf{۱. بررسی پایه (\(n=1\)):} \[1 = \frac{1(1+1)}{2} = 1\] (سمت چپ و
راست برابرند، پس درست است).
\textbf{۲. فرض استقرا (\(n=k\)):} فرض می‌کنیم حکم برای \(k\) درست است:
\[1 + \dots + k = \frac{k(k+1)}{2}\]
\textbf{۳. حکم استقرا (\(n=k+1\)):} باید ثابت کنیم تساوی برای \(k+1\) هم
برقرار است. به طرفین فرض استقرا، عدد بعدی یعنی \((k+1)\) را اضافه
می‌کنیم: \[1 + \dots + k + (k+1) = \frac{k(k+1)}{2} + (k+1)\] با مخرج
مشترک گرفتن: \[= \frac{k(k+1) + 2(k+1)}{2} = \frac{(k+1)(k+2)}{2}\] این
دقیقاً همان فرمول حکم برای \(n=k+1\) است. پس حکم ثابت شد.
\end{info}
\subsection{\texorpdfstring{۴. شبکه ارتباطی با سایر قضایا
\lr{(Analytic Map)}}{۴. شبکه ارتباطی با سایر قضایا }}\label{ux634ux628ux6a9ux647-ux627ux631ux62aux628ux627ux637ux6cc-ux628ux627-ux633ux627ux6ccux631-ux642ux636ux627ux6ccux627-analytic-map}
\subsubsection{\texorpdfstring{۱. ارتباط با
\autoref{قضیه-۶---قواعد-استنتاج} (قیاس
استثنایی)}{۱. ارتباط با  (قیاس استثنایی)}}\label{ux627ux631ux62aux628ux627ux637-ux628ux627-ux642ux636ux6ccux647-ux6f6---ux642ux648ux627ux639ux62f-ux627ux633ux62aux646ux62aux627ux62c-ux642ux6ccux627ux633-ux627ux633ux62aux62bux646ux627ux6ccux6cc}
\begin{itemize}
\tightlist
\item
  \textbf{موتور محرک:} در گام دوم استقرا، ما ثابت می‌کنیم
  \(P(k) \Rightarrow P(k+1)\). اما این به تنهایی کافی نیست. وقتی پایه
  استقرا (\(P(1)\)) را ثابت کردیم، عملاً از \textbf{قیاس استثنایی}
  استفاده می‌کنیم:
  \begin{itemize}
  \tightlist
  \item
    مقدم ۱: \(P(1)\) (ثابت شده در پایه)
  \item
    مقدم ۲: \(P(1) \Rightarrow P(2)\) (ثابت شده در گام)
  \item
    نتیجه: \(P(2)\) (و این زنجیره ادامه می‌یابد\ldots).
  \end{itemize}
\end{itemize}
\subsubsection{۲. ارتباط با تعاریف
بازگشتی}\label{ux627ux631ux62aux628ux627ux637-ux628ux627-ux62aux639ux627ux631ux6ccux641-ux628ux627ux632ux6afux634ux62aux6cc}
\begin{itemize}
\tightlist
\item
  \textbf{تعریف بازگشتی:} استقرا ارتباط نزدیکی با تعاریفی دارد که خودشان
  را صدا می‌زنند (مثل تعریف توان \(x^{n+1} = x^n \cdot x\) یا فاکتوریل).
  این تعاریف، مواد اولیه برای اثبات‌های استقرایی هستند.
\end{itemize}
