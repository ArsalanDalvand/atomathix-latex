% ---------------------------------------------------------------------
% Copyright (c) 2026 Arsalan Dalvand & Reyhaneh Darvishi.
% Licensed under CC BY-NC-SA 4.0.
% See LICENSE file for details.
% ---------------------------------------------------------------------

\section{تمرین ۱۷، ۱۸ و ۱۹: پخش‌پذیری حاصلضرب روی
تفاضل}\label{تمرین-۱۷-و-۱۸-و-۱۹---جبر-تفاضل-در-حاصلضرب}
\begin{tldr}{خلاصه}
این تمرین‌ها فرمول‌های تجزیه تفاضل حاصلضرب‌ها را ثابت می‌کنند. تمرین ۱۹ حالت
کلی تمرین ۱۷ و ۱۸ است.
\end{tldr}
\subsection{۱. تمرین ۱۹ (حالت
کلی)}\label{ux62aux645ux631ux6ccux646-ux6f1ux6f9-ux62dux627ux644ux62a-ux6a9ux644ux6cc}
\textbf{سوال:} ثابت کنید:
\[(A \times B) - (C \times D) = [(A - C) \times B] \cup [A \times (B - D)]\]
\textbf{اثبات:} \[(x,y) \in (A \times B) - (C \times D)\]
\[\equiv (x \in A \wedge y \in B) \wedge \sim(x \in C \wedge y \in D)\]
(نقیض ``و'' می‌شود ``یا'' - دمورگان):
\[\equiv (x \in A \wedge y \in B) \wedge (x \notin C \lor y \notin D)\]
(پخش کردن پرانتز اول روی پرانتز دوم):
\[\equiv [(x \in A \wedge y \in B) \wedge x \notin C] \lor [(x \in A \wedge y \in B) \wedge y \notin D]\]
(مرتب کردن جملات):
\[\equiv [x \in (A-C) \wedge y \in B] \lor [x \in A \wedge y \in (B-D)]\]
\[\equiv [(x,y) \in (A-C) \times B] \lor [(x,y) \in A \times (B-D)]\] که
همان اجتماع دو طرف است .
\subsection{۲. تمرین ۱۷ و
۱۸}\label{ux62aux645ux631ux6ccux646-ux6f1ux6f7-ux648-ux6f1ux6f8}
این دو تمرین حالت‌های خاص تمرین ۱۹ هستند:
\begin{itemize}
\tightlist
\item
  \textbf{تمرین ۱۷:} با قرار دادن \(B=B\) و \(D=C\) و \(C=C\) (کمی
  نامگذاری متفاوت است) فرمول برای \((A \times B) - (C \times C)\) اثبات
  می‌شود.
\item
  \textbf{تمرین ۱۸:} اثبات برای \((A \times A) - (B \times C)\) که دقیقاً
  با منطق بالا حل می‌شود.
\end{itemize}
