% ---------------------------------------------------------------------
% Copyright (c) 2026 Arsalan Dalvand & Reyhaneh Darvishi.
% Licensed under CC BY-NC-SA 4.0.
% See LICENSE file for details.
% ---------------------------------------------------------------------

\section{قضیه ۲: هم‌ارزی‌های منطقی
پایه}\label{قضیه-۲---هم‌ارزی‌های-منطقی-پایه}
\begin{tldr}{خلاصه سریع}
این قضیه مجموعه ابزارهای جبری برای تغییر شکل گزاره‌ها بدون تغییر ارزش
راستی آن‌هاست. مهم‌ترین بخش آن «قانون عکس نقیض» است که پایه بسیاری از
اثبات‌های ریاضی (برهان غیرمستقیم) را تشکیل می‌دهد.
\end{tldr}
\subsection{۱. متن ریاضی
قضیه}\label{ux645ux62aux646-ux631ux6ccux627ux636ux6cc-ux642ux636ux6ccux647}
فرض کنید \(p\) و \(q\) دو گزاره دلبخواه باشند.قوانین زیر همواره
برقرارند:
\begin{theorembox}{قضیه ۲}
\textbf{الف) قانون نفی مضاعف \lr{(Double Negation):}}
\[\sim(\sim p) \equiv p\] \textbf{ب) قانون جابجایی
\lr{(Commutative Laws):}} \[p \vee q \equiv q \vee p\]
\[p \wedge q \equiv q \wedge p\] \textbf{ج) قانون خودتوانی
\lr{(Idempotent Laws):}} \[p \vee p \equiv p\] \[p \wedge p \equiv p\]
\textbf{د) قانون عکس نقیض \lr{(Contrapositive Law):}}
\[(p \rightarrow q) \equiv (\sim q \rightarrow \sim p)\]
\end{theorembox}
\subsection{۲. اثبات و تحلیل (با جدول
ارزش)}\label{ux627ux62bux628ux627ux62a-ux648-ux62aux62dux644ux6ccux644-ux628ux627-ux62cux62fux648ux644-ux627ux631ux632ux634}
برهان قسمت‌های الف، ب و ج بدیهی است. اما قسمت (د) یعنی قانون عکس نقیض،
نیاز به اثبات دقیق با جدول ارزش دارد. ما ارزش گزاره دو شرطی
\((p \to q) \leftrightarrow (\sim q \to \sim p)\) را بررسی می‌کنیم. اگر
این گزاره در تمام حالات «راست» باشد، دو طرف با هم هم‌ارز هستند.
{\def\LTcaptype{none}
\begin{longtable}[]{@{}
  >{\centering\arraybackslash}p{(\linewidth - 12\tabcolsep) * \real{0.1429}}
  >{\centering\arraybackslash}p{(\linewidth - 12\tabcolsep) * \real{0.1429}}
  >{\centering\arraybackslash}p{(\linewidth - 12\tabcolsep) * \real{0.1429}}
  >{\centering\arraybackslash}p{(\linewidth - 12\tabcolsep) * \real{0.1429}}
  >{\centering\arraybackslash}p{(\linewidth - 12\tabcolsep) * \real{0.1429}}
  >{\centering\arraybackslash}p{(\linewidth - 12\tabcolsep) * \real{0.1429}}
  >{\centering\arraybackslash}p{(\linewidth - 12\tabcolsep) * \real{0.1429}}@{}}
\toprule\noalign{}
\begin{minipage}[b]{\linewidth}\centering
\(p\)
\end{minipage} & \begin{minipage}[b]{\linewidth}\centering
\(q\)
\end{minipage} & \begin{minipage}[b]{\linewidth}\centering
\((p \to q)\)
\end{minipage} & \begin{minipage}[b]{\linewidth}\centering
\(\leftrightarrow\)
\end{minipage} & \begin{minipage}[b]{\linewidth}\centering
\((\sim q \to \sim p)\)
\end{minipage} & \begin{minipage}[b]{\linewidth}\centering
\(\sim q\)
\end{minipage} & \begin{minipage}[b]{\linewidth}\centering
\(\sim p\)
\end{minipage} \\
\midrule\noalign{}
\endhead
\bottomrule\noalign{}
\endlastfoot
\lr{T} & \lr{T} & \lr{T} & \textbf{\lr{T}} & \lr{T} & \lr{F} & \lr{F} \\
\lr{T} & \lr{F} & \lr{F} & \textbf{\lr{T}} & \lr{F} & \lr{T} & \lr{F} \\
\lr{F} & \lr{T} & \lr{T} & \textbf{\lr{T}} & \lr{T} & \lr{F} & \lr{T} \\
\lr{F} & \lr{F} & \lr{T} & \textbf{\lr{T}} & \lr{T} & \lr{T} & \lr{T} \\
\emph{جدول ۹ - اثبات قانون عکس نقیض} & & & & & & \\
\end{longtable}
}
\begin{info}{تحلیل اثبات}
۱. ستون سوم نشان‌دهنده ارزش گزاره شرطی اصلی (\(p \to q\)) است. ۲. ستون
پنجم نشان‌دهنده ارزش عکس نقیض (\(\sim q \to \sim p\)) است که با توجه به
ستون‌های \(\sim q\) و \(\sim p\) محاسبه شده است. ۳. با مقایسه ستون ۳ و ۵،
می‌بینیم که در هر ۴ حالت منطقی، ارزش‌ها دقیقاً یکسان هستند. ۴. نتیجه: ستون
چهارم (\(\leftrightarrow\)) تماماً \textbf{\lr{T }(راستگو)} است، که اثبات
می‌کند این دو عبارت منطقاً هم‌ارز هستند
\end{info}
\subsection{\texorpdfstring{۳. شبکه ارتباطی با سایر قضایا
\lr{(Analytic Map)}}{۳. شبکه ارتباطی با سایر قضایا }}\label{ux634ux628ux6a9ux647-ux627ux631ux62aux628ux627ux637ux6cc-ux628ux627-ux633ux627ux6ccux631-ux642ux636ux627ux6ccux627-analytic-map}
این قضیه نقش حیاتی در دستکاری ساختاری گزاره‌ها در سایر قضایا ایفا می‌کند:
\subsubsection{\texorpdfstring{۱. ارتباط با
\autoref{قضیه-۱---قوانین-جمع-و-اختصار} (قوانین جمع و
اختصار)}{۱. ارتباط با  (قوانین جمع و اختصار)}}\label{ux627ux631ux62aux628ux627ux637-ux628ux627-ux642ux636ux6ccux647-ux6f1---ux642ux648ux627ux646ux6ccux646-ux62cux645ux639-ux648-ux627ux62eux62aux635ux627ux631-ux642ux648ux627ux646ux6ccux646-ux62cux645ux639-ux648-ux627ux62eux62aux635ux627ux631}
\begin{itemize}
\tightlist
\item
  \textbf{تعمیم قانون اختصار:} در قضیه ۱ داشتیم
  \((p \wedge q) \Rightarrow p\). با استفاده از \textbf{قانون جابجایی}
  این قضیه (\(p \wedge q \equiv q \wedge p\))، می‌توانیم ثابت کنیم که
  اختصار برای مؤلفه دوم هم معتبر است: \((q \wedge p) \Rightarrow q\).
\end{itemize}
\subsubsection{\texorpdfstring{۲. ارتباط با
\autoref{قضیه-۳---قوانین-دمورگان}
(دمورگان)}{۲. ارتباط با  (دمورگان)}}\label{ux627ux631ux62aux628ux627ux637-ux628ux627-ux642ux636ux64aux647-ux62fux645ux648ux631ux6afux627ux646}
\begin{itemize}
\tightlist
\item
  \textbf{ساختار نفی:} قانون عکس نقیض نوعی «جابجایی همراه با نفی» در
  گزاره‌های شرطی است. این مفهوم در قضیه ۳ توسعه می‌یابد، جایی که نفی وارد
  پرانتز شده و عملگرها را تغییر می‌دهد (نفی ترکیب عطفی/فصلی).
\end{itemize}
\subsubsection{\texorpdfstring{۳. ارتباط با
\autoref{قضیه-۵---قیاس-ها}(قیاس‌های
ذو‌الوجهین)}{۳. ارتباط با (قیاس‌های ذو‌الوجهین)}}\label{ux627ux631ux62aux628ux627ux637-ux628ux627-ux642ux636ux6ccux647-ux6f5---ux642ux6ccux627ux633-ux647ux627ux642ux6ccux627ux633ux647ux627ux6cc-ux630ux648ux627ux644ux648ux62cux647ux6ccux646}
\begin{itemize}
\tightlist
\item
  \textbf{قیاس منفی:} قضیه ۵ (ب) (قیاس ذوالوجهین منفی) مستقیماً بر اساس
  \textbf{قانون عکس نقیض} بنا شده است. در آنجا از نفیِ تالی‌ها
  (\(\sim q \vee \sim s\)) به نفیِ مقدم‌ها (\(\sim p \vee \sim r\))
  می‌رسیم، دقیقاً همان منطقی که \(p \to q\) را به \(\sim q \to \sim p\)
  تبدیل می‌کند.
\end{itemize}
\subsubsection{\texorpdfstring{۴. ارتباط با
\autoref{قضیه-۶---قواعد-استنتاج} (قیاس دفع -
\lr{Modus Tollens)}}{۴. ارتباط با  (قیاس دفع - }}\label{ux627ux631ux62aux628ux627ux637-ux628ux627-ux642ux636ux6ccux647-ux6f6---ux642ux648ux627ux639ux62f-ux627ux633ux62aux646ux62aux627ux62c-ux642ux6ccux627ux633-ux62fux641ux639---modus-tollens}
\begin{itemize}
\tightlist
\item
  \textbf{پایه نظری:} قاعده قیاس دفع
  (\([(p \to q) \wedge \sim q] \Rightarrow \sim p\)) در واقع کاربرد
  مستقیم قانون عکس نقیض در یک استدلال است. ما \(p \to q\) را با هم‌ارز آن
  \(\sim q \to \sim p\) جایگزین می‌کنیم و سپس از قاعده قیاس استثنایی
  \lr{(Modus Ponens) }استفاده می‌کنیم.
\end{itemize}
