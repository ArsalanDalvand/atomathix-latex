% ---------------------------------------------------------------------
% Copyright (c) 2026 Arsalan Dalvand & Reyhaneh Darvishi.
% Licensed under CC BY-NC-SA 4.0.
% See LICENSE file for details.
% ---------------------------------------------------------------------

\section{قضیه ۲: پخش‌پذیری حاصلضرب دکارتی بر
تفاضل}\label{قضیه-۲---پخش‌پذیری-حاصلضرب-دکارتی-بر-تفاضل}
\begin{tldr}{خلاصه سریع}
این قضیه نشان می‌دهد که عملگر حاصلضرب دکارتی (\(\times\)) نسبت به عملگر
تفاضل (\(-\)) نیز خاصیت پخش‌پذیری دارد. این ویژگی، رفتار خطی و توزیع‌پذیر
ضرب دکارتی را در تمامی عملیات اصلی جبر مجموعه‌ها (اشتراک، اجتماع و تفاضل)
تکمیل می‌کند.
\end{tldr}
\subsection{۱. متن ریاضی
قضیه}\label{ux645ux62aux646-ux631ux6ccux627ux636ux6cc-ux642ux636ux6ccux647}
فرض کنید \(A\)، \(B\) و \(C\) سه مجموعه دلبخواه باشند. همواره رابطه زیر
برقرار است:
\begin{theorembox}{قضیه ۲}
\[A \times (B - C) = (A \times B) - (A \times C)\]
\end{theorembox}
\subsection{\texorpdfstring{۲. اثبات صوری
\lr{(Formal Proof)}}{۲. اثبات صوری }}\label{ux627ux62bux628ux627ux62a-ux635ux648ux631ux6cc-formal-proof}
برای اثبات این تساوی، از روش تحلیل گزاره‌نمای عضویت و زنجیره هم‌ارزی‌های
منطقی استفاده می‌کنیم. نکته کلیدی در این اثبات، استفاده از قوانین منطق
گزاره‌ها برای مدیریت نقیض در مؤلفه دوم است.
\begin{info}{برهان}
باید نشان دهیم گزاره \((a, x) \in A \times (B - C)\) منطقاً هم‌ارز با
\((a, x) \in (A \times B) - (A \times C)\) است.
۱. \textbf{بسط تعریف سمت چپ:} \[(a, x) \in A \times (B - C)\] طبق تعریف
حاصلضرب دکارتی: \[\iff (a \in A) \wedge (x \in B - C)\] طبق تعریف تفاضل
مجموعه‌ها (\(x \in S - T \iff x \in S \wedge x \notin T\)):
\[\iff (a \in A) \wedge [(x \in B) \wedge (x \notin C)]\]
۲. \textbf{به‌کارگیری قانون خودتوانی \lr{(Idempotent Law):}} برای اینکه
بتوانیم گزاره \(a \in A\) را با هر دو بخش دیگر ترکیب کنیم، از هم‌ارزی
منطقی \(p \equiv p \wedge p\) استفاده می‌کنیم:
\[\iff [(a \in A) \wedge (a \in A)] \wedge (x \in B) \wedge (x \notin C)\]
۳. \textbf{استفاده از قوانین شرکت‌پذیری و جابجایی \lr{(Associativity }\&
\lr{Commutativity):}} گزاره‌ها را برای ساختن فرم حاصلضرب دکارتی بازآرایی
می‌کنیم:
\[\iff [(a \in A) \wedge (x \in B)] \wedge [(a \in A) \wedge (x \notin C)]\]
۴. \textbf{تحلیل منطقی بخش دوم (نکته ظریف):} بخش اول
\([(a \in A) \wedge (x \in B)]\) دقیقاً معادل \((a, x) \in A \times B\)
است. برای بخش دوم، باید نشان دهیم در حضور شرط اول (\(a \in A\))، گزاره
\(x \notin C\) هم‌ارز با \((a, x) \notin A \times C\) است.
\begin{itemize}
\tightlist
\item
  می‌دانیم
  \((a, x) \notin A \times C \iff \sim(a \in A \wedge x \in C) \iff a \notin A \vee x \notin C\).
\item
  چون در این زنجیره استدلال، \(a \in A\) مفروض است، پس گزاره
  \(a \notin A\) کاذب (\(F\)) می‌باشد.
\item
  بنابراین \(F \vee (x \notin C)\) منطقاً هم‌ارز با \(x \notin C\) است.
\end{itemize}
پس می‌توانیم بنویسیم:
\[\iff [(a, x) \in A \times B] \wedge [(a, x) \notin A \times C]\]
۵. \textbf{نتیجه‌گیری نهایی:} طبق تعریف تفاضل مجموعه‌ها:
\[\iff (a, x) \in (A \times B) - (A \times C)\]
\end{info}
\subsection{\texorpdfstring{۳. شبکه ارتباطی با سایر قضایا
\lr{(Analytic Map)}}{۳. شبکه ارتباطی با سایر قضایا }}\label{ux634ux628ux6a9ux647-ux627ux631ux62aux628ux627ux637ux6cc-ux628ux627-ux633ux627ux6ccux631-ux642ux636ux627ux6ccux627-analytic-map}
این قضیه پازل رفتار جبری حاصلضرب دکارتی را تکمیل می‌کند:
\subsubsection{\texorpdfstring{۱. ارتباط با
\autoref{قضیه-۱---پخش‌پذیری-حاصلضرب-دکارتی}}{۱. ارتباط با }}\label{ux627ux631ux62aux628ux627ux637-ux628ux627-ux642ux636ux6ccux647-ux6f1---ux67eux62eux634ux67eux630ux6ccux631ux6cc-ux62dux627ux635ux644ux636ux631ux628-ux62fux6a9ux627ux631ux62aux6cc}
\begin{itemize}
\tightlist
\item
  \textbf{تکمیل توزیع‌پذیری:} در قضیه ۱، پخش‌پذیری \(\times\) بر \(\cap\)
  و \(\cup\) اثبات شد. قضیه ۲ نشان می‌دهد که این رفتار برای تفاضل (\(-\))
  نیز صادق است. این یعنی حاصلضرب دکارتی با تمام عملیات بولی استاندارد
  سازگار است.
\end{itemize}
\subsubsection{\texorpdfstring{۲. ارتباط با
\autoref{قضیه-۵---متمم-و-زیرمجموعه}}{۲. ارتباط با }}\label{ux627ux631ux62aux628ux627ux637-ux628ux627-ux642ux636ux6ccux647-ux6f5-ux641ux635ux644-ux6f2}
\begin{itemize}
\tightlist
\item
  \textbf{ریشه تعریف تفاضل:} اثبات این قضیه (گام ۱) کاملاً وابسته به
  تعریف دقیق تفاضل است که در فصل ۲ بررسی شد (\(B - C = B \cap C'\)).
  بدون درک منطقی نقیض عضویت (\(x \notin C\))، گام ۴ اثبات قابل درک
  نخواهد بود.
\end{itemize}
\subsubsection{\texorpdfstring{۳. ارتباط با
\autoref{قضیه-۴---جبر-مجموعه-ها}}{۳. ارتباط با }}\label{ux627ux631ux62aux628ux627ux637-ux628ux627-ux642ux636ux6ccux647-ux6f4---ux62cux628ux631-ux645ux62cux645ux648ux639ux647-ux647ux627}
\begin{itemize}
\tightlist
\item
  \textbf{مبانی استنتاج:} استفاده از قانون خودتوانی (\(p \wedge p\)) و
  جابجایی در گام‌های ۲ و ۳، کاربرد مستقیم قوانین منطق گزاره‌ها در نظریه
  مجموعه‌هاست. این قضیه نمونه‌ای عالی از تبدیل یک مسئله مجموعه‌ای به مسئله
  منطقی و حل آن است.
\end{itemize}
