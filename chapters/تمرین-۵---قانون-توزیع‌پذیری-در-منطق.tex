% ---------------------------------------------------------------------
% Copyright (c) 2026 Arsalan Dalvand & Reyhaneh Darvishi.
% Licensed under CC BY-NC-SA 4.0.
% See LICENSE file for details.
% ---------------------------------------------------------------------

\section{تمرین ۵: قانون توزیع‌پذیری (یا) روی
(و)}\label{تمرین-۵---قانون-توزیع‌پذیری-در-منطق}
\begin{tldr}{خلاصه سریع}
این قانون شبیه ضرب اعداد در پرانتز است: \(a \times (b + c) = ab + ac\).
در منطق هم عملگر \(\lor\) (یا) روی \(\land\) (و) پخش می‌شود.
\[p \lor (q \land r) \equiv (p \lor q) \land (p \lor r)\]
\end{tldr}
\subsection{۱. درک
شهودی}\label{ux62fux631ux6a9-ux634ux647ux648ux62fux6cc}
فرض کن برای استخدام شدن (\(p\)) یا باید «مدرک» (\(q\)) داشته باشی و
«سابقه» (\(r\)). جمله: «یا پارتی دارم (\(p\))، یا (مدرک دارم و سابقه
دارم)». این دقیقاً معادل است با: «(یا پارتی دارم یا مدرک) \textbf{و} (یا
پارتی دارم یا سابقه)». اگر هر کدام از این دو شرطِ داخل پرانتزِ دوم برقرار
نباشد، کل شرط باطل است.
\subsection{۲. اثبات (با جدول
ارزش)}\label{ux627ux62bux628ux627ux62a-ux628ux627-ux62cux62fux648ux644-ux627ux631ux632ux634}
برای اثبات هم‌ارزی منطقی، کافیست نشان دهیم جدول ارزش دو طرف یکسان است.
{\def\LTcaptype{none}
\begin{longtable}[]{@{}
  >{\raggedright\arraybackslash}p{(\linewidth - 14\tabcolsep) * \real{0.0208}}
  >{\raggedright\arraybackslash}p{(\linewidth - 14\tabcolsep) * \real{0.0208}}
  >{\raggedright\arraybackslash}p{(\linewidth - 14\tabcolsep) * \real{0.0208}}
  >{\raggedright\arraybackslash}p{(\linewidth - 14\tabcolsep) * \real{0.0833}}
  >{\raggedright\arraybackslash}p{(\linewidth - 14\tabcolsep) * \real{0.2917}}
  >{\raggedright\arraybackslash}p{(\linewidth - 14\tabcolsep) * \real{0.0833}}
  >{\raggedright\arraybackslash}p{(\linewidth - 14\tabcolsep) * \real{0.0833}}
  >{\raggedright\arraybackslash}p{(\linewidth - 14\tabcolsep) * \real{0.3958}}@{}}
\toprule\noalign{}
\begin{minipage}[b]{\linewidth}\raggedright
\lr{p}
\end{minipage} & \begin{minipage}[b]{\linewidth}\raggedright
\lr{q}
\end{minipage} & \begin{minipage}[b]{\linewidth}\raggedright
\lr{r}
\end{minipage} & \begin{minipage}[b]{\linewidth}\raggedright
\lr{q }\(\land\) \lr{r}
\end{minipage} & \begin{minipage}[b]{\linewidth}\raggedright
\textbf{\lr{LHS: p }\(\lor\) \lr{(q }\(\land\) \lr{r)}}
\end{minipage} & \begin{minipage}[b]{\linewidth}\raggedright
\lr{p }\(\lor\) \lr{q}
\end{minipage} & \begin{minipage}[b]{\linewidth}\raggedright
\lr{p }\(\lor\) \lr{r}
\end{minipage} & \begin{minipage}[b]{\linewidth}\raggedright
\textbf{\lr{RHS: (p }\(\lor\) \lr{q) }\(\land\) \lr{(p }\(\lor\)
\lr{r)}}
\end{minipage} \\
\midrule\noalign{}
\endhead
\bottomrule\noalign{}
\endlastfoot
\lr{T} & \lr{T} & \lr{T} & \lr{T} & \textbf{\lr{T}} & \lr{T} & \lr{T} &
\textbf{\lr{T}} \\
\lr{T} & \lr{T} & \lr{F} & \lr{F} & \textbf{\lr{T}} & \lr{T} & \lr{T} &
\textbf{\lr{T}} \\
\lr{T} & \lr{F} & \lr{T} & \lr{F} & \textbf{\lr{T}} & \lr{T} & \lr{T} &
\textbf{\lr{T}} \\
\lr{T} & \lr{F} & \lr{F} & \lr{F} & \textbf{\lr{T}} & \lr{T} & \lr{T} &
\textbf{\lr{T}} \\
\lr{F} & \lr{T} & \lr{T} & \lr{T} & \textbf{\lr{T}} & \lr{T} & \lr{T} &
\textbf{\lr{T}} \\
\lr{F} & \lr{T} & \lr{F} & \lr{F} & \textbf{\lr{F}} & \lr{T} & \lr{F} &
\textbf{\lr{F}} \\
\lr{F} & \lr{F} & \lr{T} & \lr{F} & \textbf{\lr{F}} & \lr{F} & \lr{T} &
\textbf{\lr{F}} \\
\lr{F} & \lr{F} & \lr{F} & \lr{F} & \textbf{\lr{F}} & \lr{F} & \lr{F} &
\textbf{\lr{F}} \\
\end{longtable}
}
چون ستون \textbf{\lr{LHS}} و \textbf{\lr{RHS}} دقیقاً یکسان هستند، حکم
ثابت شد. \(\blacksquare\)
