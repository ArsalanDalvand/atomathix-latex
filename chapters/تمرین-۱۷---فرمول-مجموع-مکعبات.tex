% ---------------------------------------------------------------------
% Copyright (c) 2026 Arsalan Dalvand & Reyhaneh Darvishi.
% Licensed under CC BY-NC-SA 4.0.
% See LICENSE file for details.
% ---------------------------------------------------------------------

\section{تمرین ۱۷: اثبات فرمول مجموع
مکعبات}\label{تمرین-۱۷---فرمول-مجموع-مکعبات}
\begin{tldr}{خلاصه سریع}
مجموع مکعب اعداد طبیعی از ۱ تا \(n\) برابر است با توان‌دومِ فرمولِ مجموعِ
اعداد (یعنی \(\frac{n(n+1)}{2}\) به توان ۲).
\[1^3 + 2^3 + \dots + n^3 = \left[ \frac{n(n+1)}{2} \right]^2 = \frac{n^2(n+1)^2}{4}\]
\end{tldr}
\subsection{۱. صورت
تمرین}\label{ux635ux648ux631ux62a-ux62aux645ux631ux6ccux646}
\begin{info}{سوال}
با استفاده از استقرای ریاضی ثابت کنید برای هر عدد طبیعی \(n\):
\[1^3 + 2^3 + 3^3 + \dots + n^3 = \frac{n^2(n+1)^2}{4}\]
\end{info}
\subsection{۲. اثبات با
استقراء}\label{ux627ux62bux628ux627ux62a-ux628ux627-ux627ux633ux62aux642ux631ux627ux621}
\begin{info}{گام‌های اثبات}
\textbf{گام ۱: پایه استقراء (\(n=1\))}
\begin{itemize}
\tightlist
\item
  سمت چپ: \(1^3 = 1\)
\item
  سمت راست: \(\frac{1^2(1+1)^2}{4} = \frac{1 \times 4}{4} = 1\)
\item
  چون \(1=1\)، حکم برای \(n=1\) برقرار است.
\end{itemize}
\textbf{گام ۲: فرض استقراء} فرض می‌کنیم حکم برای \(n\) برقرار باشد:
\[S_n = \frac{n^2(n+1)^2}{4}\]
\textbf{گام ۳: گام استقراء (اثبات برای \(n+1\))} باید نشان دهیم اگر به
مجموع قبلی، جمله بعدی یعنی \((n+1)^3\) را اضافه کنیم، فرمول باز هم کار
می‌کند.
\[S_{n+1} = S_n + (n+1)^3\]
جایگذاری فرض استقراء: \[= \frac{n^2(n+1)^2}{4} + (n+1)^3\]
فاکتورگیری از \((n+1)^2\) (مشترک در هر دو جمله):
\[= (n+1)^2 \left[ \frac{n^2}{4} + (n+1) \right]\]
مخرج مشترک گرفتن داخل کروشه:
\[= (n+1)^2 \left[ \frac{n^2 + 4(n+1)}{4} \right]\]
ساده‌سازی صورت کسر: \[= (n+1)^2 \left[ \frac{n^2 + 4n + 4}{4} \right]\]
عبارت \(n^2+4n+4\) اتحاد مربع کامل \((n+2)^2\) است:
\[= \frac{(n+1)^2 (n+2)^2}{4}\]
\textbf{نتیجه:} این دقیقا همان فرمول اصلی است که در آن به جای \(n\)،
مقدار \(n+1\) قرار گرفته است. پس حکم ثابت شد. \(\blacksquare\)
\end{info}
