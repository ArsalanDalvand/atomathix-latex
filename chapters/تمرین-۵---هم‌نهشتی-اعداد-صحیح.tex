% ---------------------------------------------------------------------
% Copyright (c) 2026 Arsalan Dalvand & Reyhaneh Darvishi.
% Licensed under CC BY-NC-SA 4.0.
% See LICENSE file for details.
% ---------------------------------------------------------------------

\section{تمرین ۵: رابطه هم‌نهشتی (پیمانه
۵)}\label{تمرین-۵---هم‌نهشتی-اعداد-صحیح}
\subsection{۱. صورت
سوال}\label{ux635ux648ux631ux62a-ux633ux648ux627ux644}
\begin{info}{فرض کنید \(X=\mathbb{Z}\) (مجموعه اعداد صحیح) باشد. رابطه \(\mathcal{E}\) را روی \(\mathbb{Z}\) به صورت زیر تعریف می‌کنیم:}
\[x \mathcal{E} y \iff x - y = 5k \quad (k \in \mathbb{Z})\]
\textbf{الف)} ثابت کنید \(\mathcal{E}\) یک رابطه هم‌ارزی است. \textbf{ب)}
افراز \(\mathbb{Z}/\mathcal{E}\) را پیدا کنید.
\lr{[cite\_start]}\textbf{ج)} تحقیق کنید که رابطه حاصل از این افراز،
همان رابطه اولیه است \lr{[cite: }115-117{]}.
\end{info}
\subsection{۲. استراتژی
حل}\label{ux627ux633ux62aux631ux627ux62aux698ux6cc-ux62dux644}
این همان مفهوم \textbf{«هم‌نهشتی به پیمانه ۵»} است
(\(x \equiv y \pmod 5\)).
\begin{itemize}
\tightlist
\item
  \textbf{الف:} باید بازتابی (\(x-x=5k\))، تقارنی (اگر \(x-y=5k\) آنگاه
  \(y-x=5(-k)\)) و تعدی (\(x-y=5k, y-z=5k' \to x-z=5(k+k')\)) را اثبات
  کنیم.
\item
  \textbf{ب:} کلاس‌های هم‌ارزی را پیدا کنیم. هر عدد صحیح با باقی‌مانده
  تقسیمش بر ۵ هم‌ارز است. باقی‌مانده‌های ممکن \(\{0, 1, 2, 3, 4\}\) هستند.
\end{itemize}
\subsection{۳. حل
تشریحی}\label{ux62dux644-ux62aux634ux631ux6ccux62dux6cc}
\begin{info}{پاسخ}
\textbf{قسمت (الف): اثبات هم‌ارزی} ۱. \textbf{بازتابی:} برای هر
\(x \in \mathbb{Z}\)، داریم \(x - x = 0 = 5(0)\). چون
\(0 \in \mathbb{Z}\)، پس \(x \mathcal{E} x\). ۲. \textbf{تقارنی:} اگر
\(x \mathcal{E} y\)، یعنی \(x - y = 5k\).
\[y - x = -(x - y) = -(5k) = 5(-k)\] چون \(-k\) صحیح است، پس
\(y \mathcal{E} x\). ۳. \textbf{تعدی:} اگر \(x \mathcal{E} y\) و
\(y \mathcal{E} z\): \[x - y = 5k\] \[y - z = 5k'\] با جمع طرفین:
\((x - y) + (y - z) = 5k + 5k' \Rightarrow x - z = 5(k+k')\). پس
\(x \mathcal{E} z\). بنابراین \(\mathcal{E}\) یک رابطه هم‌ارزی است.
\textbf{قسمت (ب): افراز (کلاس‌های هم‌نهشتی)} این رابطه کل اعداد صحیح را به
۵ دسته تقسیم می‌کند (بر اساس باقی‌مانده تقسیم بر ۵):
\begin{itemize}
\tightlist
\item
  \(Z_0 = \{\dots, -5, 0, 5, 10, \dots\} = [0]\)
\item
  \(Z_1 = \{\dots, -4, 1, 6, 11, \dots\} = [1]\)
\item
  \(Z_2 = \{\dots, -3, 2, 7, 12, \dots\} = [2]\)
\item
  \(Z_3 = \{\dots, -2, 3, 8, 13, \dots\} = [3]\)
\item
  \(Z_4 = \{\dots, -1, 4, 9, 14, \dots\} = [4]\)
\end{itemize}
افراز نهایی: \(\mathbb{Z}/\mathcal{E} = \{Z_0, Z_1, Z_2, Z_3, Z_4\}\).
\textbf{قسمت (ج):} طبق قضیه تناظر (و آنچه در تمرین ۳ دیدیم)، چون این
افراز دقیقاً از دسته‌بندی اعداد بر اساس تفاضل مضرب ۵ به دست آمده، رابطه
حاصل از آن نیز همان \(x-y=5k\) خواهد بود.
\end{info}
