% ---------------------------------------------------------------------
% Copyright (c) 2026 Arsalan Dalvand & Reyhaneh Darvishi.
% Licensed under CC BY-NC-SA 4.0.
% See LICENSE file for details.
% ---------------------------------------------------------------------

\section{\texorpdfstring{قضیه ۹: تعمیم قوانین پخش‌پذیری
\lr{(Generalized Distributive Laws)}}{قضیه ۹: تعمیم قوانین پخش‌پذیری }}\label{قضیه-۹---تعمیم-پخش-پذیری}
\begin{tldr}{خلاصه سریع}
این قضیه نشان می‌دهد که قوانین پخش‌پذیری (توزیع‌پذیری) محدود به تعداد
متناهی از مجموعه‌ها نیستند. عملگر اشتراک روی «اجتماعِ یک خانواده نامتناهی»
پخش می‌شود و عملگر اجتماع روی «اشتراکِ یک خانواده نامتناهی» توزیع می‌گردد.
این ویژگی اساس تعریف «جبر سیگما» در نظریه اندازه است.
\end{tldr}
\subsection{۱. متن ریاضی
قضیه}\label{ux645ux62aux646-ux631ux6ccux627ux636ux6cc-ux642ux636ux6ccux647}
فرض کنید \(A\) یک مجموعه و \(\{B_\gamma\}_{\gamma \in \Gamma}\) یک
خانواده دلبخواه از مجموعه‌ها باشد. در این صورت:
\begin{theorembox}{قضیه ۹}
\textbf{الف) پخش اشتراک بر اجتماع:}
\[A \cap (\bigcup_{\gamma \in \Gamma} B_\gamma) = \bigcup_{\gamma \in \Gamma} (A \cap B_\gamma)\]
\textbf{ب) پخش اجتماع بر اشتراک:}
\[A \cup (\bigcap_{\gamma \in \Gamma} B_\gamma) = \bigcap_{\gamma \in \Gamma} (A \cup B_\gamma)\]
\end{theorembox}
\subsection{\texorpdfstring{۲. اثبات صوری
\lr{(Formal Proof)}}{۲. اثبات صوری }}\label{ux627ux62bux628ux627ux62a-ux635ux648ux631ux6cc-formal-proof}
اثبات این قضیه بر پایه «قوانین پخش‌پذیری منطق گزاره‌ها» و تعامل آن با
سورها بنا شده است. ما اثبات قسمت (الف) را بررسی می‌کنیم.
\begin{info}{اثبات}
باید نشان دهیم گزاره‌نمای عضویت (\(x \in S\)) برای طرفین تساوی هم‌ارز است:
\[x \in A \cap (\bigcup_{\gamma \in \Gamma} B_\gamma)\] ۱. طبق تعریف
اشتراک:
\[\iff (x \in A) \wedge (x \in \bigcup_{\gamma \in \Gamma} B_\gamma)\]
۲. طبق \textbf{\autoref{پیشنیاز---مفاهیم-بنیادین-مجموعه‌ها}} (سور وجودی):
\[\iff (x \in A) \wedge (\exists \gamma \in \Gamma, x \in B_\gamma)\] ۳.
نکته منطقی کلیدی: از آنجا که گزاره \((x \in A)\) مستقل از اندیس
\(\gamma\) است، می‌توانیم آن را به داخل سور وجودی ببریم (قانون پخش‌پذیری
منطق مسور):
\[p \wedge (\exists \gamma, q_\gamma) \iff \exists \gamma, (p \wedge q_\gamma)\]
\[\iff \exists \gamma \in \Gamma, (x \in A \wedge x \in B_\gamma)\] ۴.
طبق تعریف اشتراک:
\[\iff \exists \gamma \in \Gamma, (x \in A \cap B_\gamma)\] ۵. طبق تعریف
اجتماع تعمیم‌یافته:
\[\iff x \in \bigcup_{\gamma \in \Gamma} (A \cap B_\gamma)\]
نتیجه: دو مجموعه برابرند. (اثبات قسمت ب مشابه است و از پخش‌پذیری \(\vee\)
روی \(\forall\) استفاده می‌کند).
\end{info}
\subsection{\texorpdfstring{۳. شبکه ارتباطی با سایر قضایا
\lr{(Analytic Map)}}{۳. شبکه ارتباطی با سایر قضایا }}\label{ux634ux628ux6a9ux647-ux627ux631ux62aux628ux627ux637ux6cc-ux628ux627-ux633ux627ux6ccux631-ux642ux636ux627ux6ccux627-analytic-map}
این قضیه توسعه‌یافته‌ی مفاهیم جبری فصل ۱ و ۲ در ابعاد نامتناهی است:
\subsubsection{\texorpdfstring{۱. ارتباط با
\autoref{قضیه-۴---جبر-مجموعه-ها} (حالت
متناهی)}{۱. ارتباط با  (حالت متناهی)}}\label{ux627ux631ux62aux628ux627ux637-ux628ux627-ux642ux636ux6ccux647-ux6f4---ux62cux628ux631-ux645ux62cux645ux648ux639ux647-ux647ux627-ux62dux627ux644ux62a-ux645ux62aux646ux627ux647ux6cc}
\begin{itemize}
\tightlist
\item
  \textbf{تعمیم:} قضیه ۴ (بخش ه) بیان می‌کرد که
  \(A \cap (B \cup C) = (A \cap B) \cup (A \cap C)\). قضیه ۹ دقیقا همان
  قانون است با این تفاوت که تعداد مجموعه‌های داخل پرانتز از ۲ تا به
  بی‌نهایت (به تعداد اعضای \(\Gamma\)) افزایش یافته است.
\end{itemize}
\subsubsection{\texorpdfstring{۲. ارتباط با
\autoref{قضیه-۴---قوانین-شرکت-پذیری-و-پخش-پذیری} (ریشه
منطقی)}{۲. ارتباط با  (ریشه منطقی)}}\label{ux627ux631ux62aux628ux627ux637-ux628ux627-ux642ux636ux6ccux647-ux6f4-ux641ux635ux644-ux6f1-ux631ux6ccux634ux647-ux645ux646ux637ux642ux6cc}
\begin{itemize}
\tightlist
\item
  \textbf{پایه استدلال:} اثبات قضیه ۹ تماماً متکی بر ساختار منطقی
  \(p \wedge (q \vee r) \equiv (p \wedge q) \vee (p \wedge r)\) است. در
  نظریه مجموعه‌ها، سور وجودی (\(\exists\)) نقش تعمیم‌یافته‌ی «یا»
  (\(\vee\)) را بازی می‌کند؛ بنابراین پخش شدن \(\wedge\) روی \(\exists\)
  در اثبات بالا، بازتاب مستقیم پخش شدن \(\wedge\) روی \(\vee\) در منطق
  است.
\end{itemize}
\subsubsection{\texorpdfstring{۳. ارتباط با
\autoref{قضیه-۸---تعمیم-دمورگان}}{۳. ارتباط با }}\label{ux627ux631ux62aux628ux627ux637-ux628ux627-ux642ux636ux6ccux647-ux6f8---ux62aux639ux645ux6ccux645-ux62fux645ux648ux631ux6afux627ux646}
\begin{itemize}
\tightlist
\item
  \textbf{مکمل‌سازی:} اگر از طرفین تساوی‌های قضیه ۹ متمم بگیریم و از
  \textbf{\autoref{قضیه-۸---تعمیم-دمورگان}} استفاده کنیم، به دوگانِ
  \lr{(Dual) }یکدیگر تبدیل می‌شوند. یعنی متمم‌گیری از فرمول (الف) و
  استفاده از دمورگان، ما را به فرمول (ب) می‌رساند (با جایگزینی \(A\) با
  \(A'\) و \(B_\gamma\) با \(B_\gamma'\)).
\end{itemize}
\subsubsection{۴. کاربرد در توپولوژی (فصول
پیشرفته)}\label{ux6a9ux627ux631ux628ux631ux62f-ux62fux631-ux62aux648ux67eux648ux644ux648ux698ux6cc-ux641ux635ux648ux644-ux67eux6ccux634ux631ux641ux62aux647}
\begin{itemize}
\tightlist
\item
  \textbf{پیوستگی:} در توپولوژی، این قضیه نقش حیاتی دارد. مثلاً تعریف
  تابع پیوسته (\(f^{-1}(\cup U_\alpha) = \cup f^{-1}(U_\alpha)\)) و خواص
  تصویر معکوس توابع، دقیقاً از همین ساختار پخش‌پذیری پیروی می‌کنند.
\end{itemize}
