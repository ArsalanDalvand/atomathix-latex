% ---------------------------------------------------------------------
% Copyright (c) 2026 Arsalan Dalvand & Reyhaneh Darvishi.
% Licensed under CC BY-NC-SA 4.0.
% See LICENSE file for details.
% ---------------------------------------------------------------------

\section{\texorpdfstring{قضیه ۸: لم چسباندن
\lr{(Pasting Lemma)}}{قضیه ۸: لم چسباندن }}\label{قضیه-۸---لم-چسباندن-(اجتماع-توابع)}
\begin{tldr}{خلاصه سریع}
این قضیه شرط لازم و کافی برای «اجتماع دو تابع» را بیان می‌کند. اگر دو
تابع \(f\) و \(g\) داشته باشیم، اجتماع آن‌ها (\(f \cup g\)) تنها در صورتی
یک تابع خواهد بود که در دامنه مشترکشان، خروجی‌های یکسانی تولید کنند
(توافق داشته باشند).
\end{tldr}
\subsection{۱. متن ریاضی
قضیه}\label{ux645ux62aux646-ux631ux6ccux627ux636ux6cc-ux642ux636ux6ccux647}
فرض کنید \(f\) و \(g\) دو تابع باشند. رابطه \(h = f \cup g\) یک تابع است
اگر و تنها اگر \(f\) و \(g\) روی اشتراک دامنه‌هایشان هم‌مقدار باشند
(سازگار باشند).
\begin{theorembox}{قضیه ۸}
\[f \cup g \text{ is a function} \iff \forall x \in \text{dom}(f) \cap \text{dom}(g), f(x) = g(x)\]
در این صورت، دامنه تابع حاصل برابر است با:
\[\text{dom}(f \cup g) = \text{dom}(f) \cup \text{dom}(g)\]
\end{theorembox}
\subsection{\texorpdfstring{۲. اثبات صوری
\lr{(Formal Proof)}}{۲. اثبات صوری }}\label{ux627ux62bux628ux627ux62a-ux635ux648ux631ux6cc-formal-proof}
برای اثبات تابع بودن یک رابطه، باید نشان دهیم که به ازای هر ورودی، خروجی
\textbf{یکتا} است (خوش‌تعریفی).
\begin{info}{برهان}
\textbf{فرض:} شرط سازگاری برقرار است:
\(\forall x \in \text{dom}(f) \cap \text{dom}(g) \implies f(x) = g(x)\).
\textbf{حکم:} \(h = f \cup g\) یک تابع است.
۱. فرض کنید \(x\) عضوی از دامنه \(h\) باشد و دو زوج مرتب \((x, y_1)\) و
\((x, y_2)\) در \(h\) وجود داشته باشند. ۲. طبق تعریف اجتماع مجموعه‌ها:
\[[(x, y_1) \in f \lor (x, y_1) \in g] \wedge [(x, y_2) \in f \lor (x, y_2) \in g]\]
۳. حال حالات ممکن برای \(x\) را بررسی می‌کنیم:
\begin{itemize}
\tightlist
\item
  \textbf{حالت اول (\(x \in \text{dom}(f) \setminus \text{dom}(g)\)):}
  در این صورت زوج‌ها نمی‌توانند در \(g\) باشند، پس الزاماً
  \((x, y_1) \in f\) و \((x, y_2) \in f\). چون \(f\) تابع است
  \(\implies y_1 = y_2\).
\item
  \textbf{حالت دوم (\(x \in \text{dom}(g) \setminus \text{dom}(f)\)):}
  مشابه حالت قبل، هر دو زوج در \(g\) هستند. چون \(g\) تابع است
  \(\implies y_1 = y_2\).
\item
  \textbf{حالت سوم (\(x \in \text{dom}(f) \cap \text{dom}(g)\)):} در
  اینجا ممکن است یک زوج از \(f\) و دیگری از \(g\) آمده باشد. مثلاً
  \((x, y_1) \in f\) و \((x, y_2) \in g\). این یعنی \(y_1 = f(x)\) و
  \(y_2 = g(x)\). اما طبق \textbf{فرض سازگاری قضیه}، می‌دانیم در اشتراک
  دامنه‌ها \(f(x) = g(x)\). بنابراین \(y_1 = y_2\).
\end{itemize}
\textbf{نتیجه:} در تمام حالات، خروجی یکتاست. پس \(f \cup g\) تابع است.
\end{info}
\subsection{\texorpdfstring{۳. شبکه ارتباطی با سایر قضایا
\lr{(Analytic Map)}}{۳. شبکه ارتباطی با سایر قضایا }}\label{ux634ux628ux6a9ux647-ux627ux631ux62aux628ux627ux637ux6cc-ux628ux627-ux633ux627ux6ccux631-ux642ux636ux627ux6ccux627-analytic-map}
این قضیه ابزاری قدرتمند برای ساخت توابع پیچیده از توابع ساده‌تر است:
\subsubsection{\texorpdfstring{۱. ارتباط با
\autoref{تمرین-۱۳---تحدید-تابع} (عملیات
معکوس)}{۱. ارتباط با  (عملیات معکوس)}}\label{ux627ux631ux62aux628ux627ux637-ux628ux627-ux62aux645ux631ux6ccux646-ux6f1ux6f3---ux62aux62dux62fux6ccux62f-ux62aux627ux628ux639-ux639ux645ux644ux6ccux627ux62a-ux645ux639ux6a9ux648ux633}
\begin{itemize}
\tightlist
\item
  \textbf{رابطه دوطرفه:} در تمرین ۱۳ دیدیم که اگر یک تابع بزرگ را به
  زیرمجموعه‌ای از دامنه‌اش محدود کنیم \lr{(Restriction)، }حاصل همیشه تابع
  است (\(f|_A\)). قضیه ۸ \lr{(Pasting) }عکس آن عمل را انجام می‌دهد: تلاش
  می‌کند توابع کوچک (تحدیدها) را به هم بچسباند تا تابع اصلی (توسیع) را
  بسازد. در واقع اگر \(h = f \cup g\) تابع باشد، آنگاه
  \(f = h|_{\text{dom}(f)}\) و \(g = h|_{\text{dom}(g)}\).
\end{itemize}
\subsubsection{\texorpdfstring{۲. کاربرد در
\autoref{قضیه-۷---شرط-تساوی-توابع}}{۲. کاربرد در }}\label{ux6a9ux627ux631ux628ux631ux62f-ux62fux631-ux642ux636ux6ccux647-ux6f7---ux634ux631ux637-ux62aux633ux627ux648ux6cc-ux62aux648ux627ux628ux639}
\begin{itemize}
\tightlist
\item
  \textbf{سازگاری:} شرط \(f(x)=g(x)\) در قضیه ۸، یادآور شرط تساوی توابع
  در قضیه ۷ است. با این تفاوت که در قضیه ۷ دامنه‌ها برابر بودند، اما در
  اینجا دامنه‌ها متفاوت‌اند و تساوی فقط در بخش مشترک (\(A \cap B\)) بررسی
  می‌شود.
\end{itemize}
\subsubsection{۳. پیش‌نیاز توپولوژی و
آنالیز}\label{ux67eux6ccux634ux646ux6ccux627ux632-ux62aux648ux67eux648ux644ux648ux698ux6cc-ux648-ux622ux646ux627ux644ux6ccux632}
\begin{itemize}
\tightlist
\item
  \textbf{توابع چندضابطه‌ای:} تعریف توابع چندضابطه‌ای
  \lr{(Piecewise functions) }که در حساب دیفرانسیل می‌بینید (مانند قدر
  مطلق)، دقیقاً کاربرد عملی همین قضیه است. ما دو تابع \(y=x\) و \(y=-x\)
  را در نقطه \(x=0\) (اشتراک دامنه‌ها) چک می‌کنیم. چون هر دو در صفر برابر
  صفرند، می‌توانیم آن‌ها را بچسبانیم.
\end{itemize}
