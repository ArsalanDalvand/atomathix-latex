% ---------------------------------------------------------------------
% Copyright (c) 2026 Arsalan Dalvand & Reyhaneh Darvishi.
% Licensed under CC BY-NC-SA 4.0.
% See LICENSE file for details.
% ---------------------------------------------------------------------

\section{قضیه ۳: ویژگی‌های بنیادی کلاس‌های
هم‌ارزی}\label{قضیه-۳---ویژگی‌های-بنیادی-کلاس-هم‌ارزی}
\begin{tldr}{خلاصه سریع}
این قضیه هندسهٔ فضایی را که یک «رابطه هم‌ارزی» روی یک مجموعه ایجاد می‌کند،
ترسیم می‌نماید. طبق این قضیه، کلاس‌های هم‌ارزی یا کاملاً بر هم منطبق‌اند و یا
کاملاً از هم جدا \lr{(Disjoint) }هستند. این ویژگی، زیربنای مفهوم «افراز»
و ساختارهای جبری مانند گروه‌های خارج‌قسمتی است.
\end{tldr}
\subsection{۱. تعاریف و مفاهیم
پیش‌نیاز}\label{ux62aux639ux627ux631ux6ccux641-ux648-ux645ux641ux627ux647ux6ccux645-ux67eux6ccux634ux646ux6ccux627ux632}
پیش از ورود به اثبات، لازم است تعاریف صوری «کلاس هم‌ارزی» و «مجموعه
خارج‌قسمتی» را که در این قضیه محوریت دارند، مرور کنیم.
\subsubsection{\texorpdfstring{الف) کلاس هم‌ارزی
\lr{(Equivalence Class)}}{الف) کلاس هم‌ارزی }}\label{ux627ux644ux641-ux6a9ux644ux627ux633-ux647ux645ux627ux631ux632ux6cc-equivalence-class}
فرض کنید \(\xi\) یک رابطه هم‌ارزی روی مجموعه \(X\) باشد. برای هر عنصر
\(x \in X\)، کلاس هم‌ارزی \(x\) (که با \(x/\xi\) یا \([x]_\xi\) نمایش
داده می‌شود)، مجموعه‌ی تمام عناصری از \(X\) است که با \(x\) هم‌ارز هستند:
\[x/\xi = \{ y \in X \mid y \xi x \}\] در اینجا \(x\) را
\textbf{نماینده} \lr{(Representative) }کلاس می‌نامیم.
\subsubsection{\texorpdfstring{ب) مجموعه خارج‌قسمتی
\lr{(Quotient Set)}}{ب) مجموعه خارج‌قسمتی }}\label{ux628-ux645ux62cux645ux648ux639ux647-ux62eux627ux631ux62cux642ux633ux645ux62aux6cc-quotient-set}
مجموعه تمام کلاس‌های هم‌ارزی متمایز حاصل از رابطه \(\xi\) را مجموعه
خارج‌قسمتی \(X\) نسبت به \(\xi\) می‌نامند و با \(X/\xi\) نمایش می‌دهند:
\[X/\xi = \{ x/\xi \mid x \in X \}\]
\begin{center}\rule{0.5\linewidth}{0.5pt}\end{center}
\subsection{۲. متن ریاضی
قضیه}\label{ux645ux62aux646-ux631ux6ccux627ux636ux6cc-ux642ux636ux6ccux647}
فرض کنید \(\xi\) یک رابطه هم‌ارزی روی مجموعه ناتهی \(X\) باشد. احکام زیر
برقرارند:
\begin{theorembox}{قضیه ۳}
\textbf{الف) اصل ناتهی بودن:}
\[\forall x \in X, \quad x/\xi \neq \emptyset\] \textbf{ب) شرط تلاقی
(اشتراک):} \[x/\xi \cap y/\xi \neq \emptyset \iff x \xi y\] \textbf{ج)
اصل انطباق (تساوی کلاس‌ها):} \[x/\xi = y/\xi \iff x \xi y\]
\end{theorembox}
\begin{center}\rule{0.5\linewidth}{0.5pt}\end{center}
\subsection{\texorpdfstring{۳. اثبات صوری
\lr{(Formal Proof)}}{۳. اثبات صوری }}\label{ux627ux62bux628ux627ux62a-ux635ux648ux631ux6cc-formal-proof}
اثبات این قضیه کاربرد مستقیم ویژگی‌های سه‌گانه رابطه هم‌ارزی (بازتابی،
تقارنی، تعدی) در نظریه مجموعه‌هاست.
\subsubsection{اثبات قسمت
(الف)}\label{ux627ux62bux628ux627ux62a-ux642ux633ux645ux62a-ux627ux644ux641}
\begin{info}{برهان}
۱. طبق تعریف، \(\xi\) یک رابطه هم‌ارزی است، بنابراین دارای ویژگی
\textbf{بازتابی \lr{(Reflexive)}} است: \[\forall x \in X, (x \xi x)\] ۲.
طبق تعریف کلاس هم‌ارزی (\(y \in x/\xi \iff y \xi x\))، چون \(x \xi x\)
برقرار است، نتیجه می‌شود: \[x \in x/\xi\] ۳. چون حداقل خود عنصر \(x\) در
کلاسش وجود دارد، پس این مجموعه هرگز تهی نیست.
\end{info}
\subsubsection{اثبات قسمت
(ب)}\label{ux627ux62bux628ux627ux62a-ux642ux633ux645ux62a-ux628}
اثبات دوطرفه (\(\iff\)) انجام می‌شود.
\begin{info}{برهان}
\textbf{جهت رفت (\(\Rightarrow\)):} فرض کنیم اشتراک تهی نیست
(\(x/\xi \cap y/\xi \neq \emptyset\)). ۱. وجود دارد عنصری مانند \(z\) که
در هر دو کلاس باشد (\(z \in x/\xi \wedge z \in y/\xi\)). ۲. طبق تعریف
کلاس‌ها: \((z \xi x) \wedge (z \xi y)\). ۳. از ویژگی \textbf{تقارنی} برای
جمله اول استفاده می‌کنیم (\(z \xi x \implies x \xi z\)). 4. اکنون داریم:
\((x \xi z) \wedge (z \xi y)\). ۵. طبق ویژگی \textbf{تعدی}، نتیجه می‌شود:
\(x \xi y\).
\textbf{جهت برگشت (\(\Leftarrow\)):} فرض کنیم \(x \xi y\). ۱. طبق ویژگی
بازتابی، \(x \xi x\)، پس \(x \in x/\xi\). ۲. چون \(x \xi y\) (فرض) و
رابطه متقارن است، پس \(y \xi x\). طبق تعریف کلاس \(x\)، یعنی
\(y \in x/\xi\). (و همچنین \(x \in y/\xi\)). ۳. بنابراین عنصر \(x\) (و
حتی \(y\)) در هر دو کلاس حضور دارد. ۴. پس اشتراک ناتهی است
(\(x \in x/\xi \cap y/\xi\)).
\end{info}
\subsubsection{اثبات قسمت
(ج)}\label{ux627ux62bux628ux627ux62a-ux642ux633ux645ux62a-ux62c}
\begin{info}{برهان}
\textbf{جهت رفت (\(\Rightarrow\)):} فرض کنیم \(x/\xi = y/\xi\). ۱. طبق
قسمت (الف)، کلاس‌ها ناتهی هستند، پس اشتراک آن‌ها (\(x/\xi \cap y/\xi\))
نیز ناتهی است (چون با خود مجموعه‌ها برابر است). ۲. طبق قسمت (ب)، ناتهی
بودن اشتراک ایجاب می‌کند که \(x \xi y\).
\textbf{جهت برگشت (\(\Leftarrow\)):} فرض کنیم \(x \xi y\). باید ثابت
کنیم \(x/\xi = y/\xi\). (از روش شمول دوطرفه
\(A \subseteq B \wedge B \subseteq A\) استفاده می‌کنیم).
\begin{itemize}
\item
  \textbf{گام ۱ (\(x/\xi \subseteq y/\xi\)):} فرض کنیم \(z\) عضو دلخواهی
  از \(x/\xi\) باشد. پس \(z \xi x\). از طرفی طبق فرض \(x \xi y\). بنابر
  ویژگی \textbf{تعدی}: \((z \xi x \wedge x \xi y \implies z \xi y)\). پس
  \(z \in y/\xi\).
\item
  \textbf{گام ۲ (\(y/\xi \subseteq x/\xi\)):} فرض کنیم \(w\) عضو دلخواهی
  از \(y/\xi\) باشد. پس \(w \xi y\). چون \(x \xi y\)، طبق \textbf{تقارن}
  \(y \xi x\). بنابر ویژگی \textbf{تعدی}:
  \((w \xi y \wedge y \xi x \implies w \xi x)\). پس \(w \in x/\xi\).
\end{itemize}
\textbf{نتیجه:} دو مجموعه زیرمجموعه یکدیگرند، پس \(x/\xi = y/\xi\).
\end{info}
\begin{center}\rule{0.5\linewidth}{0.5pt}\end{center}
\subsection{\texorpdfstring{۴. شبکه ارتباطی با سایر قضایا
\lr{(Analytic Map)}}{۴. شبکه ارتباطی با سایر قضایا }}\label{ux634ux628ux6a9ux647-ux627ux631ux62aux628ux627ux637ux6cc-ux628ux627-ux633ux627ux6ccux631-ux642ux636ux627ux6ccux627-analytic-map}
این قضیه پل ارتباطی بین ``جبر رابطه‌ها'' و ``ساختار افراز'' است:
\subsubsection{\texorpdfstring{۱. ارتباط با
\autoref{پیشنیاز---مفاهیم-پایه-رابطه-و-تابع}}{۱. ارتباط با }}\label{ux627ux631ux62aux628ux627ux637-ux628ux627-ux67eux6ccux634ux646ux6ccux627ux632---ux645ux641ux627ux647ux6ccux645-ux67eux627ux6ccux647-ux631ux627ux628ux637ux647-ux648-ux62aux627ux628ux639}
\begin{itemize}
\tightlist
\item
  اثبات این قضیه تماماً بر پایه ویژگی‌های تعریف شده در پیش‌نیاز (بازتابی،
  تقارنی، تعدی) استوار است. این قضیه نشان می‌دهد که چگونه انتزاعی‌ترین
  ویژگی‌های یک رابطه، منجر به نتایج ملموس مجموعه‌ای (تساوی یا جدایی کامل)
  می‌شود.
\end{itemize}
\subsubsection{\texorpdfstring{۲. ارتباط با
\autoref{قضیه-۴---افراز-ناشی-از-رابطه-هم‌ارزی}}{۲. ارتباط با }}\label{ux627ux631ux62aux628ux627ux637-ux628ux627-ux642ux636ux6ccux647-ux6f4---ux627ux641ux631ux627ux632-ux646ux627ux634ux6cc-ux627ux632-ux631ux627ux628ux637ux647-ux647ux645ux627ux631ux632ux6cc}
\begin{itemize}
\tightlist
\item
  این قضیه (بخش‌های ب و ج) ``لم اصلی'' برای اثبات قضیه ۴ است. قضیه ۳
  تضمین می‌کند که کلاس‌ها ``دو به دو جدا'' \lr{(Mutually Disjoint) }هستند.
  بدون قضیه ۳، نمی‌توان ثابت کرد که \(X/\xi\) تشکیل یک افراز می‌دهد.
\end{itemize}
\subsubsection{\texorpdfstring{۳. ارتباط با
\autoref{مفهوم-پارادوکس-راسل} (مفهوم
خوش‌تعریفی)}{۳. ارتباط با  (مفهوم خوش‌تعریفی)}}\label{ux627ux631ux62aux628ux627ux637-ux628ux627-ux645ux641ux647ux648ux645-ux67eux627ux631ux627ux62fux648ux6a9ux633-ux631ux627ux633ux644-ux645ux641ux647ux648ux645-ux62eux648ux634ux62aux639ux631ux6ccux641ux6cc}
\begin{itemize}
\tightlist
\item
  در مباحث پیشرفته‌تر، وقتی روی کلاس‌های هم‌ارزی عملیاتی تعریف می‌کنیم (مثل
  جمع در \(\mathbb{Z}_n\))، بخش (ج) این قضیه
  (\(x/\xi = y/\xi \iff x \xi y\)) ابزار اصلی برای بررسی ``خوش‌تعریفی''
  \lr{(Well-definedness) }آن عملیات است؛ یعنی نتیجه عملیات نباید وابسته
  به نماینده کلاس (\(x\) یا \(y\)) باشد.
\end{itemize}
