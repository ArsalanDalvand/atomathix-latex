% ---------------------------------------------------------------------
% Copyright (c) 2026 Arsalan Dalvand & Reyhaneh Darvishi.
% Licensed under CC BY-NC-SA 4.0.
% See LICENSE file for details.
% ---------------------------------------------------------------------

\section{مفهوم تابع و توسعه هم‌دامنه (قضیه
۶)}\label{مفهوم-تابع-و-قضیه-۶---تعریف-و-دامنه}
\begin{tldr}{خلاصه سریع}
تابع، نوع خاصی از «رابطه» است که رفتار قطعی دارد: هر ورودی دقیقاً یک
خروجی تولید می‌کند. قضیه ۶ بیان می‌کند که بزرگ کردن مجموعه مقصد (هم‌دامنه)،
آسیبی به ساختار تابع نمی‌زند.
\end{tldr}
\subsection{\texorpdfstring{۱. تعریف صوری تابع
\lr{(Function Definition)}}{۱. تعریف صوری تابع }}\label{ux62aux639ux631ux6ccux641-ux635ux648ux631ux6cc-ux62aux627ux628ux639-function-definition}
در نظریه مجموعه‌ها، تابع یک مفهوم اولیه نیست، بلکه بر اساس مفهوم «رابطه»
و «زوج مرتب» تعریف می‌شود.
فرض کنید \(X\) و \(Y\) دو مجموعه باشند. یک تابع \(f\) از \(X\) به \(Y\)
(نماد: \(f: X \to Y\)) زیرمجموعه‌ای از حاصلضرب دکارتی \(X \times Y\) است
که دو شرط زیر را ارضا کند:
\begin{tldr}{تعریف تابع}
۱. \textbf{دامنه کامل \lr{(Existence):}} به ازای هر عضو در دامنه، حداقل
یک تصویر وجود داشته باشد.
\[\forall x \in X, \exists y \in Y : (x,y) \in f\] ۲. \textbf{یکتایی
\lr{(Uniqueness):}} هر عضو دامنه، حداکثر یک تصویر داشته باشد.
\[(x, y) \in f \wedge (x, z) \in f \implies y = z\]
\end{tldr}
اگر \((x,y) \in f\) باشد، معمولاً می‌نویسیم \(y = f(x)\).
\begin{center}\rule{0.5\linewidth}{0.5pt}\end{center}
\subsection{\texorpdfstring{۲. قضیه ۶: توسعه هم‌دامنه
\lr{(Codomain Extension)}}{۲. قضیه ۶: توسعه هم‌دامنه }}\label{ux642ux636ux6ccux647-ux6f6-ux62aux648ux633ux639ux647-ux647ux645ux62fux627ux645ux646ux647-codomain-extension}
این قضیه به ما اجازه می‌دهد که «ظرف مقصد» را بدون تغییر ضابطه تابع،
بزرگ‌تر کنیم.
\subsubsection{متن ریاضی
قضیه}\label{ux645ux62aux646-ux631ux6ccux627ux636ux6cc-ux642ux636ux6ccux647}
فرض کنید \(f: X \to Y\) یک تابع باشد و \(Y \subseteq W\).
\begin{theorembox}{قضیه ۶}
اگر \(Y \subseteq W\) باشد، آنگاه \(f\) یک تابع از \(X\) به \(W\) نیز
محسوب می‌شود (\(f: X \to W\)).
\end{theorembox}
\subsubsection{اثبات
صوری}\label{ux627ux62bux628ux627ux62a-ux635ux648ux631ux6cc}
برای اثبات اینکه \(f\) تابعی از \(X\) به \(W\) است، باید سه شرط را بررسی
کنیم: زیرمجموعه بودن، دامنه کامل و یکتایی.
\begin{info}{برهان}
\textbf{۱. شرط زیرمجموعه بودن:} می‌دانیم \(f \subseteq X \times Y\). چون
طبق فرض \(Y \subseteq W\)، طبق ویژگی‌های حاصلضرب دکارتی داریم
\(X \times Y \subseteq X \times W\). بنابراین با استفاده از خاصیت تعدی
زیرمجموعه‌ها (\autoref{قضیه-۲---تعدی-در-مجموعه-ها}):
\[f \subseteq X \times W\]
\textbf{۲. شرط دامنه کامل:} چون \(f: X \to Y\) تابع است، برای هر
\(x \in X\)، یک \(y \in Y\) هست که \((x,y) \in f\). چون
\(Y \subseteq W\)، پس این \(y\) در \(W\) هم هست (\(y \in W\)). پس شرط
وجود تصویر در \(W\) برقرار است.
\textbf{۳. شرط یکتایی:} ویژگی یکتایی تابع (\(y=z\)) وابسته به عناصر زوج
مرتب است و به مجموعه هم‌دامنه بستگی ندارد. چون \(f\) ذاتاً تابع است، این
ویژگی همچنان برقرار باقی می‌ماند.
\textbf{نتیجه:} \(f: X \to W\) یک تابع است.
\end{info}
\begin{center}\rule{0.5\linewidth}{0.5pt}\end{center}
\subsection{\texorpdfstring{۳. شبکه ارتباطی با سایر قضایا
\lr{(Analytic Map)}}{۳. شبکه ارتباطی با سایر قضایا }}\label{ux634ux628ux6a9ux647-ux627ux631ux62aux628ux627ux637ux6cc-ux628ux627-ux633ux627ux6ccux631-ux642ux636ux627ux6ccux627-analytic-map}
\subsubsection{\texorpdfstring{۱. ارتباط با
\autoref{پیشنیاز---مفاهیم-پایه-رابطه-و-تابع}}{۱. ارتباط با }}\label{ux627ux631ux62aux628ux627ux637-ux628ux627-ux67eux6ccux634ux646ux6ccux627ux632---ux645ux641ux627ux647ux6ccux645-ux67eux627ux6ccux647-ux631ux627ux628ux637ux647-ux648-ux62aux627ux628ux639}
\begin{itemize}
\tightlist
\item
  \textbf{رابطه والد-فرزندی:} تعریف تابع مستقیماً روی تعریف «رابطه»
  (زیرمجموعه حاصلضرب دکارتی) بنا شده است. قضیه ۶ نشان می‌دهد که تابع
  بودن، بیشتر به رفتار مولفه اول (دامنه) وابسته است تا محدودیت مولفه دوم
  (هم‌دامنه).
\end{itemize}
\subsubsection{\texorpdfstring{۲. ارتباط با
\autoref{قضیه-۱---پخش‌پذیری-حاصلضرب-دکارتی}}{۲. ارتباط با }}\label{ux627ux631ux62aux628ux627ux637-ux628ux627-ux642ux636ux6ccux647-ux6f1---ux67eux62eux634ux67eux630ux6ccux631ux6cc-ux62dux627ux635ux644ux636ux631ux628-ux62fux6a9ux627ux631ux62aux6cc}
\begin{itemize}
\tightlist
\item
  \textbf{ابزار اثبات:} در گام اول اثبات قضیه ۶، از اصلی استفاده کردیم
  که اگر \(B \subseteq C\) آنگاه \(A \times B \subseteq A \times C\).
  این نتیجه مستقیم تعاریف فصل ۳ درباره حاصلضرب دکارتی است.
\end{itemize}
