% ---------------------------------------------------------------------
% Copyright (c) 2026 Arsalan Dalvand & Reyhaneh Darvishi.
% Licensed under CC BY-NC-SA 4.0.
% See LICENSE file for details.
% ---------------------------------------------------------------------

\section{قضیه ۷: رفتار حدی اجتماع و اشتراک (خانواده
تهی)}\label{قضیه-۷.---تعمیم-اشتراک-و-اجتماع}
\begin{tldr}{خلاصه سریع}
این قضیه نتایج عملیات تعمیم‌یافته مجموعه‌ای را در «شرایط مرزی» بررسی
می‌کند. زمانی که مجموعه اندیس تهی باشد (\(\Gamma = \emptyset\))، اجتماع
برابر با عنصر خنثیِ جمع (تهی) و اشتراک برابر با عنصر خنثیِ ضرب (مجموعه
مرجع) می‌شود. این نتایج بر پایه مفهوم منطقی «صدق تُهی‌مایه»
\lr{(Vacuous Truth) }استوار هستند.
\end{tldr}
\subsection{۱. متن ریاضی
قضیه}\label{ux645ux62aux646-ux631ux6ccux627ux636ux6cc-ux642ux636ux6ccux647}
فرض کنید \(\{A_\gamma\}_{\gamma \in \Gamma}\) یک خانواده از زیرمجموعه‌های
مجموعه مرجع \(U\) باشد. اگر مجموعه اندیس تهی باشد
(\(\Gamma = \emptyset\))، آنگاه:
\begin{theorembox}{قضیه ۷}
\textbf{الف) اجتماع روی تهی:}
\[\bigcup_{\gamma \in \emptyset} A_\gamma = \emptyset\] \textbf{ب)
اشتراک روی تهی:} \[\bigcap_{\gamma \in \emptyset} A_\gamma = U\]
\end{theorembox}
\subsection{\texorpdfstring{۲. اثبات صوری
\lr{(Formal Proof)}}{۲. اثبات صوری }}\label{ux627ux62bux628ux627ux62a-ux635ux648ux631ux6cc-formal-proof}
اثبات این قضیه نیازمند تحلیل دقیق تعاریف سورهای وجودی و عمومی در دامنه
تهی است.
\subsubsection{اثبات قسمت
(الف)}\label{ux627ux62bux628ux627ux62a-ux642ux633ux645ux62a-ux627ux644ux641}
\begin{info}{اثبات}
طبق \textbf{\autoref{پیشنیاز---مفاهیم-بنیادین-مجموعه‌ها}}:
\[x \in \bigcup_{\gamma \in \emptyset} A_\gamma \iff \exists \gamma \in \emptyset, (x \in A_\gamma)\]
گزاره «\(\exists \gamma \in \emptyset\)» همواره \textbf{نادرست} (تناقض)
است، زیرا هیچ عضوی در \(\emptyset\) وجود ندارد که بخواهد در شرطی صدق
کند. بنابراین، هیچ \(x\) در جهان وجود ندارد که در این مجموعه باشد.
\[\forall x, x \notin \bigcup_{\gamma \in \emptyset} A_\gamma \implies \bigcup_{\gamma \in \emptyset} A_\gamma = \emptyset\]
\end{info}
\subsubsection{اثبات قسمت
(ب)}\label{ux627ux62bux628ux627ux62a-ux642ux633ux645ux62a-ux628}
\begin{info}{اثبات}
طبق \textbf{\autoref{پیشنیاز---مفاهیم-بنیادین-مجموعه‌ها}}:
\[x \in \bigcap_{\gamma \in \emptyset} A_\gamma \iff \forall \gamma (\gamma \in \emptyset \rightarrow x \in A_\gamma)\]
در گزاره شرطیِ داخل پرانتز، مقدم (\(\gamma \in \emptyset\)) همواره
\textbf{نادرست} است. طبق اصول منطق ریاضی (انتفای مقدم)، هر گزاره شرطی با
مقدم نادرست، دارای ارزش \textbf{راست \lr{(True)}} است (مستقل از ارزش
تالی). بنابراین شرط عضویت برای \textbf{تمام} \(x\)های موجود در مجموعه
مرجع \(U\) صادق است (صدق تهی‌مایه).
\[\forall x \in U, x \in \bigcap_{\gamma \in \emptyset} A_\gamma \implies \bigcap_{\gamma \in \emptyset} A_\gamma = U\]
\end{info}
\subsection{\texorpdfstring{۳. شبکه ارتباطی با سایر قضایا
\lr{(Analytic Map)}}{۳. شبکه ارتباطی با سایر قضایا }}\label{ux634ux628ux6a9ux647-ux627ux631ux62aux628ux627ux637ux6cc-ux628ux627-ux633ux627ux6ccux631-ux642ux636ux627ux6ccux627-analytic-map}
این قضیه یکی از زیباترین نمونه‌های تعامل منطق و نظریه مجموعه‌هاست:
\subsubsection{\texorpdfstring{۱. ارتباط با
\autoref{قضیه-۷---قوانین-تناقض} (منطق
گزاره‌ها)}{۱. ارتباط با  (منطق گزاره‌ها)}}\label{ux627ux631ux62aux628ux627ux637-ux628ux627-ux642ux636ux6ccux647-ux6f7---ux642ux648ux627ux646ux6ccux646-ux62aux646ux627ux642ux636-ux645ux646ux637ux642-ux6afux632ux627ux631ux647ux647ux627}
\begin{itemize}
\tightlist
\item
  \textbf{انتفای مقدم:} اثبات بخش (ب) تماماً متکی بر قانون منطقی
  \(c \rightarrow p \equiv t\) است که در فصل ۱ بررسی شد. در اینجا
  \(\gamma \in \emptyset\) نقش تناقض (\(c\)) را بازی می‌کند و باعث می‌شود
  کل گزاره راستگو (\(t\)) شود.
\end{itemize}
\subsubsection{\texorpdfstring{۲. ارتباط با
\autoref{پیشنیاز---مفاهیم-بنیادین-مجموعه‌ها}}{۲. ارتباط با }}\label{ux627ux631ux62aux628ux627ux637-ux628ux627-ux67eux6ccux634ux646ux6ccux627ux632---ux645ux641ux627ux647ux6ccux645-ux628ux646ux6ccux627ux62fux6ccux646-ux645ux62cux645ux648ux639ux647ux647ux627}
\begin{itemize}
\tightlist
\item
  \textbf{وابستگی تعریفی:} این قضیه بدون استفاده از تعاریف دقیق اجتماع و
  اشتراک مبتنی بر سورها (که در فایل پیشنیاز آمده است) قابل اثبات نیست.
  تعاریف شهودی در این نقطه کور \lr{(Empty Index) }کارایی ندارند.
\end{itemize}
\subsubsection{\texorpdfstring{۳. ارتباط با
\autoref{قضیه-۱---شمول-تهی}}{۳. ارتباط با }}\label{ux627ux631ux62aux628ux627ux637-ux628ux627-ux642ux636ux6ccux647-ux6f1---ux634ux645ux648ux644-ux62aux647ux6cc}
\begin{itemize}
\tightlist
\item
  \textbf{وحدت رویه:} همان منطقی که در قضیه ۱ باعث می‌شود
  \(\emptyset \subseteq A\) باشد (چون شرط عضویت در تهی دروغ است)، در
  اینجا باعث می‌شود اشتراک روی تهی برابر \(U\) شود. هر دو قضیه از
  ویژگی‌های دامنه تهی در گزاره‌های شرطی بهره می‌برند.
\end{itemize}
\subsubsection{\texorpdfstring{۴. ارتباط با
\autoref{قضیه-۵---متمم-و-زیرمجموعه}}{۴. ارتباط با }}\label{ux627ux631ux62aux628ux627ux637-ux628ux627-ux642ux636ux6ccux647-ux6f5---ux645ux62aux645ux645-ux648-ux632ux6ccux631ux645ux62cux645ux648ux639ux647}
\begin{itemize}
\tightlist
\item
  \textbf{تناظر جبری:} در قضیه ۵ دیدیم که \(U' = \emptyset\) و
  \(\emptyset' = U\). اگر قوانین دمورگان تعمیم‌یافته
  (\autoref{قضیه-۸---تعمیم-دمورگان}) را روی این قضیه اعمال کنیم، به
  نتیجه سازگاری می‌رسیم:
  \[(\bigcup_{\emptyset} A_\gamma)' = \bigcap_{\emptyset} A_\gamma' \implies (\emptyset)' = U\]
  که تاییدی بر صحت قضیه ۷ است.
\end{itemize}
