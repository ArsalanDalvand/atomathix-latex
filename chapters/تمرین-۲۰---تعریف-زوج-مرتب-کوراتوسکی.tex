% ---------------------------------------------------------------------
% Copyright (c) 2026 Arsalan Dalvand & Reyhaneh Darvishi.
% Licensed under CC BY-NC-SA 4.0.
% See LICENSE file for details.
% ---------------------------------------------------------------------

\section{تمرین ۲۰: تعریف دقیق زوج مرتب
(کوراتوسکی)}\label{تمرین-۲۰---تعریف-زوج-مرتب-کوراتوسکی}
\subsection{۱. صورت
سوال}\label{ux635ux648ux631ux62a-ux633ux648ux627ux644}
زوج مرتب \((x,y)\) را به صورت مجموعه \(\{\{x\}, \{x,y\}\}\) تعریف
می‌کنیم. ثابت کنید: \[(a,b) = (c,d) \iff a=c \land b=d\]
\subsection{۲. اثبات}\label{ux627ux62bux628ux627ux62a}
\subsubsection{\texorpdfstring{جهت اول
(\(\Leftarrow\))}{جهت اول (\textbackslash Leftarrow)}}\label{ux62cux647ux62a-ux627ux648ux644-leftarrow}
اگر \(a=c\) و \(b=d\) باشد، مجموعه‌های سازنده یکسان می‌شوند:
\[\{\{a\}, \{a,b\}\} = \{\{c\}, \{c,d\}\}\] پس \((a,b) = (c,d)\).
\subsubsection{\texorpdfstring{جهت دوم
(\(\Rightarrow\))}{جهت دوم (\textbackslash Rightarrow)}}\label{ux62cux647ux62a-ux62fux648ux645-rightarrow}
فرض کنیم \(\{\{a\}, \{a,b\}\} = \{\{c\}, \{c,d\}\}\). دو حالت داریم:
\textbf{حالت ۱: \(a=b\)} در این صورت
\((a,b) = \{\{a\}, \{a,a\}\} = \{\{a\}, \{a\}\} = \{\{a\}\}\) (مجموعه تک
عضوی). چون دو طرف تساوی برابرند، طرف راست هم باید تک‌عضوی باشد:
\(\{\{c\}, \{c,d\}\} = \{\{a\}\}\). این یعنی
\(\{c\} = \{c,d\} = \{a\}\). پس \(c=a\) و \(d=c\). در نتیجه \(a=b=c=d\)
و حکم ثابت است .
\textbf{حالت ۲: \(a \neq b\)} در این صورت طرف چپ دو عضو دارد: \(\{a\}\)
و \(\{a,b\}\). پس طرف راست هم باید دو عضو داشته باشد (پس \(c \neq d\)).
تنها راه برابری دو مجموعه دو عضوی این است که اعضا نظیر به نظیر برابر
باشند. عضو \(\{a\}\) (تک‌عضوی) باید با عضو تک‌عضوی طرف مقابل یعنی
\(\{c\}\) برابر باشد. پس \(\{a\} = \{c\} \implies a=c\). عضو دوم
\(\{a,b\}\) باید با \(\{c,d\}\) برابر باشد. چون \(a=c\)، پس
\(\{a,b\} = \{a,d\}\). چون \(a \neq b\)، پس \(b\) باید با عضو دیگر این
مجموعه یعنی \(d\) برابر باشد. پس \(b=d\).
\textbf{نتیجه:} در هر دو حالت \(a=c\) و \(b=d\) ثابت شد .
