% ---------------------------------------------------------------------
% Copyright (c) 2026 Arsalan Dalvand & Reyhaneh Darvishi.
% Licensed under CC BY-NC-SA 4.0.
% See LICENSE file for details.
% ---------------------------------------------------------------------

\section{تمرین ۱۵: تابع بازتابی و تابع
همانی}\label{تمرین-۱۵---تابع-بازتابی}
\subsection{۱. صورت
سوال}\label{ux635ux648ux631ux62a-ux633ux648ux627ux644}
\begin{info}{فرض کنید \(f: X \to X\) یک تابع از \(X\) به \(X\) باشد و هم‌زمان یک \textbf{رابطه بازتابی} \lr{(Reflexive Relation) }روی \(X\) محسوب شود. ثابت کنید در این صورت \(f\) همان تابع همانی \(I_X\) است.}
\end{info}
\subsection{۲. استراتژی
حل}\label{ux627ux633ux62aux631ux627ux62aux698ux6cc-ux62dux644}
ما با دو ویژگی طرف هستیم که باید آن‌ها را ترکیب کنیم:
\begin{enumerate}
\def\labelenumi{\arabic{enumi}.}
\tightlist
\item
  \textbf{رابطه بازتابی:} یعنی هر عضو با خودش رابطه دارد
  (\(\forall x, (x,x) \in f\)).
\item
  \textbf{تابع:} یعنی هر عضو دقیقاً با \textbf{یک} نفر رابطه دارد. ترکیب
  این دو ما را مجبور می‌کند که آن «یک نفر»، همان «خودش» باشد.
\end{enumerate}
\subsection{۳. حل
تشریحی}\label{ux62dux644-ux62aux634ux631ux6ccux62dux6cc}
\begin{info}{اثبات}
۱. \textbf{استفاده از ویژگی بازتابی:} چون \(f\) یک رابطه بازتابی روی
\(X\) است، طبق تعریف باید شامل تمام زوج‌های قطری باشد:
\[\forall x \in X, \quad (x, x) \in f\]
۲. \textbf{استفاده از ویژگی تابع بودن:} چون \(f\) تابع است، به ازای هر
ورودی \(x\)، خروجی باید \textbf{یکتا} باشد. فرض کنیم \((x, y) \in f\)
باشد.
۳. \textbf{نتیجه‌گیری:} از گام (۱) می‌دانیم که \((x, x)\) حتماً در \(f\)
هست. از گام (۲) می‌دانیم که \(x\) نمی‌تواند به بیش از یک چیز وصل شود. پس
تنها انتخابی که برای \(y\) باقی می‌ماند، خودِ \(x\) است.
\[y = x \implies f(x) = x\]
۴. \textbf{تطبیق با تعریف تابع همانی:} تابعی که در آن برای همه اعضا
\(f(x)=x\) باشد، تابع همانی (\(I_X\)) نامیده می‌شود.
\[\therefore f = I_X\]
\end{info}
