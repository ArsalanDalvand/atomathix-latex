% ---------------------------------------------------------------------
% Copyright (c) 2026 Arsalan Dalvand & Reyhaneh Darvishi.
% Licensed under CC BY-NC-SA 4.0.
% See LICENSE file for details.
% ---------------------------------------------------------------------

\section{قضیه ۷: شرط تساوی دو
تابع}\label{قضیه-۷---شرط-تساوی-توابع}
\begin{tldr}{خلاصه سریع}
دو تابع زمانی با هم «برابر» هستند که دقیقاً مجموعه زوج‌های مرتب یکسانی
باشند. این قضیه نشان می‌دهد که این شرط مجموعه‌ای، معادل است با اینکه خروجی
آن‌ها به ازای تک‌تک ورودی‌ها یکسان باشد.
\end{tldr}
\subsection{۱. متن ریاضی
قضیه}\label{ux645ux62aux646-ux631ux6ccux627ux636ux6cc-ux642ux636ux6ccux647}
فرض کنید \(f\) و \(g\) دو تابع از مجموعه \(X\) به مجموعه \(Y\) باشند
(\(f,g: X \to Y\)).
\begin{theorembox}{قضیه ۷}
دو تابع \(f\) و \(g\) برابرند (\(f=g\)) اگر و تنها اگر:
\[\forall x \in X, \quad f(x) = g(x)\]
\end{theorembox}
\subsection{\texorpdfstring{۲. اثبات صوری
\lr{(Formal Proof)}}{۲. اثبات صوری }}\label{ux627ux62bux628ux627ux62a-ux635ux648ux631ux6cc-formal-proof}
چون توابع در اصل «مجموعه» (از زوج‌های مرتب) هستند، اثبات برابری آن‌ها باید
بر اساس اصول نظریه مجموعه‌ها
(\(A=B \iff A \subseteq B \land B \subseteq A\)) باشد.
\begin{info}{برهان}
\textbf{جهت رفت (\(\Rightarrow\)):} فرض کنیم \(f = g\) (به عنوان دو
مجموعه برابرند). پس هر زوج مرتبی که در \(f\) باشد، در \(g\) هم هست. اگر
\((x, y) \in f\)، آنگاه \(y = f(x)\). چون \(f=g\)، پس \((x, y) \in g\) و
در نتیجه \(y = g(x)\). بنابراین \(f(x) = g(x)\).
\textbf{جهت برگشت (\(\Leftarrow\)):} فرض کنیم برای هر \(x \in X\) داشته
باشیم \(f(x) = g(x)\). باید ثابت کنیم مجموعه \(f\) زیرمجموعه \(g\) است و
برعکس.
۱. فرض کنید \((x, y)\) یک عضو دلخواه از \(f\) باشد. ۲. طبق تعریف تابع،
این یعنی \(y = f(x)\). ۳. طبق فرض خلف، می‌دانیم \(f(x) = g(x)\). ۴. پس
\(y = g(x)\). ۵. طبق تعریف تابع \(g\)، این یعنی \((x, y) \in g\). ۶.
نتیجه: \(f \subseteq g\).
(به طریق کاملاً مشابه ثابت می‌شود \(g \subseteq f\)). چون دو مجموعه
زیرمجموعه هم هستند، پس \(f = g\).
\end{info}
\subsection{\texorpdfstring{۳. شبکه ارتباطی با سایر قضایا
\lr{(Analytic Map)}}{۳. شبکه ارتباطی با سایر قضایا }}\label{ux634ux628ux6a9ux647-ux627ux631ux62aux628ux627ux637ux6cc-ux628ux627-ux633ux627ux6ccux631-ux642ux636ux627ux6ccux627-analytic-map}
\subsubsection{\texorpdfstring{۱. ارتباط با
\autoref{قضیه-۳---ویژگی‌های-بنیادی-کلاس-هم‌ارزی} (قیاس
ساختاری)}{۱. ارتباط با  (قیاس ساختاری)}}\label{ux627ux631ux62aux628ux627ux637-ux628ux627-ux642ux636ux6ccux647-ux6f3---ux648ux6ccux698ux6afux6ccux647ux627ux6cc-ux628ux646ux6ccux627ux62fux6cc-ux6a9ux644ux627ux633-ux647ux645ux627ux631ux632ux6cc-ux642ux6ccux627ux633-ux633ux627ux62eux62aux627ux631ux6cc}
\begin{itemize}
\tightlist
\item
  \textbf{یکتایی نمایش:} همان‌طور که در کلاس‌های هم‌ارزی دیدیم
  \(x/\xi = y/\xi \iff x \xi y\)، اینجا هم می‌بینیم که تساوی دو شیء
  ساختاری (تابع)، به تساوی مولفه‌های سازنده‌شان (مقادیر خروجی) تقلیل
  می‌یابد.
\end{itemize}
\subsubsection{\texorpdfstring{۲. کاربرد در
\autoref{تمرین-۱۵---تابع-بازتابی}}{۲. کاربرد در }}\label{ux6a9ux627ux631ux628ux631ux62f-ux62fux631-ux62aux645ux631ux6ccux646-ux6f1ux6f5---ux62aux627ux628ux639-ux628ux627ux632ux62aux627ux628ux6cc}
\begin{itemize}
\tightlist
\item
  \textbf{ابزار حل:} در تمرین ۱۵، برای اثبات اینکه \(f\) همان تابع همانی
  \(I_X\) است، دقیقاً از همین قضیه ۷ استفاده می‌کنیم. آنجا نشان می‌دهیم
  \(\forall x, f(x) = I_X(x)\) و سپس طبق قضیه ۷ نتیجه می‌گیریم
  \(f = I_X\). بدون قضیه ۷، استدلال در مورد برابری توابع ناقص می‌ماند.
\end{itemize}
