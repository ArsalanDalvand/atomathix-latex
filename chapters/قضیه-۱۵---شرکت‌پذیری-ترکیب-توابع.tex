% ---------------------------------------------------------------------
% Copyright (c) 2026 Arsalan Dalvand & Reyhaneh Darvishi.
% Licensed under CC BY-NC-SA 4.0.
% See LICENSE file for details.
% ---------------------------------------------------------------------

\section{\texorpdfstring{قضیه ۱۵: شرکت‌پذیری ترکیب توابع
\lr{(Associativity of Composition)}}{قضیه ۱۵: شرکت‌پذیری ترکیب توابع }}\label{قضیه-۱۵---شرکت‌پذیری-ترکیب-توابع}
\begin{tldr}{خلاصه سریع}
این قضیه بیان می‌کند که در زنجیره‌ای از توابع متوالی، اولویت ترکیب (محل
قرارگیری پرانتزها) اهمیتی ندارد. اگرچه ترکیب توابع خاصیت «جابجایی»
ندارد، اما همواره از خاصیت «شرکت‌پذیری» برخوردار است. این ویژگی، مجموعه
توابع را به یک «نیم‌گروه» \lr{(Semigroup) }تبدیل می‌کند.
\end{tldr}
\subsection{۱. متن ریاضی
قضیه}\label{ux645ux62aux646-ux631ux6ccux627ux636ux6cc-ux642ux636ux6ccux647}
فرض کنید سه تابع با دامنه‌ها و هم‌دامنه‌های متوالی به صورت زیر تعریف شده
باشند: \[f: X \to Y, \quad g: Y \to Z, \quad h: Z \to W\]
\begin{theorembox}{قضیه ۱۵}
ترکیب این توابع شرکت‌پذیر است، یعنی:
\[(h \circ g) \circ f = h \circ (g \circ f)\]
\end{theorembox}
\subsection{\texorpdfstring{۲. اثبات صوری
\lr{(Formal Proof)}}{۲. اثبات صوری }}\label{ux627ux62bux628ux627ux62a-ux635ux648ux631ux6cc-formal-proof}
برای اثبات تساوی دو تابع، طبق
\textbf{\autoref{قضیه-۷---شرط-تساوی-توابع}}، باید نشان دهیم که:
\begin{enumerate}
\def\labelenumi{\arabic{enumi}.}
\tightlist
\item
  دامنه‌ها و هم‌دامنه‌های دو طرف یکسان هستند.
\item
  به ازای هر ورودی یکسان، خروجی‌های یکسانی تولید می‌کنند.
\end{enumerate}
\begin{info}{برهان}
\textbf{گام ۱: بررسی خوش‌تعریفی و دامنه}
\begin{itemize}
\tightlist
\item
  تابع سمت چپ: ابتدا \(g \circ f\) تابعی از \(X\) به \(Z\) است. سپس
  ترکیب آن با \(h\)، تابعی از \(X\) به \(W\) می‌سازد.
\item
  تابع سمت راست: ابتدا \(h \circ g\) تابعی از \(Y\) به \(W\) است. سپس
  ترکیب آن با \(f\)، تابعی از \(X\) به \(W\) می‌سازد.
\item
  بنابراین هر دو تابع دارای دامنه \(X\) و هم‌دامنه \(W\) هستند.
\end{itemize}
\textbf{گام ۲: بررسی تساوی مقداری \lr{(Point-wise Equality)}} فرض کنیم
\(x\) عنصری دلخواه از \(X\) باشد (\(x \in X\)).
\begin{itemize}
\item
  \textbf{محاسبه سمت چپ:} \[((h \circ g) \circ f)(x)\] طبق تعریف ترکیب
  (\((\alpha \circ \beta)(x) = \alpha(\beta(x))\))، تابع بیرونی
  \((h \circ g)\) و درونی \(f\) است: \[= (h \circ g)(f(x))\] اکنون روی
  ترکیب \((h \circ g)\) اعمال تعریف می‌کنیم (ورودی آن \(f(x)\) است):
  \[= h(g(f(x)))\]
\item
  \textbf{محاسبه سمت راست:} \[(h \circ (g \circ f))(x)\] تابع بیرونی
  \(h\) و درونی \((g \circ f)\) است: \[= h((g \circ f)(x))\] اکنون داخل
  پرانتز را باز می‌کنیم: \[= h(g(f(x)))\]
\end{itemize}
\textbf{نتیجه:} چون برای هر \(x \in X\)، خروجی‌ها یکسان هستند
(\(h(g(f(x)))\))، طبق قضیه ۷، دو تابع برابرند.
\end{info}
\subsection{\texorpdfstring{۳. شبکه ارتباطی با سایر قضایا
\lr{(Analytic Map)}}{۳. شبکه ارتباطی با سایر قضایا }}\label{ux634ux628ux6a9ux647-ux627ux631ux62aux628ux627ux637ux6cc-ux628ux627-ux633ux627ux6ccux631-ux642ux636ux627ux6ccux627-analytic-map}
\subsubsection{\texorpdfstring{۱. وابستگی به
\autoref{قضیه-۷---شرط-تساوی-توابع}}{۱. وابستگی به }}\label{ux648ux627ux628ux633ux62aux6afux6cc-ux628ux647-ux642ux636ux6ccux647-ux6f7---ux634ux631ux637-ux62aux633ux627ux648ux6cc-ux62aux648ux627ux628ux639}
\begin{itemize}
\tightlist
\item
  \textbf{اصل موضوعی:} اثبات قضیه ۱۵ تماماً بر پایه قضیه ۷ استوار است.
  بدون قضیه ۷، رسیدن به \(h(g(f(x)))\) در هر دو طرف، لزوماً به معنای یکی
  بودن ``اشیاء'' تابع نیست. قضیه ۷ پل عبور از ``تساوی مقادیر'' به
  ``تساوی توابع'' است.
\end{itemize}
\subsubsection{۲. تضاد با خاصیت
جابجایی}\label{ux62aux636ux627ux62f-ux628ux627-ux62eux627ux635ux6ccux62a-ux62cux627ux628ux62cux627ux6ccux6cc}
\begin{itemize}
\tightlist
\item
  \textbf{هشدار جبری:} بسیار مهم است که شرکت‌پذیری را با جابجایی اشتباه
  نگیریم.
  \begin{itemize}
  \tightlist
  \item
    \textbf{شرکت‌پذیری (صحیح):}
    \((h \circ g) \circ f = h \circ (g \circ f)\)
  \item
    \textbf{جابجایی (غلط):} \(g \circ f \neq f \circ g\) (مگر در حالات
    خاص). این نشان می‌دهد که جبر توابع شبیه ضرب ماتریس‌هاست (شرکت‌پذیر اما
    غیرجابجایی).
  \end{itemize}
\end{itemize}
\subsubsection{\texorpdfstring{۳. ارتباط با
\autoref{مفهوم-ترکیب-توابع}}{۳. ارتباط با }}\label{ux627ux631ux62aux628ux627ux637-ux628ux627-ux645ux641ux647ux648ux645-ux62aux631ux6a9ux6ccux628-ux62aux648ux627ux628ux639}
\begin{itemize}
\tightlist
\item
  \textbf{تعریف بازگشتی:} اثبات بالا نشان می‌دهد که نماد
  \(h \circ g \circ f\) (بدون پرانتز) یک نماد خوش‌تعریف و بدون ابهام است.
  این نتیجه مستقیم تعریف بازگشتی ترکیب است که در یادداشت ``مفهوم ترکیب
  توابع'' معرفی شد.
\end{itemize}
