% ---------------------------------------------------------------------
% Copyright (c) 2026 Arsalan Dalvand & Reyhaneh Darvishi.
% Licensed under CC BY-NC-SA 4.0.
% See LICENSE file for details.
% ---------------------------------------------------------------------

\section{قضیه ۱: شمول مجموعه
تهی}\label{قضیه-۱---شمول-تهی}
\begin{tldr}{خلاصه سریع}
این قضیه بیان می‌کند که «مجموعه تهی» (\(\emptyset\)) عنصرِ کمین
\lr{(Minimal Element) }در رابطه شمول است. یعنی \(\emptyset\) زیرمجموعه
هر مجموعه دلبخواهی محسوب می‌شود. صدق این گزاره بر پایه مفهوم منطقی «صدق
تُهی‌مایه» \lr{(Vacuous Truth) }استوار است.
\end{tldr}
\subsection{۱. متن ریاضی
قضیه}\label{ux645ux62aux646-ux631ux6ccux627ux636ux6cc-ux642ux636ux6ccux647}
برای هر مجموعه دلبخواه \(A\):
\begin{theorembox}{قضیه ۱}
\[\forall A, (\emptyset \subseteq A)\]
\end{theorembox}
\subsection{\texorpdfstring{۲. اثبات صوری
\lr{(Formal Proof)}}{۲. اثبات صوری }}\label{ux627ux62bux628ux627ux62a-ux635ux648ux631ux6cc-formal-proof}
اثبات این قضیه کاربرد مستقیم تعاریف منطق گزاره‌ها در نظریه مجموعه‌هاست و
بر پایه تحلیل گزاره شرطی بنا شده است.
\begin{info}{مراحل اثبات}
۱. طبق \textbf{\autoref{پیشنیاز---مفاهیم-بنیادین-مجموعه‌ها}}، باید ثابت
کنیم گزاره شرطی زیر به ازای هر \(x\) صادق است:
\[(x \in \emptyset) \rightarrow (x \in A)\] ۲. طبق
\textbf{\autoref{پیشنیاز---مفاهیم-بنیادین-مجموعه‌ها}}، گزاره
\(x \in \emptyset\) همواره \textbf{نادرست} \lr{(False) }است. در منطق،
این گزاره را با نماد تناقض (\(c\)) نشان می‌دهیم. ۳. بنابراین ساختار گزاره
شرطی به فرم \(c \rightarrow P\) تبدیل می‌شود (که \(P\) گزاره \(x \in A\)
است
\end{info}
