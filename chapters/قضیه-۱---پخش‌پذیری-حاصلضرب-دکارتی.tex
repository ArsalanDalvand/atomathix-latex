% ---------------------------------------------------------------------
% Copyright (c) 2026 Arsalan Dalvand & Reyhaneh Darvishi.
% Licensed under CC BY-NC-SA 4.0.
% See LICENSE file for details.
% ---------------------------------------------------------------------

\section{قضیه ۱: پخش‌پذیری حاصلضرب دکارتی بر تقاطع و
اجتماع}\label{قضیه-۱---پخش‌پذیری-حاصلضرب-دکارتی}
\begin{tldr}{خلاصه سریع}
این قضیه بیان می‌کند که عملگر حاصلضرب دکارتی (\(\times\)) نسبت به
عملگرهای اشتراک (\(\cap\)) و اجتماع (\(\cup\)) خاصیت \textbf{پخش‌پذیری
\lr{(Distributivity)}} دارد. این رفتار، شباهت ساختاری جبر مجموعه‌ها را با
جبر اعداد (پخش ضرب روی جمع) تکمیل می‌کند.
\end{tldr}
\subsection{۱. متن ریاضی
قضیه}\label{ux645ux62aux646-ux631ux6ccux627ux636ux6cc-ux642ux636ux6ccux647}
فرض کنید \(A\)، \(B\) و \(C\) سه مجموعه دلبخواه باشند. همواره روابط زیر
برقرارند:
\begin{theorembox}{قضیه ۱}
\textbf{الف) پخش‌پذیری بر اشتراک:}
\[A \times (B \cap C) = (A \times B) \cap (A \times C)\] \textbf{ب)
پخش‌پذیری بر اجتماع:}
\[A \times (B \cup C) = (A \times B) \cup (A \times C)\]
\end{theorembox}
\subsection{\texorpdfstring{۲. اثبات صوری
\lr{(Formal Proof)}}{۲. اثبات صوری }}\label{ux627ux62bux628ux627ux62a-ux635ux648ux631ux6cc-formal-proof}
اثبات این قضیه بر پایه ترجمه تعاریف مجموعه‌ای به گزاره‌های منطقی و استفاده
از قوانین هم‌ارزی فصل ۱ بنا شده است.
\subsubsection{اثبات قسمت
(الف)}\label{ux627ux62bux628ux627ux62a-ux642ux633ux645ux62a-ux627ux644ux641}
برای اثبات برابری دو مجموعه، نشان می‌دهیم که گزاره‌نمای عضویت زوج مرتب
\((a, x)\) در هر دو طرف یکسان است.
\begin{info}{برهان}
\[(a, x) \in A \times (B \cap C)\] ۱. طبق تعریف حاصلضرب دکارتی و اشتراک:
\[\iff (a \in A) \wedge (x \in B \cap C)\]
\[\iff (a \in A) \wedge [(x \in B) \wedge (x \in C)]\]
۲. استفاده از \textbf{\autoref{قضیه-۲---هم‌ارزی‌های-منطقی-پایه}} برای
گزاره \((a \in A)\): \emph{(می‌دانیم \(p \equiv p \wedge p\))}
\[\iff [(a \in A) \wedge (a \in A)] \wedge (x \in B) \wedge (x \in C)\]
۳. استفاده از \textbf{\autoref{قضیه-۴---قوانین-شرکت-پذیری-و-پخش-پذیری}}
برای بازآرایی گزاره‌ها:
\[\iff [(a \in A) \wedge (x \in B)] \wedge [(a \in A) \wedge (x \in C)]\]
۴. طبق تعریف حاصلضرب دکارتی:
\[\iff [(a, x) \in A \times B] \wedge [(a, x) \in A \times C]\]
۵. طبق تعریف اشتراک: \[\iff (a, x) \in (A \times B) \cap (A \times C)\]
\textbf{نتیجه:} دو مجموعه با هم برابرند.
\end{info}
\subsection{\texorpdfstring{۳. شبکه ارتباطی با سایر قضایا
\lr{(Analytic Map)}}{۳. شبکه ارتباطی با سایر قضایا }}\label{ux634ux628ux6a9ux647-ux627ux631ux62aux628ux627ux637ux6cc-ux628ux627-ux633ux627ux6ccux631-ux642ux636ux627ux6ccux627-analytic-map}
این قضیه پیوند عمیقی بین ساختار دکارتی و اصول منطق گزاره‌ها برقرار می‌کند:
\subsubsection{\texorpdfstring{۱. ارتباط با
\autoref{قضیه-۴---قوانین-شرکت-پذیری-و-پخش-پذیری}}{۱. ارتباط با }}\label{ux627ux631ux62aux628ux627ux637-ux628ux627-ux642ux636ux6ccux647-ux6f4---ux642ux648ux627ux646ux6ccux646-ux634ux631ux6a9ux62a-ux67eux630ux6ccux631ux6cc-ux648-ux67eux62eux634-ux67eux630ux6ccux631ux6cc}
\begin{itemize}
\tightlist
\item
  \textbf{مبانی منطقی:} اگرچه در اثبات بالا (بخش الف) عمدتاً از خاصیت
  جابجایی و شرکت‌پذیری «و» استفاده شد، اما در اثبات بخش (ب) (پخش روی
  اجتماع)، مستقیماً از قانون پخش‌پذیری منطق
  (\(p \wedge (q \vee r) \equiv (p \wedge q) \vee (p \wedge r)\))
  استفاده می‌شود. این نشان می‌دهد که پخش‌پذیری در مجموعه‌ها، تصویرِ پخش‌پذیری
  در منطق است.
\end{itemize}
\subsubsection{\texorpdfstring{۲. ارتباط با
\autoref{قضیه-۴---جبر-مجموعه-ها}}{۲. ارتباط با }}\label{ux627ux631ux62aux628ux627ux637-ux628ux627-ux642ux636ux6ccux647-ux6f4---ux62cux628ux631-ux645ux62cux645ux648ux639ux647-ux647ux627}
\begin{itemize}
\tightlist
\item
  \textbf{توسعه ساختار جبری:} در فصل ۲، قوانین پخش‌پذیری را برای \(\cap\)
  و \(\cup\) دیدیم (\(A \cap (B \cup C)\)). قضیه فعلی، عملگر سوم
  (\(\times\)) را وارد این ساختار می‌کند و نشان می‌دهد که این عملگر جدید
  نیز با عملگرهای قبلی سازگار \lr{(Compatible) }است.
\end{itemize}
\subsubsection{\texorpdfstring{۳. ارتباط با
\autoref{قضیه-۲---پخش‌پذیری-حاصلضرب-دکارتی-بر-تفاضل}}{۳. ارتباط با }}\label{ux627ux631ux62aux628ux627ux637-ux628ux627-ux642ux636ux6ccux647-ux6f2---ux67eux62eux634ux67eux630ux6ccux631ux6cc-ux62dux627ux635ux644ux636ux631ux628-ux62fux6a9ux627ux631ux62aux6cc-ux628ux631-ux62aux641ux627ux636ux644}
\begin{itemize}
\tightlist
\item
  \textbf{تعمیم به تفاضل:} بلافاصله پس از این قضیه، خواهیم دید که
  حاصلضرب دکارتی روی تفاضل مجموعه‌ها نیز پخش می‌شود
  (\(A \times (B - C)\)). اثبات آن نیز از الگوی مشابهی پیروی می‌کند و
  نشان‌دهنده رفتار یکنواخت \(\times\) نسبت به تمام عملیات بولی است.
\end{itemize}
