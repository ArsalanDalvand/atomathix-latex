% ---------------------------------------------------------------------
% Copyright (c) 2026 Arsalan Dalvand & Reyhaneh Darvishi.
% Licensed under CC BY-NC-SA 4.0.
% See LICENSE file for details.
% ---------------------------------------------------------------------

\section{تمرین ۱: نمایش هندسی مجموعه‌ها در صفحه
دکارتی}\label{تمرین-۱---نمایش-هندسی-حاصلضرب-دکارتی}
\begin{tldr}{خلاصه سریع}
در فضای \(R \times R\) (صفحه مختصات)، رابطه \(x=y\) یک خط، \(x>y\) یک
ناحیه (نیم‌صفحه) و \(|x-y| \le 1\) یک نوار مورب است.
\end{tldr}
\subsection{۱. صورت
سوال}\label{ux635ux648ux631ux62a-ux633ux648ux627ux644}
هر یک از مجموعه‌های زیر را با رسم نمودار در صفحه دکارتی به طور هندسی
نمایش دهید : الف) \(\{(x,y) \in R \times R \mid x=y\}\) ب)
\(\{(x,y) \in R \times R \mid x > y\}\) ج)
\(\{(x,y) \in R \times R \mid |x-y| \le 1\}\)
\subsection{۲. حل تشریحی و
تحلیل}\label{ux62dux644-ux62aux634ux631ux6ccux62dux6cc-ux648-ux62aux62dux644ux6ccux644}
\subsubsection{\texorpdfstring{الف) نمودار
\(x = y\)}{الف) نمودار x = y}}\label{ux627ux644ux641-ux646ux645ux648ux62fux627ux631-x-y}
این معادله بیانگر نقاطی است که طول و عرضشان برابر است.
\begin{itemize}
\tightlist
\item
  \textbf{تحلیل:} این همان نیمساز ربع اول و سوم است. خطی که از مبدا
  می‌گذرد و زاویه ۴۵ درجه با محور افق دارد.
\end{itemize}
\subsubsection{\texorpdfstring{ب) نمودار
\(x > y\)}{ب) نمودار x \textgreater{} y}}\label{ux628-ux646ux645ux648ux62fux627ux631-x-y}
این نابرابری بیانگر ناحیه‌ای است که طول نقاط از عرضشان بیشتر است.
\begin{itemize}
\tightlist
\item
  \textbf{تحلیل:} خط \(x=y\) صفحه را به دو نیمه تقسیم می‌کند.
  \begin{itemize}
  \tightlist
  \item
    ناحیه بالای خط: \(y > x\)
  \item
    ناحیه پایین خط: \(x > y\)
  \end{itemize}
\item
  \textbf{پاسخ:} تمام ناحیه سمت راست و پایین خط \(x=y\) (بدون خود خط،
  چون تساوی نداریم).
\end{itemize}
\subsubsection{\texorpdfstring{ج) نمودار
\(|x-y| \le 1\)}{ج) نمودار \textbar x-y\textbar{} \textbackslash le 1}}\label{ux62c-ux646ux645ux648ux62fux627ux631-x-y-le-1}
این نامساوی قدرمطلقی به معنای فاصله عمودی یا افقی بین \(x\) و \(y\) است.
\begin{itemize}
\tightlist
\item
  \textbf{باز کردن قدر مطلق:} \[|x - y| \le 1 \iff -1 \le x - y \le 1\]
\item
  \textbf{تبدیل به دو نامساوی خطی:}
  \begin{enumerate}
  \def\labelenumi{\arabic{enumi}.}
  \tightlist
  \item
    \(x - y \le 1 \Rightarrow y \ge x - 1\) (ناحیه بالای خط \(y=x-1\))
  \item
    \(x - y \ge -1 \Rightarrow y \le x + 1\) (ناحیه پایین خط \(y=x+1\))
  \end{enumerate}
\item
  \textbf{پاسخ:} ناحیه محدود بین دو خط موازی \(y=x+1\) و \(y=x-1\) (یک
  نوار مورب شامل خود خطوط) .
\end{itemize}
