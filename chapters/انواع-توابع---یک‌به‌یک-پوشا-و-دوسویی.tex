% ---------------------------------------------------------------------
% Copyright (c) 2026 Arsalan Dalvand & Reyhaneh Darvishi.
% Licensed under CC BY-NC-SA 4.0.
% See LICENSE file for details.
% ---------------------------------------------------------------------

\section{انواع توابع: یک‌به‌یک، پوشا و
دوسویی}\label{انواع-توابع---یک‌به‌یک-پوشا-و-دوسویی}
\begin{tldr}{خلاصه سریع}
توابع بر اساس رفتار «نگاشت» اعضای دامنه به هم‌دامنه به سه دسته اصلی تقسیم
می‌شوند: ۱. \textbf{یک‌به‌یک \lr{(Injective):}} هیچ دو ورودی، خروجی یکسان
ندارند (تلاقی ممنوع). ۲. \textbf{پوشا \lr{(Surjective):}} تمام اعضای
مقصد، حداقل یک بار هدف قرار گرفته‌اند (هم‌دامنه = برد). ۳. \textbf{دوسویی
\lr{(Bijective):}} ترکیبی از هر دو؛ یک تناظر کامل و بی‌نقص بین دو مجموعه.
\end{tldr}
\begin{center}\rule{0.5\linewidth}{0.5pt}\end{center}
\subsection{\texorpdfstring{۱. تابع یک‌به‌یک \lr{(Injective }/
\lr{One-to-One)}}{۱. تابع یک‌به‌یک / }}\label{ux62aux627ux628ux639-ux6ccux6a9ux628ux647ux6ccux6a9-injective-one-to-one}
\subsubsection{الف) تعریف
ریاضی}\label{ux627ux644ux641-ux62aux639ux631ux6ccux641-ux631ux6ccux627ux636ux6cc}
تابع \(f: X \to Y\) را \textbf{یک‌به‌یک} گوییم هرگاه تصاویرِ عناصر متمایز،
متمایز باشند.
\begin{tldr}{تعریف ۱۰}
\[f \text{ is injective} \iff \forall x_1, x_2 \in X, \quad f(x_1) = f(x_2) \implies x_1 = x_2\]
\end{tldr}
\subsubsection{ب) تحلیل
منطقی}\label{ux628-ux62aux62dux644ux6ccux644-ux645ux646ux637ux642ux6cc}
این تعریف بر اساس \textbf{قانون عکس نقیض} (فصل ۱) معادل گزاره زیر است:
\[x_1 \neq x_2 \implies f(x_1) \neq f(x_2)\] یعنی تابع \(f\) هرگز دو عضو
مختلف دامنه را به یک نقطه در هم‌دامنه «فشرده» نمی‌کند
\lr{(Information Lossless).}
\subsubsection{ج) مثال}\label{ux62c-ux645ux62bux627ux644}
\begin{itemize}
\tightlist
\item
  تابع \(f(x) = 2x\) روی اعداد حقیقی یک‌به‌یک است.
\item
  تابع \(f(x) = x^2\) روی اعداد حقیقی یک‌به‌یک \textbf{نیست} (چون
  \(f(2) = f(-2) = 4\)).
\end{itemize}
\begin{center}\rule{0.5\linewidth}{0.5pt}\end{center}
\subsection{\texorpdfstring{۲. تابع پوشا \lr{(Surjective }/
\lr{Onto)}}{۲. تابع پوشا / }}\label{ux62aux627ux628ux639-ux67eux648ux634ux627-surjective-onto}
\subsubsection{الف) تعریف
ریاضی}\label{ux627ux644ux641-ux62aux639ux631ux6ccux641-ux631ux6ccux627ux636ux6cc-1}
تابع \(f: X \to Y\) را \textbf{پوشا} گوییم هرگاه برد تابع (\(Im(f)\))
دقیقاً برابر با هم‌دامنه (\(Y\)) باشد.
\begin{tldr}{تعریف ۱۱}
\[f \text{ is surjective} \iff \forall y \in Y, \exists x \in X : f(x) = y\]
معادل مجموعه‌ای: \[f(X) = Y\]
\end{tldr}
\subsubsection{ب) تحلیل
ساختاری}\label{ux628-ux62aux62dux644ux6ccux644-ux633ux627ux62eux62aux627ux631ux6cc}
در تابع پوشا، هیچ عنصری در \(Y\) «بی‌نصیب» نمی‌ماند. اگر \(f\) را به عنوان
یک تیراندازی از \(X\) به \(Y\) در نظر بگیریم، پوشا بودن یعنی تمام اهداف
در \(Y\) تیر خورده‌اند.
\begin{center}\rule{0.5\linewidth}{0.5pt}\end{center}
\subsection{\texorpdfstring{۳. تابع دوسویی
\lr{(Bijective)}}{۳. تابع دوسویی }}\label{ux62aux627ux628ux639-ux62fux648ux633ux648ux6ccux6cc-bijective}
\subsubsection{الف) تعریف
ریاضی}\label{ux627ux644ux641-ux62aux639ux631ux6ccux641-ux631ux6ccux627ux636ux6cc-2}
تابع \(f: X \to Y\) را \textbf{دوسویی} (یا تناظر یک‌به‌یک) گوییم اگر
هم‌زمان یک‌به‌یک و پوشا باشد.
\begin{tldr}{تعریف ۱۲}
\[f \text{ is bijective} \iff (f \text{ is injective}) \wedge (f \text{ is surjective})\]
\end{tldr}
\subsubsection{ب) اهمیت
(وارون‌پذیری)}\label{ux628-ux627ux647ux645ux6ccux62a-ux648ux627ux631ux648ux646ux67eux630ux6ccux631ux6cc}
توابع دوسویی مهم‌ترین کلاس توابع در جبر هستند زیرا ساختار دو مجموعه را
کاملاً به هم منتقل می‌کنند (ایزومورفیسم). تنها توابع دوسویی هستند که
\textbf{وارون‌پذیر} می‌باشند.
\begin{center}\rule{0.5\linewidth}{0.5pt}\end{center}
\subsection{\texorpdfstring{۴. شبکه ارتباطی با سایر قضایا
\lr{(Analytic Map)}}{۴. شبکه ارتباطی با سایر قضایا }}\label{ux634ux628ux6a9ux647-ux627ux631ux62aux628ux627ux637ux6cc-ux628ux627-ux633ux627ux6ccux631-ux642ux636ux627ux6ccux627-analytic-map}
\subsubsection{\texorpdfstring{۱. ارتباط با
\autoref{قضیه-۹---تصویر-و-تصویر-وارون-مجموعه} (رفتار با
برد)}{۱. ارتباط با  (رفتار با برد)}}\label{ux627ux631ux62aux628ux627ux637-ux628ux627-ux642ux636ux6ccux647-ux6f9---ux62aux635ux648ux6ccux631-ux648-ux62aux635ux648ux6ccux631-ux648ux627ux631ux648ux646-ux645ux62cux645ux648ux639ux647-ux631ux641ux62aux627ux631-ux628ux627-ux628ux631ux62f}
\begin{itemize}
\tightlist
\item
  \textbf{شرط پوشا بودن:} در قضیه ۹ دیدیم که \(f(X) \subseteq Y\) همواره
  برقرار است. تعریف پوشا بودن (تعریف ۱۱) این شمول را به تساوی
  \(f(X) = Y\) ارتقا می‌دهد.
\end{itemize}
\subsubsection{\texorpdfstring{۲. ارتباط با
\autoref{قضیه-۱۰---تعمیم-تصویر-اجتماع-و-اشتراک} (رفتار با
اشتراک)}{۲. ارتباط با  (رفتار با اشتراک)}}\label{ux627ux631ux62aux628ux627ux637-ux628ux627-ux642ux636ux6ccux647-ux6f1ux6f0---ux62aux639ux645ux6ccux645-ux62aux635ux648ux6ccux631-ux627ux62cux62aux645ux627ux639-ux648-ux627ux634ux62aux631ux627ux6a9-ux631ux641ux62aux627ux631-ux628ux627-ux627ux634ux62aux631ux627ux6a9}
\begin{itemize}
\tightlist
\item
  \textbf{یک‌به‌یک بودن و اشتراک:} در قضیه ۱۰ دیدیم که
  \(f(A \cap B) \subseteq f(A) \cap f(B)\) و تساوی لزوماً برقرار نیست.
  اما اگر تابع \textbf{یک‌به‌یک} باشد، تساوی برقرار می‌شود:
  \[f \text{ is 1-1} \implies f(A \cap B) = f(A) \cap f(B)\] این نتیجه
  در \textbf{\autoref{قضیه-۱۳---حفظ-اشتراک-در-توابع-یک‌به‌یک}} (که بعداً
  بررسی می‌کنیم) اثبات می‌شود.
\end{itemize}
\subsubsection{\texorpdfstring{۳. ارتباط با
\autoref{قضیه-۱۶---وارون‌های-یک‌طرفه}}{۳. ارتباط با }}\label{ux627ux631ux62aux628ux627ux637-ux628ux627-ux642ux636ux6ccux647-ux6f1ux6f6---ux648ux627ux631ux648ux646ux647ux627ux6cc-ux6ccux6a9ux637ux631ux641ux647}
\begin{itemize}
\tightlist
\item
  \textbf{وارون پذیری:}
  \begin{itemize}
  \tightlist
  \item
    اگر \(g \circ f = I_X\) (وارون چپ)، آنگاه \(f\) \textbf{یک‌به‌یک} است.
  \item
    اگر \(f \circ h = I_Y\) (وارون راست)، آنگاه \(f\) \textbf{پوشا} است.
  \item
    اگر هر دو برقرار باشند، \(f\) \textbf{دوسویی} است.
  \end{itemize}
\end{itemize}
