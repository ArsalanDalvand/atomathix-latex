% ---------------------------------------------------------------------
% Copyright (c) 2026 Arsalan Dalvand & Reyhaneh Darvishi.
% Licensed under CC BY-NC-SA 4.0.
% See LICENSE file for details.
% ---------------------------------------------------------------------

\section{\texorpdfstring{قضیه ۴: قوانین جبر مجموعه‌ها
\lr{(Set Algebra Laws)}}{قضیه ۴: قوانین جبر مجموعه‌ها }}\label{قضیه-۴---جبر-مجموعه-ها}
\begin{tldr}{خلاصه سریع}
این قضیه ساختار جبری حاکم بر مجموعه‌ها را تبیین می‌کند. این قوانین نشان
می‌دهند که \((\mathcal{P}(X), \cup, \cap, ', \emptyset, X)\) یک
\textbf{جبر بول \lr{(Boolean Algebra)}} است که دقیقاً ایزومورف با جبر
گزاره‌ها در منطق ریاضی می‌باشد.
\end{tldr}
\subsection{۱. متن ریاضی
قضیه}\label{ux645ux62aux646-ux631ux6ccux627ux636ux6cc-ux642ux636ux6ccux647}
فرض کنید \(X\) مجموعه مرجع باشد و \(A, B, C\) زیرمجموعه‌هایی از \(X\)
باشند. قوانین زیر همواره برقرارند:
\begin{theorembox}{قضیه ۴}
\textbf{الف) قوانین یکه \lr{(Identity Laws):}} \[A \cup \emptyset = A\]
\[A \cap X = A\]
\textbf{ب) قوانین خودتوانی \lr{(Idempotent Laws):}} \[A \cup A = A\]
\[A \cap A = A\]
\textbf{ج) قوانین جابجایی \lr{(Commutative Laws):}}
\[A \cup B = B \cup A\] \[A \cap B = B \cap A\]
\textbf{د) قوانین شرکت‌پذیری \lr{(Associative Laws):}}
\[A \cup (B \cup C) = (A \cup B) \cup C\]
\[A \cap (B \cap C) = (A \cap B) \cap C\]
\textbf{ه) قوانین پخش‌پذیری \lr{(Distributive Laws):}}
\[A \cap (B \cup C) = (A \cap B) \cup (A \cap C)\]
\[A \cup (B \cap C) = (A \cup B) \cap (A \cup C)\]
\end{theorembox}
\subsection{\texorpdfstring{۲. اثبات صوری
\lr{(Formal Proof)}}{۲. اثبات صوری }}\label{ux627ux62bux628ux627ux62a-ux635ux648ux631ux6cc-formal-proof}
اثبات این قوانین مبتنی بر ترجمه گزاره‌های مجموعه‌ای به گزاره‌های منطقی و
استفاده از هم‌ارزی‌های فصل ۱ است. در اینجا اثبات قانون شرکت‌پذیری اجتماع
(بخش د) به عنوان نمونه ارائه می‌شود.
\begin{info}{اثبات \(A \cup (B \cup C) = (A \cup B) \cup C\)}
برای اثبات تساوی دو مجموعه، نشان می‌دهیم تابع گزاره‌نمای عضویت
(\(x \in S\)) برای هر دو طرف هم‌ارز است:
۱. تعریف طرف چپ:
\[x \in A \cup (B \cup C) \iff x \in A \vee (x \in B \cup C)\]
\[\iff x \in A \vee (x \in B \vee x \in C)\]
۲. اعمال قانون شرکت‌پذیری در منطق (طبق
\textbf{\autoref{قضیه-۴---قوانین-شرکت-پذیری-و-پخش-پذیری}}):
\[\iff (x \in A \vee x \in B) \vee x \in C\]
۳. بازگردانی به تعریف مجموعه: \[\iff x \in (A \cup B) \vee x \in C\]
\[\iff x \in (A \cup B) \cup C\]
۴. نتیجه: چون شرط عضویت برای هر \(x\) در دو مجموعه یکسان است، پس دو
مجموعه برابرند.
\end{info}
\subsection{\texorpdfstring{۳. شبکه ارتباطی با سایر قضایا
\lr{(Analytic Map)}}{۳. شبکه ارتباطی با سایر قضایا }}\label{ux634ux628ux6a9ux647-ux627ux631ux62aux628ux627ux637ux6cc-ux628ux627-ux633ux627ux6ccux631-ux642ux636ux627ux6ccux627-analytic-map}
این قضیه تجلی مستقیم قوانین منطق در دنیای مجموعه‌هاست:
\subsubsection{\texorpdfstring{۱. ارتباط با
\autoref{قضیه-۴---قوانین-شرکت-پذیری-و-پخش-پذیری} (ساختار
مشترک)}{۱. ارتباط با  (ساختار مشترک)}}\label{ux627ux631ux62aux628ux627ux637-ux628ux627-ux642ux636ux6ccux647-ux6f4---ux642ux648ux627ux646ux6ccux646-ux634ux631ux6a9ux62a-ux67eux630ux6ccux631ux6cc-ux648-ux67eux62eux634-ux67eux630ux6ccux631ux6cc-ux633ux627ux62eux62aux627ux631-ux645ux634ux62aux631ux6a9}
\begin{itemize}
\tightlist
\item
  \textbf{ایزومورفیسم:} قوانین شرکت‌پذیری و پخش‌پذیری در مجموعه‌ها، تصویر
  دقیق قوانین متناظر در منطق هستند. اگر جای \(\cup\) را با \(\vee\) و
  جای \(\cap\) را با \(\wedge\) عوض کنیم، دقیقاً به فرمول‌های منطقی فصل ۱
  می‌رسیم. این نشان می‌دهد که منطق و نظریه مجموعه‌ها دو مدل مختلف از یک
  ساختار جبری واحد (جبر بول) هستند.
\end{itemize}
\subsubsection{\texorpdfstring{۲. ارتباط با
\autoref{قضیه-۷---قوانین-تناقض} (یکه و
خودتوانی)}{۲. ارتباط با  (یکه و خودتوانی)}}\label{ux627ux631ux62aux628ux627ux637-ux628ux627-ux642ux636ux6ccux647-ux6f7---ux642ux648ux627ux646ux6ccux646-ux62aux646ux627ux642ux636-ux6ccux6a9ux647-ux648-ux62eux648ux62fux62aux648ux627ux646ux6cc}
\begin{itemize}
\tightlist
\item
  \textbf{تناظر ثابت‌ها:} قوانین یکه \lr{(Identity) }در مجموعه‌ها
  (\(A \cup \emptyset = A\)) متناظر با قوانین همانی در منطق
  (\(p \vee c \equiv p\)) هستند. در این تناظر، مجموعه تهی
  (\(\emptyset\)) نقش تناقض (\(c\)) و مجموعه مرجع (\(X\)) نقش راستگو
  (\(t\)) را بازی می‌کند.
\item
  \textbf{خودتوانی:} قوانین \(A \cup A = A\) نیز مستقیماً از معادل منطقی
  \(p \vee p \equiv p\) (قضیه ۲ فصل ۱) نشأت می‌گیرند.
\end{itemize}
\subsubsection{\texorpdfstring{۳. ارتباط با
\autoref{قضیه-۹---تعمیم-پخش-پذیری}
(تعمیم)}{۳. ارتباط با  (تعمیم)}}\label{ux627ux631ux62aux628ux627ux637-ux628ux627-ux642ux636ux6ccux647-ux6f9---ux62aux639ux645ux6ccux645-ux67eux62eux634-ux67eux630ux6ccux631ux6cc-ux62aux639ux645ux6ccux645}
\begin{itemize}
\tightlist
\item
  \textbf{گسترش به خانواده‌ها:} قوانین پخش‌پذیری بیان شده در اینجا (بخش ه)
  محدود به دو یا سه مجموعه هستند. در قضیه ۹ همین فصل، این قوانین برای
  خانواده‌های نامتناهی از مجموعه‌ها تعمیم داده می‌شوند
  (\(A \cap (\bigcup B_i) = \bigcup (A \cap B_i)\)).
\end{itemize}
