% ---------------------------------------------------------------------
% Copyright (c) 2026 Arsalan Dalvand & Reyhaneh Darvishi.
% Licensed under CC BY-NC-SA 4.0.
% See LICENSE file for details.
% ---------------------------------------------------------------------

\section{\texorpdfstring{قضیه ۷: قوانین مربوط به راستگو و تناقض
\lr{(Identity and Domination Laws)}}{قضیه ۷: قوانین مربوط به راستگو و تناقض }}\label{قضیه-۷---قوانین-تناقض}
\begin{tldr}{خلاصه سریع}
این قضیه رفتار گزاره‌ها را در تعامل با «ثابت‌های منطقی» یعنی راستگو
(\(t\)) و تناقض (\(c\)) مشخص می‌کند. در اینجا \(t\) شبیه عدد ۱ در ضرب
(خنثی) یا بینهایت در جمع (غالب) عمل می‌کند و \(c\) شبیه ۰ در جمع (خنثی)
یا ۰ در ضرب (غالب) است.
\end{tldr}
\subsection{۱. متن ریاضی
قضیه}\label{ux645ux62aux646-ux631ux6ccux627ux636ux6cc-ux642ux636ux6ccux647}
فرض کنید \(t\) یک گزاره همیشه راست \lr{(Tautology)، }\(c\) یک گزاره
همیشه دروغ \lr{(Contradiction) }و \(p\) یک گزاره دلبخواه باشد.
\begin{theorembox}{قضیه ۷}
\textbf{الف) قوانین همانی \lr{(Identity Laws):}} \[p \wedge t \equiv p\]
\[p \vee c \equiv p\] \textbf{ب) قوانین سلطه \lr{(Domination Laws):}}
\[p \vee t \equiv t\] \[p \wedge c \equiv c\] \textbf{ج) قوانین شرطی
خاص:} \[p \rightarrow t \equiv t\] \[c \rightarrow p \equiv t\]
\end{theorembox}
\subsection{۲. اثبات و تحلیل (با جدول
ارزش)}\label{ux627ux62bux628ux627ux62a-ux648-ux62aux62dux644ux6ccux644-ux628ux627-ux62cux62fux648ux644-ux627ux631ux632ux634}
\subsubsection{\texorpdfstring{اثبات قانون همانی
(\(p \wedge t \equiv p\))}{اثبات قانون همانی (p \textbackslash wedge t \textbackslash equiv p)}}\label{ux627ux62bux628ux627ux62a-ux642ux627ux646ux648ux646-ux647ux645ux627ux646ux6cc-p-wedge-t-equiv-p}
در این جدول، ارزش \(t\) همواره \textbf{\lr{T}} در نظر گرفته می‌شود.
{\def\LTcaptype{none}
\begin{longtable}[]{@{}ccccc@{}}
\toprule\noalign{}
\(p\) & \(\wedge\) & \(t\) & \(\leftrightarrow\) & \(p\) \\
\midrule\noalign{}
\endhead
\bottomrule\noalign{}
\endlastfoot
\lr{T} & \lr{T} & \lr{T} & \textbf{\lr{T}} & \lr{T} \\
\lr{F} & \lr{F} & \lr{T} & \textbf{\lr{T}} & \lr{F} \\
\emph{(جدول اثبات قانون همانی)} & & & & \\
\end{longtable}
}
\begin{info}{تحلیل}
وقتی یکی از طرفین «و» (\(\wedge\))، حقیقت محض (\(t\)) باشد، کل عبارت
تنها زمانی درست است که طرف دیگر (\(p\)) درست باشد. اگر \(p\) غلط باشد،
کل عبارت غلط می‌شود. پس نتیجه دقیقاً تابع \(p\) است.
\end{info}
\subsubsection{\texorpdfstring{اثبات قانون سلطه
(\(p \vee t \equiv t\))}{اثبات قانون سلطه (p \textbackslash vee t \textbackslash equiv t)}}\label{ux627ux62bux628ux627ux62a-ux642ux627ux646ux648ux646-ux633ux644ux637ux647-p-vee-t-equiv-t}
در این جدول نیز ارزش \(t\) همواره \textbf{\lr{T}} است.
{\def\LTcaptype{none}
\begin{longtable}[]{@{}ccccc@{}}
\toprule\noalign{}
\(p\) & \(\vee\) & \(t\) & \(\leftrightarrow\) & \(t\) \\
\midrule\noalign{}
\endhead
\bottomrule\noalign{}
\endlastfoot
\lr{T} & \lr{T} & \lr{T} & \textbf{\lr{T}} & \lr{T} \\
\lr{F} & \lr{T} & \lr{T} & \textbf{\lr{T}} & \lr{T} \\
\emph{(جدول اثبات قانون سلطه)} & & & & \\
\end{longtable}
}
\begin{info}{تحلیل}
در ترکیب فصلی (\(\vee\))، اگر حداقل یک طرف درست باشد، کل عبارت درست است.
چون \(t\) همیشه درست است، حضور آن کافی است تا کل عبارت (\(p \vee t\))
صرف‌نظر از مقدار \(p\)، همیشه درست (\(t\)) شود.
\end{info}
\subsubsection{\texorpdfstring{اثبات قوانین شرطی
(\(c \rightarrow p\))}{اثبات قوانین شرطی (c \textbackslash rightarrow p)}}\label{ux627ux62bux628ux627ux62a-ux642ux648ux627ux646ux6ccux646-ux634ux631ux637ux6cc-c-rightarrow-p}
در این جدول، ارزش \(c\) همواره \textbf{\lr{F}} است.
{\def\LTcaptype{none}
\begin{longtable}[]{@{}ccccc@{}}
\toprule\noalign{}
\(c\) & \(\rightarrow\) & \(p\) & \(\leftrightarrow\) & \(t\) \\
\midrule\noalign{}
\endhead
\bottomrule\noalign{}
\endlastfoot
\lr{F} & \lr{T} & \lr{T} & \textbf{\lr{T}} & \lr{T} \\
\lr{F} & \lr{T} & \lr{F} & \textbf{\lr{T}} & \lr{T} \\
\emph{(جدول اثبات انتفای مقدم)} & & & & \\
\end{longtable}
}
\begin{info}{تحلیل}
در گزاره شرطی، اگر مقدم (\(c\)) نادرست باشد، کل گزاره به صورت خودکار
درست (\(t\)) می‌شود (انتفای مقدم). بنابراین، «از یک دروغ می‌توان هر
نتیجه‌ای گرفت» و کل ساختار همیشه راستگوست.
\end{info}
\subsection{\texorpdfstring{۳. شبکه ارتباطی با سایر قضایا
\lr{(Analytic Map)}}{۳. شبکه ارتباطی با سایر قضایا }}\label{ux634ux628ux6a9ux647-ux627ux631ux62aux628ux627ux637ux6cc-ux628ux627-ux633ux627ux6ccux631-ux642ux636ux627ux6ccux627-analytic-map}
این قضیه ابزارهای جبری قدرتمندی برای ساده‌سازی عبارات منطقی و مجموعه‌ای
فراهم می‌کند:
\subsubsection{\texorpdfstring{۱. ارتباط با
\autoref{قضیه-۶---قواعد-استنتاج} (برهان
خلف)}{۱. ارتباط با  (برهان خلف)}}\label{ux627ux631ux62aux628ux627ux637-ux628ux627-ux642ux636ux6ccux647-ux6f6---ux642ux648ux627ux639ux62f-ux627ux633ux62aux646ux62aux627ux62c-ux628ux631ux647ux627ux646-ux62eux644ux641}
\begin{itemize}
\tightlist
\item
  \textbf{تعریف فرمال تناقض:} در برهان خلف (قضیه ۶-ج) دیدیم که
  \((p \wedge \sim q) \to c\). قضیه ۷ ماهیت جبری \(c\) را تعریف می‌کند.
  مثلاً اگر در یک اثبات به \(A \wedge \sim A\) برسیم، می‌توانیم آن را با
  \(c\) جایگزین کنیم و سپس با استفاده از قانون سلطه
  (\(p \wedge c \equiv c\)) کل شاخه استدلال را باطل کنیم.
\end{itemize}
