% ---------------------------------------------------------------------
% Copyright (c) 2026 Arsalan Dalvand & Reyhaneh Darvishi.
% Licensed under CC BY-NC-SA 4.0.
% See LICENSE file for details.
% ---------------------------------------------------------------------

\section{مفاهیم پایه: حاصلضرب دکارتی، رابطه و
تابع}\label{پیشنیاز---مفاهیم-پایه-رابطه-و-تابع}
\begin{tldr}{خلاصه سریع}
این یادداشت پیش‌نیازهای ضروری برای ورود به قضایای فصل ۳ (به‌ویژه قضایای ۱،
۱۵ و ۱۶) را پوشش می‌دهد. مفاهیم کلیدی شامل ساختار زوج مرتب، حاصلضرب
دکارتی، تعریف صوری تابع، ترکیب توابع و انواع خاص توابع (یک‌به‌یک و پوشا)
هستند.
\end{tldr}
\subsection{\texorpdfstring{۱. حاصلضرب دکارتی
\lr{(Cartesian Product)}}{۱. حاصلضرب دکارتی }}\label{ux62dux627ux635ux644ux636ux631ux628-ux62fux6a9ux627ux631ux62aux6cc-cartesian-product}
سنگ‌بنای تعریف رابطه و تابع، مفهوم زوج مرتب و حاصلضرب دکارتی است.
\subsubsection{الف) زوج
مرتب}\label{ux627ux644ux641-ux632ux648ux62c-ux645ux631ux62aux628}
برای هر دو شیء \(a\) و \(b\)، زوج مرتب \((a,b)\) شیئی است که در آن ترتیب
مؤلفه‌ها اهمیت دارد.
\begin{itemize}
\tightlist
\item
  \textbf{ویژگی اصلی:} \((a,b) = (c,d) \iff a=c \wedge b=d\).
\end{itemize}
\subsubsection{ب) تعریف حاصلضرب
دکارتی}\label{ux628-ux62aux639ux631ux6ccux641-ux62dux627ux635ux644ux636ux631ux628-ux62fux6a9ux627ux631ux62aux6cc}
برای دو مجموعه \(A\) و \(B\)، حاصلضرب دکارتی \(A \times B\) مجموعه‌ی تمام
زوج‌های مرتبی است که مؤلفه اول از \(A\) و مؤلفه دوم از \(B\) باشد.
\begin{tldr}{تعریف}
\[A \times B = \{ (a,b) \mid a \in A \wedge b \in B \}\]
\end{tldr}
\subsection{\texorpdfstring{۲. رابطه و تابع
\lr{(Relation and Function)}}{۲. رابطه و تابع }}\label{ux631ux627ux628ux637ux647-ux648-ux62aux627ux628ux639-relation-and-function}
\subsubsection{الف)
رابطه}\label{ux627ux644ux641-ux631ux627ux628ux637ux647}
هر زیرمجموعه از حاصلضرب دکارتی \(A \times B\) یک «رابطه» از \(A\) به
\(B\) نامیده می‌شود. اگر \(R \subseteq A \times B\)، گزاره
\((a,b) \in R\) را اغلب به صورت \(aRb\) می‌نویسیم.
\subsubsection{ب) تابع (به عنوان نوع خاصی از
رابطه)}\label{ux628-ux62aux627ux628ux639-ux628ux647-ux639ux646ux648ux627ux646-ux646ux648ux639-ux62eux627ux635ux6cc-ux627ux632-ux631ux627ux628ux637ux647}
تابع \(f: X \to Y\) رابطه‌ای است که در آن هر عضو دامنه (\(X\)) دقیقاً با
یک عضو از هم‌دامنه (\(Y\)) در ارتباط است.
\begin{tldr}{تعریف صوری}
\[f \subseteq X \times Y \text{ is a function } \iff \forall x \in X, \exists! y \in Y, (x,y) \in f\]
\end{tldr}
\subsection{\texorpdfstring{۳. ترکیب توابع
\lr{(Function Composition)}}{۳. ترکیب توابع }}\label{ux62aux631ux6a9ux6ccux628-ux62aux648ux627ux628ux639-function-composition}
عملیات ترکیب، قلب تپنده جبر توابع است.
\begin{tldr}{تعریف}
فرض کنید \(f: X \to Y\) و \(g: Y \to Z\) باشند. ترکیب \(g\) و \(f\) که
با \(g \circ f\) نشان داده می‌شود، تابعی است از \(X\) به \(Z\) که به صورت
زیر تعریف می‌شود:
\[(g \circ f)(x) = g(f(x)) \quad , \quad \forall x \in X\]
\lr{[cite\_start][cite: }570{]}
\end{tldr}
\textbf{نکته مهم:} در نمادگذاری \(g \circ f\)، ابتدا تابع سمت راست
(\(f\)) و سپس تابع سمت چپ (\(g\)) اثر می‌کند.
\subsection{۴. انواع خاص توابع و تابع
همانی}\label{ux627ux646ux648ux627ux639-ux62eux627ux635-ux62aux648ux627ux628ux639-ux648-ux62aux627ux628ux639-ux647ux645ux627ux646ux6cc}
برای درک قضایای مربوط به وارون‌پذیری (قضیه ۱۶)، تعاریف زیر حیاتی هستند:
\subsubsection{\texorpdfstring{الف) تابع همانی
\lr{(Identity Function)}}{الف) تابع همانی }}\label{ux627ux644ux641-ux62aux627ux628ux639-ux647ux645ux627ux646ux6cc-identity-function}
تابع \(I_X: X \to X\) که هر عضو را به خودش می‌نگارد.
\[I_X(x) = x \quad , \quad \forall x \in X\] این تابع نقش «عنصر خنثی» را
در عمل ترکیب ایفا می‌کند.
\subsubsection{\texorpdfstring{ب) تابع یک‌به‌یک \lr{(Injective }/
\lr{One-to-One)}}{ب) تابع یک‌به‌یک / }}\label{ux628-ux62aux627ux628ux639-ux6ccux6a9ux628ux647ux6ccux6a9-injective-one-to-one}
تابعی که اعضای متمایز دامنه را به اعضای متمایز هم‌دامنه می‌برد.
\[f(x_1) = f(x_2) \implies x_1 = x_2\]
\subsubsection{\texorpdfstring{ج) تابع پوشا \lr{(Surjective }/
\lr{Onto)}}{ج) تابع پوشا / }}\label{ux62c-ux62aux627ux628ux639-ux67eux648ux634ux627-surjective-onto}
تابعی که برد آن برابر با هم‌دامنه باشد (هر عضو \(Y\) تصویر حداقل یک عضو
\(X\) باشد). \[\forall y \in Y, \exists x \in X : f(x) = y\]
\subsection{\texorpdfstring{۵. شبکه ارتباطی با قضایای فصل
\lr{(Analytic Map)}}{۵. شبکه ارتباطی با قضایای فصل }}\label{ux634ux628ux6a9ux647-ux627ux631ux62aux628ux627ux637ux6cc-ux628ux627-ux642ux636ux627ux6ccux627ux6cc-ux641ux635ux644-analytic-map}
\begin{itemize}
\tightlist
\item
  \textbf{حاصلضرب دکارتی \(\leftarrow\)
  \autoref{قضیه-۱---پخش‌پذیری-حاصلضرب-دکارتی}:} درک تعریف \(A \times B\)
  پیش‌شرط اثبات خاصیت پخش‌پذیری آن روی اشتراک و اجتماع است. بدون دانستن
  اینکه شرط عضویت در \(A \times B\) عبارت است از
  \((x \in A \wedge y \in B)\)، نمی‌توان اثبات‌های قضیه ۱ و ۲ را دنبال
  کرد.
\item
  \textbf{ترکیب توابع \(\leftarrow\)
  \autoref{قضیه-۱۵---شرکت‌پذیری-ترکیب-توابع}:} قضیه ۱۵ ثابت می‌کند که
  \((h \circ g) \circ f = h \circ (g \circ f)\). این ویژگی مستقیماً از
  تعریف جبری \(g(f(x))\) ناشی می‌شود.
\item
  \textbf{توابع خاص و همانی \(\leftarrow\)
  \autoref{قضیه-۱۶---وارون‌های-یک‌طرفه}:} قضیه ۱۶ ارتباط عمیقی بین ترکیب
  توابع و یک‌به‌یک/پوشا بودن برقرار می‌کند. مثلاً اگر \(g \circ f = I_X\)
  باشد، \(f\) لزوماً یک‌به‌یک است. درک نقش \(I_X\) در اینجا کلیدی است.
\end{itemize}
