% ---------------------------------------------------------------------
% Copyright (c) 2026 Arsalan Dalvand & Reyhaneh Darvishi.
% Licensed under CC BY-NC-SA 4.0.
% See LICENSE file for details.
% ---------------------------------------------------------------------

\section{قضیه ۱۳: حفظ اشتراک در توابع
یک‌به‌یک}\label{قضیه-۱۳---حفظ-اشتراک-در-توابع-یک‌به‌یک}
\begin{tldr}{خلاصه سریع}
در حالت کلی، تصویر تابع با اشتراک جابجا نمی‌شود
(\(f(A \cap B) \neq f(A) \cap f(B)\)). این قضیه شرط لازم و کافی برای
برقراری این تساوی را بیان می‌کند: تابع باید \textbf{یک‌به‌یک
\lr{(Injective)}} باشد. در این صورت، ساختار اشتراک دقیقاً در هم‌دامنه حفظ
می‌شود.
\end{tldr}
\subsection{۱. متن ریاضی
قضیه}\label{ux645ux62aux646-ux631ux6ccux627ux636ux6cc-ux642ux636ux6ccux647}
فرض کنید \(f: X \to Y\) یک تابع باشد و
\(\{A_\gamma\}_{\gamma \in \Gamma}\) خانواده‌ای از زیرمجموعه‌های دامنه
\(X\) باشد. اگر \(f\) \textbf{یک‌به‌یک} باشد، آنگاه:
\begin{theorembox}{قضیه ۱۳}
\[f\left(\bigcap_{\gamma \in \Gamma} A_\gamma\right) = \bigcap_{\gamma \in \Gamma} f(A_\gamma)\]
\end{theorembox}
\subsection{\texorpdfstring{۲. اثبات صوری
\lr{(Formal Proof)}}{۲. اثبات صوری }}\label{ux627ux62bux628ux627ux62a-ux635ux648ux631ux6cc-formal-proof}
برای اثبات تساوی دو مجموعه، از روش شمول دوطرفه استفاده می‌کنیم.
\subsubsection{\texorpdfstring{بخش اول: شمول مستقیم
(\(\subseteq\))}{بخش اول: شمول مستقیم (\textbackslash subseteq)}}\label{ux628ux62eux634-ux627ux648ux644-ux634ux645ux648ux644-ux645ux633ux62aux642ux6ccux645-subseteq}
این بخش برای \textbf{هر تابعی} (چه یک‌به‌یک باشد چه نباشد) صادق است و در
\textbf{\autoref{قضیه-۱۰---تعمیم-تصویر-اجتماع-و-اشتراک}} اثبات شده است:
\[f\left(\bigcap_{\gamma \in \Gamma} A_\gamma\right) \subseteq \bigcap_{\gamma \in \Gamma} f(A_\gamma)\]
\subsubsection{\texorpdfstring{بخش دوم: شمول معکوس
(\(\supseteq\))}{بخش دوم: شمول معکوس (\textbackslash supseteq)}}\label{ux628ux62eux634-ux62fux648ux645-ux634ux645ux648ux644-ux645ux639ux6a9ux648ux633-supseteq}
این بخش نیازمند فرض \textbf{یک‌به‌یک} بودن تابع است.
\begin{info}{برهان}
۱. فرض کنیم \(y\) عضو دلخواهی از سمت راست تساوی باشد:
\[y \in \bigcap_{\gamma \in \Gamma} f(A_\gamma)\]
۲. طبق تعریف اشتراک تعمیم‌یافته:
\[\forall \gamma \in \Gamma, \quad y \in f(A_\gamma)\]
۳. طبق تعریف تصویر مجموعه، برای هر اندیس \(\gamma\)، باید عنصری در
\(A_\gamma\) وجود داشته باشد که تصویرش \(y\) شود. فرض کنیم برای هر
\(\gamma\)، این عنصر \(x_\gamma\) باشد:
\[\forall \gamma \in \Gamma, \exists x_\gamma \in A_\gamma : f(x_\gamma) = y\]
۴. \textbf{کاربرد فرض یک‌به‌یک:} اکنون مجموعه‌ای از پیش‌نگاره‌ها داریم
\(\{x_\gamma\}_{\gamma \in \Gamma}\) که همگی توسط \(f\) به \(y\) نگاشته
می‌شوند. چون \(f\) تابع است و مقدار \(y\) ثابت است، و مهم‌تر از آن چون
\(f\) \textbf{یک‌به‌یک} است، تمام این پیش‌نگاره‌ها باید یکسان باشند (تابع
یک‌به‌یک نمی‌تواند دو ورودی متمایز را به یک خروجی ببرد).
\[f(x_\alpha) = y \wedge f(x_\beta) = y \implies f(x_\alpha) = f(x_\beta) \xrightarrow{\text{1-1}} x_\alpha = x_\beta\]
بنابراین یک عنصر یکتا (مثلاً \(x_0\)) وجود دارد که:
\[x_0 = x_\gamma, \quad \forall \gamma \in \Gamma\]
۵. چون \(x_0\) همان \(x_\gamma\) است و \(x_\gamma \in A_\gamma\)، پس:
\[\forall \gamma \in \Gamma, \quad x_0 \in A_\gamma\]
۶. طبق تعریف اشتراک، \(x_0\) در اشتراک خانواده است:
\[x_0 \in \bigcap_{\gamma \in \Gamma} A_\gamma\]
۷. چون \(f(x_0) = y\)، پس:
\[y \in f\left(\bigcap_{\gamma \in \Gamma} A_\gamma\right)\]
\textbf{نتیجه:} \(\text{RHS} \subseteq \text{LHS}\).
\end{info}
\subsection{\texorpdfstring{۳. شبکه ارتباطی با سایر قضایا
\lr{(Analytic Map)}}{۳. شبکه ارتباطی با سایر قضایا }}\label{ux634ux628ux6a9ux647-ux627ux631ux62aux628ux627ux637ux6cc-ux628ux627-ux633ux627ux6ccux631-ux642ux636ux627ux6ccux627-analytic-map}
\subsubsection{\texorpdfstring{۱. تکمیل
\autoref{قضیه-۱۰---تعمیم-تصویر-اجتماع-و-اشتراک}}{۱. تکمیل }}\label{ux62aux6a9ux645ux6ccux644-ux642ux636ux6ccux647-ux6f1ux6f0---ux62aux639ux645ux6ccux645-ux62aux635ux648ux6ccux631-ux627ux62cux62aux645ux627ux639-ux648-ux627ux634ux62aux631ux627ux6a9}
\begin{itemize}
\tightlist
\item
  \textbf{رفع نقص:} قضیه ۱۰ بیان می‌کرد که
  \(f(\cap A) \subseteq \cap f(A)\). قضیه ۱۳ متمم آن است و نشان می‌دهد که
  ``یک‌به‌یک بودن'' شرط کافی برای تبدیل این زیرمجموعه به تساوی است. در
  واقع، ناتوانی تابع معمولی در حفظ اشتراک، ناشی از ``تلاقی''
  \lr{(Collision) }ورودی‌های متمایز در یک خروجی مشترک است که در توابع
  یک‌به‌یک رخ نمی‌دهد.
\end{itemize}
\subsubsection{\texorpdfstring{۲. ارتباط با
\autoref{انواع-توابع---یک‌به‌یک-پوشا-و-دوسویی}}{۲. ارتباط با }}\label{ux627ux631ux62aux628ux627ux637-ux628ux627-ux627ux646ux648ux627ux639-ux62aux648ux627ux628ux639---ux6ccux6a9ux628ux647ux6ccux6a9-ux67eux648ux634ux627-ux648-ux62fux648ux633ux648ux6ccux6cc}
\begin{itemize}
\tightlist
\item
  \textbf{تعریف کاربردی:} اثبات این قضیه یکی از مهم‌ترین کاربردهای تعریف
  صوری یک‌به‌یک بودن (\(f(x_1)=f(x_2) \implies x_1=x_2\)) در نظریه
  مجموعه‌هاست. در گام ۴ اثبات، دقیقاً از همین استلزام منطقی برای یکی کردن
  تمام \(x_\gamma\)ها استفاده شد.
\end{itemize}
\subsubsection{\texorpdfstring{۳. ارتباط با
\autoref{قضیه-۱۱---رفتار-جبری-تصویر-وارون}}{۳. ارتباط با }}\label{ux627ux631ux62aux628ux627ux637-ux628ux627-ux642ux636ux6ccux647-ux6f1ux6f1---ux631ux641ux62aux627ux631-ux62cux628ux631ux6cc-ux62aux635ux648ux6ccux631-ux648ux627ux631ux648ux646}
\begin{itemize}
\tightlist
\item
  \textbf{تقارن جبری:} تصویر وارون (\(f^{-1}\)) همیشه اشتراک را حفظ
  می‌کند (\(f^{-1}(\cap B) = \cap f^{-1}(B)\)). قضیه ۱۳ نشان می‌دهد که اگر
  \(f\) یک‌به‌یک باشد، تصویر مستقیم (\(f\)) نیز رفتاری مشابه تصویر وارون
  پیدا می‌کند و ``هم‌ریختی'' \lr{(Homomorphism) }کامل نسبت به عملیات
  مجموعه‌ای برقرار می‌شود.
\end{itemize}
