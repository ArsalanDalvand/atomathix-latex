% ---------------------------------------------------------------------
% Copyright (c) 2026 Arsalan Dalvand & Reyhaneh Darvishi.
% Licensed under CC BY-NC-SA 4.0.
% See LICENSE file for details.
% ---------------------------------------------------------------------

\section{قضیه: تصویرِ اشتراک یک مجموعه با وارون یک مجموعه
دیگر}\label{تمرین-تکمیلی---تساوی-تصویر-اشتراک-A-و-وارون-B}
\begin{tldr}{خلاصه سریع}
این فرمول نشان می‌دهد که عملگر تصویر \(f\) نسبت به اشتراک با یک «نگاره
وارون» چگونه رفتار می‌کند. می‌توان \(B\) را از داخل پرانتزِ تصویر بیرون
کشید. \[f(A \cap f^{-1}(B)) = f(A) \cap B\]
\end{tldr}
\subsection{۱. صورت
ریاضی}\label{ux635ux648ux631ux62a-ux631ux6ccux627ux636ux6cc}
\begin{theorembox}{قضیه}
فرض کنید \(f:X \rightarrow Y\)، \(A \subseteq X\) و \(B \subseteq Y\).
داریم: \[f(A \cap f^{-1}(B)) = f(A) \cap B\]
\end{theorembox}
\subsection{۲. اثبات
دوطرفه}\label{ux627ux62bux628ux627ux62a-ux62fux648ux637ux631ux641ux647}
\begin{info}{اثبات}
از خواص منطقی سورها و تعاریف استفاده می‌کنیم تا تساوی را مستقیم نشان
دهیم.
\[y \in f(A \cap f^{-1}(B))\]
\(\equiv\) طبق تعریف تصویر، یعنی \(x\)ای وجود دارد که در مجموعه
\(A \cap f^{-1}(B)\) است و تصویرش \(y\) است:
\[\exists x \left[ x \in (A \cap f^{-1}(B)) \land f(x) = y \right]\]
\(\equiv\) تعریف اشتراک را باز می‌کنیم:
\[\exists x \left[ (x \in A \land x \in f^{-1}(B)) \land f(x) = y \right]\]
\(\equiv\) تعریف وارون (\(x \in f^{-1}(B) \iff f(x) \in B\)) را اعمال
می‌کنیم:
\[\exists x \left[ x \in A \land f(x) \in B \land f(x) = y \right]\]
\(\equiv\) در اینجا نکته کلیدی این است که \(f(x)=y\). پس شرط
\(f(x) \in B\) معادل است با \(y \in B\). چون \(y\) به \(x\) وابسته نیست
(در این بخش از گزاره)، می‌توانیم آن را از زیر سور وجودی بیرون بیاوریم یا
جدا کنیم:
\[\left( \exists x [x \in A \land f(x)=y] \right) \land y \in B\]
\(\equiv\) قسمت داخل پرانتز دقیقاً تعریف \(y \in f(A)\) است:
\[y \in f(A) \land y \in B\]
\(\equiv\) تعریف اشتراک: \[y \in f(A) \cap B\]
\(\blacksquare\)
\end{info}
\subsection{۳. حالت خاص
(نتیجه)}\label{ux62dux627ux644ux62a-ux62eux627ux635-ux646ux62aux6ccux62cux647}
اگر در فرمول بالا به جای \(A\)، کل مجموعه مرجع \(X\) را قرار دهیم:
\[f(X \cap f^{-1}(B)) = f(X) \cap B\] چون \(f^{-1}(B) \subseteq X\) است،
پس اشتراکشان خودِ \(f^{-1}(B)\) می‌شود: \[f(f^{-1}(B)) = f(X) \cap B\]
\emph{(این فرمول نشان می‌دهد تصویرِ وارونِ \lr{B }دقیقاً برابر است با اشتراکِ
برد تابع با \lr{B).}}
