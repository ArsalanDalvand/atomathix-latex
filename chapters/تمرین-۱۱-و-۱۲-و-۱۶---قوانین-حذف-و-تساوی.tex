% ---------------------------------------------------------------------
% Copyright (c) 2026 Arsalan Dalvand & Reyhaneh Darvishi.
% Licensed under CC BY-NC-SA 4.0.
% See LICENSE file for details.
% ---------------------------------------------------------------------

\section{تمرین ۱۱، ۱۲ و ۱۶: قوانین حذف در تساوی
حاصلضرب‌ها}\label{تمرین-۱۱-و-۱۲-و-۱۶---قوانین-حذف-و-تساوی}
\subsection{۱. تمرین ۱۱}\label{ux62aux645ux631ux6ccux646-ux6f1ux6f1}
\textbf{سوال:} ثابت کنید اگر \(A \times A = B \times B\) آنگاه
\(A = B\). \textbf{اثبات:}
\[(x,y) \in A \times A \iff x \in A \wedge y \in A\]
\[(x,y) \in B \times B \iff x \in B \wedge y \in B\] چون دو طرف برابرند،
اگر عضوی مثل \(z\) در \(A\) باشد، زوج \((z,z)\) در \(A \times A\) است،
پس در \(B \times B\) هم هست، پس \(z \in B\). و برعکس. پس \(A=B\) .
\begin{center}\rule{0.5\linewidth}{0.5pt}\end{center}
\subsection{۲. تمرین ۱۲}\label{ux62aux645ux631ux6ccux646-ux6f1ux6f2}
\textbf{سوال:} ثابت کنید اگر \(A \times C = B \times C\) و
\(C \neq \emptyset\)، آنگاه \(A = B\). \textbf{اثبات:} چون
\(C \neq \emptyset\)، پس یک عضو \(y \in C\) وجود دارد. فرض کنیم
\(x \in A\). آنگاه \((x,y) \in A \times C\). طبق فرض تساوی،
\((x,y) \in B \times C\). پس \(x \in B\) (و \(y \in C\)). بنابراین
\(A \subseteq B\). به طور مشابه \(B \subseteq A\). پس \(A=B\).
\begin{center}\rule{0.5\linewidth}{0.5pt}\end{center}
\subsection{۳. تمرین ۱۶}\label{ux62aux645ux631ux6ccux646-ux6f1ux6f6}
\textbf{سوال:} فرض کنیم \(A, B, C, D\) ناتهی باشند. ثابت کنید
\(A \times B = C \times D \iff A=C \land B=D\).
\textbf{اثبات:} اگر \(A=C\) و \(B=D\) باشد، تساوی بدیهی است. برعکس، فرض
کنیم \(A \times B = C \times D\). برای هر \(a \in A\) و هر \(b \in B\)
(چون ناتهی هستند)، زوج \((a,b) \in A \times B\) است. پس
\((a,b) \in C \times D\). نتیجه می‌دهد \(a \in C\) (پس \(A \subseteq C\))
و \(b \in D\) (پس \(B \subseteq D\)). به همین ترتیب عکس آن ثابت می‌شود.
پس \(A=C\) و \(B=D\).
