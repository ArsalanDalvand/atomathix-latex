% ---------------------------------------------------------------------
% Copyright (c) 2026 Arsalan Dalvand & Reyhaneh Darvishi.
% Licensed under CC BY-NC-SA 4.0.
% See LICENSE file for details.
% ---------------------------------------------------------------------

\section{تمرین ۲۲: یگانگی وارون (چپ و
راست)}\label{تمرین-۲۲---یگانگی-تابع-وارون}
\begin{tldr}{خلاصه سریع}
اگر تابعی هم «وارون چپ» (\(g\)) داشته باشد و هم «وارون راست» (\(h\))،
آنگاه تابع اصلی حتماً دوسویی است و آن دو وارون با هم برابرند
(\(g=h=f^{-1}\)). \emph{این قضیه ثابت می‌کند که یک تابع نمی‌تواند دو وارون
متفاوت داشته باشد.}
\end{tldr}
\subsection{۱. صورت
تمرین}\label{ux635ux648ux631ux62a-ux62aux645ux631ux6ccux646}
\begin{info}{سوال}
فرض کنید \(f: X \rightarrow Y\). اگر توابع \(g: Y \rightarrow X\) و
\(h: Y \rightarrow X\) طوری باشند که:
\begin{enumerate}
\def\labelenumi{\arabic{enumi}.}
\tightlist
\item
  \(g \circ f = I_X\) (وارون چپ)
\item
  \(f \circ h = I_Y\) (وارون راست)
\end{enumerate}
ثابت کنید \(f\) دوسویی است و \(g = h = f^{-1}\).
\end{info}
\subsection{۲. اثبات (بسیار زیبا و
جبری)}\label{ux627ux62bux628ux627ux62a-ux628ux633ux6ccux627ux631-ux632ux6ccux628ux627-ux648-ux62cux628ux631ux6cc}
\begin{info}{اثبات برابری \(g\) و \(h\)}
از خاصیت شرکت‌پذیری و خاصیت همانی استفاده می‌کنیم.
\[g = g \circ I_Y\] (چون \(I_Y\) خنثی است)
\[= g \circ (f \circ h)\] (جایگذاری \(I_Y\) با فرض \(f \circ h\))
\[= (g \circ f) \circ h\] (استفاده از خاصیت شرکت‌پذیری - پرانتز را جابجا
کردیم)
\[= I_X \circ h\] (جایگذاری \(g \circ f\) با فرض مسئله)
\[= h\] (چون \(I_X\) خنثی است)
\textbf{نتیجه:} \(g = h\). حال چون این تابع هم چپ و هم راست وارون \(f\)
است، پس \(f\) وارون‌پذیر (دوسویی) است و \(g=h=f^{-1}\).
\end{info}
\subsection{۳. اثبات یک‌به‌یک و پوشا بودن (روش
دوم)}\label{ux627ux62bux628ux627ux62a-ux6ccux6a9ux628ux647ux6ccux6a9-ux648-ux67eux648ux634ux627-ux628ux648ux62fux646-ux631ux648ux634-ux62fux648ux645}
\begin{itemize}
\tightlist
\item
  \textbf{یک‌به‌یک:} اگر \(f(x_1)=f(x_2)\)، با اعمال \(g\) به طرفین داریم
  \(g(f(x_1))=g(f(x_2)) \Rightarrow I(x_1)=I(x_2) \Rightarrow x_1=x_2\).
\item
  \textbf{پوشا:} برای هر \(y\)، اگر قرار دهیم \(x=h(y)\)، آنگاه
  \(f(x)=f(h(y))=I_Y(y)=y\). پس برای هر \(y\)، یک \(x\) وجود دارد.
\end{itemize}
