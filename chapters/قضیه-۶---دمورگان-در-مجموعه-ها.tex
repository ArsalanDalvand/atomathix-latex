% ---------------------------------------------------------------------
% Copyright (c) 2026 Arsalan Dalvand & Reyhaneh Darvishi.
% Licensed under CC BY-NC-SA 4.0.
% See LICENSE file for details.
% ---------------------------------------------------------------------

\section{\texorpdfstring{قضیه ۶: قوانین دمورگان در نظریه مجموعه‌ها
\lr{(De Morgan’s Laws)}}{قضیه ۶: قوانین دمورگان در نظریه مجموعه‌ها }}\label{قضیه-۶---دمورگان-در-مجموعه-ها}
\begin{tldr}{خلاصه سریع}
این قضیه بیانگر اصل «دوگانگی» \lr{(Duality) }بین عملگرهای اجتماع و
اشتراک تحت تأثیر عملگر متمم است. طبق این قوانین، متممِ اجتماع برابر با
اشتراک متمم‌هاست و بالعکس. این روابط، ایزومورفیسم کامل بین جبر مجموعه‌ها و
منطق گزاره‌ها را نشان می‌دهند.
\end{tldr}
\subsection{۱. متن ریاضی
قضیه}\label{ux645ux62aux646-ux631ux6ccux627ux636ux6cc-ux642ux636ux6ccux647}
فرض کنید \(U\) مجموعه مرجع باشد و \(A, B \subseteq U\). هم‌ارزی‌های
مجموعه‌ای زیر برقرارند:
\begin{theorembox}{قضیه ۶}
\textbf{الف) متمم اجتماع:} \[(A \cup B)' = A' \cap B'\] \textbf{ب) متمم
اشتراک:} \[(A \cap B)' = A' \cup B'\]
\end{theorembox}
\subsection{\texorpdfstring{۲. اثبات صوری
\lr{(Formal Proof)}}{۲. اثبات صوری }}\label{ux627ux62bux628ux627ux62a-ux635ux648ux631ux6cc-formal-proof}
اثبات این قضیه مبتنی بر ترجمه گزاره‌های عضویت به منطق صوری و استفاده از
\textbf{قوانین دمورگان در منطق} است.
\subsubsection{اثبات قسمت
(الف)}\label{ux627ux62bux628ux627ux62a-ux642ux633ux645ux62a-ux627ux644ux641}
نشان می‌دهیم تابع گزاره‌نمای عضویت برای دو طرف تساوی هم‌ارز است:
\begin{info}{اثبات}
\[x \in (A \cup B)'\] ۱. طبق تعریف متمم: \[\iff x \notin (A \cup B)\]
\[\iff \sim (x \in A \cup B)\] ۲. طبق تعریف اجتماع:
\[\iff \sim (x \in A \lor x \in B)\] ۳. طبق
\textbf{\autoref{قضیه-۳---قوانین-دمورگان}} (دمورگان منطقی
\(\sim(p \lor q) \equiv \sim p \land \sim q\)):
\[\iff \sim(x \in A) \land \sim(x \in B)\] ۴. طبق تعریف متمم:
\[\iff (x \in A') \land (x \in B')\] ۵. طبق تعریف اشتراک:
\[\iff x \in (A' \cap B')\]
نتیجه: \((A \cup B)' = A' \cap B'\).
\end{info}
\subsubsection{اثبات قسمت
(ب)}\label{ux627ux62bux628ux627ux62a-ux642ux633ux645ux62a-ux628}
روند مشابهی طی می‌شود با این تفاوت که از قانون منطقی
\(\sim(p \land q) \equiv \sim p \lor \sim q\) استفاده می‌کنیم.
\begin{info}{اثبات}
\[x \in (A \cap B)' \iff \sim(x \in A \land x \in B)\]
\[\iff \sim(x \in A) \lor \sim(x \in B)\]
\[\iff x \in A' \lor x \in B'\] \[\iff x \in A' \cup B'\]
\end{info}
\subsection{\texorpdfstring{۳. شبکه ارتباطی با سایر قضایا
\lr{(Analytic Map)}}{۳. شبکه ارتباطی با سایر قضایا }}\label{ux634ux628ux6a9ux647-ux627ux631ux62aux628ux627ux637ux6cc-ux628ux627-ux633ux627ux6ccux631-ux642ux636ux627ux6ccux627-analytic-map}
این قضیه پل ارتباطی حیاتی برای انتقال ویژگی‌های منطق کلاسیک به ساختارهای
جبری و توپولوژیک است:
\subsubsection{\texorpdfstring{۱. ارتباط با
\autoref{قضیه-۳---قوانین-دمورگان} (ریشه
منطقی)}{۱. ارتباط با  (ریشه منطقی)}}\label{ux627ux631ux62aux628ux627ux637-ux628ux627-ux642ux636ux6ccux647-ux6f3-ux641ux635ux644-ux6f1-ux631ux6ccux634ux647-ux645ux646ux637ux642ux6cc}
\begin{itemize}
\tightlist
\item
  \textbf{ایزومورفیسم ساختاری:} قضیه ۶ تصویر دقیق قضیه ۳ در جبر
  مجموعه‌هاست. تناظر زیر به طور کامل برقرار است:
  \begin{itemize}
  \tightlist
  \item
    اجتماع (\(\cup\)) \(\leftrightarrow\) فصل (\(\lor\))
  \item
    اشتراک (\(\cap\)) \(\leftrightarrow\) عطف (\(\land\))
  \item
    متمم (\('\)) \(\leftrightarrow\) نقیض (\(\sim\))
  \end{itemize}
\end{itemize}
\subsubsection{\texorpdfstring{۲. ارتباط با
\autoref{قضیه-۵---متمم-و-زیرمجموعه}}{۲. ارتباط با }}\label{ux627ux631ux62aux628ux627ux637-ux628ux627-ux642ux636ux6ccux647-ux6f5---ux645ux62aux645ux645-ux648-ux632ux6ccux631ux645ux62cux645ux648ux639ux647}
\begin{itemize}
\tightlist
\item
  \textbf{پایه جبری:} اثبات قضیه ۶ به شدت به تعریف دقیق متمم و نقیض
  گزاره‌ها که در قضیه ۵ بررسی شد، وابسته است. همچنین، ترکیب قضیه ۶ با
  قضیه ۵ (قانون عکس نقیض \(A \subseteq B \iff B' \subseteq A'\))
  ابزارهای قدرتمندی برای ساده‌سازی عبارات پیچیده مجموعه‌ای فراهم می‌کند.
\end{itemize}
\subsubsection{\texorpdfstring{۳. ارتباط با
\autoref{قضیه-۸---تعمیم-دمورگان}
(تعمیم)}{۳. ارتباط با  (تعمیم)}}\label{ux627ux631ux62aux628ux627ux637-ux628ux627-ux642ux636ux6ccux647-ux6f8---ux62aux639ux645ux6ccux645-ux62fux645ux648ux631ux6afux627ux646-ux62aux639ux645ux6ccux645}
\begin{itemize}
\tightlist
\item
  \textbf{حالت خاص:} قضیه ۶ حالت محدود \lr{(Finite Case) }برای دو مجموعه
  است. \textbf{\autoref{قضیه-۸---تعمیم-دمورگان}} این مفهوم را برای
  خانواده‌های نامتناهی از مجموعه‌ها گسترش می‌دهد:
  \((\bigcup A_i)' = \bigcap A_i'\). در آنجا، نقش عملگرهای \(\lor\) و
  \(\land\) به سورهای \(\exists\) و \(\forall\) تبدیل می‌شود.
\end{itemize}
