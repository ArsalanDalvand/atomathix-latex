% ---------------------------------------------------------------------
% Copyright (c) 2026 Arsalan Dalvand & Reyhaneh Darvishi.
% Licensed under CC BY-NC-SA 4.0.
% See LICENSE file for details.
% ---------------------------------------------------------------------

\section{قضیه: رفتار نگاره وارون نسبت به متمم
(تفاضل)}\label{تمرین-۱۰---تساوی-وارون-متمم-B-با-متمم-وارون-B}
\begin{tldr}{خلاصه سریع}
نگاره وارون (برخلافِ خودِ نگاره) رفتار بسیار خوش‌رفتاری دارد و عملگرهای
مجموعه (اجتماع، اشتراک و تفاضل) را حفظ می‌کند. این قضیه برای تفاضل (متمم)
است. \[f^{-1}(Y - B) = X - f^{-1}(B)\]
\end{tldr}
\subsection{۱. صورت ریاضی (تمرین
۱۰)}\label{ux635ux648ux631ux62a-ux631ux6ccux627ux636ux6cc-ux62aux645ux631ux6ccux646-ux6f1ux6f0}
\begin{theorembox}{قضیه}
فرض کنید \(f:X \rightarrow Y\) و \(B \subseteq Y\). ثابت کنید وارونِ متممِ
\lr{B }برابر است با متممِ وارونِ
\lr{B: }\[f^{-1}(Y - B) = f^{-1}(Y) - f^{-1}(B)\] \emph{(نکته: می‌دانیم
\(f^{-1}(Y) = X\) است)}
\end{theorembox}
\subsection{۲. اثبات
جبری}\label{ux627ux62bux628ux627ux62a-ux62cux628ux631ux6cc}
\begin{info}{اثبات}
نشان می‌دهیم یک عضو دلخواه \(x\) اگر در سمت چپ باشد، در سمت راست هم هست و
بالعکس.
\[x \in f^{-1}(Y - B)\]
\(\equiv\) طبق تعریف وارون: \[f(x) \in (Y - B)\]
\(\equiv\) طبق تعریف تفاضل مجموعه‌ها: \[f(x) \in Y \land f(x) \notin B\]
\(\equiv\) گزاره \(f(x) \in Y\) برای هر \(x \in X\) همواره صادق است (چون
\(Y\) هم‌دامنه است). پس می‌توانیم آن را به زبان وارون بنویسیم
(\(x \in f^{-1}(Y)\)). همچنین \(f(x) \notin B\) معادل است با
\(x \notin f^{-1}(B)\): \[x \in f^{-1}(Y) \land x \notin f^{-1}(B)\]
\(\equiv\) طبق تعریف تفاضل: \[x \in f^{-1}(Y) - f^{-1}(B)\]
\(\blacksquare\)
\end{info}
\begin{note}{یادآوری مهم}
چرا \(f^{-1}(Y) = X\)؟ چون دامنه تابع \(f\) برابر با \(X\) است، پس هر
\(x \in X\) قطعاً تصویری در \(Y\) دارد (\(f(x) \in Y\)). بنابراین تمام
\(x\)ها در تعریف وارونِ \(Y\) صدق می‌کنند.
\end{note}
