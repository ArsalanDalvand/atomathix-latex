% ---------------------------------------------------------------------
% Copyright (c) 2026 Arsalan Dalvand & Reyhaneh Darvishi.
% Licensed under CC BY-NC-SA 4.0.
% See LICENSE file for details.
% ---------------------------------------------------------------------

\section{قضیه ۵: تولید رابطه هم‌ارزی توسط
افراز}\label{قضیه-۵---رابطه-هم‌ارزی-ناشی-از-افراز}
\begin{tldr}{خلاصه سریع}
این قضیه عکس قضیه ۴ است: اگر شما مجموعه‌ای را تکه‌تکه (افراز) کنید، به طور
خودکار یک رابطه هم‌ارزی ساخته‌اید. تعریف رابطه این است: «دو نفر با هم
رابطه دارند، اگر و تنها اگر در یک تکه باشند».
\end{tldr}
\subsection{۱. تعریف رابطه ناشی از
افراز}\label{ux62aux639ux631ux6ccux641-ux631ux627ux628ux637ux647-ux646ux627ux634ux6cc-ux627ux632-ux627ux641ux631ux627ux632}
فرض کنید \(\mathcal{P}\) یک افراز برای مجموعه \(X\) باشد. رابطه هم‌ارزی
متناظر با آن (که با \(X/\mathcal{P}\) نشان داده می‌شود) چنین تعریف می‌شود:
\[x (X/\mathcal{P}) y \iff \exists A \in \mathcal{P} : (x \in A \wedge y \in A)\]
\emph{(ترجمه: \(x\) و \(y\) با هم رابطه دارند اگر هم‌گروهی باشند).}
\begin{center}\rule{0.5\linewidth}{0.5pt}\end{center}
\subsection{۲. متن ریاضی
قضیه}\label{ux645ux62aux646-ux631ux6ccux627ux636ux6cc-ux642ux636ux6ccux647}
\begin{theorembox}{قضیه ۵}
اگر \(\mathcal{P}\) یک افراز برای مجموعه ناتهی \(X\) باشد، آنگاه:
\textbf{الف)} رابطه \(X/\mathcal{P}\) یک \textbf{رابطه هم‌ارزی} روی \(X\)
است. \textbf{ب)} کلاس‌های هم‌ارزی حاصل از این رابطه، دقیقاً همان مجموعه‌های
افراز هستند (\(X/(X/\mathcal{P}) = \mathcal{P}\)).
\end{theorembox}
\begin{center}\rule{0.5\linewidth}{0.5pt}\end{center}
\subsection{\texorpdfstring{۳. اثبات صوری
\lr{(Formal Proof)}}{۳. اثبات صوری }}\label{ux627ux62bux628ux627ux62a-ux635ux648ux631ux6cc-formal-proof}
\subsubsection{اثبات قسمت (الف): هم‌ارزی
بودن}\label{ux627ux62bux628ux627ux62a-ux642ux633ux645ux62a-ux627ux644ux641-ux647ux645ux627ux631ux632ux6cc-ux628ux648ux62fux646}
باید سه ویژگی بازتابی، تقارنی و تعدی را برای رابطه تعریف شده چک کنیم.
\begin{info}{برهان}
\textbf{۱. بازتابی \lr{(Reflexive):}} هر \(x \in X\)، طبق تعریف افراز
(پوشش کامل)، حتماً متعلق به یکی از مجموعه‌های افراز (مثلاً \(A\)) است. پس
\(x, x \in A\) و در نتیجه \(x \sim x\).
\textbf{۲. تقارنی \lr{(Symmetric):}} اگر \(x \sim y\)، یعنی مجموعه‌ای مثل
\(A\) هست که \(x, y \in A\). واضح است که \(y, x \in A\) نیز برقرار است.
پس \(y \sim x\).
\textbf{۳. تعدی \lr{(Transitive):}} فرض کنیم \(x \sim y\) و
\(y \sim z\).
\begin{itemize}
\tightlist
\item
  \(x \sim y \implies \exists A \in \mathcal{P}, (x, y \in A)\)
\item
  \(y \sim z \implies \exists B \in \mathcal{P}, (y, z \in B)\)
\item
  اکنون \(y\) هم در \(A\) است و هم در \(B\). پس \(y \in A \cap B\).
\item
  یعنی اشتراک \(A\) و \(B\) تهی نیست (\(A \cap B \neq \emptyset\)).
\item
  طبق تعریف افراز (شرط مجزا بودن)، اگر دو مجموعه اشتراک داشته باشند،
  باید \textbf{یکی} باشند. پس \(A = B\).
\item
  نتیجه: \(x, z\) هر دو در \(A\) هستند، پس \(x \sim z\).
\end{itemize}
\end{info}
\subsubsection{اثبات قسمت (ب): بازگشت به افراز
اولیه}\label{ux627ux62bux628ux627ux62a-ux642ux633ux645ux62a-ux628-ux628ux627ux632ux6afux634ux62a-ux628ux647-ux627ux641ux631ux627ux632-ux627ux648ux644ux6ccux647}
\begin{info}{برهان}
می‌خواهیم نشان دهیم کلاس‌های هم‌ارزی تولید شده، همان مجموعه‌های
\(A \in \mathcal{P}\) هستند. فرض کنیم \(x \in X\) باشد و \(A\) آن
مجموعه‌ای از افراز باشد که شامل \(x\) است (\(x \in A\)). طبق تعریف رابطه،
تمام هم‌گروهی‌های \(x\) (یعنی اعضای کلاس \([x]\)) دقیقاً همان اعضای \(A\)
هستند. پس \([x] = A\). بنابراین مجموعه تمام کلاس‌ها، همان مجموعه
\(\mathcal{P}\) است.
\end{info}
\begin{center}\rule{0.5\linewidth}{0.5pt}\end{center}
\subsection{\texorpdfstring{۴. شبکه ارتباطی با سایر قضایا
\lr{(Analytic Map)}}{۴. شبکه ارتباطی با سایر قضایا }}\label{ux634ux628ux6a9ux647-ux627ux631ux62aux628ux627ux637ux6cc-ux628ux627-ux633ux627ux6ccux631-ux642ux636ux627ux6ccux627-analytic-map}
\subsubsection{\texorpdfstring{۱. چرخه کامل با
\autoref{قضیه-۴---افراز-ناشی-از-رابطه-هم‌ارزی}}{۱. چرخه کامل با }}\label{ux686ux631ux62eux647-ux6a9ux627ux645ux644-ux628ux627-ux642ux636ux6ccux647-ux6f4---ux627ux641ux631ux627ux632-ux646ux627ux634ux6cc-ux627ux632-ux631ux627ux628ux637ux647-ux647ux645ux627ux631ux632ux6cc}
\begin{itemize}
\tightlist
\item
  ترکیب قضیه ۴ و ۵ یک چرخه بسته \lr{(Bijection) }بین ``مجموعه تمام
  رابطه‌های هم‌ارزی روی \(X\)'' و ``مجموعه تمام افرازهای \(X\)'' ایجاد
  می‌کند.
  \begin{itemize}
  \tightlist
  \item
    از رابطه به افراز: \(\xi \to X/\xi\)
  \item
    از افراز به رابطه: \(\mathcal{P} \to X/\mathcal{P}\)
  \item
    قضیه ۵-ب تضمین می‌کند که اگر یک دور کامل بزنیم، به جای اول برمی‌گردیم:
    \(X/(X/\mathcal{P}) = \mathcal{P}\).
  \end{itemize}
\end{itemize}
\subsubsection{\texorpdfstring{۲. مثال کاربردی:
\autoref{تمرین-۵---هم‌نهشتی-اعداد-صحیح}}{۲. مثال کاربردی: }}\label{ux645ux62bux627ux644-ux6a9ux627ux631ux628ux631ux62fux6cc-ux62aux645ux631ux6ccux646-ux6f5---ux647ux645ux646ux647ux634ux62aux6cc-ux627ux639ux62fux627ux62f-ux635ux62dux6ccux62d}
\begin{itemize}
\tightlist
\item
  در تمرین ۵ فصل ۳، دیدیم که رابطه همنهشتی به پیمانه \(n\)، مجموعه اعداد
  صحیح را به \(n\) کلاس افراز می‌کند. این مثال عملی دقیقاً مصداق همین دو
  قضیه است. دسته‌بندی اعداد (افراز) \(\iff\) رابطه همنهشتی.
\end{itemize}
