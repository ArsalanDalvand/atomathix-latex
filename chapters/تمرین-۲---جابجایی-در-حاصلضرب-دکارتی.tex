% ---------------------------------------------------------------------
% Copyright (c) 2026 Arsalan Dalvand & Reyhaneh Darvishi.
% Licensed under CC BY-NC-SA 4.0.
% See LICENSE file for details.
% ---------------------------------------------------------------------

\section{تمرین ۲: شرط جابجایی حاصلضرب
دکارتی}\label{تمرین-۲---جابجایی-در-حاصلضرب-دکارتی}
\subsection{۱. صورت
سوال}\label{ux635ux648ux631ux62a-ux633ux648ux627ux644}
مجموعه‌های \(A\) و \(B\) چه شرایطی دارا باشند تا تساوی
\(A \times B = B \times A\) راست باشد؟
\subsection{۲. استراتژی و
حل}\label{ux627ux633ux62aux631ux627ux62aux698ux6cc-ux648-ux62dux644}
در حالت کلی \(A \times B \neq B \times A\) است (چون زوج مرتب \((a,b)\)
با \((b,a)\) فرق دارد). این تساوی فقط در شرایط خاص برقرار است.
\begin{info}{تحلیل حالات}
تساوی برقرار است اگر و تنها اگر:
\begin{enumerate}
\def\labelenumi{\arabic{enumi}.}
\tightlist
\item
  \textbf{یکی از مجموعه‌ها تهی باشد:} اگر \(A = \emptyset\) یا
  \(B = \emptyset\) باشد، حاصلضرب دکارتی هر دو طرف \(\emptyset\) می‌شود و
  تساوی برقرار است.
\item
  \textbf{مجموعه‌ها برابر باشند:} اگر \(A = B\) باشد، بدیهی است که
  \(A \times A = A \times A\).
\end{enumerate}
\textbf{پاسخ نهایی:} \[A = \emptyset \lor B = \emptyset \lor A = B\]
\end{info}
