% ---------------------------------------------------------------------
% Copyright (c) 2026 Arsalan Dalvand & Reyhaneh Darvishi.
% Licensed under CC BY-NC-SA 4.0.
% See LICENSE file for details.
% ---------------------------------------------------------------------

\section{تمرین ۱۴: زیرمجموعه یک
تابع}\label{تمرین-۱۴---زیرمجموعه-تابع}
\subsection{۱. صورت
سوال}\label{ux635ux648ux631ux62a-ux633ux648ux627ux644}
\begin{info}{تابع \(f: X \to Y\) مفروض است. ثابت کنید که هر زیرمجموعه از \(f\) (مانند \(g \subseteq f\)) نیز یک تابع است.}
\end{info}
\subsection{۲. استراتژی
حل}\label{ux627ux633ux62aux631ux627ux62aux698ux6cc-ux62dux644}
یک تابع، مجموعه‌ای از زوج‌های مرتب است. هر زیرمجموعه‌ای از آن، یک «رابطه»
است. برای اینکه این رابطه جدید (\(g\)) تابع باشد، باید شرط
\textbf{یکتایی} را حفظ کند. دامنه تابع جدید (\(D_g\)) زیرمجموعه‌ای از
دامنه اصلی (\(X\)) خواهد بود.
\subsection{۳. حل
تشریحی}\label{ux62dux644-ux62aux634ux631ux6ccux62dux6cc}
\begin{info}{اثبات}
فرض کنیم \(g \subseteq f\) باشد. دامنه \(g\) را \(D_g\) می‌نامیم. برای
اثبات تابع بودن \(g\)، باید نشان دهیم برای هر \(x \in D_g\)، تصویر یگانه
است.
۱. فرض کنیم \((x, y_1) \in g\) و \((x, y_2) \in g\). ۲. چون
\(g \subseteq f\) است، پس تمام اعضای \(g\) در \(f\) نیز هستند:
\[(x, y_1) \in f \quad \text{و} \quad (x, y_2) \in f\] ۳. اما می‌دانیم که
\(f\) یک تابع است. طبق تعریف تابع، یک عنصر نمی‌تواند به دو عنصر متمایز
نگاشته شود. ۴. بنابراین، الزاماً \(y_1 = y_2\).
\textbf{نتیجه:} رابطه \(g\) شرط تک‌مقداری بودن را دارد، پس \(g\) یک تابع
از \(D_g\) به \(Y\) است.
\end{info}
