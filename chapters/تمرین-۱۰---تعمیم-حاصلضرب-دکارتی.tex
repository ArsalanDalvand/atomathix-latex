% ---------------------------------------------------------------------
% Copyright (c) 2026 Arsalan Dalvand & Reyhaneh Darvishi.
% Licensed under CC BY-NC-SA 4.0.
% See LICENSE file for details.
% ---------------------------------------------------------------------

\section{\texorpdfstring{تمرین ۱۰: حاصلضرب دکارتی
\lr{nتایی}}{تمرین ۱۰: حاصلضرب دکارتی }}\label{تمرین-۱۰---تعمیم-حاصلضرب-دکارتی}
\subsection{۱. صورت
سوال}\label{ux635ux648ux631ux62a-ux633ux648ux627ux644}
آیا می‌توانید تعریف حاصلضرب دکارتی را برای سه مجموعه
\(A_1 \times A_2 \times A_3\) و سپس برای \(n\) مجموعه تعمیم دهید؟
\subsection{۲. پاسخ}\label{ux67eux627ux633ux62e}
بله، به جای «زوج مرتب»، از «سه‌تایی مرتب» و \lr{«n-تایی }مرتب» استفاده
می‌کنیم.
\begin{tldr}{تعریف تعمیم یافته}
\textbf{برای ۳ مجموعه:}
\[A_1 \times A_2 \times A_3 = \{ (a_1, a_2, a_3) \mid a_1 \in A_1, a_2 \in A_2, a_3 \in A_3 \}\]
\textbf{برای \lr{n }مجموعه:}
\[A_1 \times A_2 \times \dots \times A_n = \{ (a_1, a_2, \dots, a_n) \mid a_i \in A_i, i=1,\dots,n \}\]
\end{tldr}
