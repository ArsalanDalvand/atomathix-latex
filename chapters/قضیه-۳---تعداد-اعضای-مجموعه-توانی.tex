% ---------------------------------------------------------------------
% Copyright (c) 2026 Arsalan Dalvand & Reyhaneh Darvishi.
% Licensed under CC BY-NC-SA 4.0.
% See LICENSE file for details.
% ---------------------------------------------------------------------

\section{قضیه ۳: کاردینالیتی مجموعه
توانی}\label{قضیه-۳---تعداد-اعضای-مجموعه-توانی}
\begin{tldr}{خلاصه سریع}
این قضیه رابطه نمایی بین اندازه یک مجموعه و اندازه مجموعه توانی آن را
بیان می‌کند. اگر مجموعه‌ای \(n\) عضو داشته باشد، فضای حالت زیرمجموعه‌های آن
(مجموعه توانی) \(2^n\) عضو خواهد داشت.
\end{tldr}
\subsection{۱. متن ریاضی
قضیه}\label{ux645ux62aux646-ux631ux6ccux627ux636ux6cc-ux642ux636ux6ccux647}
فرض کنید \(A\) یک مجموعه متناهی با \(n\) عضو باشد (یعنی \(|A| = n\)). در
این صورت تعداد اعضای مجموعه توانی \(A\) برابر است با:
\begin{theorembox}{قضیه ۳}
\[|\mathcal{P}(A)| = 2^n\]
\end{theorembox}
\subsection{۲. اثبات‌های
صوری}\label{ux627ux62bux628ux627ux62aux647ux627ux6cc-ux635ux648ux631ux6cc}
\subsubsection{اثبات اول: تناظر یک‌به‌یک با رشته‌های
باینری}\label{ux627ux62bux628ux627ux62a-ux627ux648ux644-ux62aux646ux627ux638ux631-ux6ccux6a9ux628ux647ux6ccux6a9-ux628ux627-ux631ux634ux62aux647ux647ux627ux6cc-ux628ux627ux6ccux646ux631ux6cc}
این اثبات بر اساس ساختن یک تناظر یک‌به‌یک \lr{(Bijection) }بین
زیرمجموعه‌های \(A\) و توابع مشخصه \lr{(Characteristic Functions) }بنا شده
است.
\begin{info}{اثبات ترکیبیاتی}
۱. فرض کنید \(A = \{a_1, a_2, \dots, a_n\}\). ۲. هر زیرمجموعه
\(S \subseteq A\) را می‌توان با یک دنباله دوتایی
\lr{(Binary Sequence) }به طول \(n\) نمایش داد، به طوری که \(i\)-امین
مؤلفه ۱ باشد اگر \(a_i \in S\) و ۰ باشد اگر \(a_i \notin S\). ۳. برای هر
عنصر \(a_i\) دقیقاً ۲ حالت وجود دارد (حضور یا عدم حضور). ۴. طبق اصل ضرب
در ترکیبیات، تعداد کل حالت‌های ممکن برای ساختن این دنباله‌ها برابر است با:
\[\underbrace{2 \times 2 \times \dots \times 2}_{n \text{ times}} = 2^n\]
۵. بنابراین تعداد زیرمجموعه‌ها نیز \(2^n\) است.
\end{info}
\subsubsection{اثبات دوم: استفاده از بسط
دوجمله‌ای}\label{ux627ux62bux628ux627ux62a-ux62fux648ux645-ux627ux633ux62aux641ux627ux62fux647-ux627ux632-ux628ux633ux637-ux62fux648ux62cux645ux644ux647ux627ux6cc}
این اثبات از افراز مجموعه توانی بر اساس «اندازه زیرمجموعه‌ها» استفاده
می‌کند.
\begin{info}{اثبات جبری}
۱. می‌دانیم تعداد زیرمجموعه‌های \(k\)-عضوی از یک مجموعه \(n\)-عضوی برابر
با ترکیب \(k\) از \(n\) یا \(C(n,k)\) است. ۲. تعداد کل اعضای
\(\mathcal{P}(A)\) برابر است با مجموع تعداد زیرمجموعه‌های ۰ عضوی، ۱ عضوی،
\ldots{} تا \(n\) عضوی: \[|\mathcal{P}(A)| = \sum_{k=0}^{n} C(n, k)\] ۳.
طبق \textbf{\autoref{قضیه-۹---دو-جمله-ای}} در فصل ۱، داریم:
\[(x+y)^n = \sum_{k=0}^{n} C(n, k) x^{n-k} y^k\] ۴. با جایگذاری \(x=1\)
و \(y=1\) در اتحاد فوق:
\[(1+1)^n = \sum_{k=0}^{n} C(n, k) (1)^{n-k} (1)^k = \sum_{k=0}^{n} C(n, k)\]
۵. نتیجه: \(|\mathcal{P}(A)| = 2^n\).
\end{info}
