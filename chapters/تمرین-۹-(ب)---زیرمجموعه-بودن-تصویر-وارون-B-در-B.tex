% ---------------------------------------------------------------------
% Copyright (c) 2026 Arsalan Dalvand & Reyhaneh Darvishi.
% Licensed under CC BY-NC-SA 4.0.
% See LICENSE file for details.
% ---------------------------------------------------------------------

\section{\texorpdfstring{قضیه: رابطه نگاره‌یِ نگاره وارون \lr{B }با خود
\lr{B}}{قضیه: رابطه نگاره‌یِ نگاره وارون با خود }}\label{تمرین-۹-(ب)---زیرمجموعه-بودن-تصویر-وارون-B-در-B}
\begin{tldr}{خلاصه سریع}
اگر از مقصد (\(Y\))، مجموعه‌ای را با \(f^{-1}\) به مبدا بکشیم و دوباره با
\(f\) به مقصد پرتاب کنیم، حتماً درونِ مجموعه اولیه می‌افتیم.
\[f(f^{-1}(B)) \subseteq B\]
\end{tldr}
\subsection{۱. درک
شهودی}\label{ux62fux631ux6a9-ux634ux647ux648ux62fux6cc}
فرض کن \(B\) لیست «نمرات قابل قبول» (مثلاً ۱۰ تا ۲۰) است.
\begin{enumerate}
\def\labelenumi{\arabic{enumi}.}
\tightlist
\item
  \(f^{-1}(B)\): لیست تمام دانشجویانی است که نمره‌شان بین ۱۰ تا ۲۰ شده.
\item
  \(f(f^{-1}(B))\): حالا نمراتِ همین دانشجویانِ انتخاب شده را دوباره نگاه
  می‌کنیم.
\end{enumerate}
مشخص است که نمرات این افراد، قطعاً جزوی از همان بازه ۱۰ تا ۲۰ (\(B\))
است. اما ممکن است برخی نمرات در \(B\) باشد که هیچکس نگرفته باشد، پس شاید
کل \(B\) پوشش داده نشود (مگر تابع پوشا باشد).
\subsection{۲. صورت
ریاضی}\label{ux635ux648ux631ux62a-ux631ux6ccux627ux636ux6cc}
\begin{theorembox}{قضیه}
فرض کنید \(f:X \rightarrow Y\) و \(B \subseteq Y\). در این صورت:
\[f(f^{-1}(B)) \subseteq B\]
\end{theorembox}
\subsection{۳. اثبات
دقیق}\label{ux627ux62bux628ux627ux62a-ux62fux642ux6ccux642}
\begin{info}{اثبات}
برای اثبات \(S_1 \subseteq S_2\)، باید نشان دهیم اگر \(y \in S_1\) آنگاه
\(y \in S_2\).
\begin{enumerate}
\def\labelenumi{\arabic{enumi}.}
\tightlist
\item
  فرض کنید \(y \in f(f^{-1}(B))\) باشد.
\item
  طبق تعریف تصویر تابع، یعنی باید یک \(x\) در دامنه (در اینجا دامنه ما
  مجموعه \(f^{-1}(B)\) است) وجود داشته باشد که:
  \[y = f(x) \quad \text{for some } x \in f^{-1}(B)\]
\item
  حالا عبارت \(x \in f^{-1}(B)\) را باز می‌کنیم. طبق تعریف وارون، این
  یعنی: \[f(x) \in B\]
\item
  از طرفی در مرحله (۲) گفتیم \(y = f(x)\).
\item
  پس با جایگذاری داریم: \[y \in B\]
\item
  \textbf{نتیجه:} تمام اعضای سمت چپ، در سمت راست هستند.
  \[f(f^{-1}(B)) \subseteq B\]
\end{enumerate}
\end{info}
\begin{info}{نکته تکمیلی}
تساوی \(f(f^{-1}(B)) = B\) تنها زمانی برقرار است که تابع \(f\)
\textbf{پوشا \lr{(Surjective)}} باشد (یا حداقل مجموعه \(B\) زیرمجموعه‌ای
از برد تابع باشد).
\end{info}
