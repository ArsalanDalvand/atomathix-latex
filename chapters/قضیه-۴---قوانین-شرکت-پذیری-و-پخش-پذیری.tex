% ---------------------------------------------------------------------
% Copyright (c) 2026 Arsalan Dalvand & Reyhaneh Darvishi.
% Licensed under CC BY-NC-SA 4.0.
% See LICENSE file for details.
% ---------------------------------------------------------------------

\section{قضیه ۴: قوانین شرکت‌پذیری، پخش‌پذیری و
تعدی}\label{قضیه-۴---قوانین-شرکت-پذیری-و-پخش-پذیری}
\begin{tldr}{خلاصه سریع}
این قضیه معماریِ جملات مرکب را مدیریت می‌کند. «شرکت‌پذیری» به ما اجازه
می‌دهد پرانتزها را در عملگرهای هم‌جنس نادیده بگیریم، «پخش‌پذیری» تعامل دو
عملگر غیرهم‌جنس را نشان می‌دهد، و «تعدی» زنجیره‌ای از استدلال‌های شرطی را به
یک نتیجه واحد متصل می‌کند.
\end{tldr}
\subsection{۱. متن ریاضی
قضیه}\label{ux645ux62aux646-ux631ux6ccux627ux636ux6cc-ux642ux636ux6ccux647}
فرض کنید \(p\)، \(q\) و \(r\) سه گزاره دلبخواه باشند.
\begin{theorembox}{قضیه ۴}
\textbf{الف) قوانین شرکت‌پذیری \lr{(Associative Laws):}}
\[(p \vee q) \vee r \equiv p \vee (q \vee r)\]
\[(p \wedge q) \wedge r \equiv p \wedge (q \wedge r)\]
\textbf{ب) قوانین پخش‌پذیری \lr{(Distributive Laws):}}
\[p \wedge (q \vee r) \equiv (p \wedge q) \vee (p \wedge r)\]
\[p \vee (q \wedge r) \equiv (p \vee q) \wedge (p \vee r)\]
\textbf{ج) قانون تعدی \lr{(Transitive Law):}}
\[(p \rightarrow q) \wedge (q \rightarrow r) \Rightarrow (p \rightarrow r)\]
\end{theorembox}
\subsection{۲. اثبات و تحلیل (با جدول
ارزش)}\label{ux627ux62bux628ux627ux62a-ux648-ux62aux62dux644ux6ccux644-ux628ux627-ux62cux62fux648ux644-ux627ux631ux632ux634}
برای اثبات این قوانین، چون با ۳ متغیر گزاره‌ای سر و کار داریم، جدول ارزش
باید شامل \(2^3=8\) سطر باشد. در اینجا اثبات \textbf{قانون تعدی} (قسمت
ج) را با استفاده از جدول ارزش بررسی می‌کنیم. ما باید نشان دهیم که گزاره
شرطی نهایی همیشه «راستگو» \lr{(Tautology) }است.
{\def\LTcaptype{none}
\begin{longtable}[]{@{}
  >{\centering\arraybackslash}p{(\linewidth - 16\tabcolsep) * \real{0.1000}}
  >{\centering\arraybackslash}p{(\linewidth - 16\tabcolsep) * \real{0.0750}}
  >{\centering\arraybackslash}p{(\linewidth - 16\tabcolsep) * \real{0.0750}}
  >{\centering\arraybackslash}p{(\linewidth - 16\tabcolsep) * \real{0.0750}}
  >{\centering\arraybackslash}p{(\linewidth - 16\tabcolsep) * \real{0.0750}}
  >{\centering\arraybackslash}p{(\linewidth - 16\tabcolsep) * \real{0.0750}}
  >{\centering\arraybackslash}p{(\linewidth - 16\tabcolsep) * \real{0.1500}}
  >{\centering\arraybackslash}p{(\linewidth - 16\tabcolsep) * \real{0.1500}}
  >{\centering\arraybackslash}p{(\linewidth - 16\tabcolsep) * \real{0.2250}}@{}}
\toprule\noalign{}
\begin{minipage}[b]{\linewidth}\centering
ردیف
\end{minipage} & \begin{minipage}[b]{\linewidth}\centering
\(p\)
\end{minipage} & \begin{minipage}[b]{\linewidth}\centering
\(q\)
\end{minipage} & \begin{minipage}[b]{\linewidth}\centering
\(r\)
\end{minipage} & \begin{minipage}[b]{\linewidth}\centering
\((p \to q)\)
\end{minipage} & \begin{minipage}[b]{\linewidth}\centering
\((q \to r)\)
\end{minipage} & \begin{minipage}[b]{\linewidth}\centering
مقدم: \((p \to q) \wedge (q \to r)\)
\end{minipage} & \begin{minipage}[b]{\linewidth}\centering
تالی: \((p \to r)\)
\end{minipage} & \begin{minipage}[b]{\linewidth}\centering
کل گزاره \((\to)\)
\end{minipage} \\
\midrule\noalign{}
\endhead
\bottomrule\noalign{}
\endlastfoot
1 & \lr{T} & \lr{T} & \lr{T} & \lr{T} & \lr{T} & \textbf{\lr{T}} &
\lr{T} & \textbf{\lr{T}} \\
2 & \lr{T} & \lr{T} & \lr{F} & \lr{T} & \lr{F} & \lr{F} & \lr{F} &
\textbf{\lr{T}} \\
3 & \lr{T} & \lr{F} & \lr{T} & \lr{F} & \lr{T} & \lr{F} & \lr{T} &
\textbf{\lr{T}} \\
4 & \lr{T} & \lr{F} & \lr{F} & \lr{F} & \lr{T} & \lr{F} & \lr{F} &
\textbf{\lr{T}} \\
5 & \lr{F} & \lr{T} & \lr{T} & \lr{T} & \lr{T} & \textbf{\lr{T}} &
\lr{T} & \textbf{\lr{T}} \\
6 & \lr{F} & \lr{T} & \lr{F} & \lr{T} & \lr{F} & \lr{F} & \lr{T} &
\textbf{\lr{T}} \\
7 & \lr{F} & \lr{F} & \lr{T} & \lr{T} & \lr{T} & \textbf{\lr{T}} &
\lr{T} & \textbf{\lr{T}} \\
8 & \lr{F} & \lr{F} & \lr{F} & \lr{T} & \lr{T} & \textbf{\lr{T}} &
\lr{T} & \textbf{\lr{T}} \\
\emph{(جدول ۱۱ - اثبات قانون تعدی)} & & & & & & & & \\
\end{longtable}
}
\begin{info}{تحلیل اثبات}
۱. ستون‌های «مقدم» و «تالی» را محاسبه می‌کنیم. ۲. ستون آخر (کل گزاره)
رابطه شرطی بین مقدم و تالی را بررسی می‌کند. ۳. مشاهده می‌شود که در تمام ۸
حالت ممکن، ارزش نهایی \textbf{\lr{T}} است. این ثابت می‌کند که رابطه تعدی
یک قانون معتبر منطقی است.
\end{info}
\subsection{\texorpdfstring{۳. شبکه ارتباطی با سایر قضایا
\lr{(Analytic Map)}}{۳. شبکه ارتباطی با سایر قضایا }}\label{ux634ux628ux6a9ux647-ux627ux631ux62aux628ux627ux637ux6cc-ux628ux627-ux633ux627ux6ccux631-ux642ux636ux627ux6ccux627-analytic-map}
این قضیه پل ارتباطی بین منطق گزاره‌ها و نظریه مجموعه‌هاست و ابزار اصلی
برای ساختن برهان‌های چندمرحله‌ای می‌باشد.
\subsubsection{\texorpdfstring{۱. ارتباط با
\autoref{قضیه-۲---هم‌ارزی‌های-منطقی-پایه}
(جابجایی)}{۱. ارتباط با  (جابجایی)}}\label{ux627ux631ux62aux628ux627ux637-ux628ux627-ux642ux636ux6ccux647-ux6f2---ux647ux645ux627ux631ux632ux6ccux647ux627ux6cc-ux645ux646ux637ux642ux6cc-ux67eux627ux6ccux647-ux62cux627ux628ux62cux627ux6ccux6cc}
\begin{itemize}
\tightlist
\item
  \textbf{تکمیل ساختار جبری:} قضیه ۲ (جابجایی) می‌گفت ترتیب \(p\) و \(q\)
  مهم نیست (\(p \vee q \equiv q \vee p\)). قضیه ۴ (شرکت‌پذیری) مکمل آن
  است و می‌گوید اولویت پرانتزها هم مهم نیست. ترکیب این دو ویژگی به ما
  اجازه می‌دهد در زنجیره‌ای مثل \(p \vee q \vee r \vee s\)، گزاره‌ها را
  آزادانه جابجا و دسته‌بندی کنیم.
\end{itemize}
\subsubsection{\texorpdfstring{۲. ارتباط با
\autoref{قضیه-۶---قواعد-استنتاج}(قیاس
شرطی)}{۲. ارتباط با (قیاس شرطی)}}\label{ux627ux631ux62aux628ux627ux637-ux628ux627-ux642ux636ux6ccux647-ux6f6---ux642ux648ux627ux639ux62f-ux627ux633ux62aux646ux62aux627ux62cux642ux6ccux627ux633-ux634ux631ux637ux6cc}
\begin{itemize}
\tightlist
\item
  \textbf{نام دیگر:} قانون تعدی در بسیاری از متون (و در قضیه ۶ همین
  کتاب) تحت عنوان \textbf{قیاس شرطی \lr{(Hypothetical Syllogism)}}
  شناخته می‌شود. قضیه ۴ زیربنای نظری آن را با جدول ارزش ثابت می‌کند، و
  قضیه ۶ آن را به عنوان یک ابزار استنتاجی معرفی می‌کند.
\end{itemize}
\subsubsection{۴. ارتباط با نظریه مجموعه‌ها (فصل
۲)}\label{ux627ux631ux62aux628ux627ux637-ux628ux627-ux646ux638ux631ux6ccux647-ux645ux62cux645ux648ux639ux647ux647ux627-ux641ux635ux644-ux6f2}
\begin{itemize}
\tightlist
\item
  \textbf{ایزومورفیسم ساختاری:} قوانین پخش‌پذیری در منطق (\(\wedge\) روی
  \(\vee\)) دقیقاً متناظر با قوانین پخش‌پذیری در مجموعه‌ها (\(\cap\) روی
  \(\cup\)) هستند:
  \begin{itemize}
  \tightlist
  \item
    منطق: \(p \wedge (q \vee r) \equiv (p \wedge q) \vee (p \wedge r)\)
  \item
    مجموعه: \(A \cap (B \cup C) = (A \cap B) \cup (A \cap C)\) این شباهت
    نشان می‌دهد که منطق و نظریه مجموعه‌ها از یک ساختار جبری واحد (جبر بول)
    پیروی می‌کنند.
  \end{itemize}
\end{itemize}
