% ---------------------------------------------------------------------
% Copyright (c) 2026 Arsalan Dalvand & Reyhaneh Darvishi.
% Licensed under CC BY-NC-SA 4.0.
% See LICENSE file for details.
% ---------------------------------------------------------------------

\section{تمرین ۶: افزودن شرط مشترک به
استلزام}\label{تمرین-۶---اثبات-استلزام-شرطی}
\begin{tldr}{خلاصه سریع}
اگر بدانیم «اگر باران بیاید، زمین خیس می‌شود» (\(p \to q\))، می‌توانیم یک
شرط دیگر مثل «هوا سرد است» (\(r\)) را به دو طرف اضافه کنیم. حکم: «اگر
(باران بیاید و سرد باشد)، آنگاه (زمین خیس می‌شود و سرد است)».
\end{tldr}
\subsection{۱. صورت
مسأله}\label{ux635ux648ux631ux62a-ux645ux633ux623ux644ux647}
\begin{info}{سوال}
ثابت کنید:
\[(p \rightarrow q) \implies (p \land r \rightarrow q \land r)\]
\end{info}
\subsection{۲. اثبات
مستقیم}\label{ux627ux62bux628ux627ux62a-ux645ux633ux62aux642ux6ccux645}
\begin{info}{گام‌به‌گام}
فرض می‌کنیم مقدمِ گزاره‌ی اصلی یعنی \((p \rightarrow q)\) \textbf{درست}
باشد. می‌خواهیم نشان دهیم نتیجه هم درست است.
\begin{enumerate}
\def\labelenumi{\arabic{enumi}.}
\tightlist
\item
  فرض کنید سمت چپِ فلشِ دوم درست باشد: یعنی \((p \land r)\) درست است.
\item
  از درست بودن \((p \land r)\) نتیجه می‌گیریم که هم \(p\) درست است و هم
  \(r\) درست است.
\item
  چون \(p\) درست است و طبق فرض اولیه داریم \(p \rightarrow q\)، پس نتیجه
  می‌گیریم \(q\) هم باید درست باشد (قیاس استثنایی).
\item
  الان چی داریم؟
  \begin{itemize}
  \tightlist
  \item
    \(q\) درست است (از مرحله ۳).
  \item
    \(r\) درست است (از مرحله ۲).
  \end{itemize}
\item
  پس ترکیب \((q \land r)\) حتماً درست است.
\item
  چون از درستیِ مقدم (\((p \land r)\)) به درستی تالی (\((q \land r)\))
  رسیدیم، پس استلزام برقرار است.
\end{enumerate}
\(\blacksquare\)
\end{info}
