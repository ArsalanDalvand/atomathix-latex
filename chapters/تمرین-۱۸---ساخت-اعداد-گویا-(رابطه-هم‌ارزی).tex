% ---------------------------------------------------------------------
% Copyright (c) 2026 Arsalan Dalvand & Reyhaneh Darvishi.
% Licensed under CC BY-NC-SA 4.0.
% See LICENSE file for details.
% ---------------------------------------------------------------------

\section{تمرین ۱۸: رابطه هم‌ارزی برای
کسرها}\label{تمرین-۱۸---ساخت-اعداد-گویا-(رابطه-هم‌ارزی)}
\begin{tldr}{کاربرد}
این تمرین نحوه ساخت اعداد گویا (\(\mathbb{Q}\)) از اعداد صحیح
(\(\mathbb{Z}\)) را نشان می‌دهد. زوج \((a,b)\) همان کسر \(\frac{a}{b}\)
است و شرط \(ad=bc\) همان تساوی کسرهاست (\(\frac{a}{b}=\frac{c}{d}\)).
\end{tldr}
\subsection{۱. صورت
سوال}\label{ux635ux648ux631ux62a-ux633ux648ux627ux644}
مجموعه \(X = Z \times (Z - \{0\})\) را در نظر بگیرید. رابطه \(\sim\) را
چنین تعریف می‌کنیم: \[(a,b) \sim (c,d) \iff ad = bc\] ثابت کنید \(\sim\)
یک رابطه هم‌ارزی است.
\subsection{۲. اثبات}\label{ux627ux62bux628ux627ux62a}
\textbf{۱. انعکاسی:} آیا \((a,b) \sim (a,b)\)؟ بله، چون \(ab = ba\) (ضرب
جابجایی است). \lr{[cite: }992-994{]}
\textbf{۲. متقارن:} اگر \((a,b) \sim (c,d)\)، آیا \((c,d) \sim (a,b)\)؟
\[ad = bc \Rightarrow cb = da \Rightarrow (c,d) \sim (a,b)\]
\textbf{۳. متعدی:} فرض: \((a,b) \sim (c,d)\) و \((c,d) \sim (e,f)\).
یعنی \(ad = bc\) و \(cf = de\). طرفین را در هم ضرب می‌کنیم:
\((ad)(cf) = (bc)(de)\). \(c\) و \(d\) (که مخالف صفر است) را ساده
می‌کنیم؟ بهتر است ضرب کنیم: از اولی \(d = \frac{bc}{a}\) (نه در اعداد
صحیح نمی‌شود). راه کتاب: \(adf = (ad)f = (bc)f = b(cf) = b(de) = bde\).
پس \(adf = bde\). چون \(d \neq 0\) (طبق تعریف دامنه)، می‌توانیم \(d\) را
از طرفین حذف کنیم (قانون حذف در اعداد صحیح). \[\Rightarrow af = be\] که
یعنی \((a,b) \sim (e,f)\).
