% ---------------------------------------------------------------------
% Copyright (c) 2026 Arsalan Dalvand & Reyhaneh Darvishi.
% Licensed under CC BY-NC-SA 4.0.
% See LICENSE file for details.
% ---------------------------------------------------------------------

\section{مفاهیم تصویر و تصویر وارون + قضیه
۹}\label{قضیه-۹---تصویر-و-تصویر-وارون-مجموعه}
\begin{tldr}{خلاصه سریع}
این یادداشت ابتدا دو عملگر بنیادی روی مجموعه‌ها تحت اثر تابع (\(f(A)\) و
\(f^{-1}(B)\)) را تعریف می‌کند. سپس در قضیه ۹ ثابت می‌کنیم که عملگر «تصویر
مستقیم» ساختار اجتماع را حفظ می‌کند (\(=\))، اما در مورد اشتراک ضعیف‌تر
عمل کرده و فقط زیرمجموعه بودن (\(\subseteq\)) را تضمین می‌کند.
\end{tldr}
\subsection{۱. تعاریف بنیادین (تصویر و
وارون)}\label{ux62aux639ux627ux631ux6ccux641-ux628ux646ux6ccux627ux62fux6ccux646-ux62aux635ux648ux6ccux631-ux648-ux648ux627ux631ux648ux646}
پیش از بیان قضیه، لازم است تعاریف دقیق «تصویر مستقیم» \lr{(Image) }و
«تصویر وارون» \lr{(Inverse Image }/ \lr{Preimage) }را تثبیت کنیم. فرض
کنید \(f: X \to Y\) یک تابع باشد.
\subsubsection{\texorpdfstring{الف) تصویر مستقیم
\lr{(Direct Image)}}{الف) تصویر مستقیم }}\label{ux627ux644ux641-ux62aux635ux648ux6ccux631-ux645ux633ux62aux642ux6ccux645-direct-image}
اگر \(A \subseteq X\) باشد، تصویر \(A\) تحت تابع \(f\)، زیرمجموعه‌ای از
\(Y\) است که شامل خروجی‌های متناظر با اعضای \(A\) می‌باشد.
\begin{tldr}{تعریف ۱}
\[f(A) = \{ y \in Y \mid \exists x \in A, y = f(x) \}\] معادل مجموعه‌ای:
\(f(A) = \{ f(x) \mid x \in A \}\)
\end{tldr}
\subsubsection{\texorpdfstring{ب) تصویر وارون
\lr{(Inverse Image)}}{ب) تصویر وارون }}\label{ux628-ux62aux635ux648ux6ccux631-ux648ux627ux631ux648ux646-inverse-image}
اگر \(B \subseteq Y\) باشد، تصویر وارون \(B\) تحت تابع \(f\)،
زیرمجموعه‌ای از \(X\) است که تمام اعضای آن توسط \(f\) به درون \(B\) پرتاب
می‌شوند.
\begin{tldr}{تعریف ۲}
\[f^{-1}(B) = \{ x \in X \mid f(x) \in B \}\]
\end{tldr}
\begin{warning}{هشدار مهم}
نماد \(f^{-1}\) در اینجا به معنای «تابع وارون» نیست. تصویر وارون برای
\textbf{هر تابعی} (حتی اگر یک‌به‌یک یا پوشا نباشد) روی مجموعه‌ها تعریف
می‌شود.
\end{warning}
\begin{center}\rule{0.5\linewidth}{0.5pt}\end{center}
\subsection{۲. متن ریاضی قضیه
۹}\label{ux645ux62aux646-ux631ux6ccux627ux636ux6cc-ux642ux636ux6ccux647-ux6f9}
فرض کنید \(f: X \to Y\) یک تابع باشد و \(A, B \subseteq X\). ویژگی‌های
جبری تصویر مستقیم به شرح زیر است:
\begin{theorembox}{قضیه ۹}
\textbf{الف) حفظ شمول \lr{(Monotonicity):}}
\[A \subseteq B \implies f(A) \subseteq f(B)\] \textbf{ب) پخش‌پذیری روی
اجتماع:} \[f(A \cup B) = f(A) \cup f(B)\] \textbf{ج) رفتار روی اشتراک:}
\[f(A \cap B) \subseteq f(A) \cap f(B)\] \emph{(نکته: تساوی در قسمت ج
لزوماً برقرار نیست).}
\end{theorembox}
\begin{center}\rule{0.5\linewidth}{0.5pt}\end{center}
\subsection{\texorpdfstring{۳. اثبات صوری
\lr{(Formal Proof)}}{۳. اثبات صوری }}\label{ux627ux62bux628ux627ux62a-ux635ux648ux631ux6cc-formal-proof}
\subsubsection{اثبات قسمت (الف): حفظ
شمول}\label{ux627ux62bux628ux627ux62a-ux642ux633ux645ux62a-ux627ux644ux641-ux62dux641ux638-ux634ux645ux648ux644}
\begin{info}{برهان}
۱. فرض کنیم \(y \in f(A)\). ۲. طبق تعریف تصویر، یعنی \(\exists x \in A\)
به طوری که \(y = f(x)\). ۳. چون طبق فرض \(A \subseteq B\)، پس این \(x\)
متعلق به \(B\) نیز هست (\(x \in B\)). ۴. اکنون ما عنصری در \(B\) داریم
(\(x\)) که تصویرش \(y\) است. ۵. پس طبق تعریف، \(y \in f(B)\). ۶.
\lr{[cite\_start]نتیجه: }\(f(A) \subseteq f(B)\). \lr{[cite: }515-516{]}
\end{info}
\subsubsection{اثبات قسمت (ب):
اجتماع}\label{ux627ux62bux628ux627ux62a-ux642ux633ux645ux62a-ux628-ux627ux62cux62aux645ux627ux639}
باید تساوی دو مجموعه را با شمول دوطرفه ثابت کنیم.
\begin{info}{برهان}
\textbf{جهت اول (\(f(A \cup B) \subseteq f(A) \cup f(B)\)):} ۱. فرض کنیم
\(y \in f(A \cup B)\). ۲. یعنی \(\exists x \in A \cup B\) که
\(y = f(x)\). ۳. چون \(x \in A \cup B\)، پس (\(x \in A\)) یا
(\(x \in B\)). ۴. اگر \(x \in A \implies y \in f(A)\). ۵. اگر
\(x \in B \implies y \in f(B)\). ۶. در هر حال \(y \in f(A) \cup f(B)\).
\textbf{جهت دوم (\(f(A) \cup f(B) \subseteq f(A \cup B)\)):} ۱. می‌دانیم
\(A \subseteq A \cup B\) و \(B \subseteq A \cup B\). ۲. طبق قسمت (الف)
همین قضیه (حفظ شمول):
\[f(A) \subseteq f(A \cup B) \quad \text{و} \quad f(B) \subseteq f(A \cup B)\]
۳. اجتماع دو زیرمجموعه از یک مجموعه، باز هم زیرمجموعه آن است:
\[f(A) \cup f(B) \subseteq f(A \cup B)\]
\textbf{نتیجه:} تساوی برقرار است.
\end{info}
\subsubsection{اثبات قسمت (ج):
اشتراک}\label{ux627ux62bux628ux627ux62a-ux642ux633ux645ux62a-ux62c-ux627ux634ux62aux631ux627ux6a9}
\begin{info}{برهان}
۱. می‌دانیم \(A \cap B \subseteq A\) و \(A \cap B \subseteq B\). ۲. طبق
قسمت (الف) (حفظ شمول):
\[f(A \cap B) \subseteq f(A) \quad \text{و} \quad f(A \cap B) \subseteq f(B)\]
۳. چون مجموعه سمت چپ زیرمجموعه هر دو مجموعه سمت راست است، پس زیرمجموعه
اشتراک آن‌ها نیز می‌باشد: \[f(A \cap B) \subseteq f(A) \cap f(B)\]
\end{info}
\begin{center}\rule{0.5\linewidth}{0.5pt}\end{center}
\subsection{\texorpdfstring{۴. شبکه ارتباطی با سایر قضایا
\lr{(Analytic Map)}}{۴. شبکه ارتباطی با سایر قضایا }}\label{ux634ux628ux6a9ux647-ux627ux631ux62aux628ux627ux637ux6cc-ux628ux627-ux633ux627ux6ccux631-ux642ux636ux627ux6ccux627-analytic-map}
\subsubsection{\texorpdfstring{۱. ارتباط با
\autoref{قضیه-۱---پخش‌پذیری-حاصلضرب-دکارتی}}{۱. ارتباط با }}\label{ux627ux631ux62aux628ux627ux637-ux628ux627-ux642ux636ux6ccux647-ux6f1---ux67eux62eux634ux67eux630ux6ccux631ux6cc-ux62dux627ux635ux644ux636ux631ux628-ux62fux6a9ux627ux631ux62aux6cc}
\begin{itemize}
\tightlist
\item
  \textbf{چرا اشتراک برابر نمی‌شود؟} در اثبات اجتماع (بخش ب)، ما از
  هم‌ارزی
  \((\exists x, P(x) \lor Q(x)) \iff (\exists x, P(x)) \lor (\exists x, Q(x))\)
  استفاده کردیم. اما سور وجودی (\(\exists\)) روی اشتراک (\(\land\)) پخش
  نمی‌شود. به همین دلیل \(f(A \cap B) \neq f(A) \cap f(B)\).
\item
  \textbf{مثال نقض:} تابع \(f(x)=x^2\) را در نظر بگیرید. \(A=\{-2\}\) و
  \(B=\{2\}\).
  \begin{itemize}
  \tightlist
  \item
    \(A \cap B = \emptyset \implies f(A \cap B) = \emptyset\).
  \item
    \(f(A) = \{4\}\) و \(f(B) = \{4\} \implies f(A) \cap f(B) = \{4\}\).
  \item
    واضح است که \(\emptyset \neq \{4\}\).
  \end{itemize}
\end{itemize}
\subsubsection{\texorpdfstring{۲. ارتباط با
\autoref{قضیه-۵---متمم-و-زیرمجموعه} (تصویر
وارون)}{۲. ارتباط با  (تصویر وارون)}}\label{ux627ux631ux62aux628ux627ux637-ux628ux627-ux642ux636ux6ccux647-ux6f5---ux645ux62aux645ux645-ux648-ux632ux6ccux631ux645ux62cux645ux648ux639ux647-ux62aux635ux648ux6ccux631-ux648ux627ux631ux648ux646}
\begin{itemize}
\tightlist
\item
  \textbf{رفتار بهتر وارون:} در قضایای بعدی خواهیم دید که \(f^{-1}\)
  (تصویر وارون) برخلاف \(f\) (تصویر مستقیم)، رفتار بسیار منظمی دارد و با
  تمام عملیات (اجتماع، اشتراک و تفاضل) سازگار است
  (\(f^{-1}(A \cap B) = f^{-1}(A) \cap f^{-1}(B)\)). این تفاوت بنیادی
  بین «تابع» و «رابطه» را نشان می‌دهد.
\end{itemize}
\subsubsection{\texorpdfstring{۳. ارتباط با
\autoref{قضیه-۸---لم-چسباندن-(اجتماع-توابع)}}{۳. ارتباط با }}\label{ux627ux631ux62aux628ux627ux637-ux628ux627-ux642ux636ux6ccux647-ux6f8---ux644ux645-ux686ux633ux628ux627ux646ux62fux646-ux627ux62cux62aux645ux627ux639-ux62aux648ux627ux628ux639}
\begin{itemize}
\tightlist
\item
  \textbf{ساختار اجتماع:} قضیه ۹-ب نشان می‌دهد که تصویرِ اجتماع دامنه‌ها،
  برابر با اجتماعِ تصویرهاست. این مفهوم با لم چسباندن سازگار است؛ اگر
  توابع را روی دامنه‌های جداگانه تعریف کنیم، برد نهایی اجتماع بردهای
  آنهاست.
\end{itemize}
