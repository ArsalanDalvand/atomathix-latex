% ---------------------------------------------------------------------
% Copyright (c) 2026 Arsalan Dalvand & Reyhaneh Darvishi.
% Licensed under CC BY-NC-SA 4.0.
% See LICENSE file for details.
% ---------------------------------------------------------------------

\section{تمرین ۷: تعریف باز شده‌ی ترکیب
دوشرطی}\label{تمرین-۷---بازنویسی-ترکیب-دوشرطی}
\begin{tldr}{خلاصه سریع}
عبارت «اگر و تنها اگر» (\(p \leftrightarrow q\)) یعنی این دو گزاره
سرنوشت یکسانی دارند: یا \textbf{هر دو راستگو} هستند (\(p \land q\))، یا
\textbf{هر دو دروغگو} (\(\sim p \land \sim q\)).
\[(p \leftrightarrow q) \equiv (p \land q) \lor (\sim p \land \sim q)\]
\end{tldr}
\subsection{۱. اثبات}\label{ux627ux62bux628ux627ux62a}
\begin{info}{تحلیل حالت‌ها}
گزاره \(p \leftrightarrow q\) زمانی ارزش \textbf{\lr{T}} (درست) دارد که
ارزش \(p\) و \(q\) یکسان باشد. بیایید حالت‌ها را چک کنیم:
\textbf{حالت اول: هر دو درست باشند \lr{(T, T)}}
\begin{itemize}
\tightlist
\item
  سمت چپ: \(T \leftrightarrow T\) (درست).
\item
  سمت راست: \((T \land T) \lor (F \land F) \equiv T \lor F \equiv T\)
  (درست).
\end{itemize}
\textbf{حالت دوم: هر دو نادرست باشند \lr{(F, F)}}
\begin{itemize}
\tightlist
\item
  سمت چپ: \(F \leftrightarrow F\) (درست).
\item
  سمت راست: \((F \land F) \lor (T \land T) \equiv F \lor T \equiv T\)
  (درست).
\end{itemize}
\textbf{حالت سوم: یکی درست و دیگری نادرست \lr{(T, F }یا \lr{F, T)}}
\begin{itemize}
\tightlist
\item
  سمت چپ: \(T \leftrightarrow F\) (نادرست).
\item
  سمت راست: \((T \land F) \lor (F \land T) \equiv F \lor F \equiv F\)
  (نادرست).
\end{itemize}
چون در تمام حالات نتایج یکسان بود، هم‌ارزی برقرار است.
\end{info}
