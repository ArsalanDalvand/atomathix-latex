% ---------------------------------------------------------------------
% Copyright (c) 2026 Arsalan Dalvand & Reyhaneh Darvishi.
% Licensed under CC BY-NC-SA 4.0.
% See LICENSE file for details.
% ---------------------------------------------------------------------

\section{تمرین ۶ و ۷: شمارش اعضا و بازیابی
مجموعه}\label{تمرین-۶-و-۷---تعداد-اعضا-و-بازیابی-مجموعه}
\subsection{۱. صورت سوال تمرین
۶}\label{ux635ux648ux631ux62a-ux633ux648ux627ux644-ux62aux645ux631ux6ccux646-ux6f6}
\lr{[cite\_start]اگر }مجموعه \(A\) دارای \(m\) عنصر و مجموعه \(B\) دارای
\(n\) عنصر باشد، مجموعه \(A \times B\) چند عنصر (جفت مرتب) دارد؟
\lr{[cite: }676-677{]}
\subsubsection{حل تمرین
۶}\label{ux62dux644-ux62aux645ux631ux6ccux646-ux6f6}
طبق اصل ضرب، برای مولفه اول \(m\) انتخاب و برای مولفه دوم \(n\) انتخاب
داریم. \lr{[cite\_start]تعداد }اعضا برابر است با
\(m \times n\)\lr{[cite: }684{]}.
\begin{center}\rule{0.5\linewidth}{0.5pt}\end{center}
\subsection{۲. صورت سوال تمرین
۷}\label{ux635ux648ux631ux62a-ux633ux648ux627ux644-ux62aux645ux631ux6ccux646-ux6f7}
مجموعه \(A \times A\) نه (۹) عنصر دارد که \((-1,0)\) و \((0,1)\) دو عنصر
آن هستند. اعضای دیگر \(A \times A\) را بیابید.
\subsubsection{حل تمرین
۷}\label{ux62dux644-ux62aux645ux631ux6ccux646-ux6f7}
\begin{enumerate}
\def\labelenumi{\arabic{enumi}.}
\tightlist
\item
  \textbf{یافتن تعداد اعضای \lr{A:}} چون
  \(|A \times A| = |A| \cdot |A| = 9\)، پس \(|A|^2 = 9\) و در نتیجه
  \(A\) باید \textbf{۳ عنصر} داشته باشد.
\item
  \textbf{یافتن اعضای \lr{A:}} زوج‌های \((-1, 0)\) و \((0, 1)\) در
  \(A \times A\) هستند. در حاصلضرب دکارتی \(A \times A\)، هم مولفه اول و
  هم مولفه دوم باید از \(A\) باشند.
  \begin{itemize}
  \tightlist
  \item
    از \((-1, 0)\) می‌فهمیم: \(-1 \in A\) و \(0 \in A\).
  \item
    از \((0, 1)\) می‌فهمیم: \(0 \in A\) و \(1 \in A\).
  \item
    مجموعه اعضای یافت شده: \(\{-1, 0, 1\}\). چون \(A\) باید ۳ عضو داشته
    باشد، همین‌ها تمام اعضای \(A\) هستند. \[A = \{-1, 0, 1\}\]
  \end{itemize}
\item
  \textbf{تشکیل \(A \times A\):} تمام ترکیب‌های دوتایی از این ۳ عدد را
  می‌نویسیم:
  \[A \times A = \{ (-1,-1), (-1,0), (-1,1), (0,-1), (0,0), (0,1), (1,-1), (1,0), (1,1) \}\]
\end{enumerate}
