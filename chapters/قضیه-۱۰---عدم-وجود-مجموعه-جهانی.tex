% ---------------------------------------------------------------------
% Copyright (c) 2026 Arsalan Dalvand & Reyhaneh Darvishi.
% Licensed under CC BY-NC-SA 4.0.
% See LICENSE file for details.
% ---------------------------------------------------------------------

\section{قضیه ۱۰: عدم وجود مجموعه جهانی
مطلق}\label{قضیه-۱۰---عدم-وجود-مجموعه-جهانی}
\begin{tldr}{خلاصه سریع}
این قضیه بیان می‌کند که چیزی به نام «مجموعه همه مجموعه‌ها» وجود ندارد. اگر
چنین مجموعه‌ای وجود داشته باشد، ریاضیات دچار فروپاشی منطقی می‌شود. این
حکم، راه حلی برای گریز از پارادوکس راسل است.
\end{tldr}
\subsection{۱. متن ریاضی
قضیه}\label{ux645ux62aux646-ux631ux6ccux627ux636ux6cc-ux642ux636ux6ccux647}
هیچ مجموعه‌ای به نام \(\mathcal{U}\) (مجموعه جهانی) وجود ندارد که شامل
تمام مجموعه‌ها باشد.
\begin{theorembox}{قضیه ۱۰}
\[\nexists \mathcal{U} : \forall S, (S \in \mathcal{U})\]
\end{theorembox}
\subsection{۲. اثبات صوری (برهان
خلف)}\label{ux627ux62bux628ux627ux62a-ux635ux648ux631ux6cc-ux628ux631ux647ux627ux646-ux62eux644ux641}
این اثبات از \textbf{\autoref{مفهوم-پارادوکس-راسل}} به عنوان موتور محرک
استفاده می‌کند.
\begin{info}{مراحل اثبات}
۱. \textbf{فرض خلف:} فرض کنید مجموعه جهانی \(\mathcal{U}\) وجود دارد که
شامل تمام مجموعه‌هاست. ۲. طبق اصل تصریح
\lr{(Axiom of Separation)، }می‌توانیم زیرمجموعه‌ای از \(\mathcal{U}\) را
با یک ویژگی خاص جدا کنیم. زیرمجموعه \(R\) را چنین تعریف می‌کنیم:
\[R = \{ S \in \mathcal{U} \mid S \notin S \}\] ۳. چون \(R\) یک
زیرمجموعه خوش‌تعریف از \(\mathcal{U}\) است و \(\mathcal{U}\) شامل همه
مجموعه‌هاست، پس خود \(R\) نیز باید عضوی از \(\mathcal{U}\) باشد
(\(R \in \mathcal{U}\)). ۴. اکنون وضعیت عضویت \(R\) در خودش را بررسی
می‌کنیم:
\begin{itemize}
\tightlist
\item
  اگر \(R \in R \implies R \notin R\) (طبق تعریف \(R\)).
\item
  اگر \(R \notin R \implies R \in R\) (طبق تعریف \(R\)). ۵. به تناقض
  \((R \in R \iff R \notin R)\) می‌رسیم. ۶. \textbf{نتیجه:} فرض اولیه
  (وجود \(\mathcal{U}\)) باطل است.
\end{itemize}
\end{info}
\subsection{\texorpdfstring{۳. شبکه ارتباطی با سایر قضایا
\lr{(Analytic Map)}}{۳. شبکه ارتباطی با سایر قضایا }}\label{ux634ux628ux6a9ux647-ux627ux631ux62aux628ux627ux637ux6cc-ux628ux627-ux633ux627ux6ccux631-ux642ux636ux627ux6ccux627-analytic-map}
این قضیه مرزهای نظریه مجموعه‌ها را مشخص می‌کند:
\subsubsection{\texorpdfstring{۱. ارتباط با
\autoref{مفهوم-پارادوکس-راسل}}{۱. ارتباط با }}\label{ux627ux631ux62aux628ux627ux637-ux628ux627-ux645ux641ux647ux648ux645-ux67eux627ux631ux627ux62fux648ux6a9ux633-ux631ux627ux633ux644}
\begin{itemize}
\tightlist
\item
  \textbf{رابطه علت و معلولی:} \autoref{مفهوم-پارادوکس-راسل} «مشکل» را
  نشان می‌دهد و قضیه ۱۰ «راه حل» (حذف مجموعه جهانی) را ارائه می‌دهد. بدون
  تعریف \(R\) در پارادوکس راسل، اثبات این قضیه ممکن نیست.
\end{itemize}
\subsubsection{\texorpdfstring{۲. ارتباط با
\autoref{پیشنیاز---تعریف-مجموعه-توانی}}{۲. ارتباط با }}\label{ux627ux631ux62aux628ux627ux637-ux628ux627-ux67eux6ccux634ux646ux6ccux627ux632---ux62aux639ux631ux6ccux641-ux645ux62cux645ux648ux639ux647-ux62aux648ux627ux646ux6cc}
\begin{itemize}
\tightlist
\item
  \textbf{قضیه کانتور:} قضیه ۱۰ با قضیه کانتور (که می‌گوید
  \(|A| < |\mathcal{P}(A)|\)) هم‌خوانی دارد. اگر \(\mathcal{U}\) وجود
  داشت، باید \(\mathcal{P}(\mathcal{U})\) زیرمجموعه‌ای از \(\mathcal{U}\)
  می‌شد (چون \(\mathcal{U}\) شامل همه چیز است). این یعنی اندازه کل کوچکتر
  از جزء می‌شود که محال است.
\end{itemize}
\subsubsection{\texorpdfstring{۳. ارتباط با
\autoref{قضیه-۱---شمول-تهی}}{۳. ارتباط با }}\label{ux627ux631ux62aux628ux627ux637-ux628ux627-ux642ux636ux6ccux647-ux6f1---ux634ux645ux648ux644-ux62aux647ux6cc}
\begin{itemize}
\tightlist
\item
  \textbf{تضاد در کران‌ها:} در \autoref{قضیه-۱---شمول-تهی} ثابت کردیم
  «کوچکترین» مجموعه (\(\emptyset\)) وجود دارد و یکتاست. قضیه ۱۰ ثابت
  می‌کند که «بزرگترین» مجموعه (\(\mathcal{U}\)) وجود ندارد. ساختار
  مجموعه‌ها از پایین بسته و از بالا باز است.
\end{itemize}
