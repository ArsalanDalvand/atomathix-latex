% ---------------------------------------------------------------------
% Copyright (c) 2026 Arsalan Dalvand & Reyhaneh Darvishi.
% Licensed under CC BY-NC-SA 4.0.
% See LICENSE file for details.
% ---------------------------------------------------------------------

\section{\texorpdfstring{قواعد تسویر
\lr{(Quantification Rules)}}{قواعد تسویر }}\label{قواعد-تسویر}
\begin{tldr}{خلاصه سریع}
این بخش زبان منطق را از «گزاره‌های ساده» به «مجموعه‌ها» گسترش می‌دهد.
\begin{itemize}
\tightlist
\item
  \textbf{سور عمومی (\(\forall\)):} یعنی «برای همه» (مثل عملگر «و» روی
  تمام اعضا).
\item
  \textbf{سور وجودی (\(\exists\)):} یعنی «حداقل یکی هست» (مثل عملگر «یا»
  روی تمام اعضا).
\item
  \textbf{نفی:} نفیِ «همه»، «بعضی» است و نفیِ «بعضی»، «همه» است (با تغییر
  گزاره).
\end{itemize}
\end{tldr}
\subsection{۱. تعاریف
پایه}\label{ux62aux639ux627ux631ux6ccux641-ux67eux627ux6ccux647}
در گزاره‌هایی که مربوط به یک مجموعه (عالم سخن - \lr{Universe) }هستند، از
دو نماد اصلی استفاده می‌کنیم:
\subsubsection{\texorpdfstring{الف) سور عمومی
\lr{(Universal Quantifier)}}{الف) سور عمومی }}\label{ux627ux644ux641-ux633ux648ux631-ux639ux645ux648ux645ux6cc-universal-quantifier}
عبارت «برای تمام \(x\)های در عالم» را با نماد \(\forall x\) نشان می‌دهیم.
\begin{itemize}
\tightlist
\item
  \textbf{نماد:} \((\forall x)(p(x))\)
\item
  \textbf{معنی:} گزاره \(p(x)\) برای تک‌تک اعضای مجموعه مرجع درست است.
\end{itemize}
\subsubsection{\texorpdfstring{ب) سور وجودی
\lr{(Existential Quantifier)}}{ب) سور وجودی }}\label{ux628-ux633ux648ux631-ux648ux62cux648ux62fux6cc-existential-quantifier}
عبارت «حداقل یک \(x\) وجود دارد که\ldots» را با نماد \(\exists x\) نشان
می‌دهیم.
\begin{itemize}
\tightlist
\item
  \textbf{نماد:} \((\exists x)(p(x))\)
\item
  \textbf{معنی:} حداقل یک عضو در مجموعه پیدا می‌شود که \(p(x)\) برای آن
  درست باشد (لازم نیست برای همه درست باشد).
\end{itemize}
\begin{center}\rule{0.5\linewidth}{0.5pt}\end{center}
\subsection{\texorpdfstring{۲. قواعد نقیض سور
\lr{(Quantifier Negation Laws)}}{۲. قواعد نقیض سور }}\label{ux642ux648ux627ux639ux62f-ux646ux642ux6ccux636-ux633ux648ux631-quantifier-negation-laws}
چگونه جملات کلی را منفی کنیم؟ این قواعد بسیار شبیه به
\textbf{\autoref{قضیه-۳---قوانین-دمورگان}} عمل می‌کنند.
\begin{theorembox}{قاعده نقیض سور}
\textbf{۱. نفی سور عمومی:}
\[\sim [(\forall x)(p(x))] \equiv (\exists x)(\sim p(x))\] \emph{(ترجمه:
اگر «همه خوب نیستند»، یعنی «حداقل یک نفر بد است»)}.
\textbf{۲. نفی سور وجودی:}
\[\sim [(\exists x)(p(x))] \equiv (\forall x)(\sim p(x))\] \emph{(ترجمه:
اگر «چنین نیست که کسی آمده باشد»، یعنی «همه نیامده‌اند»)}.
\end{theorembox}
\begin{center}\rule{0.5\linewidth}{0.5pt}\end{center}
\subsection{\texorpdfstring{۳. شبکه ارتباطی با سایر قضایا
\lr{(Analytic Map)}}{۳. شبکه ارتباطی با سایر قضایا }}\label{ux634ux628ux6a9ux647-ux627ux631ux62aux628ux627ux637ux6cc-ux628ux627-ux633ux627ux6ccux631-ux642ux636ux627ux6ccux627-analytic-map}
این بخش نشان می‌دهد که سورها موجودات جدیدی نیستند، بلکه تعمیم‌یافته‌ی همان
عملگرهای منطقی فصل ۱ هستند.
\subsubsection{\texorpdfstring{۱. ارتباط با
\autoref{قضیه-۳---قوانین-دمورگان} (ریشه نفی
سورها)}{۱. ارتباط با  (ریشه نفی سورها)}}\label{ux627ux631ux62aux628ux627ux637-ux628ux627-ux642ux636ux6ccux647-ux6f3---ux642ux648ux627ux646ux6ccux646-ux62fux645ux648ux631ux6afux627ux646-ux631ux6ccux634ux647-ux646ux641ux6cc-ux633ux648ux631ux647ux627}
اگر عالم سخن ما محدود باشد (مثلاً \(\{a_1, a_2, \dots, a_n\}\))، می‌توان
سورها را باز کرد:
\begin{itemize}
\tightlist
\item
  \textbf{سور عمومی \(\equiv\) ترکیب عطفی:}
  \[(\forall x)p(x) \equiv p(a_1) \wedge p(a_2) \wedge \dots \wedge p(a_n)\]
\item
  \textbf{سور وجودی \(\equiv\) ترکیب فصلی:}
  \[(\exists x)p(x) \equiv p(a_1) \vee p(a_2) \vee \dots \vee p(a_n)\]
\end{itemize}
حال اگر از \textbf{\autoref{قضیه-۳---قوانین-دمورگان}} استفاده کنیم:
\begin{itemize}
\tightlist
\item
  نفیِ «ترکیب عطفی» (\(\sim(p \wedge q \dots)\)) تبدیل به «ترکیب فصلی
  نقیض‌ها» (\(\sim p \vee \sim q \dots\)) می‌شود.
\item
  این دقیقاً همان قاعده تبدیل \(\sim \forall\) به \(\exists \sim\) است.
\end{itemize}
\subsubsection{\texorpdfstring{۲. ارتباط با
\autoref{قضیه-۱---قوانین-جمع-و-اختصار} (ساختار
استنتاج)}{۲. ارتباط با  (ساختار استنتاج)}}\label{ux627ux631ux62aux628ux627ux637-ux628ux627-ux642ux636ux6ccux647-ux6f1---ux642ux648ux627ux646ux6ccux646-ux62cux645ux639-ux648-ux627ux62eux62aux635ux627ux631-ux633ux627ux62eux62aux627ux631-ux627ux633ux62aux646ux62aux627ux62c}
\begin{itemize}
\item
  \textbf{قانون اختصار تعمیم‌یافته:} چون \(\forall\) ماهیت «عطفی»
  \lr{(AND) }دارد، قانون اختصار (\((p \wedge q) \to p\)) برای آن صادق
  است. یعنی: \[(\forall x)p(x) \Rightarrow p(a_i)\] \emph{(اگر حکمی برای
  همه درست باشد، برای تک‌تک افراد هم درست است).}
\item
  \textbf{قانون جمع تعمیم‌یافته:} چون \(\exists\) ماهیت «فصلی»
  \lr{(OR) }دارد، قانون جمع (\(p \to p \vee q\)) برای آن صادق است. یعنی:
  \[p(a_i) \Rightarrow (\exists x)p(x)\] *(اگر حکمی برای یک نفر درست
  باشد، پس وجود دارد کسی
\end{itemize}
