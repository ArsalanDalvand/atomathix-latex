% ---------------------------------------------------------------------
% Copyright (c) 2026 Arsalan Dalvand & Reyhaneh Darvishi.
% Licensed under CC BY-NC-SA 4.0.
% See LICENSE file for details.
% ---------------------------------------------------------------------

\section{تمرین ۳: بازگشت از افراز به رابطه
(تناظر)}\label{تمرین-۳---تناظر-یک‌به‌یک-افراز-و-رابطه}
\subsection{۱. صورت
سوال}\label{ux635ux648ux631ux62a-ux633ux648ux627ux644}
\begin{info}{فرض کنید افراز \(X/\mathcal{P}\) همان افراز به‌دست آمده در مسئله ۲ باشد (یعنی \(\{ \{a,b\}, \{c,d\} \}\)).}
رابطه هم‌ارزی متناظر با این افراز را روی \(X\) بیابید.
\end{info}
\subsection{۲. استراتژی
حل}\label{ux627ux633ux62aux631ux627ux62aux698ux6cc-ux62dux644}
این تمرین نکته ظریفی دارد: در تمرین ۲، ما از رابطه \(\mathcal{R}\) به
افراز رسیدیم. حالا می‌خواهد از افراز دوباره رابطه را بسازیم. طبق
\textbf{قضیه اصلی رابطه‌های هم‌ارزی}، تناظر بین افرازها و روابط هم‌ارزی
یک‌به‌یک است. یعنی اگر از یک رابطه افراز بگیریم و از آن افراز دوباره رابطه
بسازیم، باید به همان رابطه اولیه برسیم.
\subsection{۳. حل
تشریحی}\label{ux62dux644-ux62aux634ux631ux6ccux62dux6cc}
\begin{info}{پاسخ}
افراز داده شده \(\mathcal{P} = \{ \{a,b\}, \{c,d\} \}\) است. رابطه
متناظر با این افراز (\(R_\mathcal{P}\)) شامل تمام زوج‌هایی است که
مولفه‌هایشان در ``یک بسته'' قرار دارند.
۱. \textbf{بسته اول \(\{a,b\}\):} حاصل‌ضرب دکارتی این بسته در خودش:
\[\{(a,a), (a,b), (b,a), (b,b)\}\]
۲. \textbf{بسته دوم \(\{c,d\}\):} حاصل‌ضرب دکارتی این بسته در خودش:
\[\{(c,c), (c,d), (d,c), (d,d)\}\]
\textbf{نتیجه:} رابطه نهایی اجتماع این دو مجموعه است:
\[\{(a,a), (b,b), (a,b), (b,a), (c,c), (d,d), (c,d), (d,c)\}\] مشاهده
می‌کنیم که این دقیقاً همان رابطه \(\mathcal{R}\) در تمرین ۲ است.
\end{info}
