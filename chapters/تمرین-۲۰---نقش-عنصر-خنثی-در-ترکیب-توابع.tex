% ---------------------------------------------------------------------
% Copyright (c) 2026 Arsalan Dalvand & Reyhaneh Darvishi.
% Licensed under CC BY-NC-SA 4.0.
% See LICENSE file for details.
% ---------------------------------------------------------------------

\section{تمرین ۲۰: تابع همانی به عنوان عنصر
خنثی}\label{تمرین-۲۰---نقش-عنصر-خنثی-در-ترکیب-توابع}
\begin{tldr}{خلاصه سریع}
تابع همانی (\(I(x)=x\)) در دنیای توابع، نقش عدد «۱» در ضرب را دارد.
ترکیب هر تابع با تابع همانی، خودِ آن تابع را نتیجه می‌دهد.
\[f \circ I_X = f = I_Y \circ f\]
\end{tldr}
\subsection{۱. صورت
تمرین}\label{ux635ux648ux631ux62a-ux62aux645ux631ux6ccux646}
\begin{info}{سوال}
فرض کنید \(f: X \rightarrow Y\) یک تابع باشد. ثابت کنید ترکیب \(f\) با
توابع همانیِ دامنه (\(I_X\)) و هم‌دامنه (\(I_Y\))، خود تابع \(f\) می‌شود.
\end{info}
\subsection{۲. درک
شهودی}\label{ux62fux631ux6a9-ux634ux647ux648ux62fux6cc}
\begin{itemize}
\tightlist
\item
  \(f \circ I_X\): یعنی اول «هیچ کاری روی ورودی نکن» (\(I_X\))، بعد
  \(f\) را اعمال کن. نتیجه: همان \(f\) اعمال شده.
\item
  \(I_Y \circ f\): یعنی اول \(f\) را اعمال کن، بعد روی خروجی «هیچ کاری
  نکن» (\(I_Y\)). نتیجه: همان خروجی \(f\).
\end{itemize}
\subsection{۳. اثبات
دقیق}\label{ux627ux62bux628ux627ux62a-ux62fux642ux6ccux642}
\begin{info}{اثبات}
برای اثبات برابری دو تابع، باید نشان دهیم برای هر \(x\)، خروجی‌ها یکسان
است.
\textbf{بخش اول (\(f \circ I_X = f\)):} برای هر \(x \in X\):
\[(f \circ I_X)(x) = f(I_X(x))\] چون \(I_X(x) = x\) (تعریف تابع همانی):
\[= f(x)\] \(\therefore f \circ I_X = f\)
\textbf{بخش دوم (\(I_Y \circ f = f\)):} برای هر \(x \in X\)، می‌دانیم
\(y = f(x) \in Y\): \[(I_Y \circ f)(x) = I_Y(f(x))\] چون برای هر عضو در
\(Y\) (مثل \(f(x)\))، تابع همانی آن را تغییر نمی‌دهد: \[= f(x)\]
\(\therefore I_Y \circ f = f\)
\end{info}
