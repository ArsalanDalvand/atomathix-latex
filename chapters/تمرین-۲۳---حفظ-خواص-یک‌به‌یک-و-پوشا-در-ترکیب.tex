% ---------------------------------------------------------------------
% Copyright (c) 2026 Arsalan Dalvand & Reyhaneh Darvishi.
% Licensed under CC BY-NC-SA 4.0.
% See LICENSE file for details.
% ---------------------------------------------------------------------

\section{تمرین ۲۳: انتقال خواص در ترکیب
توابع}\label{تمرین-۲۳---حفظ-خواص-یک‌به‌یک-و-پوشا-در-ترکیب}
\begin{tldr}{خلاصه سریع}
\begin{itemize}
\tightlist
\item
  ترکیبِ توابع \textbf{یک‌به‌یک}، حتماً \textbf{یک‌به‌یک} است.
\item
  ترکیبِ توابع \textbf{پوشا}، حتماً \textbf{پوشا} است.
\item
  (و در نتیجه: ترکیب توابع دوسویی، دوسویی است).
\end{itemize}
\end{tldr}
\subsection{۱. صورت
تمرین}\label{ux635ux648ux631ux62a-ux62aux645ux631ux6ccux646}
\begin{info}{سوال}
فرض کنید \(f: X \rightarrow Y\) و \(g: Y \rightarrow Z\). الف) اگر
\(f, g\) یک‌به‌یک باشند، ثابت کنید \(g \circ f\) یک‌به‌یک است. ب) اگر
\(f, g\) پوشا باشند، ثابت کنید \(g \circ f\) پوشا است.
\end{info}
\subsection{۲. اثبات (الف) -
یک‌به‌یک}\label{ux627ux62bux628ux627ux62a-ux627ux644ux641---ux6ccux6a9ux628ux647ux6ccux6a9}
\begin{info}{اثبات}
فرض کنید \((g \circ f)(x_1) = (g \circ f)(x_2)\). باید ثابت کنیم
\(x_1 = x_2\).
\begin{enumerate}
\def\labelenumi{\arabic{enumi}.}
\tightlist
\item
  طبق تعریف: \(g(f(x_1)) = g(f(x_2))\).
\item
  چون \(g\) \textbf{یک‌به‌یک} است، می‌توانیم \(g\) را از طرفین برداریم:
  \[f(x_1) = f(x_2)\]
\item
  حالا چون \(f\) \textbf{یک‌به‌یک} است، می‌توانیم \(f\) را برداریم:
  \[x_1 = x_2\] \(\blacksquare\)
\end{enumerate}
\end{info}
\subsection{۳. اثبات (ب) -
پوشا}\label{ux627ux62bux628ux627ux62a-ux628---ux67eux648ux634ux627}
\begin{info}{اثبات}
باید ثابت کنیم بردِ تابع ترکیبی، کل مجموعه \(Z\) است
(\((g \circ f)(X) = Z\)).
\begin{enumerate}
\def\labelenumi{\arabic{enumi}.}
\tightlist
\item
  چون \(g\) \textbf{پوشا} است، برد آن کل \(Z\) است (\(g(Y) = Z\)).
\item
  چون \(f\) \textbf{پوشا} است، برد آن کل \(Y\) است (\(f(X) = Y\)).
\item
  حال برد تابع مرکب را حساب می‌کنیم: \[(g \circ f)(X) = g(f(X))\]
\item
  جایگذاری رابطه (۲): \[= g(Y)\]
\item
  جایگذاری رابطه (۱): \[= Z\] \(\blacksquare\)
\end{enumerate}
\end{info}
