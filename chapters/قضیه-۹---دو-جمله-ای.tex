% ---------------------------------------------------------------------
% Copyright (c) 2026 Arsalan Dalvand & Reyhaneh Darvishi.
% Licensed under CC BY-NC-SA 4.0.
% See LICENSE file for details.
% ---------------------------------------------------------------------

\section{\texorpdfstring{قضیه ۹: قضیه دوجمله‌ای
\lr{(Binomial Theorem)}}{قضیه ۹: قضیه دوجمله‌ای }}\label{قضیه-۹---دو-جمله-ای}
\begin{tldr}{خلاصه سریع}
این قضیه الگوی باز کردن اتحاد \((x+y)^n\) را برای هر توان طبیعی \(n\)
بیان می‌کند. ضرایب هر جمله، همان اعداد ترکیب (\(C(n,r)\)) هستند که در
قضیه ۸ معرفی شدند.
\end{tldr}
\subsection{۱. متن ریاضی
قضیه}\label{ux645ux62aux646-ux631ux6ccux627ux636ux6cc-ux642ux636ux6ccux647}
اگر \(x\) و \(y\) دو متغیر و \(n\) یک عدد طبیعی باشد، آنگاه:
\begin{theorembox}{قضیه ۹}
\[(x+y)^n = C(n,0)x^n + C(n,1)x^{n-1}y + \dots + C(n,r)x^{n-r}y^r + \dots + C(n,n)y^n\]
یا به فرم سیگما (جمع): \[(x+y)^n = \sum_{r=0}^{n} C(n,r) x^{n-r} y^r\]
\end{theorembox}
\subsection{۲. اثبات (با استفاده از
استقرا)}\label{ux627ux62bux628ux627ux62a-ux628ux627-ux627ux633ux62aux641ux627ux62fux647-ux627ux632-ux627ux633ux62aux642ux631ux627}
از روش \textbf{\autoref{مفهوم-استقرا}} استفاده می‌کنیم:
\begin{info}{مراحل اثبات}
\textbf{۱. پایه استقرا (\(n=1\)):}
\[(x+y)^1 = C(1,0)x + C(1,1)y = 1x + 1y = x+y\] (حکم برقرار است).
\textbf{۲. فرض استقرا (\(n=k\)):} فرض می‌کنیم حکم برای \(k\) درست باشد:
\[(x+y)^k = \sum_{r=0}^{k} C(k,r)x^{k-r}y^r\]
\textbf{۳. گام استقرا (\(n=k+1\)):} طرفین فرض را در \((x+y)\) ضرب
می‌کنیم: \[(x+y)^{k+1} = (x+y) [C(k,0)x^k + \dots + C(k,k)y^k]\] با ضرب
کردن \(x\) در تمام جملات و سپس \(y\) در تمام جملات و فاکتورگیری از
توان‌های مشابه \(x^{k+1-r}y^r\)، ضرایب به صورت مجموع دو جمله قبلی ظاهر
می‌شوند: \[C(k+1, r) = C(k, r) + C(k, r-1)\] (این همان تعریف بازگشتی
\textbf{\autoref{قضیه-۸---ترکیب}} است). بدین ترتیب فرمول برای \(k+1\)
ساخته می‌شود.
\end{info}
\subsection{\texorpdfstring{۳. شبکه ارتباطی با سایر قضایا
\lr{(Analytic Map)}}{۳. شبکه ارتباطی با سایر قضایا }}\label{ux634ux628ux6a9ux647-ux627ux631ux62aux628ux627ux637ux6cc-ux628ux627-ux633ux627ux6ccux631-ux642ux636ux627ux6ccux627-analytic-map}
\subsubsection{\texorpdfstring{۱. ارتباط با
\autoref{قضیه-۸---ترکیب}}{۱. ارتباط با }}\label{ux627ux631ux62aux628ux627ux637-ux628ux627-ux642ux636ux6ccux647-ux6f8---ux62aux631ux6a9ux6ccux628}
\begin{itemize}
\tightlist
\item
  \textbf{تامین ضرایب:} قضیه دوجمله‌ای بدون قضیه ۸ معنا ندارد. قضیه ۸
  ابزار محاسبه‌ی «وزن» هر جمله در بسط دوجمله‌ای است.
\end{itemize}
\subsubsection{\texorpdfstring{۲. ارتباط با
\autoref{مفهوم-استقرا}}{۲. ارتباط با }}\label{ux627ux631ux62aux628ux627ux637-ux628ux627-ux645ux641ux647ux648ux645-ux627ux633ux62aux642ux631ux627}
\begin{itemize}
\tightlist
\item
  \textbf{کاربرد عملی:} قضیه ۹ یکی از کلاسیک‌ترین و مهم‌ترین مثال‌های
  کاربرد استقرای ریاضی در جبر است. این نشان می‌دهد که چگونه ابزار منطقی
  (استقرا) برای اثبات حقایق جبری (اتحادها) به کار می‌رود.
\end{itemize}
