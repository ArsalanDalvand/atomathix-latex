% ---------------------------------------------------------------------
% Copyright (c) 2026 Arsalan Dalvand & Reyhaneh Darvishi.
% Licensed under CC BY-NC-SA 4.0.
% See LICENSE file for details.
% ---------------------------------------------------------------------

\section{قضیه ۴: تولید افراز توسط رابطه
هم‌ارزی}\label{قضیه-۴---افراز-ناشی-از-رابطه-هم‌ارزی}
\begin{tldr}{خلاصه سریع}
این قضیه بیان می‌کند که هر «رابطه هم‌ارزی» به طور طبیعی مجموعه را تکه‌تکه
(افراز) می‌کند. به عبارت دیگر، کلاس‌های هم‌ارزی همان قطعات پازلی هستند که
بدون هم‌پوشانی، کل مجموعه را می‌سازند.
\end{tldr}
\subsection{۱. تعاریف بنیادین
(پیش‌نیاز)}\label{ux62aux639ux627ux631ux6ccux641-ux628ux646ux6ccux627ux62fux6ccux646-ux67eux6ccux634ux646ux6ccux627ux632}
برای درک عمیق این قضیه، باید تعریف دقیق \textbf{افراز} را مرور کنیم.
\subsubsection{\texorpdfstring{تعریف افراز
\lr{(Partition)}}{تعریف افراز }}\label{ux62aux639ux631ux6ccux641-ux627ux641ux631ux627ux632-partition}
فرض کنید \(X\) یک مجموعه ناتهی باشد. خانواده‌ای از زیرمجموعه‌های \(X\)
مانند \(\mathcal{P} = \{A_\gamma\}_{\gamma \in \Gamma}\) را یک
\textbf{افراز} برای \(X\) می‌نامیم اگر سه شرط زیر برقرار باشد:
\begin{enumerate}
\def\labelenumi{\arabic{enumi}.}
\tightlist
\item
  \textbf{ناتهی بودن:} هیچ‌کدام از مجموعه‌ها خالی نباشند
  (\(\forall A \in \mathcal{P}, A \neq \emptyset\)).
\item
  \textbf{مجزا بودن \lr{(Disjointness):}} هیچ دو مجموعه متمایزی اشتراک
  نداشته باشند (\(A_i \neq A_j \implies A_i \cap A_j = \emptyset\)).
\item
  \textbf{پوشش کامل \lr{(Covering):}} اجتماع تمام مجموعه‌ها برابر با کل
  \(X\) باشد (\(\bigcup A_\gamma = X\)).
\end{enumerate}
\begin{center}\rule{0.5\linewidth}{0.5pt}\end{center}
\subsection{۲. متن ریاضی
قضیه}\label{ux645ux62aux646-ux631ux6ccux627ux636ux6cc-ux642ux636ux6ccux647}
فرض کنید \(\xi\) یک رابطه هم‌ارزی روی مجموعه ناتهی \(X\) باشد. مجموعه
خارج‌قسمتی \(X/\xi\) (شامل تمام کلاس‌های هم‌ارزی) یک \textbf{افراز} برای
\(X\) است.
\begin{theorembox}{قضیه ۴}
اگر \(\xi\) یک رابطه هم‌ارزی روی \(X\) باشد، آنگاه \(X/\xi\) یک افراز
برای \(X\) است.
\end{theorembox}
\begin{center}\rule{0.5\linewidth}{0.5pt}\end{center}
\subsection{\texorpdfstring{۳. اثبات صوری
\lr{(Formal Proof)}}{۳. اثبات صوری }}\label{ux627ux62bux628ux627ux62a-ux635ux648ux631ux6cc-formal-proof}
برای اثبات اینکه \(X/\xi\) یک افراز است، باید نشان دهیم سه شرط تعریف
افراز (بخش ۱) را دارد. ما از
\textbf{\autoref{قضیه-۳---ویژگی‌های-بنیادی-کلاس-هم‌ارزی}} به عنوان ابزار
اصلی استفاده می‌کنیم.
\begin{info}{برهان}
\textbf{گام ۱: بررسی شرط ناتهی بودن} طبق
\autoref{قضیه-۳---ویژگی‌های-بنیادی-کلاس-هم‌ارزی}، برای هر \(x \in X\)،
کلاس \(x/\xi\) یک زیرمجموعه ناتهی از \(X\) است (حداقل شامل خود \(x\)
است).
\textbf{گام ۲: بررسی شرط مجزا بودن} باید ثابت کنیم:
\(x/\xi \neq y/\xi \implies (x/\xi) \cap (y/\xi) = \emptyset\). از برهان
خلف استفاده می‌کنیم (یا عکس نقیض قضیه ۳-ب): طبق
\autoref{قضیه-۳---ویژگی‌های-بنیادی-کلاس-هم‌ارزی}، اگر اشتراک دو کلاس تهی
نباشد (\(x/\xi \cap y/\xi \neq \emptyset\))، آنگاه \(x \xi y\). طبق
\autoref{قضیه-۳---ویژگی‌های-بنیادی-کلاس-هم‌ارزی}، اگر \(x \xi y\)، آنگاه
\(x/\xi = y/\xi\). بنابراین، اگر دو کلاس برابر نباشند، محال است اشتراک
داشته باشند.
\textbf{گام ۳: بررسی شرط پوشش} باید ثابت کنیم
\(\bigcup_{x \in X} x/\xi = X\).
\begin{itemize}
\tightlist
\item
  (شمول \(\subseteq\)): واضح است که هر کلاس زیرمجموعه \(X\) است، پس
  اجتماعشان هم زیرمجموعه \(X\) است.
\item
  (شمول \(\supseteq\)): هر عضو دلخواه \(y \in X\) را در نظر بگیرید.
  می‌دانیم هر عضو متعلق به کلاس خودش است (\(y \in y/\xi\)). پس \(y\) در
  اجتماع کلاس‌ها حضور دارد.
\end{itemize}
\textbf{نتیجه:} هر سه شرط برقرار است، پس \(X/\xi\) یک افراز است.
\end{info}
\begin{center}\rule{0.5\linewidth}{0.5pt}\end{center}
\subsection{\texorpdfstring{۴. شبکه ارتباطی با سایر قضایا
\lr{(Analytic Map)}}{۴. شبکه ارتباطی با سایر قضایا }}\label{ux634ux628ux6a9ux647-ux627ux631ux62aux628ux627ux637ux6cc-ux628ux627-ux633ux627ux6ccux631-ux642ux636ux627ux6ccux627-analytic-map}
\subsubsection{\texorpdfstring{۱. وابستگی به
\autoref{قضیه-۳---ویژگی‌های-بنیادی-کلاس-هم‌ارزی}}{۱. وابستگی به }}\label{ux648ux627ux628ux633ux62aux6afux6cc-ux628ux647-ux642ux636ux6ccux647-ux6f3---ux648ux6ccux698ux6afux6ccux647ux627ux6cc-ux628ux646ux6ccux627ux62fux6cc-ux6a9ux644ux627ux633-ux647ux645ux627ux631ux632ux6cc}
\begin{itemize}
\tightlist
\item
  این قضیه بدون قضیه ۳ قابل اثبات نیست. قضیه ۳ ``لم'' \lr{(Lemma) }یا
  ابزار تکنیکی است که تضمین می‌کند کلاس‌ها یا ``یکی هستند'' یا ``کاملاً
  جدا''. این خاصیت دوگانه \lr{(Dichotomy) }جوهره‌ی اصلی افراز است.
\end{itemize}
\subsubsection{\texorpdfstring{۲. هم‌ارزی با
\autoref{قضیه-۵---رابطه-هم‌ارزی-ناشی-از-افراز}}{۲. هم‌ارزی با }}\label{ux647ux645ux627ux631ux632ux6cc-ux628ux627-ux642ux636ux6ccux647-ux6f5---ux631ux627ux628ux637ux647-ux647ux645ux627ux631ux632ux6cc-ux646ux627ux634ux6cc-ux627ux632-ux627ux641ux631ux627ux632}
\begin{itemize}
\tightlist
\item
  قضیه ۴ مسیر ``رابطه \(\to\) افراز'' را طی می‌کند. قضیه ۵ مسیر عکس آن،
  یعنی ``افراز \(\to\) رابطه'' را طی می‌کند. این دو قضیه با هم نشان
  می‌دهند که مفاهیم ``رابطه هم‌ارزی'' و ``افراز'' از نظر منطقی
  \textbf{هم‌ارز} \lr{(Equivalent) }هستند (دو روی یک سکه).
\end{itemize}
\subsubsection{۳. کاربرد در جبر (فصل‌های
آینده)}\label{ux6a9ux627ux631ux628ux631ux62f-ux62fux631-ux62cux628ux631-ux641ux635ux644ux647ux627ux6cc-ux622ux6ccux646ux62fux647}
\begin{itemize}
\tightlist
\item
  در جبر مجرد، این قضیه زیربنای ساختن \(Z_n\) (اعداد صحیح به پیمانه
  \(n\)) است. رابطه ``همنهشتی'' اعداد را افراز می‌کند و ما با این قطعات
  افراز (کلاس‌ها) به عنوان اعداد جدید کار می‌کنیم.
\end{itemize}
