% ---------------------------------------------------------------------
% Copyright (c) 2026 Arsalan Dalvand & Reyhaneh Darvishi.
% Licensed under CC BY-NC-SA 4.0.
% See LICENSE file for details.
% ---------------------------------------------------------------------

\section{\texorpdfstring{تمرین ۱: خاصیت وارونِ وارون
(\(\mathcal{R}^{-1})^{-1} = \mathcal{R}\))}{تمرین ۱: خاصیت وارونِ وارون (\textbackslash mathcal\{R\}\^{}\{-1\})\^{}\{-1\} = \textbackslash mathcal\{R\})}}\label{تمرین-۱---وارونِ-وارون-رابطه}
\begin{tldr}{خلاصه سریع}
اگر جای مؤلفه‌های یک رابطه را دو بار عوض کنیم، به حالت اول برمی‌گردد. مثل
اینکه یک لباس را پشت‌رو کنید و دوباره پشت‌رو کنید؛ به حالت اصلی برمی‌گردد.
\end{tldr}
\subsection{۱. صورت
سوال}\label{ux635ux648ux631ux62a-ux633ux648ux627ux644}
فرض کنید \(\mathcal{R}\) رابطه‌ای از \(A\) به \(B\) است. ثابت کنید:
\[(\mathcal{R}^{-1})^{-1} = \mathcal{R}\]
\subsection{۲. اثبات
دقیق}\label{ux627ux62bux628ux627ux62a-ux62fux642ux6ccux642}
برای اثبات تساوی دو مجموعه، نشان می‌دهیم عضو بودن در یکی معادل عضو بودن
در دیگری است.
\begin{info}{مراحل اثبات}
\[(x,y) \in \mathcal{R}\] طبق تعریف وارون رابطه
(\((x,y) \in \mathcal{R} \iff (y,x) \in \mathcal{R}^{-1}\)):
\[\iff (y,x) \in \mathcal{R}^{-1}\] حالا دوباره تعریف وارون را روی
\(\mathcal{R}^{-1}\) اعمال می‌کنیم:
\[\iff (x,y) \in (\mathcal{R}^{-1})^{-1}\] پس دو مجموعه برابرند.
\end{info}
