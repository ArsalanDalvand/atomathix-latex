% ---------------------------------------------------------------------
% Copyright (c) 2026 Arsalan Dalvand & Reyhaneh Darvishi.
% Licensed under CC BY-NC-SA 4.0.
% See LICENSE file for details.
% ---------------------------------------------------------------------

\section{\texorpdfstring{قضیه: رابطه مجموعه \lr{A }و نگاره وارونِ نگاره
آن}{قضیه: رابطه مجموعه و نگاره وارونِ نگاره آن}}\label{تمرین-۹-(الف)---زیرمجموعه-بودن-A-در-وارون-تصویر-A}
\begin{tldr}{خلاصه سریع}
اگر یک مجموعه را با تابع \(f\) بفرستیم به مقصد و دوباره با \(f^{-1}\)
برگردانیم، لزوماً به همان مجموعه اولیه نمی‌رسیم؛ بلکه ممکن است مجموعه‌ای
\textbf{بزرگتر} شود که مجموعه اصلی ما زیرمجموعه آن است.
\[A \subseteq f^{-1}(f(A))\]
\end{tldr}
\subsection{۱. درک
شهودی}\label{ux62fux631ux6a9-ux634ux647ux648ux62fux6cc}
فرض کن \(A\) مجموعه‌ای از دانش‌آموزان یک کلاس خاص در مدرسه است و تابع
\(f\) به هر دانش‌آموز، «شماره پلاک منزل» او را نسبت می‌دهد.
\begin{enumerate}
\def\labelenumi{\arabic{enumi}.}
\tightlist
\item
  \(f(A)\): می‌شود مجموعه پلاک‌های این دانش‌آموزان.
\item
  \(f^{-1}(f(A))\): یعنی تمام کسانی در کل شهر (دامنه \(X\)) که پلاکشان
  جزو لیست بالا باشد.
\end{enumerate}
ممکن است در شهر، افراد دیگری (خارج از کلاس \(A\)) هم باشند که پلاکشان
مشابه یکی از بچه‌های کلاس باشد (اگر تابع یک‌به‌یک نباشد). پس وقتی
برمی‌گردیم، جمعیت ما برابر یا بزرگتر از کلاس \(A\) خواهد بود.
\subsection{۲. صورت
ریاضی}\label{ux635ux648ux631ux62a-ux631ux6ccux627ux636ux6cc}
\begin{theorembox}{قضیه}
فرض کنید \(f:X \rightarrow Y\) یک تابع باشد و \(A \subseteq X\). در این
صورت: \[A \subseteq f^{-1}(f(A))\]
\end{theorembox}
\subsection{۳. اثبات
دقیق}\label{ux627ux62bux628ux627ux62a-ux62fux642ux6ccux642}
\begin{info}{اثبات}
می‌خواهیم نشان دهیم هر عضوی که در \(A\) است، در طرف راست هم هست.
\begin{enumerate}
\def\labelenumi{\arabic{enumi}.}
\tightlist
\item
  فرض کنید \(x\) عضوی دلخواه از \(A\) باشد (\(x \in A\)).
\item
  طبق تعریف تابع، تصویر این عضو یعنی \(f(x)\) باید در مجموعه تصاویر
  \(A\) باشد: \[f(x) \in f(A)\]
\item
  حال به تعریف \textbf{نگاره وارون} دقت کنید:
  \(f^{-1}(S) = \{x \in X | f(x) \in S\}\).
\item
  چون \(f(x)\) متعلق به مجموعه \(f(A)\) است، پس خودِ \(x\) باید متعلق به
  وارونِ آن مجموعه باشد: \[x \in f^{-1}(f(A))\]
\item
  \textbf{نتیجه:} چون \(x\) نماینده هر عضو دلخواه \(A\) بود، پس:
  \[A \subseteq f^{-1}(f(A))\]
\end{enumerate}
\end{info}
\begin{info}{نکته تکمیلی}
تساوی \(A = f^{-1}(f(A))\) تنها زمانی برقرار است که تابع \(f\)
\textbf{یک‌به‌یک \lr{(Injective)}} باشد.
\end{info}
