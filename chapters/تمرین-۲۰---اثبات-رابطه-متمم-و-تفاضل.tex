% ---------------------------------------------------------------------
% Copyright (c) 2026 Arsalan Dalvand & Reyhaneh Darvishi.
% Licensed under CC BY-NC-SA 4.0.
% See LICENSE file for details.
% ---------------------------------------------------------------------

\section{تمرین ۲۰: بازنویسی مجموعه بر اساس
تفاضل}\label{تمرین-۲۰---اثبات-رابطه-متمم-و-تفاضل}
\begin{tldr}{خلاصه سریع}
اگر مجموعه‌ی بزرگ \(C\) به دو تکه جداگانه \(A\) و \(B\) تقسیم شده باشد
(افراز شده باشد)، آنگاه \(A\) دقیقاً همان «کل منهای \lr{B» }است. شرط‌ها:
\(A \cup B = C\) و \(A \cap B = \emptyset\).
\end{tldr}
\subsection{۱. صورت
مسأله}\label{ux635ux648ux631ux62a-ux645ux633ux623ux644ux647}
\begin{info}{سوال}
ثابت کنید اگر \(A \subseteq C\) و \(B \subseteq C\) و داشته باشیم
\(A \cup B = C\) و \(A \cap B = \emptyset\)، آنگاه: \[A = C - B\]
\end{info}
\subsection{۲. اثبات دو
طرفه}\label{ux627ux62bux628ux627ux62a-ux62fux648-ux637ux631ux641ux647}
\begin{info}{اثبات}
\textbf{مسیر رفت (\(A \subseteq C - B\)):}
\begin{enumerate}
\def\labelenumi{\arabic{enumi}.}
\tightlist
\item
  فرض کنید \(x \in A\).
\item
  چون \(A \subseteq C\)، پس \(x \in C\).
\item
  چون \(A \cap B = \emptyset\) (اشتراک ندارند)، و \(x\) در \(A\) است، پس
  قطعاً \(x \notin B\).
\item
  از (۲) و (۳) داریم: \(x \in C\) و \(x \notin B\).
\item
  طبق تعریف تفاضل: \(x \in C - B\).
\end{enumerate}
\textbf{مسیر برگشت (\(C - B \subseteq A\)):} 6. فرض کنید
\(x \in C - B\). 7. یعنی \(x \in C\) و \(x \notin B\). 8. طبق فرض مسأله
\(C = A \cup B\). پس چون \(x \in C\)، باید یا در \(A\) باشد یا در \(B\).
9. اما می‌دانیم \(x \notin B\) (از مرحله ۲). 10. پس تنها گزینه باقی‌مانده
این است که \(x \in A\).
\textbf{نتیجه:} چون هر دو طرف زیرمجموعه هم شدند، پس \(A = C - B\).
\(\blacksquare\)
\end{info}
