% ---------------------------------------------------------------------
% Copyright (c) 2026 Arsalan Dalvand & Reyhaneh Darvishi.
% Licensed under CC BY-NC-SA 4.0.
% See LICENSE file for details.
% ---------------------------------------------------------------------

\section{قضیه ۱۶: وارون‌های یک‌طرفه (چپ و
راست)}\label{قضیه-۱۶---وارون‌های-یک‌طرفه}
\begin{tldr}{خلاصه سریع}
این قضیه ارتباط عمیق بین «ساختار جبری» (ترکیب توابع) و «ویژگی‌های نگاشتی»
(یک‌به‌یک و پوشا بودن) را نشان می‌دهد.
\begin{itemize}
\tightlist
\item
  اگر بتوان اثر تابع را از \textbf{چپ} خنثی کرد \(\implies\) تابع
  \textbf{یک‌به‌یک} است.
\item
  اگر بتوان اثر تابع را از \textbf{راست} خنثی کرد \(\implies\) تابع
  \textbf{پوشا} است.
\end{itemize}
\end{tldr}
\subsection{۱. متن ریاضی
قضیه}\label{ux645ux62aux646-ux631ux6ccux627ux636ux6cc-ux642ux636ux6ccux647}
فرض کنید \(f: X \to Y\) یک تابع باشد.
\begin{theorembox}{قضیه ۱۶}
\textbf{الف) شرط یک‌به‌یک بودن (وارون چپ):} اگر تابعی مانند \(g: Y \to X\)
وجود داشته باشد که \(g \circ f = I_X\)، آنگاه \(f\) \textbf{یک‌به‌یک}
\lr{(Injective) }است.
\textbf{ب) شرط پوشا بودن (وارون راست):} اگر تابعی مانند \(h: Y \to X\)
وجود داشته باشد که \(f \circ h = I_Y\)، آنگاه \(f\) \textbf{پوشا}
\lr{(Surjective) }است.
\end{theorembox}
\emph{(تذکر: \(I_X\) و \(I_Y\) توابع همانی روی دامنه‌های مربوطه هستند).}
\subsection{\texorpdfstring{۲. اثبات صوری
\lr{(Formal Proof)}}{۲. اثبات صوری }}\label{ux627ux62bux628ux627ux62a-ux635ux648ux631ux6cc-formal-proof}
\subsubsection{\texorpdfstring{اثبات قسمت (الف): وارون چپ \(\implies\)
یک‌به‌یک}{اثبات قسمت (الف): وارون چپ \textbackslash implies یک‌به‌یک}}\label{ux627ux62bux628ux627ux62a-ux642ux633ux645ux62a-ux627ux644ux641-ux648ux627ux631ux648ux646-ux686ux67e-implies-ux6ccux6a9ux628ux647ux6ccux6a9}
برای اثبات یک‌به‌یک بودن، باید نشان دهیم
\(f(x_1) = f(x_2) \implies x_1 = x_2\).
\begin{info}{برهان}
۱. فرض کنید \(x_1, x_2 \in X\) باشند و تصاویرشان برابر باشد:
\[f(x_1) = f(x_2)\] ۲. روی طرفین تساوی، تابع \(g\) را اعمال می‌کنیم (چون
\(g\) تابع است، خروجی‌های یکسان برای ورودی‌های یکسان دارد):
\[g(f(x_1)) = g(f(x_2))\] ۳. طبق تعریف ترکیب توابع:
\[(g \circ f)(x_1) = (g \circ f)(x_2)\] ۴. طبق فرض قضیه
(\(g \circ f = I_X\)): \[I_X(x_1) = I_X(x_2)\] ۵. طبق تعریف تابع همانی
(\(I_X(x) = x\)): \[x_1 = x_2\]
\textbf{نتیجه:} تابع \(f\) یک‌به‌یک است.
\end{info}
\subsubsection{\texorpdfstring{اثبات قسمت (ب): وارون راست \(\implies\)
پوشا}{اثبات قسمت (ب): وارون راست \textbackslash implies پوشا}}\label{ux627ux62bux628ux627ux62a-ux642ux633ux645ux62a-ux628-ux648ux627ux631ux648ux646-ux631ux627ux633ux62a-implies-ux67eux648ux634ux627}
برای اثبات پوشا بودن، باید نشان دهیم برای هر \(y \in Y\)، پیش‌نگاره‌ای
وجود دارد.
\begin{info}{برهان}
۱. فرض کنید \(y\) عضوی دلخواه از هم‌دامنه \(Y\) باشد (\(y \in Y\)). ۲. ما
به دنبال یک \(x \in X\) هستیم که \(f(x) = y\). ۳. عنصر \(x\) را به صورت
\(x = h(y)\) تعریف می‌کنیم. (چون \(h: Y \to X\) است، پس \(x \in X\)
می‌باشد). ۴. حال مقدار تابع \(f\) را در این نقطه محاسبه می‌کنیم:
\[f(x) = f(h(y))\] ۵. طبق تعریف ترکیب توابع: \[= (f \circ h)(y)\] ۶. طبق
فرض قضیه (\(f \circ h = I_Y\)): \[= I_Y(y)\] ۷. طبق تعریف تابع همانی:
\[= y\]
\textbf{نتیجه:} برای هر \(y\)، یک \(x\) (همان \(h(y)\)) یافت شد که
\(f(x)=y\). پس \(f\) پوشا است.
\end{info}
\subsection{\texorpdfstring{۳. شبکه ارتباطی با سایر قضایا
\lr{(Analytic Map)}}{۳. شبکه ارتباطی با سایر قضایا }}\label{ux634ux628ux6a9ux647-ux627ux631ux62aux628ux627ux637ux6cc-ux628ux627-ux633ux627ux6ccux631-ux642ux636ux627ux6ccux627-analytic-map}
\subsubsection{\texorpdfstring{۱. تکمیل
\autoref{قضیه-۱۴---وجود-و-ویژگی‌های-تابع-وارون}}{۱. تکمیل }}\label{ux62aux6a9ux645ux6ccux644-ux642ux636ux6ccux647-ux6f1ux6f4---ux648ux62cux648ux62f-ux648-ux648ux6ccux698ux6afux6ccux647ux627ux6cc-ux62aux627ux628ux639-ux648ux627ux631ux648ux646}
\begin{itemize}
\tightlist
\item
  \textbf{تحلیل ساختاری:} قضیه ۱۴ بیان می‌کرد که اگر \(f\)
  \textbf{دوسویی} باشد، وارون‌پذیر است. قضیه ۱۶ این شرط را تجزیه می‌کند:
  \begin{itemize}
  \tightlist
  \item
    بخش «یک‌به‌یک» بودن \(f\) ناشی از وجود وارون چپ است.
  \item
    بخش «پوشا» بودن \(f\) ناشی از وجود وارون راست است.
  \item
    اگر \(f\) هم وارون چپ داشته باشد و هم راست (و این دو برابر باشند)،
    آنگاه وارون‌پذیر کامل (\(f^{-1}\)) است.
  \end{itemize}
\end{itemize}
\subsubsection{\texorpdfstring{۲. ارتباط با
\autoref{انواع-توابع---یک‌به‌یک-پوشا-و-دوسویی}}{۲. ارتباط با }}\label{ux627ux631ux62aux628ux627ux637-ux628ux627-ux627ux646ux648ux627ux639-ux62aux648ux627ux628ux639---ux6ccux6a9ux628ux647ux6ccux6a9-ux67eux648ux634ux627-ux648-ux62fux648ux633ux648ux6ccux6cc}
\begin{itemize}
\tightlist
\item
  \textbf{آزمون جبری:} این قضیه یک روش ``عملیاتی'' برای تست یک‌به‌یک یا
  پوشا بودن ارائه می‌دهد. به جای چک کردن تک‌تک اعضا (تعریف اصلی)، کافی است
  سعی کنیم تابعی بسازیم که اثر \(f\) را خنثی کند. این روش در معادلات
  دیفرانسیل و جبر خطی (معکوس ماتریس) بسیار کاربرد دارد.
\end{itemize}
\subsubsection{\texorpdfstring{۳. ارتباط با
\autoref{توابع-خاص---همانی-و-ثابت}}{۳. ارتباط با }}\label{ux627ux631ux62aux628ux627ux637-ux628ux627-ux62aux648ux627ux628ux639-ux62eux627ux635---ux647ux645ux627ux646ux6cc-ux648-ux62bux627ux628ux62a}
\begin{itemize}
\tightlist
\item
  \textbf{نقش عنصر خنثی:} بدون تعریف دقیق تابع همانی (\(I_X\)) و درک نقش
  آن به عنوان ``عنصر خنثی'' در ترکیب توابع، بیان این قضیه ممکن نیست.
  تفاوت \(I_X\) و \(I_Y\) در صورت قضیه بسیار حیاتی است (چون دامنه‌ها
  متفاوت‌اند).
\end{itemize}
