% ---------------------------------------------------------------------
% Copyright (c) 2026 Arsalan Dalvand & Reyhaneh Darvishi.
% Licensed under CC BY-NC-SA 4.0.
% See LICENSE file for details.
% ---------------------------------------------------------------------

\section{قضیه ۱۱: رفتار جبری تصویر وارون (هم‌ریختی
کامل)}\label{قضیه-۱۱---رفتار-جبری-تصویر-وارون}
\begin{tldr}{خلاصه سریع}
برخلاف «تصویر مستقیم» (\(f\)) که در برخورد با اشتراک دچار ضعف می‌شود،
«تصویر وارون» (\(f^{-1}\)) یک عملگر ایده‌آل است. این عملگر ساختار جبری
اجتماع و اشتراک را دقیقاً حفظ می‌کند و با هر دو عملگر جابجا می‌شود.
\end{tldr}
\subsection{۱. متن ریاضی
قضیه}\label{ux645ux62aux646-ux631ux6ccux627ux636ux6cc-ux642ux636ux6ccux647}
فرض کنید \(f: X \to Y\) یک تابع باشد و
\(\{B_\gamma\}_{\gamma \in \Gamma}\) خانواده‌ای از زیرمجموعه‌های هم‌دامنه
\(Y\) باشد.
\begin{theorembox}{قضیه ۱۱}
\textbf{الف) پخش‌پذیری کامل بر اجتماع:}
\[f^{-1}\left(\bigcup_{\gamma \in \Gamma} B_\gamma\right) = \bigcup_{\gamma \in \Gamma} f^{-1}(B_\gamma)\]
\textbf{ب) پخش‌پذیری کامل بر اشتراک:}
\[f^{-1}\left(\bigcap_{\gamma \in \Gamma} B_\gamma\right) = \bigcap_{\gamma \in \Gamma} f^{-1}(B_\gamma)\]
\end{theorembox}
\subsection{\texorpdfstring{۲. اثبات صوری
\lr{(Formal Proof)}}{۲. اثبات صوری }}\label{ux627ux62bux628ux627ux62a-ux635ux648ux631ux6cc-formal-proof}
اثبات این قضیه بر پایه «تعریف تصویر وارون» و «قوانین منطق مرتبه اول»
استوار است. زیبایی این اثبات در این است که تصویر وارون، گزاره‌های عضویت
را بدون تغییر ساختار منطقی منتقل می‌کند.
\subsubsection{اثبات قسمت (الف):
اجتماع}\label{ux627ux62bux628ux627ux62a-ux642ux633ux645ux62a-ux627ux644ux641-ux627ux62cux62aux645ux627ux639}
\begin{info}{برهان}
\[x \in f^{-1}\left(\bigcup_{\gamma \in \Gamma} B_\gamma\right)\] ۱. طبق
تعریف تصویر وارون (\(x \in f^{-1}(S) \iff f(x) \in S\)):
\[\iff f(x) \in \bigcup_{\gamma \in \Gamma} B_\gamma\]
۲. طبق تعریف اجتماع تعمیم‌یافته:
\[\iff \exists \gamma \in \Gamma, (f(x) \in B_\gamma)\]
۳. طبق تعریف تصویر وارون (بازگشت به دامنه):
\[\iff \exists \gamma \in \Gamma, (x \in f^{-1}(B_\gamma))\]
۴. طبق تعریف اجتماع تعمیم‌یافته:
\[\iff x \in \bigcup_{\gamma \in \Gamma} f^{-1}(B_\gamma)\]
\textbf{نتیجه:} تساوی برقرار است.
\end{info}
\subsubsection{اثبات قسمت (ب):
اشتراک}\label{ux627ux62bux628ux627ux62a-ux642ux633ux645ux62a-ux628-ux627ux634ux62aux631ux627ux6a9}
\begin{info}{برهان}
\[x \in f^{-1}\left(\bigcap_{\gamma \in \Gamma} B_\gamma\right)\] ۱. طبق
تعریف تصویر وارون:
\[\iff f(x) \in \bigcap_{\gamma \in \Gamma} B_\gamma\]
۲. طبق تعریف اشتراک تعمیم‌یافته:
\[\iff \forall \gamma \in \Gamma, (f(x) \in B_\gamma)\]
۳. طبق تعریف تصویر وارون:
\[\iff \forall \gamma \in \Gamma, (x \in f^{-1}(B_\gamma))\]
۴. طبق تعریف اشتراک تعمیم‌یافته:
\[\iff x \in \bigcap_{\gamma \in \Gamma} f^{-1}(B_\gamma)\]
\textbf{نتیجه:} تساوی برقرار است.
\end{info}
\subsection{\texorpdfstring{۳. شبکه ارتباطی با سایر قضایا
\lr{(Analytic Map)}}{۳. شبکه ارتباطی با سایر قضایا }}\label{ux634ux628ux6a9ux647-ux627ux631ux62aux628ux627ux637ux6cc-ux628ux627-ux633ux627ux6ccux631-ux642ux636ux627ux6ccux627-analytic-map}
\subsubsection{\texorpdfstring{۱. مقایسه با
\autoref{قضیه-۱۰---تعمیم-تصویر-اجتماع-و-اشتراک}}{۱. مقایسه با }}\label{ux645ux642ux627ux6ccux633ux647-ux628ux627-ux642ux636ux6ccux647-ux6f1ux6f0---ux62aux639ux645ux6ccux645-ux62aux635ux648ux6ccux631-ux627ux62cux62aux645ux627ux639-ux648-ux627ux634ux62aux631ux627ux6a9}
\begin{itemize}
\tightlist
\item
  \textbf{تفاوت بنیادین:} در قضیه ۱۰ دیدیم که
  \(f(\cap A_\gamma) \subseteq \cap f(A_\gamma)\) و تساوی لزوماً برقرار
  نیست. اما قضیه ۱۱ نشان می‌دهد که \(f^{-1}\) هیچ‌گونه اطلاعاتی را در
  فرایند اشتراک‌گیری از دست نمی‌دهد.
\item
  \textbf{علت منطقی:} دلیل این تفاوت در ساختار سورهاست. تعریف \(f(A)\)
  شامل یک سور وجودی (\(\exists x\)) است که با سور عمومی (\(\forall\)) در
  اشتراک سازگار نیست. اما تعریف \(f^{-1}(B)\) هیچ سور مخفی‌ای ندارد و
  صرفاً یک ترجمه شرطی (\(P(f(x))\)) است، لذا با همه سورها جابجا می‌شود.
\end{itemize}
\subsubsection{\texorpdfstring{۲. ارتباط با
\autoref{قضیه-۹---تصویر-و-تصویر-وارون-مجموعه}}{۲. ارتباط با }}\label{ux627ux631ux62aux628ux627ux637-ux628ux627-ux642ux636ux6ccux647-ux6f9---ux62aux635ux648ux6ccux631-ux648-ux62aux635ux648ux6ccux631-ux648ux627ux631ux648ux646-ux645ux62cux645ux648ux639ux647}
\begin{itemize}
\tightlist
\item
  \textbf{تعمیم:} قضیه ۱۱ تعمیم مستقیم قضیه ۹ (اگر آن را برای وارون
  بنویسیم) به حالت نامتناهی است. این نشان می‌دهد ویژگی‌های توپولوژیکی
  وارون تابع (مانند پیوستگی که با باز بودن وارون مجموعه‌های باز تعریف
  می‌شود) بسیار پایدارتر از تصویر مستقیم هستند.
\end{itemize}
