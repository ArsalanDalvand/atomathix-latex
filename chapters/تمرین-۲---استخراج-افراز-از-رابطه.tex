% ---------------------------------------------------------------------
% Copyright (c) 2026 Arsalan Dalvand & Reyhaneh Darvishi.
% Licensed under CC BY-NC-SA 4.0.
% See LICENSE file for details.
% ---------------------------------------------------------------------

\section{تمرین ۲: استخراج افراز از رابطه
هم‌ارزی}\label{تمرین-۲---استخراج-افراز-از-رابطه}
\subsection{۱. صورت
سوال}\label{ux635ux648ux631ux62a-ux633ux648ux627ux644}
\begin{info}{فرض کنید \(X=\{a,b,c,d\}\) و رابطه \(\mathcal{R}\) به صورت زیر داده شده است:}
\[\mathcal{R}=\{(a,b),(b,a),(a,a),(b,b),(c,d),(d,c),(c,c),(d,d)\}\]
\textbf{الف)} تحقیق کنید که \(\mathcal{R}\) یک رابطه هم‌ارزی روی \(X\)
است. \textbf{ب)} افراز \(X/\mathcal{R}\) را که از \(\mathcal{R}\) پدید
آمده است، بیابید.
\end{info}
\subsection{۲. استراتژی
حل}\label{ux627ux633ux62aux631ux627ux62aux698ux6cc-ux62dux644}
برای (الف) باید سه ویژگی رابطه هم ارزی را روی زوج‌مرتب‌ها چک کنیم:
\begin{enumerate}
\def\labelenumi{\arabic{enumi}.}
\tightlist
\item
  \textbf{بازتابی:} آیا \((x,x)\) برای همه اعضا هست؟
\item
  \textbf{تقارنی:} آیا هر جا \((x,y)\) هست، \((y,x)\) هم هست؟
\item
  \textbf{تعدی:} آیا زنجیره‌های \((x,y)\) و \((y,z)\) به \((x,z)\) ختم
  می‌شوند؟
\end{enumerate}
برای (ب) باید \textbf{\autoref{قضیه-۳---ویژگی‌های-بنیادی-کلاس-هم‌ارزی}} هر
عضو را پیدا کنیم (\([x] = \{y \mid x\mathcal{R}y\}\)). مجموعه این
کلاس‌های متمایز، افراز مطلوب است.
\subsection{۳. حل
تشریحی}\label{ux62dux644-ux62aux634ux631ux6ccux62dux6cc}
\begin{info}{پاسخ}
\textbf{قسمت (الف): بررسی ویژگی‌ها} ۱. \textbf{بازتابی:} اعضای \(X\)
عبارتند از \(a,b,c,d\). در رابطه \(\mathcal{R}\) زوج‌های
\((a,a), (b,b), (c,c), (d,d)\) وجود دارند. \(\checkmark\) ۲.
\textbf{تقارنی:}
\begin{itemize}
\tightlist
\item
  برای \((a,b)\)، معکوسش \((b,a)\) وجود دارد.
\item
  برای \((c,d)\)، معکوسش \((d,c)\) وجود دارد.
\item
  زوج‌های قطری (مثل \(a,a\)) متقارن خودشان هستند. \(\checkmark\) ۳.
  \textbf{تعدی:}
\item
  ترکیب \((a,b)\) و \((b,a)\) می‌دهد \((a,a)\) که موجود است.
\item
  ترکیب \((c,d)\) و \((d,c)\) می‌دهد \((c,c)\) که موجود است.
\item
  سایر ترکیب‌ها هم بررسی می‌شوند و برقرارند. \(\checkmark\) پس
  \(\mathcal{R}\) یک رابطه هم‌ارزی است .
\end{itemize}
\textbf{قسمت (ب): یافتن افراز (کلاس‌های هم‌ارزی)} باید ببینیم هر عضو با چه
کسانی رابطه دارد:
\begin{itemize}
\tightlist
\item
  کلاس \(a\): تمام عناصری که با \(a\) در رابطه‌اند
  \(\to [a]_\mathcal{R} = \{a, b\}\)
\item
  کلاس \(b\): تمام عناصری که با \(b\) در رابطه‌اند
  \(\to [b]_\mathcal{R} = \{a, b\}\) (تکراری)
\item
  کلاس \(c\): تمام عناصری که با \(c\) در رابطه‌اند
  \(\to [c]_\mathcal{R} = \{c, d\}\)
\item
  کلاس \(d\): تمام عناصری که با \(d\) در رابطه‌اند
  \(\to [d]_\mathcal{R} = \{c, d\}\) (تکراری)
\end{itemize}
بنابراین افراز نهایی شامل کلاس‌های متمایز است:
\[X/\mathcal{R} = \{ \{a,b\}, \{c,d\} \}\]
\end{info}
