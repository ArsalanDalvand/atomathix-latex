% ---------------------------------------------------------------------
% Copyright (c) 2026 Arsalan Dalvand & Reyhaneh Darvishi.
% Licensed under CC BY-NC-SA 4.0.
% See LICENSE file for details.
% ---------------------------------------------------------------------

\section{تمرین ۵: ساختن رابطه بازتابی و متعدی اما
غیرمتقارن}\label{تمرین-۵---مثال-بازتابی-و-متعدی-غیرمتقارن}
\subsection{۱. صورت
سوال}\label{ux635ux648ux631ux62a-ux633ux648ux627ux644}
رابطه‌ای مثال بزنید که انعکاسی (بازتابی) و متعدی باشد، اما متقارن نباشد.
\subsection{۲. حل}\label{ux62dux644}
رابطه «کوچکتر مساوی» (\(\le\)) یا «زیرمجموعه بودن» (\(\subseteq\))
بهترین مثال‌های ذهنی هستند. روی مجموعه \(A=\{a,b\}\):
\[\mathcal{R} = \{(a,a), (b,b), (a,b)\}\]
\begin{itemize}
\tightlist
\item
  \textbf{انعکاسی:} \((a,a), (b,b)\) دارد.
\item
  \textbf{متعدی:} تنها ترکیب ممکن \((a,a)\wedge(a,b)\to(a,b)\) است که
  برقرار است.
\item
  \textbf{غیرمتقارن:} \((a,b)\) هست ولی \((b,a)\) نیست.
\end{itemize}
