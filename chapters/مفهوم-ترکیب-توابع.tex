% ---------------------------------------------------------------------
% Copyright (c) 2026 Arsalan Dalvand & Reyhaneh Darvishi.
% Licensed under CC BY-NC-SA 4.0.
% See LICENSE file for details.
% ---------------------------------------------------------------------

\section{\texorpdfstring{مفهوم ترکیب توابع
\lr{(Function Composition)}}{مفهوم ترکیب توابع }}\label{مفهوم-ترکیب-توابع}
\begin{tldr}{خلاصه سریع}
ترکیب توابع یعنی اتصال سریالی دو ماشین پردازشگر. خروجی ماشین اول،
مستقیماً به عنوان ورودی وارد ماشین دوم می‌شود. اگر \(f\) کارش «شستن» و
\(g\) کارش «خشک کردن» باشد، \(g \circ f\) ماشینی است که لباس چرک را
می‌گیرد و لباس شسته و خشک شده تحویل می‌دهد.
\end{tldr}
\subsection{۱. درک شهودی: آنالیز
ماشین‌ها}\label{ux62fux631ux6a9-ux634ux647ux648ux62fux6cc-ux622ux646ux627ux644ux6ccux632-ux645ux627ux634ux6ccux646ux647ux627}
بهترین راه برای درک ترکیب توابع، تصور آن‌ها به عنوان \textbf{ماشین} است:
فرض کنید دو ماشین داریم:
\begin{enumerate}
\def\labelenumi{\arabic{enumi}.}
\tightlist
\item
  \textbf{ماشین \(f\) (لباسشویی):} لباس چرک (\(x\)) را می‌گیرد و لباس خیس
  و تمیز (\(f(x)\)) تحویل می‌دهد.
  \begin{itemize}
  \tightlist
  \item
    \(f: X \to Y\)
  \end{itemize}
\item
  \textbf{ماشین \(g\) (خشک‌کن):} لباس خیس (\(y\)) را می‌گیرد و لباس خشک و
  تمیز (\(g(y)\)) تحویل می‌دهد.
  \begin{itemize}
  \tightlist
  \item
    \(g: Y \to Z\)
  \end{itemize}
\end{enumerate}
اگر این دو ماشین را به هم وصل کنیم (خروجی اولی به ورودی دومی)، یک ماشین
جدید و پیشرفته \(h\) ساخته‌ایم که لباس چرک می‌گیرد و لباس آماده پوشیدن
تحویل می‌دهد. این ماشین جدید را \textbf{ترکیب \(g\) با \(f\)} می‌نامیم و
با نماد \textbf{\(g \circ f\)} نشان می‌دهیم.
\begin{warning}{نکته مهم در نمادگذاری}
در نماد \(g \circ f\)، با اینکه \(g\) سمت چپ نوشته شده، اما در عمل
\textbf{دومین} تابعی است که اجرا می‌شود.
\[(\underbrace{g}_{\text{دوم}} \circ \underbrace{f}_{\text{اول}})(x) = g(f(x))\]
چون \(x\) ابتدا وارد \(f\) می‌شود.
\end{warning}
\begin{center}\rule{0.5\linewidth}{0.5pt}\end{center}
\subsection{\texorpdfstring{۲. تعریف ریاضی
\lr{(Definition }13)}{۲. تعریف ریاضی 13)}}\label{ux62aux639ux631ux6ccux641-ux631ux6ccux627ux636ux6cc-definition-13}
فرض کنید \(f: X \to Y\) و \(g: Y \to Z\) دو تابع باشند (دقت کنید که
هم‌دامنه اولی باید با دامنه دومی سازگار باشد).
\begin{tldr}{تعریف ترکیب}
تابع \(g \circ f: X \to Z\) به صورت زیر تعریف می‌شود:
\[\forall x \in X, \quad (g \circ f)(x) = g(f(x))\]
\end{tldr}
\subsubsection{تعریف مجموعه‌ای
(دقیق)}\label{ux62aux639ux631ux6ccux641-ux645ux62cux645ux648ux639ux647ux627ux6cc-ux62fux642ux6ccux642}
از آنجا که تابع مجموعه‌ای از زوج‌های مرتب است، تعریف دقیق مجموعه‌ای
\(g \circ f\) چنین است:
\[g \circ f = \{ (x, z) \in X \times Z \mid \exists y \in Y : (x, y) \in f \wedge (y, z) \in g \}\]
\emph{(ترجمه: زوج \((x,z)\) در ترکیب است اگر واسطه‌ای مثل \(y\) وجود
داشته باشد که \(x\) را به \(y\) (توسط \(f\)) و \(y\) را به \(z\) (توسط
\(g\)) وصل کند).}
\begin{center}\rule{0.5\linewidth}{0.5pt}\end{center}
\subsection{\texorpdfstring{۳. شبکه ارتباطی با سایر قضایا
\lr{(Analytic Map)}}{۳. شبکه ارتباطی با سایر قضایا }}\label{ux634ux628ux6a9ux647-ux627ux631ux62aux628ux627ux637ux6cc-ux628ux627-ux633ux627ux6ccux631-ux642ux636ux627ux6ccux627-analytic-map}
\subsubsection{\texorpdfstring{۱. ارتباط با
\autoref{مفهوم-تابع-و-قضیه-۶---تعریف-و-دامنه}}{۱. ارتباط با }}\label{ux627ux631ux62aux628ux627ux637-ux628ux627-ux645ux641ux647ux648ux645-ux62aux627ux628ux639-ux648-ux642ux636ux6ccux647-ux6f6---ux62aux639ux631ux6ccux641-ux648-ux62fux627ux645ux646ux647}
\begin{itemize}
\tightlist
\item
  \textbf{شرط وجود:} برای اینکه \(g \circ f\) قابل تعریف باشد، خروجی‌های
  \(f\) (یعنی \(Im(f)\)) باید حتماً زیرمجموعه‌ای از ورودی‌های مجاز \(g\)
  (یعنی \(Dom(g)\)) باشند. اگر \(Range(f) \not\subseteq Dom(g)\)، ماشین
  دوم ممکن است ورودی نامعتبر دریافت کند و خراب شود.
\end{itemize}
\subsubsection{\texorpdfstring{۲. پیش‌نیاز
\autoref{قضیه-۱۵---شرکت‌پذیری-ترکیب-توابع}}{۲. پیش‌نیاز }}\label{ux67eux6ccux634ux646ux6ccux627ux632-ux642ux636ux6ccux647-ux6f1ux6f5---ux634ux631ux6a9ux62aux67eux630ux6ccux631ux6cc-ux62aux631ux6a9ux6ccux628-ux62aux648ux627ux628ux639}
\begin{itemize}
\tightlist
\item
  \textbf{مقدمه:} در قضایای بعدی خواهیم دید که ترکیب توابع خاصیت جابجایی
  ندارد (\(f \circ g \neq g \circ f\))، اما خاصیت شرکت‌پذیری دارد
  (\(h \circ (g \circ f) = (h \circ g) \circ f\)).
\end{itemize}
\subsubsection{\texorpdfstring{۳. ارتباط با
\autoref{قضیه-۱۴---وجود-و-ویژگی‌های-تابع-وارون}}{۳. ارتباط با }}\label{ux627ux631ux62aux628ux627ux637-ux628ux627-ux642ux636ux6ccux647-ux6f1ux6f4---ux648ux62cux648ux62f-ux648-ux648ux6ccux698ux6afux6ccux647ux627ux6cc-ux62aux627ux628ux639-ux648ux627ux631ux648ux646}
\begin{itemize}
\tightlist
\item
  \textbf{خنثی‌سازی:} مفهوم «تابع وارون» دقیقاً بر عکس کردنِ عملیات ترکیب
  است. اگر \(f\) لباسی را بشوید، \(f^{-1}\) باید بتواند آن را دوباره چرک
  کند! رابطه آن‌ها چنین است:
  \[f^{-1} \circ f = I_X \quad (\text{تابع همانی})\]
\end{itemize}
