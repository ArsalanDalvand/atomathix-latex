% ---------------------------------------------------------------------
% Copyright (c) 2026 Arsalan Dalvand & Reyhaneh Darvishi.
% Licensed under CC BY-NC-SA 4.0.
% See LICENSE file for details.
% ---------------------------------------------------------------------

\section{\texorpdfstring{تعریف مجموعه توانی
\lr{(Power Set)}}{تعریف مجموعه توانی }}\label{پیشنیاز---تعریف-مجموعه-توانی}
\begin{tldr}{خلاصه سریع}
مجموعه‌ای که اعضایش، «تمام زیرمجموعه‌های ممکن» مجموعه اصلی هستند. آن را با
\(\mathcal{P}(A)\) یا \(2^A\) نشان می‌دهند.
\end{tldr}
\subsection{۱. مثال
ساده}\label{ux645ux62bux627ux644-ux633ux627ux62fux647}
اگر \(A = \{1, 2\}\) باشد، زیرمجموعه‌هایش عبارتند از:
\begin{itemize}
\tightlist
\item
  \(\emptyset\) (هیچ‌کدام)
\item
  \(\{1\}\)
\item
  \(\{2\}\)
\item
  \(\{1, 2\}\) (هر دو) پس مجموعه توانی می‌شود:
  \[\mathcal{P}(A) = \{ \emptyset, \{1\}, \{2\}, \{1, 2\} \}\]
\end{itemize}
\subsection{۲. تعریف
ریاضی}\label{ux62aux639ux631ux6ccux641-ux631ux6ccux627ux636ux6cc}
\begin{tldr}{تعریف}
\[\mathcal{P}(A) = \{ S \mid S \subseteq A \}\]
\end{tldr}
