% ---------------------------------------------------------------------
% Copyright (c) 2026 Arsalan Dalvand & Reyhaneh Darvishi.
% Licensed under CC BY-NC-SA 4.0.
% See LICENSE file for details.
% ---------------------------------------------------------------------

\section{\texorpdfstring{تمرین ۱۵: بستار متقارن
(\(\mathcal{R} \cup \mathcal{R}^{-1}\))}{تمرین ۱۵: بستار متقارن (\textbackslash mathcal\{R\} \textbackslash cup \textbackslash mathcal\{R\}\^{}\{-1\})}}\label{تمرین-۱۵---بستار-متقارن-(Symmetric-Closure)}
\begin{tldr}{مفهوم}
اگر یک رابطه متقارن نباشد، با اضافه کردن وارونش به خودش، آن را متقارن
می‌کنیم. این «کوچکترین» رابطه متقارنی است که شامل رابطه اصلی است.
\end{tldr}
\subsection{۱. صورت
سوال}\label{ux635ux648ux631ux62a-ux633ux648ux627ux644}
ثابت کنید \(\mathcal{T} = \mathcal{R} \cup \mathcal{R}^{-1}\): ۱. یک
رابطه متقارن است. ۲. اگر \(\mathcal{S}\) هر رابطه متقارنی باشد که
\(\mathcal{R} \subseteq \mathcal{S}\)، آنگاه
\(\mathcal{T} \subseteq \mathcal{S}\) (یعنی \(\mathcal{T}\) کوچکترین
است).
\subsection{۲. اثبات}\label{ux627ux62bux628ux627ux62a}
\textbf{۱. متقارن بودن:}
\[(\mathcal{R} \cup \mathcal{R}^{-1})^{-1} = \mathcal{R}^{-1} \cup (\mathcal{R}^{-1})^{-1} = \mathcal{R}^{-1} \cup \mathcal{R} = \mathcal{R} \cup \mathcal{R}^{-1}\]
چون وارونش با خودش برابر شد، پس متقارن است.
\textbf{۲. کوچکترین بودن:} فرض کنیم \(\mathcal{S}\) متقارن باشد
(\(\mathcal{S}=\mathcal{S}^{-1}\)) و
\(\mathcal{R} \subseteq \mathcal{S}\). چون
\(\mathcal{R} \subseteq \mathcal{S}\)، پس
\(\mathcal{R}^{-1} \subseteq \mathcal{S}^{-1} = \mathcal{S}\). حالا چون
هم \(\mathcal{R}\) و هم \(\mathcal{R}^{-1}\) زیرمجموعه \(\mathcal{S}\)
هستند، اجتماعشان هم هست:
\[\mathcal{R} \cup \mathcal{R}^{-1} \subseteq \mathcal{S}\]
