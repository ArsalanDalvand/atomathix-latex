% ---------------------------------------------------------------------
% Copyright (c) 2026 Arsalan Dalvand & Reyhaneh Darvishi.
% Licensed under CC BY-NC-SA 4.0.
% See LICENSE file for details.
% ---------------------------------------------------------------------

\section{توابع خاص: تابع همانی و تابع
ثابت}\label{توابع-خاص---همانی-و-ثابت}
\begin{tldr}{خلاصه سریع}
در دنیای توابع، دو بازیگر بسیار مهم وجود دارند: ۱. \textbf{تابع همانی
(\(I_X\)):} آینه‌ای که هر کس را به خودش نشان می‌دهد. ۲. \textbf{تابع ثابت
(\(C_c\)):} سیاه‌چاله‌ای که همه ورودی‌ها را به یک نقطه واحد می‌فرستد.
\end{tldr}
\subsection{\texorpdfstring{۱. تابع همانی
\lr{(Identity Function)}}{۱. تابع همانی }}\label{ux62aux627ux628ux639-ux647ux645ux627ux646ux6cc-identity-function}
\subsubsection{الف) درک
شهودی}\label{ux627ux644ux641-ux62fux631ux6a9-ux634ux647ux648ux62fux6cc}
این «بی‌اثرترین» تابع ممکن است. مثل ضرب در عدد ۱ یا جمع با عدد ۰. هر چه
به آن بدهید، همان را پس می‌گیرید.
\subsubsection{ب) تعریف
ریاضی}\label{ux628-ux62aux639ux631ux6ccux641-ux631ux6ccux627ux636ux6cc}
فرض کنید \(X\) یک مجموعه باشد. تابع همانی روی \(X\) را با \(I_X\) (یا
\(1_X\)) نمایش می‌دهیم.
\begin{tldr}{تعریف}
تابع \(I_X: X \to X\) به صورت زیر تعریف می‌شود:
\[\forall x \in X, \quad I_X(x) = x\] یا به زبان زوج مرتب:
\[I_X = \{ (x, x) \mid x \in X \}\] (دقت کنید که این همان \textbf{رابطه
قطری} \(\Delta_X\) است).
\end{tldr}
\subsubsection{ج) ویژگی مهم (عنصر
خنثی)}\label{ux62c-ux648ux6ccux698ux6afux6cc-ux645ux647ux645-ux639ux646ux635ux631-ux62eux646ux62bux6cc}
تابع همانی نقش «عنصر خنثی» در عمل ترکیب توابع را دارد. برای هر تابع
\(f: X \to Y\): \[f \circ I_X = f \quad \text{و} \quad I_Y \circ f = f\]
\begin{center}\rule{0.5\linewidth}{0.5pt}\end{center}
\subsection{\texorpdfstring{۲. تابع ثابت
\lr{(Constant Function)}}{۲. تابع ثابت }}\label{ux62aux627ux628ux639-ux62bux627ux628ux62a-constant-function}
\subsubsection{الف) درک
شهودی}\label{ux627ux644ux641-ux62fux631ux6a9-ux634ux647ux648ux62fux6cc-1}
این تابع تمام تنوع ورودی‌ها را نادیده می‌گیرد و همه را مجبور می‌کند به یک
خروجی خاص تبدیل شوند. برد \lr{(Range) }این تابع همیشه تک‌عضوی است.
\lr{[Image of constant function graph horizontal line]}
\subsubsection{ب) تعریف
ریاضی}\label{ux628-ux62aux639ux631ux6ccux641-ux631ux6ccux627ux636ux6cc-1}
فرض کنید \(f: X \to Y\) یک تابع باشد. \(f\) را تابع ثابت می‌نامیم اگر
عنصری مانند \(c \in Y\) وجود داشته باشد که:
\begin{tldr}{تعریف}
\[\exists c \in Y, \forall x \in X, \quad f(x) = c\]
\end{tldr}
\subsubsection{ج) مثال}\label{ux62c-ux645ux62bux627ux644}
اگر \(X = \mathbb{R}\) و \(f(x) = 5\)، این یک تابع ثابت است. نمودار آن
خطی موازی محور افقی است.
\begin{center}\rule{0.5\linewidth}{0.5pt}\end{center}
\subsection{\texorpdfstring{۳. شبکه ارتباطی با سایر قضایا
\lr{(Analytic Map)}}{۳. شبکه ارتباطی با سایر قضایا }}\label{ux634ux628ux6a9ux647-ux627ux631ux62aux628ux627ux637ux6cc-ux628ux627-ux633ux627ux6ccux631-ux642ux636ux627ux6ccux627-analytic-map}
\subsubsection{\texorpdfstring{۱. ارتباط با
\autoref{تمرین-۱۵---تابع-بازتابی}}{۱. ارتباط با }}\label{ux627ux631ux62aux628ux627ux637-ux628ux627-ux62aux645ux631ux6ccux646-ux6f1ux6f5---ux62aux627ux628ux639-ux628ux627ux632ux62aux627ux628ux6cc}
\begin{itemize}
\tightlist
\item
  \textbf{تابع همانی:} در تمرین ۱۵ ثابت کردیم که اگر تابعی خاصیت بازتابی
  داشته باشد (یعنی \((x,x) \in f\))، آن تابع لزوماً همان \textbf{تابع
  همانی} (\(I_X\)) است. \(I_X\) تنها تابعی است که تماماً روی قطر اصلی
  قرار دارد.
\end{itemize}
\subsubsection{\texorpdfstring{۲. ارتباط با
\autoref{قضیه-۱۶---وارون‌های-یک‌طرفه}}{۲. ارتباط با }}\label{ux627ux631ux62aux628ux627ux637-ux628ux627-ux642ux636ux6ccux647-ux6f1ux6f6---ux648ux627ux631ux648ux646ux647ux627ux6cc-ux6ccux6a9ux637ux631ux641ux647}
\begin{itemize}
\tightlist
\item
  \textbf{نقش کلیدی \(I_X\):} در قضایای مربوط به وارون‌پذیری (که در ادامه
  می‌خوانیم)، شرط اینکه \(g\) وارون چپ \(f\) باشد، این است که ترکیب آن‌ها
  تابع همانی شود (\(g \circ f = I_X\)). \lr{[cite\_start]بدون }تعریف
  دقیق \(I_X\)، تعریف وارون تابع ممکن \lr{نیست[cite: }93{]}.
\end{itemize}
\subsubsection{\texorpdfstring{۳. ارتباط با
\autoref{مفهوم-تابع-و-قضیه-۶---تعریف-و-دامنه}}{۳. ارتباط با }}\label{ux627ux631ux62aux628ux627ux637-ux628ux627-ux645ux641ux647ux648ux645-ux62aux627ux628ux639-ux648-ux642ux636ux6ccux647-ux6f6---ux62aux639ux631ux6ccux641-ux648-ux62fux627ux645ux646ux647}
\begin{itemize}
\tightlist
\item
  \textbf{تفاوت در برد:}
  \begin{itemize}
  \tightlist
  \item
    در تابع همانی، برد دقیقاً برابر با دامنه است (\(R_f = D_f = X\)).
  \item
    در تابع ثابت، برد کوچک‌ترین اندازه ممکن را دارد (تک‌عضوی)، در حالی که
    دامنه می‌تواند بسیار بزرگ (حتی نامتناهی) باشد.
  \end{itemize}
\end{itemize}
