% ---------------------------------------------------------------------
% Copyright (c) 2026 Arsalan Dalvand & Reyhaneh Darvishi.
% Licensed under CC BY-NC-SA 4.0.
% See LICENSE file for details.
% ---------------------------------------------------------------------

\section{تمرین ۴: حاصلضرب دکارتی چه زمانی تهی
است؟}\label{تمرین-۴---شرط-تهی-بودن-حاصلضرب}
\subsection{۱. صورت
سوال}\label{ux635ux648ux631ux62a-ux633ux648ux627ux644}
ثابت کنید:
\[A \times B = \emptyset \iff (A = \emptyset \lor B = \emptyset)\]
\subsection{۲. حل
تشریحی}\label{ux62dux644-ux62aux634ux631ux6ccux62dux6cc}
این اثبات دو طرفه است (\(\Rightarrow\) و \(\Leftarrow\)).
\begin{info}{اثبات}
\textbf{طرف اول (\(\Leftarrow\)):} اگر \(A = \emptyset\) یا
\(B = \emptyset\) باشد، هیچ زوج مرتبی \((x,y)\) نمی‌توان ساخت که
\(x \in A\) و \(y \in B\) باشد (چون یکی از آن‌ها عضو ندارد). پس
\(A \times B = \emptyset\).
\textbf{طرف دوم (\(\Rightarrow\)):} (از برهان خلف استفاده می‌کنیم). فرض
کنیم \(A \times B = \emptyset\) باشد اما حکم غلط باشد (یعنی هم
\(A \neq \emptyset\) و هم \(B \neq \emptyset\)). چون \(A\) ناتهی است، پس
عنصری مثل \(x\) دارد. چون \(B\) ناتهی است، پس عنصری مثل \(y\) دارد. پس
زوج \((x,y)\) وجود دارد که در \(A \times B\) باشد. این یعنی
\(A \times B \neq \emptyset\) که با فرض تناقض دارد. پس حتماً باید
\(A=\emptyset\) یا \(B=\emptyset\) باشد.
\end{info}
