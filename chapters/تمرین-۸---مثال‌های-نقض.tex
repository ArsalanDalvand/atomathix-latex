% ---------------------------------------------------------------------
% Copyright (c) 2026 Arsalan Dalvand & Reyhaneh Darvishi.
% Licensed under CC BY-NC-SA 4.0.
% See LICENSE file for details.
% ---------------------------------------------------------------------

\section{تمرین ۸: بررسی گزاره‌ها با مثال
نقض}\label{تمرین-۸---مثال‌های-نقض}
\subsection{۱. صورت
سوال}\label{ux635ux648ux631ux62a-ux633ux648ux627ux644}
\lr{[cite\_start]درستی }یا نادرستی حکم‌های زیر را با آوردن مثال نقض بررسی
\lr{کنید[cite: }693{]}: الف)
\(A \times B \subseteq C \times D \iff A \subseteq C \land B \subseteq D\)
ب) \(\mathcal{P}(A \times B) = \mathcal{P}(A) \times \mathcal{P}(B)\)
\subsection{۲. حل
تشریحی}\label{ux62dux644-ux62aux634ux631ux6ccux62dux6cc}
\subsubsection{الف) بررسی شرط زیرمجموعه
بودن}\label{ux627ux644ux641-ux628ux631ux631ux633ux6cc-ux634ux631ux637-ux632ux6ccux631ux645ux62cux645ux648ux639ux647-ux628ux648ux62fux646}
این حکم \textbf{نادرست} است.
\begin{itemize}
\tightlist
\item
  \textbf{دلیل:} اگر یکی از مجموعه‌ها (مثلاً \(B\)) تهی باشد، حاصلضرب
  دکارتی تهی می‌شود و تهی زیرمجموعه هر چیزی است، حتی اگر شرط
  \(A \subseteq C\) برقرار نباشد.
\item
  \textbf{مثال نقض:} فرض کنید \(A=\{1, 2\}, B=\emptyset\) و
  \(C=\{1\}, D=\{a\}\). \[A \times B = \emptyset\]
  \[\emptyset \subseteq C \times D\] (این درست است) اما
  \(A \nsubseteq C\) (چون ۲ در \(A\) هست ولی در \(C\) نیست). پس شرط
  برقرار نیست.
\end{itemize}
\subsubsection{ب) مجموعه توانی
حاصلضرب}\label{ux628-ux645ux62cux645ux648ux639ux647-ux62aux648ux627ux646ux6cc-ux62dux627ux635ux644ux636ux631ux628}
این حکم \textbf{نادرست} است.
\begin{itemize}
\tightlist
\item
  \textbf{دلیل:} جنس اعضای دو طرف فرق می‌کند. \(\mathcal{P}(A \times B)\)
  شامل زیرمجموعه‌هایی از زوج‌مرتب‌هاست، اما
  \(\mathcal{P}(A) \times \mathcal{P}(B)\) شامل زوج‌مرتب‌هایی از
  مجموعه‌هاست. همچنین تعداد اعضایشان برابر نیست.
\item
  \textbf{تحلیل تعدادی:} اگر \(|A|=m, |B|=n\):
  \begin{itemize}
  \tightlist
  \item
    تعداد اعضای سمت راست: \(2^m \times 2^n = 2^{m+n}\)
  \item
    تعداد اعضای سمت چپ: \(2^{mn}\)
  \item
    معمولاً \(2^{mn} \neq 2^{m+n}\) (مثلاً \(m=n=2 \to 16 \neq 256\)) .
  \end{itemize}
\end{itemize}
