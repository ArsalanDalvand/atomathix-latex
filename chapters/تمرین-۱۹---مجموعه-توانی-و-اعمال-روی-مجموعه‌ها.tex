% ---------------------------------------------------------------------
% Copyright (c) 2026 Arsalan Dalvand & Reyhaneh Darvishi.
% Licensed under CC BY-NC-SA 4.0.
% See LICENSE file for details.
% ---------------------------------------------------------------------

\section{تمرین ۱۹: رفتار مجموعه توانی با اشتراک و
اجتماع}\label{تمرین-۱۹---مجموعه-توانی-و-اعمال-روی-مجموعه‌ها}
\begin{tldr}{خلاصه سریع}
مجموعه توانی (\(\mathcal{P}\)) با \textbf{اشتراک} دوست است (تساوی دارد)،
اما با \textbf{اجتماع} سر ناسازگاری دارد (تساوی ندارد).
\begin{itemize}
\tightlist
\item
  الف) \(\mathcal{P}(A) \cap \mathcal{P}(B) = \mathcal{P}(A \cap B)\) ✅
\item
  ب) \(\mathcal{P}(A) \cup \mathcal{P}(B) \neq \mathcal{P}(A \cup B)\)
  ❌
\end{itemize}
\end{tldr}
\subsection{۱. حل قسمت (الف) -
اشتراک}\label{ux62dux644-ux642ux633ux645ux62a-ux627ux644ux641---ux627ux634ux62aux631ux627ux6a9}
\begin{info}{اثبات تساوی}
باید نشان دهیم \(X\) عضو سمت چپ است اگر و تنها اگر عضو سمت راست باشد.
\[X \in \mathcal{P}(A) \cap \mathcal{P}(B)\]
\(\equiv X \in \mathcal{P}(A) \land X \in \mathcal{P}(B)\) (تعریف
اشتراک) \(\equiv X \subseteq A \land X \subseteq B\) (تعریف مجموعه
توانی) \(\equiv X \subseteq (A \cap B)\) (اگر مجموعه‌ای زیرمجموعه دو
مجموعه باشد، زیرمجموعه اشتراک آن‌هاست)
\(\equiv X \in \mathcal{P}(A \cap B)\) (تعریف مجموعه توانی)
پس حکم \textbf{درست} است.
\end{info}
\subsection{۲. حل قسمت (ب) -
اجتماع}\label{ux62dux644-ux642ux633ux645ux62a-ux628---ux627ux62cux62aux645ux627ux639}
\begin{note}{مثال نقض (اثبات نادرستی)}
فرض کنید:
\begin{itemize}
\tightlist
\item
  \(A = \{1\}\)
\item
  \(B = \{2\}\)
\end{itemize}
\textbf{سمت چپ:}
\begin{itemize}
\tightlist
\item
  \(\mathcal{P}(A) = \{\emptyset, \{1\}\}\)
\item
  \(\mathcal{P}(B) = \{\emptyset, \{2\}\}\)
\item
  اجتماع آن‌ها: \(\{\emptyset, \{1\}, \{2\}\}\)
\end{itemize}
\textbf{سمت راست:}
\begin{itemize}
\tightlist
\item
  \(A \cup B = \{1, 2\}\)
\item
  \(\mathcal{P}(A \cup B) = \{\emptyset, \{1\}, \{2\}, \{1, 2\}\}\)
\end{itemize}
\textbf{نتیجه:} مجموعه \(\{1, 2\}\) در سمت راست هست اما در سمت چپ نیست.
پس حکم \textbf{نادرست} است.
\[\mathcal{P}(A) \cup \mathcal{P}(B) \subseteq \mathcal{P}(A \cup B)\]
(رابطه فقط به صورت زیرمجموعه برقرار است، نه تساوی).
\end{note}
