% ---------------------------------------------------------------------
% Copyright (c) 2026 Arsalan Dalvand & Reyhaneh Darvishi.
% Licensed under CC BY-NC-SA 4.0.
% See LICENSE file for details.
% ---------------------------------------------------------------------

\section{\texorpdfstring{مفهوم پارادوکس راسل
\lr{(Russell’s Paradox)}}{مفهوم پارادوکس راسل }}\label{مفهوم-پارادوکس-راسل}
\begin{tldr}{خلاصه سریع}
پارادوکس راسل نشان می‌دهد که «اصل تصریح نامحدود»
\lr{(Unrestricted Comprehension) }در نظریه مجموعه‌های کانتور منجر به
تناقض می‌شود. اگر اجازه دهیم هر ویژگی دلخواهی یک مجموعه بسازد، می‌توانیم
مجموعه‌ای مثل \(R\) بسازیم که همزمان باید «عضو خودش باشد» و «عضو خودش
نباشد».
\end{tldr}
\subsection{۱. ساختار صوری
پارادوکس}\label{ux633ux627ux62eux62aux627ux631-ux635ux648ux631ux6cc-ux67eux627ux631ux627ux62fux648ux6a9ux633}
در نظریه مجموعه‌های کلاسیک \lr{(Naive Set Theory)، }فرض بر این بود که
برای هر ویژگی \(P(x)\)، مجموعه‌ای وجود دارد شامل تمام عناصری که در \(P\)
صدق می‌کنند: \(\{x \mid P(x)\}\). برتراند راسل در سال ۱۹۰۲ با تعریف ویژگی
\(P(x): x \notin x\) این فرض را به چالش کشید.
مجموعه \(R\) را به صورت زیر تعریف می‌کنیم:
\[R = \{ x \mid x \notin x \}\] \emph{(مجموعه تمام مجموعه‌هایی که عضو
خودشان نیستند)}.
\subsection{۲. تحلیل منطقی (تضاد
درونی)}\label{ux62aux62dux644ux6ccux644-ux645ux646ux637ux642ux6cc-ux62aux636ux627ux62f-ux62fux631ux648ux646ux6cc}
سوال اساسی این است: \textbf{آیا \(R \in R\)؟} برای پاسخ، دو حالت ممکن در
منطق دوارزشی را بررسی می‌کنیم:
\subsubsection{\texorpdfstring{حالت اول: فرض کنیم
\(R \in R\)}{حالت اول: فرض کنیم R \textbackslash in R}}\label{ux62dux627ux644ux62a-ux627ux648ux644-ux641ux631ux636-ux6a9ux646ux6ccux645-r-in-r}
\begin{enumerate}
\def\labelenumi{\arabic{enumi}.}
\tightlist
\item
  اگر \(R \in R\) باشد، طبق تعریف مجموعه \(R\)، باید ویژگی شرطی مجموعه
  (\(x \notin x\)) را داشته باشد.
\item
  بنابراین \(R \notin R\).
\item
  نتیجه: \((R \in R) \implies (R \notin R)\). (تناقض)
\end{enumerate}
\subsubsection{\texorpdfstring{حالت دوم: فرض کنیم
\(R \notin R\)}{حالت دوم: فرض کنیم R \textbackslash notin R}}\label{ux62dux627ux644ux62a-ux62fux648ux645-ux641ux631ux636-ux6a9ux646ux6ccux645-r-notin-r}
\begin{enumerate}
\def\labelenumi{\arabic{enumi}.}
\tightlist
\item
  اگر \(R \notin R\) باشد، پس \(R\) ویژگی لازم برای عضویت در مجموعه
  \(R\) (که همان عضو خود نبودن است) را دارد.
\item
  بنابراین \(R \in R\).
\item
  نتیجه: \((R \notin R) \implies (R \in R)\). (تناقض)
\end{enumerate}
\subsubsection{نتیجه
نهایی}\label{ux646ux62aux6ccux62cux647-ux646ux647ux627ux6ccux6cc}
گزاره \((R \in R) \iff (R \notin R)\) یک تناقض منطقی
\lr{(Contradiction) }است که نشان می‌دهد سیستم اصل موضوعی ما ایراد دارد.
\subsection{\texorpdfstring{۳. شبکه ارتباطی با سایر قضایا
\lr{(Analytic Map)}}{۳. شبکه ارتباطی با سایر قضایا }}\label{ux634ux628ux6a9ux647-ux627ux631ux62aux628ux627ux637ux6cc-ux628ux627-ux633ux627ux6ccux631-ux642ux636ux627ux6ccux627-analytic-map}
این پارادوکس نقطه عطفی است که نیاز به بازبینی در تعاریف پایه (فصل ۱ و ۲)
را ایجاد می‌کند:
\subsubsection{\texorpdfstring{۱. ارتباط با
\autoref{قضیه-۱۰---عدم-وجود-مجموعه-جهانی}}{۱. ارتباط با }}\label{ux627ux631ux62aux628ux627ux637-ux628ux627-ux642ux636ux6ccux647-ux6f1ux6f0---ux639ux62fux645-ux648ux62cux648ux62f-ux645ux62cux645ux648ux639ux647-ux62cux647ux627ux646ux6cc}
\begin{itemize}
\tightlist
\item
  \textbf{نتیجه مستقیم:} پارادوکس راسل ابزار اصلی اثبات
  \autoref{قضیه-۱۰---عدم-وجود-مجموعه-جهانی} است. آن قضیه نشان می‌دهد که
  برای رفع این پارادوکس، باید فرض «وجود مجموعه تمام مجموعه‌ها» را کنار
  بگذاریم.
\end{itemize}
\subsubsection{\texorpdfstring{۲. ارتباط با
\autoref{قضیه-۵---متمم-و-زیرمجموعه} (قانون عکس
نقیض)}{۲. ارتباط با  (قانون عکس نقیض)}}\label{ux627ux631ux62aux628ux627ux637-ux628ux627-ux642ux636ux6ccux647-ux6f5---ux645ux62aux645ux645-ux648-ux632ux6ccux631ux645ux62cux645ux648ux639ux647-ux642ux627ux646ux648ux646-ux639ux6a9ux633-ux646ux642ux6ccux636}
\begin{itemize}
\tightlist
\item
  \textbf{تحلیل ساختاری:} استدلال راسل شبیه به استفاده از قطری‌سازی
  کانتور و قانون عکس نقیض است. اگر نگاشتی وجود داشته باشد که ساختار را
  حفظ کند، با منفی کردن آن (متمم) به تناقض می‌رسیم.
\end{itemize}
\subsubsection{\texorpdfstring{۳. ارتباط با
\autoref{قضیه-۶---قواعد-استنتاج} (برهان
خلف)}{۳. ارتباط با  (برهان خلف)}}\label{ux627ux631ux62aux628ux627ux637-ux628ux627-ux642ux636ux6ccux647-ux6f6---ux642ux648ux627ux639ux62f-ux627ux633ux62aux646ux62aux627ux62c-ux628ux631ux647ux627ux646-ux62eux644ux641}
\begin{itemize}
\tightlist
\item
  \textbf{متدولوژی:} کل این پارادوکس یک مثال کلاسیک از رسیدن به
  \(p \land \sim p\) است که در منطق کلاسیک باطل است. این امر ریاضیدانان
  را مجبور کرد اصول موضوعی جدیدی (مانند \lr{ZFC) }تدوین کنند که در آن
  تشکیل مجموعه \(R\) غیرقانونی است.
\end{itemize}
